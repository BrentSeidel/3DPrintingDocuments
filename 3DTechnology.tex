\documentclass[english,10pt]{beamer}

\title{3D Printing Technologies}
\subtitle{3D Printing Technologies}
\author{Brent Seidel\\modestconsulting@gmail.com\\Modest Consulting}
\institute{Modest Consulting}
\date{\today}


\begin{document}
\begin{frame}
  \titlepage
\end{frame}

\begin{frame}
  \frametitle{Outline}
  \tableofcontents
\end{frame}

\section{Introduction}
\begin{frame}
  \frametitle{Introduction}
  There are a number of different technologies used for 3D printing.  These range from low end consumer and hobbyist units that fuse melted plastic to high end units that can make metal parts of jet and rocket engines.

  There are likely new methods being developed in the lab so this presentation can only be an overview of some of the more common methods right now.  The best way to find out about the latest developments is to search online.  

  Note that the boundaries between the technologies aren't always as cut and dried as one might expect.  There are printers that straddle the boundaries or use hybrid technologies.
\end{frame}

\section{Technologies}
\begin{frame}
  \frametitle{Technologies}
  \begin{itemize}
    \item Binder Jetting (BJ)
    \item Digital Light Processing (DLP) aka Direct Light Processing (DLP)
    \item Direct Metal Laser Sintering (DMLS)
    \item Drop on Demand (DOD) aka Material Jetting (MJ)
    \item Fused Deposition Modeling (FDM) aka Fused Filament Fabrication (FFF)
    \item Electron Beam Melting (EBM)
    \item Laminated Object Manufacturing (LOM)
    \item Stereolithography (SLA)
    \item Selective Laser Melting (SLM)
    \item Selective Laser Sintering (SLS)
  \end{itemize}
\end{frame}

\subsection{Binder Jetting (BJ)}
\begin{frame}
  \frametitle{Binder Jetting (BJ)}
  Developed in 1996, Binder Jetting (BJ) uses a thin layer of powder is placed on a platform and a binding agent is selectively sprayed to join the powder particles.  The platform is lowered by the layer height and a new layer of powder is deposited.  The advantage of this method is that no support structure is required since the unprocessed powder provides support.  Depending on the material used, post processing may be required.
\end{frame}

\subsection{Digital Light Processing (DLP) aka Direct Light Processing (DLP)}
\begin{frame}
  \frametitle{Digital Light Processing (DLP) aka Direct Light Processing (DLP)}
  Developed in 1987, it is similar to the earlier Stereolithography (SLA) method, except that instead of scanning each layer with a laser, it uses a system similar to a video projector to expose a whole layer at a time.  Because of this, it can be faster than SLA.
\end{frame}

\subsection{Drop on Demand (DOD) aka Material Jetting (MJ)}
\begin{frame}
  \frametitle{Drop on Demand (DOD) aka Material Jetting (MJ)}
  Think of a cross between an inkjet printer and Stereolithography (SLA).  This method originated in 1994.  It primarily uses a wax or photopolymer which is sprayed onto the platform.  If a photopolymer is used, it is then cured by an ultraviolet light.  This method is under development with new materials being developed for ceramic or metal objects.  It is also used is some biological applications to deposit cells to build up 3D printed tissues.  Metal or biological applications tend to be called Drop on Demand while other applications tend to be called Material Jetting.
\end{frame}

\subsection{Fused Deposition Modeling (FDM) aka Fused Filament Fabrication (FFF)}
\begin{frame}
  \frametitle{Fused Deposition Modeling (FDM) aka Fused Filament Fabrication (FFF)}
  Developed in 1988, this is the method used by most consumer and hobbyist printers.  It extrudes melted plastic to build up the object layer by layer.  Depending on the printer, a variety of different plastics are available for use.
\end{frame}

%\subsection{Electron Beam Melting (EBM)}
\begin{frame}
  \frametitle{Electron Beam Melting (EBM)}
  Filler
\end{frame}

%\subsection{Laminated Object Manufacturing (LOM)}
\begin{frame}
  \frametitle{Laminated Object Manufacturing (LOM)}
  Filler
\end{frame}

\subsection{Stereolithography (SLA)}
\begin{frame}
  \frametitle{Stereolithography (SLA)}
  This may be the first 3D printing method.  Developed in 1981, it uses a liquid photopolymer that hardens when exposed to light.  The light is provided by a laser that scans each layer.  Some processing may be required after printing.
\end{frame}

\subsection{Selective Laser Melting (SLM) aka Direct Metal Laser Sintering (DMLS)}
\begin{frame}
  \frametitle{Selective Laser Melting (SLM) aka Direct Metal Laser Sintering (DMLS)}
  Developed in 1995, this method is similar to Selective Laser Sintering (SLS), except in this case the material is fully melted instead of just being sintered.    A thin layer of powder is placed on a platform and is scanned by a high powered laser to form a layer of the object.  The platform is lowered by the layer height and a new layer of powder is deposited.  An advantage of this method is that the finished product can be made of metals that are fully fused.
\end{frame}

\subsection{Selective Laser Sintering (SLS)}
\begin{frame}
  \frametitle{Selective Laser Sintering (SLS)}
  This method has a similarity with Stereolithography (SLA), except that a powder is used instead of a liquid.  A thin layer of powder is placed on a platform and is scanned by a laser to sinter the powder grains together to form a layer of the object.  The platform is lowered by the layer height and a new layer of powder is deposited.  The advantage of this method is that no support structure is required since the unprocessed powder provides support.
\end{frame}

\section{References}
\begin{frame}
  \frametitle{References}
  For more information, please refer to the references below.
  \begin{itemize}
    \item \url{https://www.google.com/search?q=3d+printing+technologies}
    \item \url{http://3dprintingfromscratch.com/common/types-of-3d-printers-or-3d-printing-technologies-overview/}
    \item \url{https://all3dp.com/1/types-of-3d-printers-3d-printing-technology/}
  \end{itemize}
\end{frame}

\end{document}

