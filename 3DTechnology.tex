\documentclass[english,10pt]{beamer}

\title{3D Printing Technologies}
\subtitle{3D Printing Technologies}
\usepackage[yyyymmdd]{datetime}
\renewcommand{\dateseparator}{-}

\author{Brent Seidel\\modestconsulting@gmail.com\\Modest Consulting}
\institute{Modest Consulting}
\date{Copyright \today\ CC-BY-SA}


\begin{document}
\begin{frame}
  \titlepage
\end{frame}

\begin{frame}
  \frametitle{Outline}
  \tableofcontents
\end{frame}

\section{Introduction}
\begin{frame}
  \frametitle{Introduction}
  There are a number of different technologies used for 3D printing.  These range from low end consumer and hobbyist units that fuse melted plastic to high end units that can make metal parts of jet and rocket engines.

  There are likely new methods being developed in the lab so this presentation can only be an overview of some of the more common methods right now.  The best way to find out about the latest developments is to search online.  

  Note that the boundaries between the technologies aren't always as cut and dried as one might expect.  There are printers that straddle the boundaries or use hybrid technologies.
\end{frame}

\section{Technologies}
\begin{frame}
  \frametitle{Technologies}
  \begin{itemize}
    \item Binder Jetting (BJ)
    \item Digital Light Processing (DLP) aka Direct Light Processing (DLP)
    \item Direct Metal Laser Sintering (DMLS)
    \item Drop on Demand (DOD)
    \item Fused Deposition Modeling (FDM) aka Fused Filament Fabrication (FFF)
    \item Electron Beam Melting (EBM)
    \item Laminated Object Manufacturing (LOM)
    \item Material Jetting (MJ)
    \item Stereolithography (SLA)
    \item Selective Laser Melting (SLM)
    \item Selective Laser Sintering (SLS)
  \end{itemize}
\end{frame}

%\subsection{Binder Jetting (BJ)}
\begin{frame}
  \frametitle{Binder Jetting (BJ)}
  Filler
\end{frame}

\subsection{Digital Light Processing (DLP) aka Direct Light Processing (DLP)}
\begin{frame}
  \frametitle{Digital Light Processing (DLP) aka Direct Light Processing (DLP)}
  Developed in 1987, it is similar to the earlier Stereolithography (SLA) method, except that instead of scanning each layer with a laser, it uses a system similar to a video projector to expose a whole layer at a time.  Because of this, it can be faster than SLA.
\end{frame}

%\subsection{Direct Metal Laser Sintering (DMLS)}
\begin{frame}
  \frametitle{Direct Metal Laser Sintering (DMLS)}
  Filler
\end{frame}

%\subsection{Drop on Demand (DOD)}
\begin{frame}
  \frametitle{Drop on Demand (DOD)}
  Filler
\end{frame}

\subsection{Fused Deposition Modeling (FDM) aka Fused Filament Fabrication (FFF)}
\begin{frame}
  \frametitle{Fused Deposition Modeling (FDM) aka Fused Filament Fabrication (FFF)}
  Developed in 1988, this is the method used by most consumer and hobbyist printers.  It extrudes melted plastic to build up the object layer by layer.  Depending on the printer, a variety of different plastics are available for use.
\end{frame}

%\subsection{Electron Beam Melting (EBM)}
\begin{frame}
  \frametitle{Electron Beam Melting (EBM)}
  Filler
\end{frame}

%\subsection{Laminated Object Manufacturing (LOM)}
\begin{frame}
  \frametitle{Laminated Object Manufacturing (LOM)}
  Filler
\end{frame}

\subsection{Material Jetting (MJ)}
\begin{frame}
  \frametitle{Material Jetting (MJ)}
  Think of a cross between an inkjet printer and Stereolithography (SLA).  This method expels tiny drops of a photopolymer which are then cured by an ultraviolet light.
\end{frame}

\subsection{Stereolithography (SLA)}
\begin{frame}
  \frametitle{Stereolithography (SLA)}
  This may be the first 3D printing method.  Developed in 1981, it uses a liquid photopolymer that hardens when exposed to light.  The light is provided by a laser that scans each layer.  Some processing may be required after printing.
\end{frame}

%\subsection{Selective Laser Melting (SLM)}
\begin{frame}
  \frametitle{Selective Laser Melting (SLM)}
  Filler
\end{frame}

\subsection{Selective Laser Sintering (SLS)}
\begin{frame}
  \frametitle{Selective Laser Sintering (SLS)}
  This method has a similarity with Stereolithography (SLA), except that a powder is used instead of a liquid.  A thin layer of powder is placed on a platform and is scanned by a laser to for a layer of the object.  The platform is lowered by the layer height and a new layer of powder is deposited.  The advantage of this method is that no support structure is required since the unprocessed powder provides support.
\end{frame}

\section{References}
\begin{frame}
  \frametitle{References}
  For more information, please refer to the references below.
  \begin{itemize}
    \item \url{https://www.google.com/search?q=3d+printing+technologies}
    \item \url{http://3dprintingfromscratch.com/common/types-of-3d-printers-or-3d-printing-technologies-overview/}
    \item \url{https://all3dp.com/1/types-of-3d-printers-3d-printing-technology/}
  \end{itemize}
\end{frame}

\end{document}

