\documentclass[english,10pt]{beamer}

\title{3D Printing Overview}
\subtitle{3D Overview}
\author{Brent Seidel\\modestconsulting@gmail.com\\Modest Consulting}
\institute{Modest Consulting}
\date{\today}


\begin{document}
\begin{frame}
  \titlepage
\end{frame}

\begin{frame}
  \frametitle{Outline}
  \tableofcontents
\end{frame}

\section{Introduction}
\begin{frame}
  \frametitle{Introduction}
  While this presentation primarily focusses on consumer grade (Fused Deposition Modeling) FDM printers, much of it applies to other printers.  There are three basic stages in the 3D printing pipeline.
  \begin{itemize}
    \item 3D Modeling
    \item Preparing for Printing
    \item Printing
  \end{itemize}
\end{frame}

\begin{frame}
  \frametitle{3D Printing Pipeline Overview}
  \begin{figure}
    %LaTeX with PSTricks extensions
%%Creator: inkscape 0.91
%%Please note this file requires PSTricks extensions
\psset{xunit=.5pt,yunit=.5pt,runit=.5pt}
\begin{pspicture}(371.67807752,304.72465275)
{
\newrgbcolor{curcolor}{0.98823529 0.98823529 0.98823529}
\pscustom[linestyle=none,fillstyle=solid,fillcolor=curcolor]
{
\newpath
\moveto(100.78741079,141.47769842)
\lineto(260.78740126,141.47769842)
\lineto(260.78740126,81.47770557)
\lineto(100.78741079,81.47770557)
\closepath
}
}
{
\newrgbcolor{curcolor}{0 0 0}
\pscustom[linewidth=1.99999982,linecolor=curcolor]
{
\newpath
\moveto(100.78741079,141.47769842)
\lineto(260.78740126,141.47769842)
\lineto(260.78740126,81.47770557)
\lineto(100.78741079,81.47770557)
\closepath
}
}
{
\newrgbcolor{curcolor}{0 0 0}
\pscustom[linestyle=none,fillstyle=solid,fillcolor=curcolor]
{
\newpath
\moveto(155.11016146,118.1002598)
\lineto(160.91973143,118.1002598)
\curveto(162.06816886,118.1002598)(162.99395006,117.77506453)(163.69707501,117.12467398)
\curveto(164.40019997,116.48014281)(164.75176245,115.57193979)(164.75176245,114.40006493)
\curveto(164.75176245,113.39225255)(164.43828591,112.5133464)(163.81133282,111.76334649)
\curveto(163.18437973,111.01920596)(162.2205126,110.64713569)(160.91973143,110.64713569)
\lineto(156.8591848,110.64713569)
\lineto(156.8591848,105.18912853)
\lineto(155.11016146,105.18912853)
\lineto(155.11016146,118.1002598)
\closepath
\moveto(162.98516099,114.39127587)
\curveto(162.98516099,115.3404945)(162.63359851,115.98502568)(161.93047356,116.32486939)
\curveto(161.54375483,116.50650999)(161.01348142,116.59733029)(160.33965334,116.59733029)
\lineto(156.8591848,116.59733029)
\lineto(156.8591848,112.12369801)
\lineto(160.33965334,112.12369801)
\curveto(161.12480954,112.12369801)(161.76055169,112.29069018)(162.24687979,112.62467452)
\curveto(162.73906726,112.95865885)(162.98516099,113.54752597)(162.98516099,114.39127587)
\closepath
}
}
{
\newrgbcolor{curcolor}{0 0 0}
\pscustom[linestyle=none,fillstyle=solid,fillcolor=curcolor]
{
\newpath
\moveto(166.79961389,114.60221334)
\lineto(168.30254349,114.60221334)
\lineto(168.30254349,112.97623697)
\curveto(168.42559036,113.29264319)(168.72734815,113.6764322)(169.20781687,114.12760402)
\curveto(169.68828559,114.58463522)(170.2419965,114.81315082)(170.86894959,114.81315082)
\curveto(170.89824646,114.81315082)(170.94805114,114.81022113)(171.01836364,114.80436176)
\curveto(171.08867614,114.79850238)(171.20879332,114.78678363)(171.37871518,114.76920551)
\lineto(171.37871518,113.09928383)
\curveto(171.28496519,113.11686196)(171.19707457,113.12858071)(171.11504332,113.13444008)
\curveto(171.03887145,113.14029945)(170.95391052,113.14322914)(170.86016052,113.14322914)
\curveto(170.06328557,113.14322914)(169.45098092,112.88541667)(169.02324657,112.36979173)
\curveto(168.59551222,111.86002617)(168.38164505,111.27115905)(168.38164505,110.60319038)
\lineto(168.38164505,105.18912853)
\lineto(166.79961389,105.18912853)
\lineto(166.79961389,114.60221334)
\closepath
}
}
{
\newrgbcolor{curcolor}{0 0 0}
\pscustom[linestyle=none,fillstyle=solid,fillcolor=curcolor]
{
\newpath
\moveto(172.76738697,114.55826804)
\lineto(174.37578531,114.55826804)
\lineto(174.37578531,105.18912853)
\lineto(172.76738697,105.18912853)
\lineto(172.76738697,114.55826804)
\closepath
\moveto(172.76738697,118.1002598)
\lineto(174.37578531,118.1002598)
\lineto(174.37578531,116.30729126)
\lineto(172.76738697,116.30729126)
\lineto(172.76738697,118.1002598)
\closepath
}
}
{
\newrgbcolor{curcolor}{0 0 0}
\pscustom[linestyle=none,fillstyle=solid,fillcolor=curcolor]
{
\newpath
\moveto(176.77519923,114.60221334)
\lineto(178.27812883,114.60221334)
\lineto(178.27812883,113.266276)
\curveto(178.72344131,113.81705719)(179.19512096,114.21256495)(179.69316781,114.4527993)
\curveto(180.19121466,114.69303364)(180.74492556,114.81315082)(181.35430052,114.81315082)
\curveto(182.69023794,114.81315082)(183.59258164,114.34733056)(184.06133161,113.41569005)
\curveto(184.3191441,112.90592448)(184.44805034,112.17643238)(184.44805034,111.22721374)
\lineto(184.44805034,105.18912853)
\lineto(182.839652,105.18912853)
\lineto(182.839652,111.12174501)
\curveto(182.839652,111.69596369)(182.75469106,112.15885426)(182.5847692,112.51041672)
\curveto(182.30351922,113.09635415)(181.79375362,113.38932286)(181.05547242,113.38932286)
\curveto(180.68047244,113.38932286)(180.37285527,113.35123693)(180.13262091,113.27506506)
\curveto(179.69902718,113.14615883)(179.31816783,112.88834636)(178.99004285,112.50162766)
\curveto(178.72637099,112.19108082)(178.55351944,111.86881523)(178.4714882,111.5348309)
\curveto(178.39531632,111.20670593)(178.35723039,110.7350263)(178.35723039,110.119792)
\lineto(178.35723039,105.18912853)
\lineto(176.77519923,105.18912853)
\lineto(176.77519923,114.60221334)
\closepath
\moveto(180.49297245,114.83072894)
\lineto(180.49297245,114.83072894)
\closepath
}
}
{
\newrgbcolor{curcolor}{0 0 0}
\pscustom[linestyle=none,fillstyle=solid,fillcolor=curcolor]
{
\newpath
\moveto(187.11113612,117.23014272)
\lineto(188.7107454,117.23014272)
\lineto(188.7107454,114.60221334)
\lineto(190.213675,114.60221334)
\lineto(190.213675,113.31022131)
\lineto(188.7107454,113.31022131)
\lineto(188.7107454,107.16666735)
\curveto(188.7107454,106.83854239)(188.82207352,106.61881586)(189.04472975,106.50748774)
\curveto(189.16777662,106.44303463)(189.37285473,106.41080807)(189.65996409,106.41080807)
\lineto(189.90605783,106.41080807)
\curveto(189.99394845,106.41666744)(190.0964875,106.4254565)(190.213675,106.43717525)
\lineto(190.213675,105.18912853)
\curveto(190.03203438,105.13639416)(189.84160471,105.09830823)(189.64238597,105.07487073)
\curveto(189.4490266,105.05143323)(189.23808912,105.03971448)(189.0095735,105.03971448)
\curveto(188.2712923,105.03971448)(187.77031577,105.22721446)(187.50664391,105.60221442)
\curveto(187.24297205,105.98307374)(187.11113612,106.47526119)(187.11113612,107.07877674)
\lineto(187.11113612,113.31022131)
\lineto(185.83672213,113.31022131)
\lineto(185.83672213,114.60221334)
\lineto(187.11113612,114.60221334)
\lineto(187.11113612,117.23014272)
\closepath
}
}
{
\newrgbcolor{curcolor}{0 0 0}
\pscustom[linestyle=none,fillstyle=solid,fillcolor=curcolor]
{
\newpath
\moveto(195.74199498,114.81315082)
\curveto(196.40996369,114.81315082)(197.05742459,114.65494771)(197.68437768,114.3385415)
\curveto(198.31133076,114.02799466)(198.7888698,113.62369783)(199.11699478,113.12565102)
\curveto(199.43340101,112.6510417)(199.6443385,112.09733083)(199.74980724,111.4645184)
\curveto(199.84355723,111.03092471)(199.89043223,110.33951854)(199.89043223,109.3902999)
\lineto(192.99101858,109.3902999)
\curveto(193.02031545,108.43522189)(193.24590138,107.66764386)(193.66777635,107.0875658)
\curveto(194.08965133,106.51334712)(194.7429716,106.22623778)(195.62773717,106.22623778)
\curveto(196.453909,106.22623778)(197.11308865,106.49869868)(197.60527612,107.04362049)
\curveto(197.8865261,107.36002671)(198.08574484,107.7262376)(198.20293233,108.14225318)
\lineto(199.7585963,108.14225318)
\curveto(199.71758068,107.79655009)(199.57988537,107.40983139)(199.34551039,106.98209706)
\curveto(199.11699478,106.56022211)(198.85918229,106.21451903)(198.57207293,105.94498781)
\curveto(198.09160421,105.47623787)(197.49687769,105.15983166)(196.78789335,104.99576918)
\curveto(196.407034,104.90201919)(195.97636996,104.85514419)(195.49590124,104.85514419)
\curveto(194.32402631,104.85514419)(193.33086231,105.27994883)(192.51640923,106.1295581)
\curveto(191.70195616,106.98502675)(191.29472962,108.18033911)(191.29472962,109.71549518)
\curveto(191.29472962,111.22721374)(191.70488584,112.45475266)(192.5251983,113.39811192)
\curveto(193.34551075,114.34147119)(194.41777631,114.81315082)(195.74199498,114.81315082)
\closepath
\moveto(198.26445577,110.64713569)
\curveto(198.20000264,111.33268248)(198.05058859,111.88053398)(197.8162136,112.29069018)
\curveto(197.38261988,113.05240884)(196.65898711,113.43326817)(195.6453153,113.43326817)
\curveto(194.91875284,113.43326817)(194.30937788,113.16959633)(193.81719041,112.64225264)
\curveto(193.32500293,112.12076833)(193.06426076,111.45572934)(193.03496389,110.64713569)
\lineto(198.26445577,110.64713569)
\closepath
\moveto(195.59258092,114.83072894)
\lineto(195.59258092,114.83072894)
\closepath
}
}
{
\newrgbcolor{curcolor}{0 0 0}
\pscustom[linestyle=none,fillstyle=solid,fillcolor=curcolor]
{
\newpath
\moveto(201.8855493,114.60221334)
\lineto(203.3884789,114.60221334)
\lineto(203.3884789,112.97623697)
\curveto(203.51152577,113.29264319)(203.81328356,113.6764322)(204.29375228,114.12760402)
\curveto(204.774221,114.58463522)(205.32793191,114.81315082)(205.95488499,114.81315082)
\curveto(205.98418187,114.81315082)(206.03398655,114.81022113)(206.10429905,114.80436176)
\curveto(206.17461154,114.79850238)(206.29472872,114.78678363)(206.46465059,114.76920551)
\lineto(206.46465059,113.09928383)
\curveto(206.37090059,113.11686196)(206.28300998,113.12858071)(206.20097873,113.13444008)
\curveto(206.12480686,113.14029945)(206.03984593,113.14322914)(205.94609593,113.14322914)
\curveto(205.14922098,113.14322914)(204.53691633,112.88541667)(204.10918198,112.36979173)
\curveto(203.68144763,111.86002617)(203.46758046,111.27115905)(203.46758046,110.60319038)
\lineto(203.46758046,105.18912853)
\lineto(201.8855493,105.18912853)
\lineto(201.8855493,114.60221334)
\closepath
}
}
{
\newrgbcolor{curcolor}{0.98823529 0.98823529 0.98823529}
\pscustom[linestyle=none,fillstyle=solid,fillcolor=curcolor]
{
\newpath
\moveto(99.93113586,60.99999579)
\lineto(259.93112632,60.99999579)
\lineto(259.93112632,1.00000294)
\lineto(99.93113586,1.00000294)
\closepath
}
}
{
\newrgbcolor{curcolor}{0 0 0}
\pscustom[linewidth=1.99999982,linecolor=curcolor]
{
\newpath
\moveto(99.93113586,60.99999579)
\lineto(259.93112632,60.99999579)
\lineto(259.93112632,1.00000294)
\lineto(99.93113586,1.00000294)
\closepath
}
}
{
\newrgbcolor{curcolor}{0 0 0}
\pscustom[linestyle=none,fillstyle=solid,fillcolor=curcolor]
{
\newpath
\moveto(116.66306845,39.44628742)
\lineto(122.47263842,39.44628742)
\curveto(123.62107585,39.44628742)(124.54685704,39.12109215)(125.249982,38.4707016)
\curveto(125.95310696,37.82617042)(126.30466944,36.91796741)(126.30466944,35.74609255)
\curveto(126.30466944,34.73828017)(125.9911929,33.85937402)(125.36423981,33.10937411)
\curveto(124.73728672,32.36523358)(123.77341959,31.99316331)(122.47263842,31.99316331)
\lineto(118.41209178,31.99316331)
\lineto(118.41209178,26.53515615)
\lineto(116.66306845,26.53515615)
\lineto(116.66306845,39.44628742)
\closepath
\moveto(124.53806798,35.73730349)
\curveto(124.53806798,36.68652212)(124.1865055,37.3310533)(123.48338055,37.67089701)
\curveto(123.09666182,37.85253761)(122.56638841,37.94335791)(121.89256033,37.94335791)
\lineto(118.41209178,37.94335791)
\lineto(118.41209178,33.46972563)
\lineto(121.89256033,33.46972563)
\curveto(122.67771653,33.46972563)(123.31345868,33.6367178)(123.79978678,33.97070213)
\curveto(124.29197425,34.30468647)(124.53806798,34.89355359)(124.53806798,35.73730349)
\closepath
}
}
{
\newrgbcolor{curcolor}{0 0 0}
\pscustom[linestyle=none,fillstyle=solid,fillcolor=curcolor]
{
\newpath
\moveto(132.04392691,27.55468727)
\curveto(133.09275497,27.55468727)(133.81052837,27.95019504)(134.19724709,28.74121057)
\curveto(134.5898252,29.53808548)(134.78611425,30.42285099)(134.78611425,31.39550713)
\curveto(134.78611425,32.27441327)(134.64548925,32.98925694)(134.36423927,33.54003812)
\curveto(133.9189268,34.40722552)(133.15134872,34.84081922)(132.06150503,34.84081922)
\curveto(131.09470822,34.84081922)(130.39158326,34.47167864)(129.95213016,33.73339747)
\curveto(129.51267706,32.99511631)(129.29295051,32.10449142)(129.29295051,31.06152279)
\curveto(129.29295051,30.05956979)(129.51267706,29.22460895)(129.95213016,28.55664028)
\curveto(130.39158326,27.88867161)(131.08884884,27.55468727)(132.04392691,27.55468727)
\closepath
\moveto(132.10545034,36.22070187)
\curveto(133.3183409,36.22070187)(134.34373146,35.81640504)(135.18162204,35.00781139)
\curveto(136.01951261,34.19921773)(136.4384579,33.00976475)(136.4384579,31.43945244)
\curveto(136.4384579,29.92187449)(136.06931729,28.66796839)(135.33103609,27.67773413)
\curveto(134.59275488,26.68749988)(133.44724714,26.19238275)(131.89451286,26.19238275)
\curveto(130.59959106,26.19238275)(129.57127081,26.62890613)(128.8095521,27.50195291)
\curveto(128.0478334,28.38085905)(127.66697405,29.55859329)(127.66697405,31.03515561)
\curveto(127.66697405,32.61718667)(128.06834121,33.87695215)(128.87107554,34.81445203)
\curveto(129.67380986,35.75195192)(130.7519348,36.22070187)(132.10545034,36.22070187)
\closepath
\moveto(132.05271597,36.17675656)
\lineto(132.05271597,36.17675656)
\closepath
}
}
{
\newrgbcolor{curcolor}{0 0 0}
\pscustom[linestyle=none,fillstyle=solid,fillcolor=curcolor]
{
\newpath
\moveto(139.26853585,29.48828079)
\curveto(139.31541085,28.96093711)(139.44724678,28.55664028)(139.66404364,28.27539031)
\curveto(140.06248112,27.76562475)(140.75388733,27.51074197)(141.73826227,27.51074197)
\curveto(142.32419973,27.51074197)(142.8398247,27.63671851)(143.28513718,27.88867161)
\curveto(143.73044965,28.14648408)(143.95310589,28.54199184)(143.95310589,29.07519491)
\curveto(143.95310589,29.47949173)(143.77439496,29.78710888)(143.41697311,29.99804636)
\curveto(143.1884575,30.12695259)(142.73728565,30.27636664)(142.06345756,30.44628849)
\lineto(140.8066217,30.7626947)
\curveto(140.00388737,30.96191343)(139.41209053,31.18456965)(139.03123118,31.43066337)
\curveto(138.35154372,31.8583977)(138.01169999,32.4501945)(138.01169999,33.20605379)
\curveto(138.01169999,34.09667868)(138.33103591,34.81738172)(138.96970775,35.36816291)
\curveto(139.61423896,35.91894409)(140.47849672,36.19433468)(141.56248103,36.19433468)
\curveto(142.9804497,36.19433468)(144.00291057,35.77831911)(144.62986366,34.94628796)
\curveto(145.02244176,34.41894427)(145.21287144,33.85058496)(145.20115269,33.24121003)
\lineto(143.70701215,33.24121003)
\curveto(143.67771528,33.59863187)(143.55173872,33.92382714)(143.32908249,34.21679585)
\curveto(142.96580126,34.63281143)(142.33591848,34.84081922)(141.43943416,34.84081922)
\curveto(140.84177795,34.84081922)(140.38767641,34.72656142)(140.07712956,34.49804582)
\curveto(139.77244207,34.26953022)(139.62009833,33.96777245)(139.62009833,33.59277249)
\curveto(139.62009833,33.18261629)(139.82224676,32.85449133)(140.22654361,32.60839761)
\curveto(140.4609186,32.46191325)(140.8066217,32.33300702)(141.26365292,32.22167891)
\lineto(142.3095513,31.96679612)
\curveto(143.44626998,31.69140553)(144.20798868,31.424804)(144.59470741,31.16699153)
\curveto(145.20994175,30.7626947)(145.51755892,30.12695259)(145.51755892,29.2597652)
\curveto(145.51755892,28.42187467)(145.198223,27.69824194)(144.55955116,27.08886702)
\curveto(143.9267387,26.47949209)(142.95994188,26.17480463)(141.65916071,26.17480463)
\curveto(140.25877017,26.17480463)(139.26560617,26.49121084)(138.6796687,27.12402326)
\curveto(138.09959061,27.76269506)(137.78904375,28.55078091)(137.74802813,29.48828079)
\lineto(139.26853585,29.48828079)
\closepath
\moveto(141.60642634,36.17675656)
\lineto(141.60642634,36.17675656)
\closepath
}
}
{
\newrgbcolor{curcolor}{0 0 0}
\pscustom[linestyle=none,fillstyle=solid,fillcolor=curcolor]
{
\newpath
\moveto(147.67966817,38.57617034)
\lineto(149.27927745,38.57617034)
\lineto(149.27927745,35.94824096)
\lineto(150.78220704,35.94824096)
\lineto(150.78220704,34.65624893)
\lineto(149.27927745,34.65624893)
\lineto(149.27927745,28.51269497)
\curveto(149.27927745,28.18457001)(149.39060556,27.96484348)(149.6132618,27.85351536)
\curveto(149.73630867,27.78906225)(149.94138678,27.75683569)(150.22849614,27.75683569)
\lineto(150.47458987,27.75683569)
\curveto(150.56248049,27.76269506)(150.66501955,27.77148412)(150.78220704,27.78320287)
\lineto(150.78220704,26.53515615)
\curveto(150.60056643,26.48242178)(150.41013675,26.44433584)(150.21091801,26.42089835)
\curveto(150.01755865,26.39746085)(149.80662116,26.3857421)(149.57810555,26.3857421)
\curveto(148.83982435,26.3857421)(148.33884781,26.57324208)(148.07517595,26.94824203)
\curveto(147.8115041,27.32910136)(147.67966817,27.8212888)(147.67966817,28.42480436)
\lineto(147.67966817,34.65624893)
\lineto(146.40525418,34.65624893)
\lineto(146.40525418,35.94824096)
\lineto(147.67966817,35.94824096)
\lineto(147.67966817,38.57617034)
\closepath
}
}
{
\newrgbcolor{curcolor}{0 0 0}
\pscustom[linestyle=none,fillstyle=solid,fillcolor=curcolor]
{
\newpath
\moveto(151.97751947,32.36230389)
\lineto(156.38962858,32.36230389)
\lineto(156.38962858,30.73632752)
\lineto(151.97751947,30.73632752)
\lineto(151.97751947,32.36230389)
\closepath
}
}
{
\newrgbcolor{curcolor}{0 0 0}
\pscustom[linestyle=none,fillstyle=solid,fillcolor=curcolor]
{
\newpath
\moveto(162.37497979,27.59863258)
\curveto(163.113261,27.59863258)(163.72556565,27.90624973)(164.21189374,28.52148403)
\curveto(164.70408121,29.14257771)(164.95017495,30.06835885)(164.95017495,31.29882745)
\curveto(164.95017495,32.04882736)(164.84177652,32.69335854)(164.62497966,33.23242097)
\curveto(164.21482343,34.26953022)(163.46482347,34.78808485)(162.37497979,34.78808485)
\curveto(161.27927673,34.78808485)(160.52927677,34.24023335)(160.12497992,33.14453036)
\curveto(159.90818306,32.55859293)(159.79978463,31.81445239)(159.79978463,30.91210875)
\curveto(159.79978463,30.18554634)(159.90818306,29.56738235)(160.12497992,29.05761678)
\curveto(160.53513615,28.08496065)(161.2851361,27.59863258)(162.37497979,27.59863258)
\closepath
\moveto(158.27927691,35.90429565)
\lineto(159.81736275,35.90429565)
\lineto(159.81736275,34.65624893)
\curveto(160.13376899,35.08398325)(160.47947209,35.4150379)(160.85447207,35.64941287)
\curveto(161.38767516,36.00097533)(162.01462825,36.17675656)(162.73533133,36.17675656)
\curveto(163.80173752,36.17675656)(164.7070109,35.76660036)(165.45115148,34.94628796)
\curveto(166.19529206,34.13183493)(166.56736235,32.96581944)(166.56736235,31.4482415)
\curveto(166.56736235,29.39746049)(166.03122957,27.93261692)(164.95896401,27.05371077)
\curveto(164.27927655,26.49707021)(163.48826097,26.21874993)(162.58591728,26.21874993)
\curveto(161.87693294,26.21874993)(161.28220642,26.37402335)(160.8017377,26.68457019)
\curveto(160.52048771,26.86035142)(160.20701117,27.1621092)(159.86130806,27.58984352)
\lineto(159.86130806,22.78222691)
\lineto(158.27927691,22.78222691)
\lineto(158.27927691,35.90429565)
\closepath
}
}
{
\newrgbcolor{curcolor}{0 0 0}
\pscustom[linestyle=none,fillstyle=solid,fillcolor=curcolor]
{
\newpath
\moveto(168.46579974,35.94824096)
\lineto(169.96872934,35.94824096)
\lineto(169.96872934,34.32226459)
\curveto(170.0917762,34.6386708)(170.393534,35.02245982)(170.87400272,35.47363164)
\curveto(171.35447144,35.93066284)(171.90818235,36.15917844)(172.53513543,36.15917844)
\curveto(172.56443231,36.15917844)(172.61423699,36.15624875)(172.68454949,36.15038937)
\curveto(172.75486198,36.14453)(172.87497916,36.13281125)(173.04490103,36.11523313)
\lineto(173.04490103,34.44531145)
\curveto(172.95115103,34.46288958)(172.86326041,34.47460832)(172.78122917,34.4804677)
\curveto(172.7050573,34.48632707)(172.62009637,34.48925676)(172.52634637,34.48925676)
\curveto(171.72947142,34.48925676)(171.11716677,34.23144429)(170.68943242,33.71581935)
\curveto(170.26169807,33.20605379)(170.04783089,32.61718667)(170.04783089,31.949218)
\lineto(170.04783089,26.53515615)
\lineto(168.46579974,26.53515615)
\lineto(168.46579974,35.94824096)
\closepath
}
}
{
\newrgbcolor{curcolor}{0 0 0}
\pscustom[linestyle=none,fillstyle=solid,fillcolor=curcolor]
{
\newpath
\moveto(178.16892416,27.55468727)
\curveto(179.21775222,27.55468727)(179.93552562,27.95019504)(180.32224434,28.74121057)
\curveto(180.71482245,29.53808548)(180.9111115,30.42285099)(180.9111115,31.39550713)
\curveto(180.9111115,32.27441327)(180.77048651,32.98925694)(180.48923652,33.54003812)
\curveto(180.04392405,34.40722552)(179.27634597,34.84081922)(178.18650228,34.84081922)
\curveto(177.21970547,34.84081922)(176.51658051,34.47167864)(176.07712741,33.73339747)
\curveto(175.63767431,32.99511631)(175.41794776,32.10449142)(175.41794776,31.06152279)
\curveto(175.41794776,30.05956979)(175.63767431,29.22460895)(176.07712741,28.55664028)
\curveto(176.51658051,27.88867161)(177.21384609,27.55468727)(178.16892416,27.55468727)
\closepath
\moveto(178.23044759,36.22070187)
\curveto(179.44333815,36.22070187)(180.46872871,35.81640504)(181.30661929,35.00781139)
\curveto(182.14450986,34.19921773)(182.56345515,33.00976475)(182.56345515,31.43945244)
\curveto(182.56345515,29.92187449)(182.19431455,28.66796839)(181.45603334,27.67773413)
\curveto(180.71775213,26.68749988)(179.57224439,26.19238275)(178.01951011,26.19238275)
\curveto(176.72458831,26.19238275)(175.69626806,26.62890613)(174.93454935,27.50195291)
\curveto(174.17283065,28.38085905)(173.7919713,29.55859329)(173.7919713,31.03515561)
\curveto(173.7919713,32.61718667)(174.19333846,33.87695215)(174.99607279,34.81445203)
\curveto(175.79880711,35.75195192)(176.87693205,36.22070187)(178.23044759,36.22070187)
\closepath
\moveto(178.17771322,36.17675656)
\lineto(178.17771322,36.17675656)
\closepath
}
}
{
\newrgbcolor{curcolor}{0 0 0}
\pscustom[linestyle=none,fillstyle=solid,fillcolor=curcolor]
{
\newpath
\moveto(188.08298607,36.22070187)
\curveto(189.14353288,36.22070187)(190.00486096,35.9628894)(190.66697029,35.44726446)
\curveto(191.334939,34.93163952)(191.73630616,34.04394431)(191.87107178,32.78417884)
\lineto(190.33298594,32.78417884)
\curveto(190.23923594,33.36425689)(190.02536877,33.84472559)(189.69138441,34.22558492)
\curveto(189.35740006,34.61230362)(188.82126728,34.80566297)(188.08298607,34.80566297)
\curveto(187.07517363,34.80566297)(186.35447055,34.31347553)(185.92087682,33.32910065)
\curveto(185.63962684,32.69042885)(185.49900185,31.90234301)(185.49900185,30.96484312)
\curveto(185.49900185,30.02148385)(185.69822059,29.22753864)(186.09665806,28.58300746)
\curveto(186.49509554,27.93847629)(187.12204863,27.6162107)(187.97751733,27.6162107)
\curveto(188.63376729,27.6162107)(189.15232194,27.81542943)(189.5331813,28.21386688)
\curveto(189.91990002,28.61816371)(190.18650157,29.16894489)(190.33298594,29.86621044)
\lineto(191.87107178,29.86621044)
\curveto(191.69529054,28.61816371)(191.25583744,27.70410132)(190.55271248,27.12402326)
\curveto(189.84958753,26.54980458)(188.95017352,26.26269524)(187.85447046,26.26269524)
\curveto(186.62400178,26.26269524)(185.64255653,26.71093737)(184.9101347,27.60742164)
\curveto(184.17771286,28.50976529)(183.81150195,29.63476515)(183.81150195,30.98242124)
\curveto(183.81150195,32.63476479)(184.21286911,33.92089745)(185.01560344,34.84081922)
\curveto(185.81833777,35.76074098)(186.84079864,36.22070187)(188.08298607,36.22070187)
\closepath
\moveto(187.83689233,36.17675656)
\lineto(187.83689233,36.17675656)
\closepath
}
}
{
\newrgbcolor{curcolor}{0 0 0}
\pscustom[linestyle=none,fillstyle=solid,fillcolor=curcolor]
{
\newpath
\moveto(197.37302458,36.15917844)
\curveto(198.04099329,36.15917844)(198.68845419,36.00097533)(199.31540728,35.68456912)
\curveto(199.94236036,35.37402228)(200.4198994,34.96972545)(200.74802438,34.47167864)
\curveto(201.06443061,33.99706932)(201.2753681,33.44335845)(201.38083684,32.81054602)
\curveto(201.47458683,32.37695232)(201.52146183,31.68554616)(201.52146183,30.73632752)
\lineto(194.62204818,30.73632752)
\curveto(194.65134505,29.78124951)(194.87693098,29.01367148)(195.29880595,28.43359342)
\curveto(195.72068093,27.85937474)(196.3740012,27.5722654)(197.25876677,27.5722654)
\curveto(198.0849386,27.5722654)(198.74411825,27.8447263)(199.23630572,28.38964811)
\curveto(199.5175557,28.70605432)(199.71677444,29.07226522)(199.83396193,29.48828079)
\lineto(201.3896259,29.48828079)
\curveto(201.34861028,29.14257771)(201.21091497,28.75585901)(200.97653999,28.32812468)
\curveto(200.74802438,27.90624973)(200.49021189,27.56054665)(200.20310253,27.29101543)
\curveto(199.72263381,26.82226549)(199.12790729,26.50585927)(198.41892295,26.34179679)
\curveto(198.0380636,26.2480468)(197.60739956,26.20117181)(197.12693084,26.20117181)
\curveto(195.95505591,26.20117181)(194.96189191,26.62597645)(194.14743883,27.47558572)
\curveto(193.33298576,28.33105437)(192.92575922,29.52636673)(192.92575922,31.06152279)
\curveto(192.92575922,32.57324136)(193.33591544,33.80078028)(194.1562279,34.74413954)
\curveto(194.97654035,35.6874988)(196.04880591,36.15917844)(197.37302458,36.15917844)
\closepath
\moveto(199.89548537,31.99316331)
\curveto(199.83103224,32.6787101)(199.68161819,33.2265616)(199.4472432,33.6367178)
\curveto(199.01364948,34.39843646)(198.29001671,34.77929579)(197.2763449,34.77929579)
\curveto(196.54978244,34.77929579)(195.94040748,34.51562394)(195.44822001,33.98828026)
\curveto(194.95603253,33.46679594)(194.69529036,32.80175696)(194.66599349,31.99316331)
\lineto(199.89548537,31.99316331)
\closepath
\moveto(197.22361052,36.17675656)
\lineto(197.22361052,36.17675656)
\closepath
}
}
{
\newrgbcolor{curcolor}{0 0 0}
\pscustom[linestyle=none,fillstyle=solid,fillcolor=curcolor]
{
\newpath
\moveto(204.41306322,29.48828079)
\curveto(204.45993822,28.96093711)(204.59177415,28.55664028)(204.80857101,28.27539031)
\curveto(205.20700849,27.76562475)(205.8984147,27.51074197)(206.88278964,27.51074197)
\curveto(207.4687271,27.51074197)(207.98435207,27.63671851)(208.42966454,27.88867161)
\curveto(208.87497702,28.14648408)(209.09763325,28.54199184)(209.09763325,29.07519491)
\curveto(209.09763325,29.47949173)(208.91892233,29.78710888)(208.56150047,29.99804636)
\curveto(208.33298486,30.12695259)(207.88181301,30.27636664)(207.20798493,30.44628849)
\lineto(205.95114907,30.7626947)
\curveto(205.14841474,30.96191343)(204.5566179,31.18456965)(204.17575855,31.43066337)
\curveto(203.49607109,31.8583977)(203.15622736,32.4501945)(203.15622736,33.20605379)
\curveto(203.15622736,34.09667868)(203.47556328,34.81738172)(204.11423511,35.36816291)
\curveto(204.75876633,35.91894409)(205.62302409,36.19433468)(206.7070084,36.19433468)
\curveto(208.12497706,36.19433468)(209.14743794,35.77831911)(209.77439103,34.94628796)
\curveto(210.16696913,34.41894427)(210.3573988,33.85058496)(210.34568006,33.24121003)
\lineto(208.85153952,33.24121003)
\curveto(208.82224265,33.59863187)(208.69626609,33.92382714)(208.47360985,34.21679585)
\curveto(208.11032863,34.63281143)(207.48044585,34.84081922)(206.58396153,34.84081922)
\curveto(205.98630532,34.84081922)(205.53220378,34.72656142)(205.22165692,34.49804582)
\curveto(204.91696944,34.26953022)(204.7646257,33.96777245)(204.7646257,33.59277249)
\curveto(204.7646257,33.18261629)(204.96677413,32.85449133)(205.37107098,32.60839761)
\curveto(205.60544596,32.46191325)(205.95114907,32.33300702)(206.40818029,32.22167891)
\lineto(207.45407867,31.96679612)
\curveto(208.59079735,31.69140553)(209.35251605,31.424804)(209.73923478,31.16699153)
\curveto(210.35446912,30.7626947)(210.66208629,30.12695259)(210.66208629,29.2597652)
\curveto(210.66208629,28.42187467)(210.34275037,27.69824194)(209.70407853,27.08886702)
\curveto(209.07126607,26.47949209)(208.10446925,26.17480463)(206.80368808,26.17480463)
\curveto(205.40329754,26.17480463)(204.41013353,26.49121084)(203.82419607,27.12402326)
\curveto(203.24411798,27.76269506)(202.93357112,28.55078091)(202.8925555,29.48828079)
\lineto(204.41306322,29.48828079)
\closepath
\moveto(206.75095371,36.17675656)
\lineto(206.75095371,36.17675656)
\closepath
}
}
{
\newrgbcolor{curcolor}{0 0 0}
\pscustom[linestyle=none,fillstyle=solid,fillcolor=curcolor]
{
\newpath
\moveto(213.44821893,29.48828079)
\curveto(213.49509393,28.96093711)(213.62692986,28.55664028)(213.84372672,28.27539031)
\curveto(214.2421642,27.76562475)(214.93357041,27.51074197)(215.91794535,27.51074197)
\curveto(216.50388281,27.51074197)(217.01950778,27.63671851)(217.46482026,27.88867161)
\curveto(217.91013273,28.14648408)(218.13278897,28.54199184)(218.13278897,29.07519491)
\curveto(218.13278897,29.47949173)(217.95407804,29.78710888)(217.59665619,29.99804636)
\curveto(217.36814057,30.12695259)(216.91696873,30.27636664)(216.24314064,30.44628849)
\lineto(214.98630478,30.7626947)
\curveto(214.18357045,30.96191343)(213.59177361,31.18456965)(213.21091426,31.43066337)
\curveto(212.5312268,31.8583977)(212.19138307,32.4501945)(212.19138307,33.20605379)
\curveto(212.19138307,34.09667868)(212.51071899,34.81738172)(213.14939083,35.36816291)
\curveto(213.79392204,35.91894409)(214.6581798,36.19433468)(215.74216411,36.19433468)
\curveto(217.16013277,36.19433468)(218.18259365,35.77831911)(218.80954674,34.94628796)
\curveto(219.20212484,34.41894427)(219.39255452,33.85058496)(219.38083577,33.24121003)
\lineto(217.88669523,33.24121003)
\curveto(217.85739836,33.59863187)(217.7314218,33.92382714)(217.50876557,34.21679585)
\curveto(217.14548434,34.63281143)(216.51560156,34.84081922)(215.61911724,34.84081922)
\curveto(215.02146103,34.84081922)(214.56735949,34.72656142)(214.25681263,34.49804582)
\curveto(213.95212515,34.26953022)(213.79978141,33.96777245)(213.79978141,33.59277249)
\curveto(213.79978141,33.18261629)(214.00192984,32.85449133)(214.40622669,32.60839761)
\curveto(214.64060167,32.46191325)(214.98630478,32.33300702)(215.443336,32.22167891)
\lineto(216.48923438,31.96679612)
\curveto(217.62595306,31.69140553)(218.38767176,31.424804)(218.77439049,31.16699153)
\curveto(219.38962483,30.7626947)(219.697242,30.12695259)(219.697242,29.2597652)
\curveto(219.697242,28.42187467)(219.37790608,27.69824194)(218.73923424,27.08886702)
\curveto(218.10642178,26.47949209)(217.13962496,26.17480463)(215.83884379,26.17480463)
\curveto(214.43845325,26.17480463)(213.44528925,26.49121084)(212.85935178,27.12402326)
\curveto(212.27927369,27.76269506)(211.96872683,28.55078091)(211.92771121,29.48828079)
\lineto(213.44821893,29.48828079)
\closepath
\moveto(215.78610942,36.17675656)
\lineto(215.78610942,36.17675656)
\closepath
}
}
{
\newrgbcolor{curcolor}{0 0 0}
\pscustom[linestyle=none,fillstyle=solid,fillcolor=curcolor]
{
\newpath
\moveto(221.54294501,35.90429565)
\lineto(223.15134335,35.90429565)
\lineto(223.15134335,26.53515615)
\lineto(221.54294501,26.53515615)
\lineto(221.54294501,35.90429565)
\closepath
\moveto(221.54294501,39.44628742)
\lineto(223.15134335,39.44628742)
\lineto(223.15134335,37.65331888)
\lineto(221.54294501,37.65331888)
\lineto(221.54294501,39.44628742)
\closepath
}
}
{
\newrgbcolor{curcolor}{0 0 0}
\pscustom[linestyle=none,fillstyle=solid,fillcolor=curcolor]
{
\newpath
\moveto(225.55075727,35.94824096)
\lineto(227.05368687,35.94824096)
\lineto(227.05368687,34.61230362)
\curveto(227.49899935,35.1630848)(227.970679,35.55859257)(228.46872585,35.79882692)
\curveto(228.9667727,36.03906126)(229.5204836,36.15917844)(230.12985856,36.15917844)
\curveto(231.46579598,36.15917844)(232.36813968,35.69335818)(232.83688965,34.76171766)
\curveto(233.09470214,34.2519521)(233.22360838,33.52246)(233.22360838,32.57324136)
\lineto(233.22360838,26.53515615)
\lineto(231.61521004,26.53515615)
\lineto(231.61521004,32.46777263)
\curveto(231.61521004,33.04199131)(231.5302491,33.50488188)(231.36032724,33.85644434)
\curveto(231.07907726,34.44238177)(230.56931166,34.73535048)(229.83103046,34.73535048)
\curveto(229.45603048,34.73535048)(229.14841331,34.69726455)(228.90817895,34.62109268)
\curveto(228.47458522,34.49218645)(228.09372587,34.23437398)(227.76560089,33.84765527)
\curveto(227.50192903,33.53710844)(227.32907748,33.21484285)(227.24704624,32.88085851)
\curveto(227.17087436,32.55273355)(227.13278843,32.08105392)(227.13278843,31.46581962)
\lineto(227.13278843,26.53515615)
\lineto(225.55075727,26.53515615)
\lineto(225.55075727,35.94824096)
\closepath
\moveto(229.26853049,36.17675656)
\lineto(229.26853049,36.17675656)
\closepath
}
}
{
\newrgbcolor{curcolor}{0 0 0}
\pscustom[linestyle=none,fillstyle=solid,fillcolor=curcolor]
{
\newpath
\moveto(238.89255335,36.11523313)
\curveto(239.63083456,36.11523313)(240.27536577,35.93359253)(240.82614699,35.57031132)
\curveto(241.1249751,35.36523322)(241.42966258,35.06640513)(241.74020943,34.67382705)
\lineto(241.74020943,35.86035035)
\lineto(243.19919372,35.86035035)
\lineto(243.19919372,27.29980449)
\curveto(243.19919372,26.10449213)(243.02341248,25.16113287)(242.67185,24.4697267)
\curveto(242.01560004,23.19238311)(240.7763423,22.55371131)(238.95407679,22.55371131)
\curveto(237.94040497,22.55371131)(237.08786596,22.78222691)(236.39645975,23.2392581)
\curveto(235.70505354,23.69042992)(235.31833482,24.39941421)(235.23630357,25.36621097)
\lineto(236.84470191,25.36621097)
\curveto(236.92087378,24.94433602)(237.07321752,24.61914075)(237.30173314,24.39062515)
\curveto(237.65915499,24.03906269)(238.22165496,23.86328146)(238.98923304,23.86328146)
\curveto(240.20212359,23.86328146)(240.99606885,24.29101579)(241.37106883,25.14648444)
\curveto(241.59372507,25.65039063)(241.69626412,26.54980458)(241.678686,27.8447263)
\curveto(241.36227977,27.36425761)(240.98142042,27.00683578)(240.53610794,26.7724608)
\curveto(240.09079547,26.53808583)(239.50192832,26.42089835)(238.76950649,26.42089835)
\curveto(237.7499753,26.42089835)(236.85642066,26.78124987)(236.08884258,27.50195291)
\curveto(235.32712388,28.22851532)(234.94626453,29.42675736)(234.94626453,31.09667904)
\curveto(234.94626453,32.67285073)(235.33005357,33.90331933)(236.09763165,34.78808485)
\curveto(236.8710691,35.67285037)(237.80270967,36.11523313)(238.89255335,36.11523313)
\closepath
\moveto(241.74020943,31.28124933)
\curveto(241.74020943,32.44726482)(241.49997507,33.31152253)(241.01950635,33.87402246)
\curveto(240.53903763,34.43652239)(239.92673298,34.71777236)(239.1825924,34.71777236)
\curveto(238.06931122,34.71777236)(237.30759251,34.19628804)(236.89743629,33.15331942)
\curveto(236.68063942,32.59667886)(236.57224099,31.86718676)(236.57224099,30.96484312)
\curveto(236.57224099,29.90429637)(236.78610817,29.09570272)(237.21384252,28.53906216)
\curveto(237.64743624,27.98828097)(238.22751433,27.71289038)(238.95407679,27.71289038)
\curveto(240.09079547,27.71289038)(240.89060011,28.22558563)(241.35349071,29.25097613)
\curveto(241.61130319,29.83105419)(241.74020943,30.50781192)(241.74020943,31.28124933)
\closepath
\moveto(239.07712366,36.17675656)
\lineto(239.07712366,36.17675656)
\closepath
}
}
{
\newrgbcolor{curcolor}{0.98823529 0.98823529 0.98823529}
\pscustom[linestyle=none,fillstyle=solid,fillcolor=curcolor]
{
\newpath
\moveto(102.4999459,222.71451475)
\lineto(262.49993636,222.71451475)
\lineto(262.49993636,162.7145219)
\lineto(102.4999459,162.7145219)
\closepath
}
}
{
\newrgbcolor{curcolor}{0 0 0}
\pscustom[linewidth=1.99999982,linecolor=curcolor]
{
\newpath
\moveto(102.4999459,222.71451475)
\lineto(262.49993636,222.71451475)
\lineto(262.49993636,162.7145219)
\lineto(102.4999459,162.7145219)
\closepath
}
}
{
\newrgbcolor{curcolor}{0 0 0}
\pscustom[linestyle=none,fillstyle=solid,fillcolor=curcolor]
{
\newpath
\moveto(162.17547945,190.43815141)
\curveto(162.21649507,189.70572962)(162.38934662,189.11100313)(162.69403411,188.65397194)
\curveto(163.2741122,187.79850329)(164.29657307,187.37076896)(165.76141674,187.37076896)
\curveto(166.4176667,187.37076896)(167.01532291,187.46451895)(167.55438538,187.65201893)
\curveto(168.59735407,188.01530014)(169.11883841,188.66569069)(169.11883841,189.60319057)
\curveto(169.11883841,190.30631549)(168.89911186,190.80729199)(168.45965876,191.10612008)
\curveto(168.01434629,191.3990888)(167.31708071,191.65397158)(166.36786201,191.87076843)
\lineto(164.61883868,192.26627619)
\curveto(163.47626062,192.52408866)(162.66766692,192.80826832)(162.19305757,193.11881515)
\curveto(161.37274512,193.65787759)(160.9625889,194.46354156)(160.9625889,195.53580705)
\curveto(160.9625889,196.69596317)(161.36395606,197.64811149)(162.16669039,198.39225203)
\curveto(162.96942471,199.13639256)(164.1061434,199.50846283)(165.57684643,199.50846283)
\curveto(166.93036198,199.50846283)(168.07879941,199.18033787)(169.02215873,198.52408795)
\curveto(169.97137742,197.8736974)(170.44598677,196.83072877)(170.44598677,195.39518207)
\lineto(168.80243218,195.39518207)
\curveto(168.71454156,196.08658824)(168.52704157,196.61686161)(168.23993221,196.98600219)
\curveto(167.70672912,197.65983024)(166.80145574,197.99674426)(165.52411206,197.99674426)
\curveto(164.49286212,197.99674426)(163.75165123,197.77994741)(163.30047938,197.34635371)
\curveto(162.84930753,196.91276001)(162.62372161,196.40885382)(162.62372161,195.83463514)
\curveto(162.62372161,195.20182272)(162.88739347,194.73893215)(163.41473719,194.44596343)
\curveto(163.76044029,194.25846346)(164.54266681,194.02408848)(165.76141674,193.74283852)
\lineto(167.5719635,193.32975263)
\curveto(168.44501033,193.1305339)(169.11883841,192.858073)(169.59344776,192.51236991)
\curveto(170.41376021,191.90885436)(170.82391643,191.0328779)(170.82391643,189.88444054)
\curveto(170.82391643,188.45475321)(170.30243209,187.43229239)(169.2594634,186.81705809)
\curveto(168.22235409,186.20182379)(167.01532291,185.89420664)(165.63836987,185.89420664)
\curveto(164.03290121,185.89420664)(162.77606535,186.30436284)(161.86786228,187.12467524)
\curveto(160.95965921,187.93912827)(160.51434674,189.04362033)(160.53192486,190.43815141)
\lineto(162.17547945,190.43815141)
\closepath
\moveto(165.70868236,199.53483001)
\lineto(165.70868236,199.53483001)
\closepath
}
}
{
\newrgbcolor{curcolor}{0 0 0}
\pscustom[linestyle=none,fillstyle=solid,fillcolor=curcolor]
{
\newpath
\moveto(172.85418975,199.18326756)
\lineto(174.43622091,199.18326756)
\lineto(174.43622091,186.27213628)
\lineto(172.85418975,186.27213628)
\lineto(172.85418975,199.18326756)
\closepath
}
}
{
\newrgbcolor{curcolor}{0 0 0}
\pscustom[linestyle=none,fillstyle=solid,fillcolor=curcolor]
{
\newpath
\moveto(176.8180567,195.64127579)
\lineto(178.42645504,195.64127579)
\lineto(178.42645504,186.27213628)
\lineto(176.8180567,186.27213628)
\lineto(176.8180567,195.64127579)
\closepath
\moveto(176.8180567,199.18326756)
\lineto(178.42645504,199.18326756)
\lineto(178.42645504,197.39029902)
\lineto(176.8180567,197.39029902)
\lineto(176.8180567,199.18326756)
\closepath
}
}
{
\newrgbcolor{curcolor}{0 0 0}
\pscustom[linestyle=none,fillstyle=solid,fillcolor=curcolor]
{
\newpath
\moveto(184.45575156,195.957682)
\curveto(185.51629837,195.957682)(186.37762644,195.69986953)(187.03973578,195.1842446)
\curveto(187.70770449,194.66861966)(188.10907165,193.78092445)(188.24383727,192.52115898)
\lineto(186.70575143,192.52115898)
\curveto(186.61200143,193.10123703)(186.39813426,193.58170572)(186.0641499,193.96256505)
\curveto(185.73016555,194.34928376)(185.19403277,194.54264311)(184.45575156,194.54264311)
\curveto(183.44793912,194.54264311)(182.72723604,194.05045567)(182.29364231,193.06608079)
\curveto(182.01239233,192.42740899)(181.87176734,191.63932314)(181.87176734,190.70182326)
\curveto(181.87176734,189.75846399)(182.07098608,188.96451877)(182.46942355,188.3199876)
\curveto(182.86786103,187.67545643)(183.49481412,187.35319084)(184.35028282,187.35319084)
\curveto(185.00653278,187.35319084)(185.52508743,187.55240957)(185.90594679,187.95084702)
\curveto(186.29266551,188.35514385)(186.55926706,188.90592503)(186.70575143,189.60319057)
\lineto(188.24383727,189.60319057)
\curveto(188.06805603,188.35514385)(187.62860293,187.44108146)(186.92547797,186.8610034)
\curveto(186.22235302,186.28678472)(185.32293901,185.99967538)(184.22723595,185.99967538)
\curveto(182.99676727,185.99967538)(182.01532202,186.44791751)(181.28290019,187.34440178)
\curveto(180.55047835,188.24674542)(180.18426744,189.37174529)(180.18426744,190.71940138)
\curveto(180.18426744,192.37174493)(180.5856346,193.65787759)(181.38836893,194.57779936)
\curveto(182.19110326,195.49772112)(183.21356413,195.957682)(184.45575156,195.957682)
\closepath
\moveto(184.20965782,195.9137367)
\lineto(184.20965782,195.9137367)
\closepath
}
}
{
\newrgbcolor{curcolor}{0 0 0}
\pscustom[linestyle=none,fillstyle=solid,fillcolor=curcolor]
{
\newpath
\moveto(193.74579007,195.89615857)
\curveto(194.41375878,195.89615857)(195.06121968,195.73795547)(195.68817276,195.42154925)
\curveto(196.31512585,195.11100242)(196.79266489,194.70670559)(197.12078987,194.20865877)
\curveto(197.4371961,193.73404946)(197.64813359,193.18033858)(197.75360233,192.54752616)
\curveto(197.84735232,192.11393246)(197.89422732,191.42252629)(197.89422732,190.47330766)
\lineto(190.99481367,190.47330766)
\curveto(191.02411054,189.51822965)(191.24969647,188.75065161)(191.67157144,188.17057356)
\curveto(192.09344642,187.59635488)(192.74676669,187.30924553)(193.63153226,187.30924553)
\curveto(194.45770409,187.30924553)(195.11688374,187.58170644)(195.60907121,188.12662825)
\curveto(195.89032119,188.44303446)(196.08953993,188.80924536)(196.20672742,189.22526093)
\lineto(197.76239139,189.22526093)
\curveto(197.72137577,188.87955785)(197.58368046,188.49283914)(197.34930548,188.06510482)
\curveto(197.12078987,187.64322987)(196.86297738,187.29752679)(196.57586802,187.02799557)
\curveto(196.0953993,186.55924562)(195.50067278,186.24283941)(194.79168844,186.07877693)
\curveto(194.41082909,185.98502694)(193.98016505,185.93815195)(193.49969633,185.93815195)
\curveto(192.3278214,185.93815195)(191.3346574,186.36295658)(190.52020432,187.21256586)
\curveto(189.70575125,188.06803451)(189.29852471,189.26334686)(189.29852471,190.79850293)
\curveto(189.29852471,192.3102215)(189.70868093,193.53776042)(190.52899338,194.48111968)
\curveto(191.34930584,195.42447894)(192.4215714,195.89615857)(193.74579007,195.89615857)
\closepath
\moveto(196.26825086,191.73014344)
\curveto(196.20379773,192.41569024)(196.05438368,192.96354174)(195.82000869,193.37369794)
\curveto(195.38641497,194.1354166)(194.6627822,194.51627593)(193.64911039,194.51627593)
\curveto(192.92254793,194.51627593)(192.31317297,194.25260408)(191.8209855,193.72526039)
\curveto(191.32879802,193.20377608)(191.06805585,192.5387371)(191.03875898,191.73014344)
\lineto(196.26825086,191.73014344)
\closepath
\moveto(193.59637601,195.9137367)
\lineto(193.59637601,195.9137367)
\closepath
}
}
{
\newrgbcolor{curcolor}{0 0 0}
\pscustom[linestyle=none,fillstyle=solid,fillcolor=curcolor]
{
\newpath
\moveto(199.88934439,195.6852211)
\lineto(201.39227399,195.6852211)
\lineto(201.39227399,194.05924473)
\curveto(201.51532085,194.37565094)(201.81707865,194.75943996)(202.29754737,195.21061178)
\curveto(202.77801609,195.66764298)(203.331727,195.89615857)(203.95868008,195.89615857)
\curveto(203.98797696,195.89615857)(204.03778164,195.89322889)(204.10809414,195.88736951)
\curveto(204.17840663,195.88151014)(204.29852381,195.86979139)(204.46844568,195.85221327)
\lineto(204.46844568,194.18229159)
\curveto(204.37469568,194.19986971)(204.28680506,194.21158846)(204.20477382,194.21744784)
\curveto(204.12860195,194.22330721)(204.04364102,194.2262369)(203.94989102,194.2262369)
\curveto(203.15301607,194.2262369)(202.54071142,193.96842443)(202.11297707,193.45279949)
\curveto(201.68524272,192.94303393)(201.47137555,192.35416681)(201.47137555,191.68619814)
\lineto(201.47137555,186.27213628)
\lineto(199.88934439,186.27213628)
\lineto(199.88934439,195.6852211)
\closepath
}
}
{
\newrgbcolor{curcolor}{0.98823529 0.98823529 0.98823529}
\pscustom[linestyle=none,fillstyle=solid,fillcolor=curcolor]
{
\newpath
\moveto(210.67807625,302.34206773)
\lineto(370.67806672,302.34206773)
\lineto(370.67806672,242.34207488)
\lineto(210.67807625,242.34207488)
\closepath
}
}
{
\newrgbcolor{curcolor}{0 0 0}
\pscustom[linewidth=1.99999982,linecolor=curcolor]
{
\newpath
\moveto(210.67807625,302.34206773)
\lineto(370.67806672,302.34206773)
\lineto(370.67806672,242.34207488)
\lineto(210.67807625,242.34207488)
\closepath
}
}
{
\newrgbcolor{curcolor}{0 0 0}
\pscustom[linestyle=none,fillstyle=solid,fillcolor=curcolor]
{
\newpath
\moveto(241.613133,280.96411007)
\curveto(243.24789853,280.96411007)(244.51645314,280.53344606)(245.41879684,279.67211804)
\curveto(246.32114053,278.81079002)(246.82211707,277.83227451)(246.92172644,276.73657151)
\lineto(245.21664841,276.73657151)
\curveto(245.02328905,277.56860267)(244.63657032,278.22778227)(244.05649223,278.71411034)
\curveto(243.48227352,279.20043841)(242.67367981,279.44360244)(241.63071113,279.44360244)
\curveto(240.35922683,279.44360244)(239.33090658,278.99536031)(238.54575037,278.09887604)
\curveto(237.76645354,277.20825115)(237.37680513,275.84008725)(237.37680513,273.99438434)
\curveto(237.37680513,272.48266577)(237.72836761,271.25512686)(238.43149257,270.31176759)
\curveto(239.1404769,269.3742677)(240.19516434,268.90551776)(241.59555488,268.90551776)
\curveto(242.8846173,268.90551776)(243.86606256,269.40063489)(244.53989064,270.39086915)
\curveto(244.89731249,270.91235346)(245.16391404,271.59790025)(245.33969528,272.44750953)
\lineto(247.0447733,272.44750953)
\curveto(246.89242956,271.08813469)(246.38852334,269.94848639)(245.53305464,269.02856462)
\curveto(244.50766408,267.92114288)(243.12485166,267.36743201)(241.38461739,267.36743201)
\curveto(239.88461748,267.36743201)(238.62485193,267.82153351)(237.60532074,268.72973653)
\curveto(236.26352395,269.93090826)(235.59262555,271.78540023)(235.59262555,274.29321243)
\curveto(235.59262555,276.19750908)(236.09653177,277.75903233)(237.10434421,278.97778218)
\curveto(238.19418789,280.30200078)(239.69711749,280.96411007)(241.613133,280.96411007)
\closepath
\moveto(241.26157052,280.96411007)
\lineto(241.26157052,280.96411007)
\closepath
}
}
{
\newrgbcolor{curcolor}{0 0 0}
\pscustom[linestyle=none,fillstyle=solid,fillcolor=curcolor]
{
\newpath
\moveto(255.80746809,272.99243134)
\lineto(253.84750727,278.69653222)
\lineto(251.76449958,272.99243134)
\lineto(255.80746809,272.99243134)
\closepath
\moveto(252.93344483,280.61254761)
\lineto(254.91098377,280.61254761)
\lineto(259.59555381,267.70141634)
\lineto(257.67953829,267.70141634)
\lineto(256.36996806,271.56860338)
\lineto(251.26352305,271.56860338)
\lineto(249.8660622,267.70141634)
\lineto(248.07309355,267.70141634)
\lineto(252.93344483,280.61254761)
\closepath
\moveto(253.83871821,280.61254761)
\lineto(253.83871821,280.61254761)
\closepath
}
}
{
\newrgbcolor{curcolor}{0 0 0}
\pscustom[linestyle=none,fillstyle=solid,fillcolor=curcolor]
{
\newpath
\moveto(266.1609831,269.19555679)
\curveto(266.75277994,269.19555679)(267.23910804,269.25708022)(267.61996739,269.38012708)
\curveto(268.29965485,269.60864268)(268.85629544,270.04809575)(269.28988916,270.6984863)
\curveto(269.63559227,271.21997061)(269.88461569,271.88793928)(270.03695943,272.70239231)
\curveto(270.12485005,273.18872037)(270.16879536,273.6398922)(270.16879536,274.05590777)
\curveto(270.16879536,275.65551696)(269.84945944,276.89770431)(269.21078761,277.78246983)
\curveto(268.57797514,278.66723535)(267.55551427,279.10961811)(266.14340498,279.10961811)
\lineto(263.0408661,279.10961811)
\lineto(263.0408661,269.19555679)
\lineto(266.1609831,269.19555679)
\closepath
\moveto(261.2830537,280.61254761)
\lineto(266.51254558,280.61254761)
\curveto(268.2879361,280.61254761)(269.66488914,279.98266488)(270.64340471,278.7228994)
\curveto(271.51645153,277.58618079)(271.95297494,276.13012627)(271.95297494,274.35473586)
\curveto(271.95297494,272.98364227)(271.69516246,271.74438461)(271.17953749,270.63696287)
\curveto(270.27133442,268.67993185)(268.70981107,267.70141634)(266.49496746,267.70141634)
\lineto(261.2830537,267.70141634)
\lineto(261.2830537,280.61254761)
\closepath
}
}
{
\newrgbcolor{curcolor}{0 0 0}
\pscustom[linestyle=none,fillstyle=solid,fillcolor=curcolor]
{
\newpath
\moveto(279.37094325,280.61254761)
\lineto(285.18051322,280.61254761)
\curveto(286.32895065,280.61254761)(287.25473184,280.28735234)(287.9578568,279.63696179)
\curveto(288.66098176,278.99243062)(289.01254424,278.0842276)(289.01254424,276.91235274)
\curveto(289.01254424,275.90454036)(288.6990677,275.02563422)(288.07211461,274.27563431)
\curveto(287.44516152,273.53149377)(286.48129439,273.1594235)(285.18051322,273.1594235)
\lineto(281.11996658,273.1594235)
\lineto(281.11996658,267.70141634)
\lineto(279.37094325,267.70141634)
\lineto(279.37094325,280.61254761)
\closepath
\moveto(287.24594278,276.90356368)
\curveto(287.24594278,277.85278232)(286.8943803,278.49731349)(286.19125534,278.8371572)
\curveto(285.80453662,279.0187978)(285.27426321,279.10961811)(284.60043513,279.10961811)
\lineto(281.11996658,279.10961811)
\lineto(281.11996658,274.63598583)
\lineto(284.60043513,274.63598583)
\curveto(285.38559133,274.63598583)(286.02133348,274.802978)(286.50766158,275.13696233)
\curveto(286.99984905,275.47094667)(287.24594278,276.05981378)(287.24594278,276.90356368)
\closepath
}
}
{
\newrgbcolor{curcolor}{0 0 0}
\pscustom[linestyle=none,fillstyle=solid,fillcolor=curcolor]
{
\newpath
\moveto(291.06039568,277.11450116)
\lineto(292.56332528,277.11450116)
\lineto(292.56332528,275.48852479)
\curveto(292.68637215,275.804931)(292.98812994,276.18872002)(293.46859866,276.63989184)
\curveto(293.94906738,277.09692303)(294.50277829,277.32543863)(295.12973137,277.32543863)
\curveto(295.15902825,277.32543863)(295.20883293,277.32250894)(295.27914543,277.31664957)
\curveto(295.34945792,277.3107902)(295.4695751,277.29907145)(295.63949697,277.28149332)
\lineto(295.63949697,275.61157165)
\curveto(295.54574697,275.62914977)(295.45785636,275.64086852)(295.37582511,275.64672789)
\curveto(295.29965324,275.65258727)(295.21469231,275.65551696)(295.12094231,275.65551696)
\curveto(294.32406736,275.65551696)(293.71176271,275.39770449)(293.28402836,274.88207955)
\curveto(292.85629401,274.37231398)(292.64242684,273.78344687)(292.64242684,273.1154782)
\lineto(292.64242684,267.70141634)
\lineto(291.06039568,267.70141634)
\lineto(291.06039568,277.11450116)
\closepath
}
}
{
\newrgbcolor{curcolor}{0 0 0}
\pscustom[linestyle=none,fillstyle=solid,fillcolor=curcolor]
{
\newpath
\moveto(300.7635201,268.72094747)
\curveto(301.81234816,268.72094747)(302.53012156,269.11645524)(302.91684029,269.90747077)
\curveto(303.30941839,270.70434567)(303.50570744,271.58911119)(303.50570744,272.56176732)
\curveto(303.50570744,273.44067347)(303.36508245,274.15551713)(303.08383246,274.70629832)
\curveto(302.63851999,275.57348572)(301.87094191,276.00707941)(300.78109823,276.00707941)
\curveto(299.81430141,276.00707941)(299.11117645,275.63793883)(298.67172335,274.89965767)
\curveto(298.23227025,274.16137651)(298.0125437,273.27075162)(298.0125437,272.22778299)
\curveto(298.0125437,271.22582998)(298.23227025,270.39086915)(298.67172335,269.72290048)
\curveto(299.11117645,269.05493181)(299.80844203,268.72094747)(300.7635201,268.72094747)
\closepath
\moveto(300.82504354,277.38696206)
\curveto(302.03793409,277.38696206)(303.06332465,276.98266524)(303.90121523,276.17407158)
\curveto(304.7391058,275.36547793)(305.15805109,274.17602494)(305.15805109,272.60571263)
\curveto(305.15805109,271.08813469)(304.78891049,269.83422859)(304.05062928,268.84399433)
\curveto(303.31234807,267.85376007)(302.16684033,267.35864294)(300.61410605,267.35864294)
\curveto(299.31918425,267.35864294)(298.290864,267.79516633)(297.52914529,268.6682131)
\curveto(296.76742659,269.54711925)(296.38656724,270.72485348)(296.38656724,272.20141581)
\curveto(296.38656724,273.78344687)(296.7879344,275.04321234)(297.59066873,275.98071223)
\curveto(298.39340306,276.91821212)(299.47152799,277.38696206)(300.82504354,277.38696206)
\closepath
\moveto(300.77230916,277.34301675)
\lineto(300.77230916,277.34301675)
\closepath
}
}
{
\newrgbcolor{curcolor}{0 0 0}
\pscustom[linestyle=none,fillstyle=solid,fillcolor=curcolor]
{
\newpath
\moveto(310.36996484,277.28149332)
\curveto(311.10824605,277.28149332)(311.75277726,277.09985272)(312.30355848,276.73657151)
\curveto(312.60238658,276.53149341)(312.90707407,276.23266532)(313.21762092,275.84008725)
\lineto(313.21762092,277.02661054)
\lineto(314.67660521,277.02661054)
\lineto(314.67660521,268.46606469)
\curveto(314.67660521,267.27075233)(314.50082397,266.32739307)(314.14926149,265.6359869)
\curveto(313.49301153,264.3586433)(312.25375379,263.7199715)(310.43148828,263.7199715)
\curveto(309.41781646,263.7199715)(308.56527745,263.9484871)(307.87387124,264.4055183)
\curveto(307.18246503,264.85669012)(306.7957463,265.56567441)(306.71371506,266.53247117)
\lineto(308.3221134,266.53247117)
\curveto(308.39828527,266.11059622)(308.55062901,265.78540095)(308.77914462,265.55688535)
\curveto(309.13656648,265.20532289)(309.69906644,265.02954166)(310.46664452,265.02954166)
\curveto(311.67953508,265.02954166)(312.47348034,265.45727598)(312.84848032,266.31274463)
\curveto(313.07113656,266.81665082)(313.17367561,267.71606478)(313.15609749,269.0109865)
\curveto(312.83969126,268.53051781)(312.4588319,268.17309597)(312.01351943,267.938721)
\curveto(311.56820696,267.70434603)(310.97933981,267.58715854)(310.24691797,267.58715854)
\curveto(309.22738678,267.58715854)(308.33383215,267.94751006)(307.56625407,268.6682131)
\curveto(306.80453537,269.39477551)(306.42367601,270.59301756)(306.42367601,272.26293924)
\curveto(306.42367601,273.83911092)(306.80746505,275.06957953)(307.57504313,275.95434505)
\curveto(308.34848059,276.83911056)(309.28012116,277.28149332)(310.36996484,277.28149332)
\closepath
\moveto(313.21762092,272.44750953)
\curveto(313.21762092,273.61352501)(312.97738656,274.47778272)(312.49691784,275.04028265)
\curveto(312.01644912,275.60278259)(311.40414447,275.88403255)(310.66000389,275.88403255)
\curveto(309.5467227,275.88403255)(308.785004,275.36254824)(308.37484777,274.31957962)
\curveto(308.15805091,273.76293906)(308.04965248,273.03344696)(308.04965248,272.13110331)
\curveto(308.04965248,271.07055656)(308.26351965,270.26196291)(308.691254,269.70532235)
\curveto(309.12484773,269.15454117)(309.70492582,268.87915058)(310.43148828,268.87915058)
\curveto(311.56820696,268.87915058)(312.3680116,269.39184583)(312.83090219,270.41723633)
\curveto(313.08871468,270.99731439)(313.21762092,271.67407212)(313.21762092,272.44750953)
\closepath
\moveto(310.55453514,277.34301675)
\lineto(310.55453514,277.34301675)
\closepath
}
}
{
\newrgbcolor{curcolor}{0 0 0}
\pscustom[linestyle=none,fillstyle=solid,fillcolor=curcolor]
{
\newpath
\moveto(317.11117538,277.11450116)
\lineto(318.61410497,277.11450116)
\lineto(318.61410497,275.48852479)
\curveto(318.73715184,275.804931)(319.03890964,276.18872002)(319.51937836,276.63989184)
\curveto(319.99984708,277.09692303)(320.55355798,277.32543863)(321.18051107,277.32543863)
\curveto(321.20980795,277.32543863)(321.25961263,277.32250894)(321.32992513,277.31664957)
\curveto(321.40023762,277.3107902)(321.5203548,277.29907145)(321.69027667,277.28149332)
\lineto(321.69027667,275.61157165)
\curveto(321.59652667,275.62914977)(321.50863605,275.64086852)(321.42660481,275.64672789)
\curveto(321.35043294,275.65258727)(321.265472,275.65551696)(321.17172201,275.65551696)
\curveto(320.37484706,275.65551696)(319.76254241,275.39770449)(319.33480806,274.88207955)
\curveto(318.90707371,274.37231398)(318.69320653,273.78344687)(318.69320653,273.1154782)
\lineto(318.69320653,267.70141634)
\lineto(317.11117538,267.70141634)
\lineto(317.11117538,277.11450116)
\closepath
}
}
{
\newrgbcolor{curcolor}{0 0 0}
\pscustom[linestyle=none,fillstyle=solid,fillcolor=curcolor]
{
\newpath
\moveto(324.29183901,270.20629886)
\curveto(324.29183901,269.74926766)(324.45883119,269.38891614)(324.79281554,269.1252443)
\curveto(325.1267999,268.86157245)(325.52230769,268.72973653)(325.97933891,268.72973653)
\curveto(326.5359795,268.72973653)(327.07504197,268.85864277)(327.59652631,269.11645524)
\curveto(328.47543251,269.54418956)(328.91488561,270.24438479)(328.91488561,271.21704092)
\lineto(328.91488561,272.49145483)
\curveto(328.72152625,272.36840797)(328.47250282,272.26586892)(328.16781534,272.18383768)
\curveto(327.86312786,272.10180644)(327.56429975,272.0432127)(327.27133102,272.00805645)
\lineto(326.31332327,271.88500959)
\curveto(325.73910455,271.80883773)(325.30844051,271.68872055)(325.02133116,271.52465807)
\curveto(324.53500306,271.24926748)(324.29183901,270.80981441)(324.29183901,270.20629886)
\closepath
\moveto(328.12387003,273.40551722)
\curveto(328.48715126,273.45239222)(328.73031531,273.60473595)(328.85336218,273.86254842)
\curveto(328.92367467,274.0031734)(328.95883092,274.20532182)(328.95883092,274.46899366)
\curveto(328.95883092,275.0080561)(328.76547156,275.39770449)(328.37875283,275.63793883)
\curveto(327.99789348,275.88403255)(327.45004195,276.00707941)(326.73519824,276.00707941)
\curveto(325.90902642,276.00707941)(325.32308895,275.78442319)(324.97738585,275.33911074)
\curveto(324.78402648,275.09301702)(324.65804993,274.72680613)(324.59945618,274.24047806)
\lineto(323.12289377,274.24047806)
\curveto(323.15219064,275.40063417)(323.52719062,276.20629814)(324.2478937,276.65746996)
\curveto(324.97445616,277.11450116)(325.81527642,277.34301675)(326.77035449,277.34301675)
\curveto(327.8777763,277.34301675)(328.77719031,277.13207928)(329.46859652,276.71020433)
\curveto(330.15414335,276.28832938)(330.49691677,275.63207946)(330.49691677,274.74145456)
\lineto(330.49691677,269.31860365)
\curveto(330.49691677,269.15454117)(330.52914333,269.02270525)(330.59359645,268.92309588)
\curveto(330.66390894,268.82348652)(330.80746362,268.77368184)(331.02426049,268.77368184)
\curveto(331.09457298,268.77368184)(331.17367454,268.77661153)(331.26156516,268.7824709)
\curveto(331.34945578,268.79418965)(331.44320577,268.80883808)(331.54281514,268.82641621)
\lineto(331.54281514,267.65747103)
\curveto(331.29672141,267.58715854)(331.10922142,267.54321324)(330.98031518,267.52563511)
\curveto(330.85140893,267.50805699)(330.67562769,267.49926793)(330.45297146,267.49926793)
\curveto(329.90804961,267.49926793)(329.51254183,267.69262728)(329.26644809,268.07934598)
\curveto(329.13754185,268.28442408)(329.04672154,268.57446311)(328.99398717,268.94946307)
\curveto(328.67172156,268.52758812)(328.20883097,268.16137722)(327.60531538,267.85083039)
\curveto(327.00179979,267.54028355)(326.33676076,267.38501013)(325.61019831,267.38501013)
\curveto(324.73715148,267.38501013)(324.02230778,267.64868197)(323.46566719,268.17602566)
\curveto(322.91488597,268.70922872)(322.63949536,269.3742677)(322.63949536,270.17114261)
\curveto(322.63949536,271.04418938)(322.91195628,271.72094711)(323.45687812,272.20141581)
\curveto(324.00179997,272.6818845)(324.71664367,272.9777829)(325.60140925,273.08911101)
\lineto(328.12387003,273.40551722)
\closepath
\moveto(326.8142998,277.34301675)
\lineto(326.8142998,277.34301675)
\closepath
}
}
{
\newrgbcolor{curcolor}{0 0 0}
\pscustom[linestyle=none,fillstyle=solid,fillcolor=curcolor]
{
\newpath
\moveto(333.09847911,277.11450116)
\lineto(334.66293214,277.11450116)
\lineto(334.66293214,275.77856382)
\curveto(335.03793212,276.24145439)(335.37777585,276.57836841)(335.68246333,276.78930588)
\curveto(336.20394768,277.14672772)(336.79574452,277.32543863)(337.45785385,277.32543863)
\curveto(338.20785381,277.32543863)(338.8113694,277.14086834)(339.26840062,276.77172776)
\curveto(339.5262131,276.56079029)(339.76058809,276.25024345)(339.97152558,275.84008725)
\curveto(340.32308806,276.34399344)(340.73617397,276.7160637)(341.21078332,276.95629805)
\curveto(341.68539266,277.20239177)(342.21859576,277.32543863)(342.8103926,277.32543863)
\curveto(344.07601752,277.32543863)(344.93734559,276.86840744)(345.39437682,275.95434505)
\curveto(345.64047055,275.4621576)(345.76351742,274.80004831)(345.76351742,273.96801716)
\lineto(345.76351742,267.70141634)
\lineto(344.11996283,267.70141634)
\lineto(344.11996283,274.24047806)
\curveto(344.11996283,274.86743111)(343.96175971,275.29809512)(343.64535348,275.5324701)
\curveto(343.33480663,275.76684507)(342.95394727,275.88403255)(342.50277543,275.88403255)
\curveto(341.88168171,275.88403255)(341.34554893,275.67602477)(340.89437708,275.26000919)
\curveto(340.44906461,274.84399362)(340.22640837,274.14965776)(340.22640837,273.17700163)
\lineto(340.22640837,267.70141634)
\lineto(338.61801003,267.70141634)
\lineto(338.61801003,273.8449703)
\curveto(338.61801003,274.4836421)(338.54183816,274.94946235)(338.38949442,275.24243107)
\curveto(338.14926006,275.68188414)(337.7010179,275.90161068)(337.04476794,275.90161068)
\curveto(336.44711172,275.90161068)(335.90218988,275.67016539)(335.41000241,275.20727482)
\curveto(334.92367432,274.74438425)(334.68051027,273.90649373)(334.68051027,272.69360325)
\lineto(334.68051027,267.70141634)
\lineto(333.09847911,267.70141634)
\lineto(333.09847911,277.11450116)
\closepath
}
}
{
\newrgbcolor{curcolor}{0.98823529 0.98823529 0.98823529}
\pscustom[linestyle=none,fillstyle=solid,fillcolor=curcolor]
{
\newpath
\moveto(1.0000002,303.72464973)
\lineto(160.99999066,303.72464973)
\lineto(160.99999066,243.72465689)
\lineto(1.0000002,243.72465689)
\closepath
}
}
{
\newrgbcolor{curcolor}{0 0 0}
\pscustom[linewidth=1.99999982,linecolor=curcolor]
{
\newpath
\moveto(1.0000002,303.72464973)
\lineto(160.99999066,303.72464973)
\lineto(160.99999066,243.72465689)
\lineto(1.0000002,243.72465689)
\closepath
}
}
{
\newrgbcolor{curcolor}{0 0 0}
\pscustom[linestyle=none,fillstyle=solid,fillcolor=curcolor]
{
\newpath
\moveto(19.17333896,280.32523846)
\lineto(20.93994042,280.32523846)
\lineto(20.93994042,267.41410719)
\lineto(19.17333896,267.41410719)
\lineto(19.17333896,280.32523846)
\closepath
}
}
{
\newrgbcolor{curcolor}{0 0 0}
\pscustom[linestyle=none,fillstyle=solid,fillcolor=curcolor]
{
\newpath
\moveto(23.55908089,276.827192)
\lineto(25.06201048,276.827192)
\lineto(25.06201048,275.49125466)
\curveto(25.50732296,276.04203585)(25.97900262,276.43754361)(26.47704946,276.67777796)
\curveto(26.97509631,276.9180123)(27.52880721,277.03812948)(28.13818218,277.03812948)
\curveto(29.4741196,277.03812948)(30.37646329,276.57230922)(30.84521326,275.64066871)
\curveto(31.10302575,275.13090314)(31.23193199,274.40141104)(31.23193199,273.4521924)
\lineto(31.23193199,267.41410719)
\lineto(29.62353365,267.41410719)
\lineto(29.62353365,273.34672367)
\curveto(29.62353365,273.92094235)(29.53857272,274.38383292)(29.36865085,274.73539538)
\curveto(29.08740087,275.32133281)(28.57763527,275.61430152)(27.83935407,275.61430152)
\curveto(27.46435409,275.61430152)(27.15673692,275.57621559)(26.91650256,275.50004372)
\curveto(26.48290884,275.37113749)(26.10204948,275.11332502)(25.7739245,274.72660631)
\curveto(25.51025264,274.41605948)(25.33740109,274.09379389)(25.25536985,273.75980955)
\curveto(25.17919798,273.43168459)(25.14111204,272.96000496)(25.14111204,272.34477066)
\lineto(25.14111204,267.41410719)
\lineto(23.55908089,267.41410719)
\lineto(23.55908089,276.827192)
\closepath
\moveto(27.2768541,277.0557076)
\lineto(27.2768541,277.0557076)
\closepath
}
}
{
\newrgbcolor{curcolor}{0 0 0}
\pscustom[linestyle=none,fillstyle=solid,fillcolor=curcolor]
{
\newpath
\moveto(33.89501777,279.45512138)
\lineto(35.49462705,279.45512138)
\lineto(35.49462705,276.827192)
\lineto(36.99755665,276.827192)
\lineto(36.99755665,275.53519997)
\lineto(35.49462705,275.53519997)
\lineto(35.49462705,269.39164601)
\curveto(35.49462705,269.06352105)(35.60595517,268.84379452)(35.8286114,268.7324664)
\curveto(35.95165827,268.66801329)(36.15673638,268.63578673)(36.44384574,268.63578673)
\lineto(36.68993948,268.63578673)
\curveto(36.7778301,268.6416461)(36.88036915,268.65043516)(36.99755665,268.66215391)
\lineto(36.99755665,267.41410719)
\curveto(36.81591603,267.36137282)(36.62548636,267.32328688)(36.42626762,267.29984939)
\curveto(36.23290826,267.27641189)(36.02197077,267.26469314)(35.79345516,267.26469314)
\curveto(35.05517395,267.26469314)(34.55419742,267.45219312)(34.29052556,267.82719307)
\curveto(34.0268537,268.2080524)(33.89501777,268.70023985)(33.89501777,269.3037554)
\lineto(33.89501777,275.53519997)
\lineto(32.62060378,275.53519997)
\lineto(32.62060378,276.827192)
\lineto(33.89501777,276.827192)
\lineto(33.89501777,279.45512138)
\closepath
}
}
{
\newrgbcolor{curcolor}{0 0 0}
\pscustom[linestyle=none,fillstyle=solid,fillcolor=curcolor]
{
\newpath
\moveto(42.52587663,277.03812948)
\curveto(43.19384534,277.03812948)(43.84130624,276.87992637)(44.46825933,276.56352016)
\curveto(45.09521241,276.25297332)(45.57275145,275.84867649)(45.90087643,275.35062968)
\curveto(46.21728266,274.87602036)(46.42822015,274.32230949)(46.53368889,273.68949706)
\curveto(46.62743889,273.25590337)(46.67431388,272.5644972)(46.67431388,271.61527856)
\lineto(39.77490023,271.61527856)
\curveto(39.80419711,270.66020055)(40.02978303,269.89262252)(40.451658,269.31254446)
\curveto(40.87353298,268.73832578)(41.52685325,268.45121644)(42.41161882,268.45121644)
\curveto(43.23779065,268.45121644)(43.8969703,268.72367734)(44.38915777,269.26859915)
\curveto(44.67040775,269.58500537)(44.86962649,269.95121626)(44.98681398,270.36723183)
\lineto(46.54247795,270.36723183)
\curveto(46.50146233,270.02152875)(46.36376703,269.63481005)(46.12939204,269.20707572)
\curveto(45.90087643,268.78520077)(45.64306394,268.43949769)(45.35595459,268.16996647)
\curveto(44.87548587,267.70121653)(44.28075934,267.38481031)(43.57177501,267.22074783)
\curveto(43.19091565,267.12699785)(42.76025162,267.08012285)(42.27978289,267.08012285)
\curveto(41.10790796,267.08012285)(40.11474396,267.50492749)(39.30029089,268.35453676)
\curveto(38.48583781,269.21000541)(38.07861127,270.40531777)(38.07861127,271.94047383)
\curveto(38.07861127,273.4521924)(38.4887675,274.67973132)(39.30907995,275.62309058)
\curveto(40.1293924,276.56644985)(41.20165796,277.03812948)(42.52587663,277.03812948)
\closepath
\moveto(45.04833742,272.87211435)
\curveto(44.9838843,273.55766114)(44.83447024,274.10551264)(44.60009526,274.51566884)
\curveto(44.16650153,275.2773875)(43.44286876,275.65824683)(42.42919695,275.65824683)
\curveto(41.70263449,275.65824683)(41.09325953,275.39457499)(40.60107206,274.8672313)
\curveto(40.10888459,274.34574699)(39.84814241,273.680708)(39.81884554,272.87211435)
\lineto(45.04833742,272.87211435)
\closepath
\moveto(42.37646258,277.0557076)
\lineto(42.37646258,277.0557076)
\closepath
}
}
{
\newrgbcolor{curcolor}{0 0 0}
\pscustom[linestyle=none,fillstyle=solid,fillcolor=curcolor]
{
\newpath
\moveto(48.66943095,276.827192)
\lineto(50.17236055,276.827192)
\lineto(50.17236055,275.20121563)
\curveto(50.29540742,275.51762185)(50.59716521,275.90141086)(51.07763393,276.35258268)
\curveto(51.55810265,276.80961388)(52.11181356,277.03812948)(52.73876665,277.03812948)
\curveto(52.76806352,277.03812948)(52.8178682,277.03519979)(52.8881807,277.02934042)
\curveto(52.9584932,277.02348104)(53.07861038,277.01176229)(53.24853224,276.99418417)
\lineto(53.24853224,275.32426249)
\curveto(53.15478225,275.34184062)(53.06689163,275.35355936)(52.98486038,275.35941874)
\curveto(52.90868851,275.36527811)(52.82372758,275.3682078)(52.72997758,275.3682078)
\curveto(51.93310263,275.3682078)(51.32079798,275.11039533)(50.89306363,274.59477039)
\curveto(50.46532928,274.08500483)(50.25146211,273.49613771)(50.25146211,272.82816904)
\lineto(50.25146211,267.41410719)
\lineto(48.66943095,267.41410719)
\lineto(48.66943095,276.827192)
\closepath
}
}
{
\newrgbcolor{curcolor}{0 0 0}
\pscustom[linestyle=none,fillstyle=solid,fillcolor=curcolor]
{
\newpath
\moveto(54.63720403,276.827192)
\lineto(56.14013363,276.827192)
\lineto(56.14013363,275.49125466)
\curveto(56.5854461,276.04203585)(57.05712576,276.43754361)(57.55517261,276.67777796)
\curveto(58.05321945,276.9180123)(58.60693036,277.03812948)(59.21630532,277.03812948)
\curveto(60.55224274,277.03812948)(61.45458644,276.57230922)(61.92333641,275.64066871)
\curveto(62.1811489,275.13090314)(62.31005514,274.40141104)(62.31005514,273.4521924)
\lineto(62.31005514,267.41410719)
\lineto(60.7016568,267.41410719)
\lineto(60.7016568,273.34672367)
\curveto(60.7016568,273.92094235)(60.61669586,274.38383292)(60.446774,274.73539538)
\curveto(60.16552402,275.32133281)(59.65575842,275.61430152)(58.91747722,275.61430152)
\curveto(58.54247724,275.61430152)(58.23486007,275.57621559)(57.99462571,275.50004372)
\curveto(57.56103198,275.37113749)(57.18017263,275.11332502)(56.85204765,274.72660631)
\curveto(56.58837579,274.41605948)(56.41552424,274.09379389)(56.33349299,273.75980955)
\curveto(56.25732112,273.43168459)(56.21923519,272.96000496)(56.21923519,272.34477066)
\lineto(56.21923519,267.41410719)
\lineto(54.63720403,267.41410719)
\lineto(54.63720403,276.827192)
\closepath
\moveto(58.35497725,277.0557076)
\lineto(58.35497725,277.0557076)
\closepath
}
}
{
\newrgbcolor{curcolor}{0 0 0}
\pscustom[linestyle=none,fillstyle=solid,fillcolor=curcolor]
{
\newpath
\moveto(68.57665633,277.03812948)
\curveto(69.24462504,277.03812948)(69.89208594,276.87992637)(70.51903902,276.56352016)
\curveto(71.14599211,276.25297332)(71.62353115,275.84867649)(71.95165613,275.35062968)
\curveto(72.26806236,274.87602036)(72.47899984,274.32230949)(72.58446859,273.68949706)
\curveto(72.67821858,273.25590337)(72.72509358,272.5644972)(72.72509358,271.61527856)
\lineto(65.82567993,271.61527856)
\curveto(65.8549768,270.66020055)(66.08056273,269.89262252)(66.5024377,269.31254446)
\curveto(66.92431268,268.73832578)(67.57763295,268.45121644)(68.46239852,268.45121644)
\curveto(69.28857035,268.45121644)(69.94775,268.72367734)(70.43993747,269.26859915)
\curveto(70.72118745,269.58500537)(70.92040619,269.95121626)(71.03759368,270.36723183)
\lineto(72.59325765,270.36723183)
\curveto(72.55224203,270.02152875)(72.41454672,269.63481005)(72.18017174,269.20707572)
\curveto(71.95165613,268.78520077)(71.69384364,268.43949769)(71.40673428,268.16996647)
\curveto(70.92626556,267.70121653)(70.33153904,267.38481031)(69.6225547,267.22074783)
\curveto(69.24169535,267.12699785)(68.81103131,267.08012285)(68.33056259,267.08012285)
\curveto(67.15868766,267.08012285)(66.16552366,267.50492749)(65.35107058,268.35453676)
\curveto(64.53661751,269.21000541)(64.12939097,270.40531777)(64.12939097,271.94047383)
\curveto(64.12939097,273.4521924)(64.53954719,274.67973132)(65.35985964,275.62309058)
\curveto(66.1801721,276.56644985)(67.25243766,277.03812948)(68.57665633,277.03812948)
\closepath
\moveto(71.09911711,272.87211435)
\curveto(71.03466399,273.55766114)(70.88524994,274.10551264)(70.65087495,274.51566884)
\curveto(70.21728123,275.2773875)(69.49364846,275.65824683)(68.47997665,275.65824683)
\curveto(67.75341419,275.65824683)(67.14403923,275.39457499)(66.65185175,274.8672313)
\curveto(66.15966428,274.34574699)(65.89892211,273.680708)(65.86962524,272.87211435)
\lineto(71.09911711,272.87211435)
\closepath
\moveto(68.42724227,277.0557076)
\lineto(68.42724227,277.0557076)
\closepath
}
}
{
\newrgbcolor{curcolor}{0 0 0}
\pscustom[linestyle=none,fillstyle=solid,fillcolor=curcolor]
{
\newpath
\moveto(74.99267157,279.45512138)
\lineto(76.59228085,279.45512138)
\lineto(76.59228085,276.827192)
\lineto(78.09521045,276.827192)
\lineto(78.09521045,275.53519997)
\lineto(76.59228085,275.53519997)
\lineto(76.59228085,269.39164601)
\curveto(76.59228085,269.06352105)(76.70360897,268.84379452)(76.9262652,268.7324664)
\curveto(77.04931207,268.66801329)(77.25439019,268.63578673)(77.54149954,268.63578673)
\lineto(77.78759328,268.63578673)
\curveto(77.8754839,268.6416461)(77.97802295,268.65043516)(78.09521045,268.66215391)
\lineto(78.09521045,267.41410719)
\curveto(77.91356983,267.36137282)(77.72314016,267.32328688)(77.52392142,267.29984939)
\curveto(77.33056206,267.27641189)(77.11962457,267.26469314)(76.89110896,267.26469314)
\curveto(76.15282775,267.26469314)(75.65185122,267.45219312)(75.38817936,267.82719307)
\curveto(75.1245075,268.2080524)(74.99267157,268.70023985)(74.99267157,269.3037554)
\lineto(74.99267157,275.53519997)
\lineto(73.71825758,275.53519997)
\lineto(73.71825758,276.827192)
\lineto(74.99267157,276.827192)
\lineto(74.99267157,279.45512138)
\closepath
}
}
{
\newrgbcolor{curcolor}{0 0 0}
\pscustom[linestyle=none,fillstyle=solid,fillcolor=curcolor]
{
\newpath
\moveto(91.53368621,272.70512218)
\lineto(89.57372539,278.40922306)
\lineto(87.4907177,272.70512218)
\lineto(91.53368621,272.70512218)
\closepath
\moveto(88.65966294,280.32523846)
\lineto(90.63720189,280.32523846)
\lineto(95.32177192,267.41410719)
\lineto(93.40575641,267.41410719)
\lineto(92.09618618,271.28129423)
\lineto(86.98974117,271.28129423)
\lineto(85.59228031,267.41410719)
\lineto(83.79931167,267.41410719)
\lineto(88.65966294,280.32523846)
\closepath
\moveto(89.56493633,280.32523846)
\lineto(89.56493633,280.32523846)
\closepath
}
}
{
\newrgbcolor{curcolor}{0 0 0}
\pscustom[linestyle=none,fillstyle=solid,fillcolor=curcolor]
{
\newpath
\moveto(96.76317809,276.827192)
\lineto(98.26610768,276.827192)
\lineto(98.26610768,275.20121563)
\curveto(98.38915455,275.51762185)(98.69091235,275.90141086)(99.17138107,276.35258268)
\curveto(99.65184979,276.80961388)(100.20556069,277.03812948)(100.83251378,277.03812948)
\curveto(100.86181065,277.03812948)(100.91161534,277.03519979)(100.98192783,277.02934042)
\curveto(101.05224033,277.02348104)(101.17235751,277.01176229)(101.34227937,276.99418417)
\lineto(101.34227937,275.32426249)
\curveto(101.24852938,275.34184062)(101.16063876,275.35355936)(101.07860752,275.35941874)
\curveto(101.00243564,275.36527811)(100.91747471,275.3682078)(100.82372472,275.3682078)
\curveto(100.02684977,275.3682078)(99.41454511,275.11039533)(98.98681076,274.59477039)
\curveto(98.55907642,274.08500483)(98.34520924,273.49613771)(98.34520924,272.82816904)
\lineto(98.34520924,267.41410719)
\lineto(96.76317809,267.41410719)
\lineto(96.76317809,276.827192)
\closepath
}
}
{
\newrgbcolor{curcolor}{0 0 0}
\pscustom[linestyle=none,fillstyle=solid,fillcolor=curcolor]
{
\newpath
\moveto(106.36083376,277.09965291)
\curveto(107.42138057,277.09965291)(108.28270865,276.84184044)(108.94481798,276.3262155)
\curveto(109.61278669,275.81059056)(110.01415386,274.92289535)(110.14891947,273.66312988)
\lineto(108.61083363,273.66312988)
\curveto(108.51708363,274.24320793)(108.30321646,274.72367663)(107.9692321,275.10453596)
\curveto(107.63524775,275.49125466)(107.09911497,275.68461401)(106.36083376,275.68461401)
\curveto(105.35302132,275.68461401)(104.63231824,275.19242657)(104.19872452,274.20805169)
\curveto(103.91747453,273.56937989)(103.77684954,272.78129405)(103.77684954,271.84379416)
\curveto(103.77684954,270.9004349)(103.97606828,270.10648968)(104.37450576,269.4619585)
\curveto(104.77294323,268.81742733)(105.39989632,268.49516175)(106.25536502,268.49516175)
\curveto(106.91161498,268.49516175)(107.43016964,268.69438047)(107.81102899,269.09281792)
\curveto(108.19774772,269.49711475)(108.46434926,270.04789593)(108.61083363,270.74516148)
\lineto(110.14891947,270.74516148)
\curveto(109.97313824,269.49711475)(109.53368514,268.58305236)(108.83056018,268.0029743)
\curveto(108.12743522,267.42875562)(107.22802121,267.14164628)(106.13231815,267.14164628)
\curveto(104.90184947,267.14164628)(103.92040422,267.58988842)(103.18798239,268.48637268)
\curveto(102.45556056,269.38871633)(102.08934964,270.51371619)(102.08934964,271.86137228)
\curveto(102.08934964,273.51371583)(102.49071681,274.79984849)(103.29345113,275.71977026)
\curveto(104.09618546,276.63969202)(105.11864634,277.09965291)(106.36083376,277.09965291)
\closepath
\moveto(106.11474003,277.0557076)
\lineto(106.11474003,277.0557076)
\closepath
}
}
{
\newrgbcolor{curcolor}{0 0 0}
\pscustom[linestyle=none,fillstyle=solid,fillcolor=curcolor]
{
\newpath
\moveto(111.73095063,280.36918377)
\lineto(113.31298179,280.36918377)
\lineto(113.31298179,275.55277809)
\curveto(113.68798176,276.02738741)(114.02489581,276.36137174)(114.32372391,276.5547311)
\curveto(114.83348951,276.88871543)(115.46923166,277.0557076)(116.23095036,277.0557076)
\curveto(117.59618466,277.0557076)(118.52196585,276.57816859)(119.00829395,275.62309058)
\curveto(119.27196581,275.10160627)(119.40380174,274.37797354)(119.40380174,273.4521924)
\lineto(119.40380174,267.41410719)
\lineto(117.77782527,267.41410719)
\lineto(117.77782527,273.34672367)
\curveto(117.77782527,274.03812983)(117.68993465,274.54496571)(117.51415341,274.8672313)
\curveto(117.22704405,275.38285624)(116.68798158,275.64066871)(115.89696601,275.64066871)
\curveto(115.24071605,275.64066871)(114.64598952,275.4150828)(114.11278643,274.96391097)
\curveto(113.57958333,274.51273915)(113.31298179,273.66020019)(113.31298179,272.40629409)
\lineto(113.31298179,267.41410719)
\lineto(111.73095063,267.41410719)
\lineto(111.73095063,280.36918377)
\closepath
}
}
{
\newrgbcolor{curcolor}{0 0 0}
\pscustom[linestyle=none,fillstyle=solid,fillcolor=curcolor]
{
\newpath
\moveto(121.75048128,276.78324669)
\lineto(123.35887962,276.78324669)
\lineto(123.35887962,267.41410719)
\lineto(121.75048128,267.41410719)
\lineto(121.75048128,276.78324669)
\closepath
\moveto(121.75048128,280.32523846)
\lineto(123.35887962,280.32523846)
\lineto(123.35887962,278.53226992)
\lineto(121.75048128,278.53226992)
\lineto(121.75048128,280.32523846)
\closepath
}
}
{
\newrgbcolor{curcolor}{0 0 0}
\pscustom[linestyle=none,fillstyle=solid,fillcolor=curcolor]
{
\newpath
\moveto(126.531731,276.827192)
\lineto(129.04540272,269.16313042)
\lineto(131.67333225,276.827192)
\lineto(133.40477746,276.827192)
\lineto(129.85399643,267.41410719)
\lineto(128.16649653,267.41410719)
\lineto(124.69481705,276.827192)
\lineto(126.531731,276.827192)
\closepath
}
}
{
\newrgbcolor{curcolor}{0 0 0}
\pscustom[linestyle=none,fillstyle=solid,fillcolor=curcolor]
{
\newpath
\moveto(138.67821465,277.03812948)
\curveto(139.34618336,277.03812948)(139.99364426,276.87992637)(140.62059735,276.56352016)
\curveto(141.24755043,276.25297332)(141.72508947,275.84867649)(142.05321445,275.35062968)
\curveto(142.36962068,274.87602036)(142.58055817,274.32230949)(142.68602691,273.68949706)
\curveto(142.7797769,273.25590337)(142.8266519,272.5644972)(142.8266519,271.61527856)
\lineto(135.92723825,271.61527856)
\curveto(135.95653512,270.66020055)(136.18212105,269.89262252)(136.60399602,269.31254446)
\curveto(137.025871,268.73832578)(137.67919127,268.45121644)(138.56395684,268.45121644)
\curveto(139.39012867,268.45121644)(140.04930832,268.72367734)(140.54149579,269.26859915)
\curveto(140.82274577,269.58500537)(141.02196451,269.95121626)(141.139152,270.36723183)
\lineto(142.69481597,270.36723183)
\curveto(142.65380035,270.02152875)(142.51610505,269.63481005)(142.28173006,269.20707572)
\curveto(142.05321445,268.78520077)(141.79540196,268.43949769)(141.50829261,268.16996647)
\curveto(141.02782388,267.70121653)(140.43309736,267.38481031)(139.72411302,267.22074783)
\curveto(139.34325367,267.12699785)(138.91258964,267.08012285)(138.43212091,267.08012285)
\curveto(137.26024598,267.08012285)(136.26708198,267.50492749)(135.4526289,268.35453676)
\curveto(134.63817583,269.21000541)(134.23094929,270.40531777)(134.23094929,271.94047383)
\curveto(134.23094929,273.4521924)(134.64110551,274.67973132)(135.46141797,275.62309058)
\curveto(136.28173042,276.56644985)(137.35399598,277.03812948)(138.67821465,277.03812948)
\closepath
\moveto(141.20067544,272.87211435)
\curveto(141.13622232,273.55766114)(140.98680826,274.10551264)(140.75243328,274.51566884)
\curveto(140.31883955,275.2773875)(139.59520678,275.65824683)(138.58153497,275.65824683)
\curveto(137.85497251,275.65824683)(137.24559755,275.39457499)(136.75341008,274.8672313)
\curveto(136.26122261,274.34574699)(136.00048043,273.680708)(135.97118356,272.87211435)
\lineto(141.20067544,272.87211435)
\closepath
\moveto(138.5288006,277.0557076)
\lineto(138.5288006,277.0557076)
\closepath
}
}
{
\newrgbcolor{curcolor}{0 0 0}
\pscustom[linewidth=0.99999991,linecolor=curcolor]
{
\newpath
\moveto(160.99999066,273.72465331)
\lineto(182.49994113,273.72465331)
\lineto(182.49994113,222.71451475)
}
}
{
\newrgbcolor{curcolor}{0 0 0}
\pscustom[linestyle=none,fillstyle=solid,fillcolor=curcolor]
{
\newpath
\moveto(182.49994113,232.71451356)
\lineto(178.49994137,236.71451308)
\lineto(182.49994113,222.71451475)
\lineto(186.49994089,236.71451308)
\lineto(182.49994113,232.71451356)
\closepath
}
}
{
\newrgbcolor{curcolor}{0 0 0}
\pscustom[linewidth=0.99999991,linecolor=curcolor]
{
\newpath
\moveto(182.49994113,232.71451356)
\lineto(178.49994137,236.71451308)
\lineto(182.49994113,222.71451475)
\lineto(186.49994089,236.71451308)
\lineto(182.49994113,232.71451356)
\closepath
}
}
{
\newrgbcolor{curcolor}{0 0 0}
\pscustom[linewidth=0.99999991,linecolor=curcolor]
{
\newpath
\moveto(290.67807148,242.34207488)
\lineto(290.67807148,190.71451857)
\lineto(262.49993636,190.71451857)
}
}
{
\newrgbcolor{curcolor}{0 0 0}
\pscustom[linestyle=none,fillstyle=solid,fillcolor=curcolor]
{
\newpath
\moveto(272.49993576,190.71451857)
\lineto(276.49993553,194.71451809)
\lineto(262.49993636,190.71451857)
\lineto(276.49993553,186.71451904)
\lineto(272.49993576,190.71451857)
\closepath
}
}
{
\newrgbcolor{curcolor}{0 0 0}
\pscustom[linewidth=0.99999991,linecolor=curcolor]
{
\newpath
\moveto(272.49993576,190.71451857)
\lineto(276.49993553,194.71451809)
\lineto(262.49993636,190.71451857)
\lineto(276.49993553,186.71451904)
\lineto(272.49993576,190.71451857)
\closepath
}
}
{
\newrgbcolor{curcolor}{0 0 0}
\pscustom[linewidth=0.99999991,linecolor=curcolor]
{
\newpath
\moveto(181.88271369,162.7145219)
\lineto(181.44578196,141.47769842)
}
}
{
\newrgbcolor{curcolor}{0 0 0}
\pscustom[linestyle=none,fillstyle=solid,fillcolor=curcolor]
{
\newpath
\moveto(181.65148091,151.4755814)
\lineto(177.73460706,155.55701417)
\lineto(181.44578196,141.47769842)
\lineto(185.73291392,155.39245502)
\lineto(181.65148091,151.4755814)
\closepath
}
}
{
\newrgbcolor{curcolor}{0 0 0}
\pscustom[linewidth=0.99999991,linecolor=curcolor]
{
\newpath
\moveto(181.65148091,151.4755814)
\lineto(177.73460706,155.55701417)
\lineto(181.44578196,141.47769842)
\lineto(185.73291392,155.39245502)
\lineto(181.65148091,151.4755814)
\closepath
}
}
{
\newrgbcolor{curcolor}{0 0 0}
\pscustom[linewidth=0.99999991,linecolor=curcolor]
{
\newpath
\moveto(180.78740948,81.47770936)
\lineto(180.78740948,60.9999918)
}
}
{
\newrgbcolor{curcolor}{0 0 0}
\pscustom[linestyle=none,fillstyle=solid,fillcolor=curcolor]
{
\newpath
\moveto(180.78740948,70.99999061)
\lineto(176.78740972,74.99999013)
\lineto(180.78740948,60.9999918)
\lineto(184.78740924,74.99999013)
\lineto(180.78740948,70.99999061)
\closepath
}
}
{
\newrgbcolor{curcolor}{0 0 0}
\pscustom[linewidth=0.99999991,linecolor=curcolor]
{
\newpath
\moveto(180.78740948,70.99999061)
\lineto(176.78740972,74.99999013)
\lineto(180.78740948,60.9999918)
\lineto(184.78740924,74.99999013)
\lineto(180.78740948,70.99999061)
\closepath
}
}
\end{pspicture}

    \caption{The 3D Printing Pipeline}
  \end{figure}
\end{frame}

\section{3D Models}
\begin{frame}
  \frametitle{3D Models}
  To print a 3D object, you first need to get a 3D model.  There are a couple of ways to do it.
  \begin{itemize}
    \item Use an existing model
    \item Design it yourself (or get someone do design it for you)
  \end{itemize}
\end{frame}

\begin{frame}
  \frametitle{Existing Models}
  \begin{itemize}
    \item There are a number of sites on the web where you can download existing 3D models for free or for money.  A couple of the more prominent free sites are Thingiverse (https://www.thingiverse.com) and MyMiniFactory (https://www.myminifactory.com).  If you have something specific in mind, try using a search engine to find a model.

    \item  On the other hand, you may have already created the model, in which case, just follow the design it yourself part of the process.
  \end{itemize}
\end{frame}

\begin{frame}
  \frametitle{Design it Yourself}
  \begin{itemize}
    \item If you can't find the model you want, are good at 3D modeling, or want to learn 3D modeling, you can try designing it yourself.
  
    \item There is a vast number of free and pay CAD packages.  Try to find one that suits your needs.  I use a program called OpenSCAD (http://www.openscad.org), some people use Blender (https://www.blender.org), others, any of a variety of other programs.
  
    \item If you don't already have a favorite, download several of the free ones and try them to see which one you like.  The right one for you will depend on how your mind works and what sort of things you want to design.  The right one for you is not necessarily the right one for me.
  \end{itemize}
\end{frame}

\begin{frame}
  \frametitle{Getting Ready for the Next Step}
  \begin{itemize}
    \item Since each CAD program has its own file format and stores information that it not needed for printing, your 3D model needs to be converted to a standard format.
    \item The thing about standards is that there are so many to choose from.
    \item The most common one is  ``.stl''.  Many, if not most, downloadable models use it.
    \item Just make sure that the next step will accept your chosen format.
  \end{itemize}
\end{frame}

\section{Preparation for Printing}
\begin{frame}
  \frametitle{Preparation for Printing}
  \begin{itemize}
    \item Once you have your model in a standard format, it needs to be converted to a form for your printer and print job.
    \item This applies information that is specific to your printer and the material that you are printing with.
    \item You can also select print specific parameters like resolution or supports for overhangs.
    \item The program that does this is called a slicer.
  \end{itemize}
\end{frame}

\begin{frame}
  \frametitle{Slicing}
  \begin{itemize}
    \item The slicer program converts the 3D model into a series of instructions to the printer.
    \item There are a number of slicers available--check with your printer manufacturer to see what they recommend.
    \item Your printer manufacturer may provide their own slicer or settings that can be used by another common slicer.
    \item The output of the slicer is specific to the printer and the material being used.
  \end{itemize}
\end{frame}

\section{Printing}
\begin{frame}
  \frametitle{Printing}
  \begin{itemize}
    \item Load the sliced file into the printer and press ``Start''.
    \item The printer manufacturer should provide instructions for using their printer.
  \end{itemize}
\end{frame}

\section{Post Processing}
\begin{frame}
  \frametitle{Post Processing}
  Depending on the printer and the use for the object, some post processing may be required after printing.  This can include things like machining, sanding, chemical washes, and removal of supporting material.  These are done to improve the surface finish, attain proper dimensions, cure material to improve durability, or remove extraneous material.
\end{frame}

\section{Final Thoughts}
\begin{frame}
  \frametitle{Final Thoughts}
  \begin{itemize}
    \item This technology is still in its infancy so things are changing quickly.
    \begin{itemize}
      \item There are apparently 3D printers that can print a house out of concrete.
    \end{itemize}
    \item There are probably exceptions to everything said in this presentation.
  \end{itemize}
\end{frame}

\section{Glossary}
\begin{frame}
  \frametitle{Glossary}
  \begin{description}
    \item [g-code] A common instruction set for numerically controlled machine tools.  Also commonly used by 3D printers.
    \item [FDM] Fused Deposition Modeling--a popular 3D printing technology
    \item [slicer] A program that reads a standard 3D model and produces specific instructions (typically g-code) for your printer.
    \item [.stl] A standard file format for 3D models.  Originally it stood for ``stereolithography'',  but since then ``standard triangle list'' and ``standard tessellation libary'' have also been advanced as the meaning, but for our purposes, it doesn't really matter.
  \end{description}
\end{frame}

\end{document}
