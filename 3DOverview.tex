\documentclass[english,10pt]{beamer}

\title{3D Printing Overview}
\author{Brent Seidel\\modestconsulting@gmail.com\\Modest Consulting}
\institute{Modest Consulting}
\date{\today}


\subtitle{3D Overview}
\begin{document}
\begin{frame}
  \titlepage
\end{frame}

\begin{frame}
  \frametitle{Outline}
  \tableofcontents
\end{frame}

\section{Introduction}
\begin{frame}
  \frametitle{Introduction}
  While this presentation primarily focusses on consumer grade FDM printers, much of it should apply to other printers.  There are three basic stages in the 3D printing pipeline.
  \begin{itemize}
    \item 3D Modeling
    \item Preparing for Printing
    \item Printing
  \end{itemize}
\end{frame}

\section{3D Models}
\begin{frame}
  \frametitle{3D Models}
  To print a 3D object, you first need to get a 3D models.  There are a couple of ways to do it.
  \begin{itemize}
    \item Use an existing model
    \item Design it yourself (or get someone do design it for you)
  \end{itemize}
\end{frame}

\subsection{Existing Models}
\begin{frame}
  \frametitle{Existing Models}
  \begin{itemize}
    \item There are a number of sites on the web where you can download existing 3D models for free or for money.  A couple of the more prominent free sites are Thingiverse (https://www.thingiverse.com) and MyMiniFactory (https://www.myminifactory.com).  If you have something specific in mind, try using a search engine to find a model.

    \item  On the other hand, you may have already created the model, in which case, just follow the design it yourself part of the process.
  \end{itemize}
\end{frame}

\subsection{Design it Yourself}
\begin{frame}
  \frametitle{Design it Yourself}
  \begin{itemize}
    \item If you can't find the model you want, are good at 3D modeling, or want to learn 3D modeling, you can try designing it yourself.
  
    \item There is a vast number of free and pay CAD packages.  Try to find one that suits your needs.  I use a program called OpenSCAD (http://www.openscad.org), some people use Blender (https://www.blender.org), others, any of a variety of other programs.
  
    \item If you don't already have a favorite, download several of the free ones and try them to see which one you like.  The right one for you will depend on how your mind works and what sort of things you want to design.  The right one for you is not necessarily the right one for me.
  \end{itemize}
\end{frame}

\subsection{Getting Ready for the Next Step}
\begin{frame}
  \frametitle{Getting Ready for the Next Step}
  \begin{itemize}
    \item Since each CAD program has its own file format and stores information that it not needed for printing, your 3D model needs to be converted to a standard format.
    \item The thing about standards is that there are so many to choose from.
    \item The one that I use is .stl.  It seems to be fairly common and many downloadable models use it.
    \item Just make sure that the next step will accept your chosen format.
  \end{itemize}
\end{frame}

\section{Preparation for Printing}
\begin{frame}
  \frametitle{Preparation for Printing}
  \begin{itemize}
    \item Once you have your model in a standard format, it needs to be converted to a form for your printer and print job.
    \item This applies information that is specific to your printer and the material that you are printing with.
    \item You can also select print specific parameters like resolution or supports for overhangs.
    \item The program that does this is called a slicer.
  \end{itemize}
\end{frame}

\subsection{Slicing}
\begin{frame}
  \frametitle{Slicing}
  \begin{itemize}
    \item The slicer program converts the 3D model into a series of instructions to the printer.
    \item There are a number of slicers available--check with your printer manufacturer to see what they recommend.
    \item Your printer manufacturer may provide their own slicer or settings that can be used by another common slicer.
    \item The output of the slicer is specific to the printer and the material being used.
  \end{itemize}
\end{frame}

\section{Printing}
\begin{frame}
  \frametitle{Printing}
  \begin{itemize}
    \item Load the sliced file into the printer and press ``Start''.
    \item The printer manufacturer should provide instructions for using their printer.
  \end{itemize}
\end{frame}

\section{Final Thoughts}
\begin{frame}
  \frametitle{Final Thoughts}
  \begin{itemize}
    \item This technology is still in its infancy so things are changing quickly.
    \item There are probably exceptions to everything I said in this presentation.
  \end{itemize}
\end{frame}

\end{document}