\documentclass[english,10pt]{beamer}

\title{Design for 3D Printing}
\author{Brent Seidel\\modestconsulting@gmail.com\\Modest Consulting}
\institute{Modest Consulting}
\date{\today}


\subtitle{Design for 3D Printing}
\begin{document}
\begin{frame}
  \titlepage
\end{frame}

\begin{frame}
  \frametitle{Outline}
  \tableofcontents
\end{frame}

\section{Introduction}
\begin{frame}
  \frametitle{Introduction}
  A number of items need to be considered when designing an object for 3D printing.  Some of them are generic while others may depend on the 3D printing technology used.
\end{frame}

\section{Measurements}
\begin{frame}
  \frametitle{Measurements (units)}
  A good question is, ``What does one unit in the CAD program translate to when printed out?''.  Depending on your application, the answer may range from mildly interesting to vitally important.  If you are creating a stand alone decorative object, you may not care much as long as the result is a reasonable size.  If, however, you are designing a part to physically interface with some existing items, the answer is vitally important.
\end{frame}

\begin{frame}
  \frametitle{Measurements (accuracy)}
  There are three different items that impact the detail and accuracy of the print.
  \begin{itemize}
    \item What is the resolution for printing bits of material? (note that x and y resolution may be different)
    \item What is the smallest bit of material that can be printed?
    \item What is the layer height?
  \end{itemize}
  Some or all of these may be adjustable on your printer.  In general, reducing the resolution makes the print go faster.
\end{frame}

\begin{frame}
  \frametitle{Measurements (Resolution)}
  \begin{figure}
    %LaTeX with PSTricks extensions
%%Creator: inkscape 0.91
%%Please note this file requires PSTricks extensions
\psset{xunit=.5pt,yunit=.5pt,runit=.5pt}
\begin{pspicture}(435.04902397,317.45960206)
{
\newrgbcolor{curcolor}{0 0 0}
\pscustom[linestyle=none,fillstyle=solid,fillcolor=curcolor]
{
\newpath
\moveto(20.5,253.99787418)
\curveto(20.5,248.47502668)(16.0228475,243.99787418)(10.5,243.99787418)
\curveto(4.9771525,243.99787418)(0.5,248.47502668)(0.5,253.99787418)
\curveto(0.5,259.52072168)(4.9771525,263.99787418)(10.5,263.99787418)
\curveto(16.0228475,263.99787418)(20.5,259.52072168)(20.5,253.99787418)
\closepath
}
}
{
\newrgbcolor{curcolor}{0 0 0}
\pscustom[linewidth=1,linecolor=curcolor]
{
\newpath
\moveto(20.5,253.99787418)
\curveto(20.5,248.47502668)(16.0228475,243.99787418)(10.5,243.99787418)
\curveto(4.9771525,243.99787418)(0.5,248.47502668)(0.5,253.99787418)
\curveto(0.5,259.52072168)(4.9771525,263.99787418)(10.5,263.99787418)
\curveto(16.0228475,263.99787418)(20.5,259.52072168)(20.5,253.99787418)
\closepath
}
}
{
\newrgbcolor{curcolor}{0 0 0}
\pscustom[linestyle=none,fillstyle=solid,fillcolor=curcolor]
{
\newpath
\moveto(61.60121155,253.99787418)
\curveto(61.60121155,248.47502668)(57.12405904,243.99787418)(51.60121155,243.99787418)
\curveto(46.07836405,243.99787418)(41.60121155,248.47502668)(41.60121155,253.99787418)
\curveto(41.60121155,259.52072168)(46.07836405,263.99787418)(51.60121155,263.99787418)
\curveto(57.12405904,263.99787418)(61.60121155,259.52072168)(61.60121155,253.99787418)
\closepath
}
}
{
\newrgbcolor{curcolor}{0 0 0}
\pscustom[linewidth=1,linecolor=curcolor]
{
\newpath
\moveto(61.60121155,253.99787418)
\curveto(61.60121155,248.47502668)(57.12405904,243.99787418)(51.60121155,243.99787418)
\curveto(46.07836405,243.99787418)(41.60121155,248.47502668)(41.60121155,253.99787418)
\curveto(41.60121155,259.52072168)(46.07836405,263.99787418)(51.60121155,263.99787418)
\curveto(57.12405904,263.99787418)(61.60121155,259.52072168)(61.60121155,253.99787418)
\closepath
}
}
{
\newrgbcolor{curcolor}{0 0 0}
\pscustom[linestyle=none,fillstyle=solid,fillcolor=curcolor]
{
\newpath
\moveto(100.98986816,253.14159671)
\curveto(100.98986816,247.61874922)(96.51271566,243.14159671)(90.98986816,243.14159671)
\curveto(85.46702067,243.14159671)(80.98986816,247.61874922)(80.98986816,253.14159671)
\curveto(80.98986816,258.66444421)(85.46702067,263.14159671)(90.98986816,263.14159671)
\curveto(96.51271566,263.14159671)(100.98986816,258.66444421)(100.98986816,253.14159671)
\closepath
}
}
{
\newrgbcolor{curcolor}{0 0 0}
\pscustom[linewidth=1,linecolor=curcolor]
{
\newpath
\moveto(100.98986816,253.14159671)
\curveto(100.98986816,247.61874922)(96.51271566,243.14159671)(90.98986816,243.14159671)
\curveto(85.46702067,243.14159671)(80.98986816,247.61874922)(80.98986816,253.14159671)
\curveto(80.98986816,258.66444421)(85.46702067,263.14159671)(90.98986816,263.14159671)
\curveto(96.51271566,263.14159671)(100.98986816,258.66444421)(100.98986816,253.14159671)
\closepath
}
}
{
\newrgbcolor{curcolor}{0 0 0}
\pscustom[linestyle=none,fillstyle=solid,fillcolor=curcolor]
{
\newpath
\moveto(142.09107971,253.99787418)
\curveto(142.09107971,248.47502668)(137.61392721,243.99787418)(132.09107971,243.99787418)
\curveto(126.56823222,243.99787418)(122.09107971,248.47502668)(122.09107971,253.99787418)
\curveto(122.09107971,259.52072168)(126.56823222,263.99787418)(132.09107971,263.99787418)
\curveto(137.61392721,263.99787418)(142.09107971,259.52072168)(142.09107971,253.99787418)
\closepath
}
}
{
\newrgbcolor{curcolor}{0 0 0}
\pscustom[linewidth=1,linecolor=curcolor]
{
\newpath
\moveto(142.09107971,253.99787418)
\curveto(142.09107971,248.47502668)(137.61392721,243.99787418)(132.09107971,243.99787418)
\curveto(126.56823222,243.99787418)(122.09107971,248.47502668)(122.09107971,253.99787418)
\curveto(122.09107971,259.52072168)(126.56823222,263.99787418)(132.09107971,263.99787418)
\curveto(137.61392721,263.99787418)(142.09107971,259.52072168)(142.09107971,253.99787418)
\closepath
}
}
{
\newrgbcolor{curcolor}{0 0 0}
\pscustom[linestyle=none,fillstyle=solid,fillcolor=curcolor]
{
\newpath
\moveto(181.47973633,253.99787418)
\curveto(181.47973633,248.47502668)(177.00258382,243.99787418)(171.47973633,243.99787418)
\curveto(165.95688883,243.99787418)(161.47973633,248.47502668)(161.47973633,253.99787418)
\curveto(161.47973633,259.52072168)(165.95688883,263.99787418)(171.47973633,263.99787418)
\curveto(177.00258382,263.99787418)(181.47973633,259.52072168)(181.47973633,253.99787418)
\closepath
}
}
{
\newrgbcolor{curcolor}{0 0 0}
\pscustom[linewidth=1,linecolor=curcolor]
{
\newpath
\moveto(181.47973633,253.99787418)
\curveto(181.47973633,248.47502668)(177.00258382,243.99787418)(171.47973633,243.99787418)
\curveto(165.95688883,243.99787418)(161.47973633,248.47502668)(161.47973633,253.99787418)
\curveto(161.47973633,259.52072168)(165.95688883,263.99787418)(171.47973633,263.99787418)
\curveto(177.00258382,263.99787418)(181.47973633,259.52072168)(181.47973633,253.99787418)
\closepath
}
}
{
\newrgbcolor{curcolor}{0 0 0}
\pscustom[linestyle=none,fillstyle=solid,fillcolor=curcolor]
{
\newpath
\moveto(221.47973633,253.99787418)
\curveto(221.47973633,248.47502668)(217.00258382,243.99787418)(211.47973633,243.99787418)
\curveto(205.95688883,243.99787418)(201.47973633,248.47502668)(201.47973633,253.99787418)
\curveto(201.47973633,259.52072168)(205.95688883,263.99787418)(211.47973633,263.99787418)
\curveto(217.00258382,263.99787418)(221.47973633,259.52072168)(221.47973633,253.99787418)
\closepath
}
}
{
\newrgbcolor{curcolor}{0 0 0}
\pscustom[linewidth=1,linecolor=curcolor]
{
\newpath
\moveto(221.47973633,253.99787418)
\curveto(221.47973633,248.47502668)(217.00258382,243.99787418)(211.47973633,243.99787418)
\curveto(205.95688883,243.99787418)(201.47973633,248.47502668)(201.47973633,253.99787418)
\curveto(201.47973633,259.52072168)(205.95688883,263.99787418)(211.47973633,263.99787418)
\curveto(217.00258382,263.99787418)(221.47973633,259.52072168)(221.47973633,253.99787418)
\closepath
}
}
{
\newrgbcolor{curcolor}{0 0 0}
\pscustom[linestyle=none,fillstyle=solid,fillcolor=curcolor]
{
\newpath
\moveto(20.5,213.99787418)
\curveto(20.5,208.47502668)(16.0228475,203.99787418)(10.5,203.99787418)
\curveto(4.9771525,203.99787418)(0.5,208.47502668)(0.5,213.99787418)
\curveto(0.5,219.52072168)(4.9771525,223.99787418)(10.5,223.99787418)
\curveto(16.0228475,223.99787418)(20.5,219.52072168)(20.5,213.99787418)
\closepath
}
}
{
\newrgbcolor{curcolor}{0 0 0}
\pscustom[linewidth=1,linecolor=curcolor]
{
\newpath
\moveto(20.5,213.99787418)
\curveto(20.5,208.47502668)(16.0228475,203.99787418)(10.5,203.99787418)
\curveto(4.9771525,203.99787418)(0.5,208.47502668)(0.5,213.99787418)
\curveto(0.5,219.52072168)(4.9771525,223.99787418)(10.5,223.99787418)
\curveto(16.0228475,223.99787418)(20.5,219.52072168)(20.5,213.99787418)
\closepath
}
}
{
\newrgbcolor{curcolor}{0 0 0}
\pscustom[linestyle=none,fillstyle=solid,fillcolor=curcolor]
{
\newpath
\moveto(61.60121155,213.99787418)
\curveto(61.60121155,208.47502668)(57.12405904,203.99787418)(51.60121155,203.99787418)
\curveto(46.07836405,203.99787418)(41.60121155,208.47502668)(41.60121155,213.99787418)
\curveto(41.60121155,219.52072168)(46.07836405,223.99787418)(51.60121155,223.99787418)
\curveto(57.12405904,223.99787418)(61.60121155,219.52072168)(61.60121155,213.99787418)
\closepath
}
}
{
\newrgbcolor{curcolor}{0 0 0}
\pscustom[linewidth=1,linecolor=curcolor]
{
\newpath
\moveto(61.60121155,213.99787418)
\curveto(61.60121155,208.47502668)(57.12405904,203.99787418)(51.60121155,203.99787418)
\curveto(46.07836405,203.99787418)(41.60121155,208.47502668)(41.60121155,213.99787418)
\curveto(41.60121155,219.52072168)(46.07836405,223.99787418)(51.60121155,223.99787418)
\curveto(57.12405904,223.99787418)(61.60121155,219.52072168)(61.60121155,213.99787418)
\closepath
}
}
{
\newrgbcolor{curcolor}{0 0 0}
\pscustom[linestyle=none,fillstyle=solid,fillcolor=curcolor]
{
\newpath
\moveto(100.98986816,213.14159671)
\curveto(100.98986816,207.61874922)(96.51271566,203.14159671)(90.98986816,203.14159671)
\curveto(85.46702067,203.14159671)(80.98986816,207.61874922)(80.98986816,213.14159671)
\curveto(80.98986816,218.66444421)(85.46702067,223.14159671)(90.98986816,223.14159671)
\curveto(96.51271566,223.14159671)(100.98986816,218.66444421)(100.98986816,213.14159671)
\closepath
}
}
{
\newrgbcolor{curcolor}{0 0 0}
\pscustom[linewidth=1,linecolor=curcolor]
{
\newpath
\moveto(100.98986816,213.14159671)
\curveto(100.98986816,207.61874922)(96.51271566,203.14159671)(90.98986816,203.14159671)
\curveto(85.46702067,203.14159671)(80.98986816,207.61874922)(80.98986816,213.14159671)
\curveto(80.98986816,218.66444421)(85.46702067,223.14159671)(90.98986816,223.14159671)
\curveto(96.51271566,223.14159671)(100.98986816,218.66444421)(100.98986816,213.14159671)
\closepath
}
}
{
\newrgbcolor{curcolor}{0 0 0}
\pscustom[linestyle=none,fillstyle=solid,fillcolor=curcolor]
{
\newpath
\moveto(142.09107971,213.99787418)
\curveto(142.09107971,208.47502668)(137.61392721,203.99787418)(132.09107971,203.99787418)
\curveto(126.56823222,203.99787418)(122.09107971,208.47502668)(122.09107971,213.99787418)
\curveto(122.09107971,219.52072168)(126.56823222,223.99787418)(132.09107971,223.99787418)
\curveto(137.61392721,223.99787418)(142.09107971,219.52072168)(142.09107971,213.99787418)
\closepath
}
}
{
\newrgbcolor{curcolor}{0 0 0}
\pscustom[linewidth=1,linecolor=curcolor]
{
\newpath
\moveto(142.09107971,213.99787418)
\curveto(142.09107971,208.47502668)(137.61392721,203.99787418)(132.09107971,203.99787418)
\curveto(126.56823222,203.99787418)(122.09107971,208.47502668)(122.09107971,213.99787418)
\curveto(122.09107971,219.52072168)(126.56823222,223.99787418)(132.09107971,223.99787418)
\curveto(137.61392721,223.99787418)(142.09107971,219.52072168)(142.09107971,213.99787418)
\closepath
}
}
{
\newrgbcolor{curcolor}{0 0 0}
\pscustom[linestyle=none,fillstyle=solid,fillcolor=curcolor]
{
\newpath
\moveto(181.47973633,213.99787418)
\curveto(181.47973633,208.47502668)(177.00258382,203.99787418)(171.47973633,203.99787418)
\curveto(165.95688883,203.99787418)(161.47973633,208.47502668)(161.47973633,213.99787418)
\curveto(161.47973633,219.52072168)(165.95688883,223.99787418)(171.47973633,223.99787418)
\curveto(177.00258382,223.99787418)(181.47973633,219.52072168)(181.47973633,213.99787418)
\closepath
}
}
{
\newrgbcolor{curcolor}{0 0 0}
\pscustom[linewidth=1,linecolor=curcolor]
{
\newpath
\moveto(181.47973633,213.99787418)
\curveto(181.47973633,208.47502668)(177.00258382,203.99787418)(171.47973633,203.99787418)
\curveto(165.95688883,203.99787418)(161.47973633,208.47502668)(161.47973633,213.99787418)
\curveto(161.47973633,219.52072168)(165.95688883,223.99787418)(171.47973633,223.99787418)
\curveto(177.00258382,223.99787418)(181.47973633,219.52072168)(181.47973633,213.99787418)
\closepath
}
}
{
\newrgbcolor{curcolor}{0 0 0}
\pscustom[linestyle=none,fillstyle=solid,fillcolor=curcolor]
{
\newpath
\moveto(221.47973633,213.99787418)
\curveto(221.47973633,208.47502668)(217.00258382,203.99787418)(211.47973633,203.99787418)
\curveto(205.95688883,203.99787418)(201.47973633,208.47502668)(201.47973633,213.99787418)
\curveto(201.47973633,219.52072168)(205.95688883,223.99787418)(211.47973633,223.99787418)
\curveto(217.00258382,223.99787418)(221.47973633,219.52072168)(221.47973633,213.99787418)
\closepath
}
}
{
\newrgbcolor{curcolor}{0 0 0}
\pscustom[linewidth=1,linecolor=curcolor]
{
\newpath
\moveto(221.47973633,213.99787418)
\curveto(221.47973633,208.47502668)(217.00258382,203.99787418)(211.47973633,203.99787418)
\curveto(205.95688883,203.99787418)(201.47973633,208.47502668)(201.47973633,213.99787418)
\curveto(201.47973633,219.52072168)(205.95688883,223.99787418)(211.47973633,223.99787418)
\curveto(217.00258382,223.99787418)(221.47973633,219.52072168)(221.47973633,213.99787418)
\closepath
}
}
{
\newrgbcolor{curcolor}{0 0 0}
\pscustom[linestyle=none,fillstyle=solid,fillcolor=curcolor]
{
\newpath
\moveto(20.5,173.99788944)
\curveto(20.5,168.47504194)(16.0228475,163.99788944)(10.5,163.99788944)
\curveto(4.9771525,163.99788944)(0.5,168.47504194)(0.5,173.99788944)
\curveto(0.5,179.52073693)(4.9771525,183.99788944)(10.5,183.99788944)
\curveto(16.0228475,183.99788944)(20.5,179.52073693)(20.5,173.99788944)
\closepath
}
}
{
\newrgbcolor{curcolor}{0 0 0}
\pscustom[linewidth=1,linecolor=curcolor]
{
\newpath
\moveto(20.5,173.99788944)
\curveto(20.5,168.47504194)(16.0228475,163.99788944)(10.5,163.99788944)
\curveto(4.9771525,163.99788944)(0.5,168.47504194)(0.5,173.99788944)
\curveto(0.5,179.52073693)(4.9771525,183.99788944)(10.5,183.99788944)
\curveto(16.0228475,183.99788944)(20.5,179.52073693)(20.5,173.99788944)
\closepath
}
}
{
\newrgbcolor{curcolor}{0 0 0}
\pscustom[linestyle=none,fillstyle=solid,fillcolor=curcolor]
{
\newpath
\moveto(61.60121155,173.99788944)
\curveto(61.60121155,168.47504194)(57.12405904,163.99788944)(51.60121155,163.99788944)
\curveto(46.07836405,163.99788944)(41.60121155,168.47504194)(41.60121155,173.99788944)
\curveto(41.60121155,179.52073693)(46.07836405,183.99788944)(51.60121155,183.99788944)
\curveto(57.12405904,183.99788944)(61.60121155,179.52073693)(61.60121155,173.99788944)
\closepath
}
}
{
\newrgbcolor{curcolor}{0 0 0}
\pscustom[linewidth=1,linecolor=curcolor]
{
\newpath
\moveto(61.60121155,173.99788944)
\curveto(61.60121155,168.47504194)(57.12405904,163.99788944)(51.60121155,163.99788944)
\curveto(46.07836405,163.99788944)(41.60121155,168.47504194)(41.60121155,173.99788944)
\curveto(41.60121155,179.52073693)(46.07836405,183.99788944)(51.60121155,183.99788944)
\curveto(57.12405904,183.99788944)(61.60121155,179.52073693)(61.60121155,173.99788944)
\closepath
}
}
{
\newrgbcolor{curcolor}{0 0 0}
\pscustom[linestyle=none,fillstyle=solid,fillcolor=curcolor]
{
\newpath
\moveto(100.98986816,173.14159671)
\curveto(100.98986816,167.61874922)(96.51271566,163.14159671)(90.98986816,163.14159671)
\curveto(85.46702067,163.14159671)(80.98986816,167.61874922)(80.98986816,173.14159671)
\curveto(80.98986816,178.66444421)(85.46702067,183.14159671)(90.98986816,183.14159671)
\curveto(96.51271566,183.14159671)(100.98986816,178.66444421)(100.98986816,173.14159671)
\closepath
}
}
{
\newrgbcolor{curcolor}{0 0 0}
\pscustom[linewidth=1,linecolor=curcolor]
{
\newpath
\moveto(100.98986816,173.14159671)
\curveto(100.98986816,167.61874922)(96.51271566,163.14159671)(90.98986816,163.14159671)
\curveto(85.46702067,163.14159671)(80.98986816,167.61874922)(80.98986816,173.14159671)
\curveto(80.98986816,178.66444421)(85.46702067,183.14159671)(90.98986816,183.14159671)
\curveto(96.51271566,183.14159671)(100.98986816,178.66444421)(100.98986816,173.14159671)
\closepath
}
}
{
\newrgbcolor{curcolor}{0 0 0}
\pscustom[linestyle=none,fillstyle=solid,fillcolor=curcolor]
{
\newpath
\moveto(142.09107971,173.99788944)
\curveto(142.09107971,168.47504194)(137.61392721,163.99788944)(132.09107971,163.99788944)
\curveto(126.56823222,163.99788944)(122.09107971,168.47504194)(122.09107971,173.99788944)
\curveto(122.09107971,179.52073693)(126.56823222,183.99788944)(132.09107971,183.99788944)
\curveto(137.61392721,183.99788944)(142.09107971,179.52073693)(142.09107971,173.99788944)
\closepath
}
}
{
\newrgbcolor{curcolor}{0 0 0}
\pscustom[linewidth=1,linecolor=curcolor]
{
\newpath
\moveto(142.09107971,173.99788944)
\curveto(142.09107971,168.47504194)(137.61392721,163.99788944)(132.09107971,163.99788944)
\curveto(126.56823222,163.99788944)(122.09107971,168.47504194)(122.09107971,173.99788944)
\curveto(122.09107971,179.52073693)(126.56823222,183.99788944)(132.09107971,183.99788944)
\curveto(137.61392721,183.99788944)(142.09107971,179.52073693)(142.09107971,173.99788944)
\closepath
}
}
{
\newrgbcolor{curcolor}{0 0 0}
\pscustom[linestyle=none,fillstyle=solid,fillcolor=curcolor]
{
\newpath
\moveto(181.47973633,173.99788944)
\curveto(181.47973633,168.47504194)(177.00258382,163.99788944)(171.47973633,163.99788944)
\curveto(165.95688883,163.99788944)(161.47973633,168.47504194)(161.47973633,173.99788944)
\curveto(161.47973633,179.52073693)(165.95688883,183.99788944)(171.47973633,183.99788944)
\curveto(177.00258382,183.99788944)(181.47973633,179.52073693)(181.47973633,173.99788944)
\closepath
}
}
{
\newrgbcolor{curcolor}{0 0 0}
\pscustom[linewidth=1,linecolor=curcolor]
{
\newpath
\moveto(181.47973633,173.99788944)
\curveto(181.47973633,168.47504194)(177.00258382,163.99788944)(171.47973633,163.99788944)
\curveto(165.95688883,163.99788944)(161.47973633,168.47504194)(161.47973633,173.99788944)
\curveto(161.47973633,179.52073693)(165.95688883,183.99788944)(171.47973633,183.99788944)
\curveto(177.00258382,183.99788944)(181.47973633,179.52073693)(181.47973633,173.99788944)
\closepath
}
}
{
\newrgbcolor{curcolor}{0 0 0}
\pscustom[linestyle=none,fillstyle=solid,fillcolor=curcolor]
{
\newpath
\moveto(221.47973633,173.99788944)
\curveto(221.47973633,168.47504194)(217.00258382,163.99788944)(211.47973633,163.99788944)
\curveto(205.95688883,163.99788944)(201.47973633,168.47504194)(201.47973633,173.99788944)
\curveto(201.47973633,179.52073693)(205.95688883,183.99788944)(211.47973633,183.99788944)
\curveto(217.00258382,183.99788944)(221.47973633,179.52073693)(221.47973633,173.99788944)
\closepath
}
}
{
\newrgbcolor{curcolor}{0 0 0}
\pscustom[linewidth=1,linecolor=curcolor]
{
\newpath
\moveto(221.47973633,173.99788944)
\curveto(221.47973633,168.47504194)(217.00258382,163.99788944)(211.47973633,163.99788944)
\curveto(205.95688883,163.99788944)(201.47973633,168.47504194)(201.47973633,173.99788944)
\curveto(201.47973633,179.52073693)(205.95688883,183.99788944)(211.47973633,183.99788944)
\curveto(217.00258382,183.99788944)(221.47973633,179.52073693)(221.47973633,173.99788944)
\closepath
}
}
{
\newrgbcolor{curcolor}{0 0 0}
\pscustom[linestyle=none,fillstyle=solid,fillcolor=curcolor]
{
\newpath
\moveto(20.5,133.99788944)
\curveto(20.5,128.47504194)(16.0228475,123.99788944)(10.5,123.99788944)
\curveto(4.9771525,123.99788944)(0.5,128.47504194)(0.5,133.99788944)
\curveto(0.5,139.52073693)(4.9771525,143.99788944)(10.5,143.99788944)
\curveto(16.0228475,143.99788944)(20.5,139.52073693)(20.5,133.99788944)
\closepath
}
}
{
\newrgbcolor{curcolor}{0 0 0}
\pscustom[linewidth=1,linecolor=curcolor]
{
\newpath
\moveto(20.5,133.99788944)
\curveto(20.5,128.47504194)(16.0228475,123.99788944)(10.5,123.99788944)
\curveto(4.9771525,123.99788944)(0.5,128.47504194)(0.5,133.99788944)
\curveto(0.5,139.52073693)(4.9771525,143.99788944)(10.5,143.99788944)
\curveto(16.0228475,143.99788944)(20.5,139.52073693)(20.5,133.99788944)
\closepath
}
}
{
\newrgbcolor{curcolor}{0 0 0}
\pscustom[linestyle=none,fillstyle=solid,fillcolor=curcolor]
{
\newpath
\moveto(61.60121155,133.99788944)
\curveto(61.60121155,128.47504194)(57.12405904,123.99788944)(51.60121155,123.99788944)
\curveto(46.07836405,123.99788944)(41.60121155,128.47504194)(41.60121155,133.99788944)
\curveto(41.60121155,139.52073693)(46.07836405,143.99788944)(51.60121155,143.99788944)
\curveto(57.12405904,143.99788944)(61.60121155,139.52073693)(61.60121155,133.99788944)
\closepath
}
}
{
\newrgbcolor{curcolor}{0 0 0}
\pscustom[linewidth=1,linecolor=curcolor]
{
\newpath
\moveto(61.60121155,133.99788944)
\curveto(61.60121155,128.47504194)(57.12405904,123.99788944)(51.60121155,123.99788944)
\curveto(46.07836405,123.99788944)(41.60121155,128.47504194)(41.60121155,133.99788944)
\curveto(41.60121155,139.52073693)(46.07836405,143.99788944)(51.60121155,143.99788944)
\curveto(57.12405904,143.99788944)(61.60121155,139.52073693)(61.60121155,133.99788944)
\closepath
}
}
{
\newrgbcolor{curcolor}{0 0 0}
\pscustom[linestyle=none,fillstyle=solid,fillcolor=curcolor]
{
\newpath
\moveto(100.98986816,133.14159671)
\curveto(100.98986816,127.61874922)(96.51271566,123.14159671)(90.98986816,123.14159671)
\curveto(85.46702067,123.14159671)(80.98986816,127.61874922)(80.98986816,133.14159671)
\curveto(80.98986816,138.66444421)(85.46702067,143.14159671)(90.98986816,143.14159671)
\curveto(96.51271566,143.14159671)(100.98986816,138.66444421)(100.98986816,133.14159671)
\closepath
}
}
{
\newrgbcolor{curcolor}{0 0 0}
\pscustom[linewidth=1,linecolor=curcolor]
{
\newpath
\moveto(100.98986816,133.14159671)
\curveto(100.98986816,127.61874922)(96.51271566,123.14159671)(90.98986816,123.14159671)
\curveto(85.46702067,123.14159671)(80.98986816,127.61874922)(80.98986816,133.14159671)
\curveto(80.98986816,138.66444421)(85.46702067,143.14159671)(90.98986816,143.14159671)
\curveto(96.51271566,143.14159671)(100.98986816,138.66444421)(100.98986816,133.14159671)
\closepath
}
}
{
\newrgbcolor{curcolor}{0 0 0}
\pscustom[linestyle=none,fillstyle=solid,fillcolor=curcolor]
{
\newpath
\moveto(142.09107971,133.99788944)
\curveto(142.09107971,128.47504194)(137.61392721,123.99788944)(132.09107971,123.99788944)
\curveto(126.56823222,123.99788944)(122.09107971,128.47504194)(122.09107971,133.99788944)
\curveto(122.09107971,139.52073693)(126.56823222,143.99788944)(132.09107971,143.99788944)
\curveto(137.61392721,143.99788944)(142.09107971,139.52073693)(142.09107971,133.99788944)
\closepath
}
}
{
\newrgbcolor{curcolor}{0 0 0}
\pscustom[linewidth=1,linecolor=curcolor]
{
\newpath
\moveto(142.09107971,133.99788944)
\curveto(142.09107971,128.47504194)(137.61392721,123.99788944)(132.09107971,123.99788944)
\curveto(126.56823222,123.99788944)(122.09107971,128.47504194)(122.09107971,133.99788944)
\curveto(122.09107971,139.52073693)(126.56823222,143.99788944)(132.09107971,143.99788944)
\curveto(137.61392721,143.99788944)(142.09107971,139.52073693)(142.09107971,133.99788944)
\closepath
}
}
{
\newrgbcolor{curcolor}{0 0 0}
\pscustom[linestyle=none,fillstyle=solid,fillcolor=curcolor]
{
\newpath
\moveto(181.47973633,133.99788944)
\curveto(181.47973633,128.47504194)(177.00258382,123.99788944)(171.47973633,123.99788944)
\curveto(165.95688883,123.99788944)(161.47973633,128.47504194)(161.47973633,133.99788944)
\curveto(161.47973633,139.52073693)(165.95688883,143.99788944)(171.47973633,143.99788944)
\curveto(177.00258382,143.99788944)(181.47973633,139.52073693)(181.47973633,133.99788944)
\closepath
}
}
{
\newrgbcolor{curcolor}{0 0 0}
\pscustom[linewidth=1,linecolor=curcolor]
{
\newpath
\moveto(181.47973633,133.99788944)
\curveto(181.47973633,128.47504194)(177.00258382,123.99788944)(171.47973633,123.99788944)
\curveto(165.95688883,123.99788944)(161.47973633,128.47504194)(161.47973633,133.99788944)
\curveto(161.47973633,139.52073693)(165.95688883,143.99788944)(171.47973633,143.99788944)
\curveto(177.00258382,143.99788944)(181.47973633,139.52073693)(181.47973633,133.99788944)
\closepath
}
}
{
\newrgbcolor{curcolor}{0 0 0}
\pscustom[linestyle=none,fillstyle=solid,fillcolor=curcolor]
{
\newpath
\moveto(221.47973633,133.99788944)
\curveto(221.47973633,128.47504194)(217.00258382,123.99788944)(211.47973633,123.99788944)
\curveto(205.95688883,123.99788944)(201.47973633,128.47504194)(201.47973633,133.99788944)
\curveto(201.47973633,139.52073693)(205.95688883,143.99788944)(211.47973633,143.99788944)
\curveto(217.00258382,143.99788944)(221.47973633,139.52073693)(221.47973633,133.99788944)
\closepath
}
}
{
\newrgbcolor{curcolor}{0 0 0}
\pscustom[linewidth=1,linecolor=curcolor]
{
\newpath
\moveto(221.47973633,133.99788944)
\curveto(221.47973633,128.47504194)(217.00258382,123.99788944)(211.47973633,123.99788944)
\curveto(205.95688883,123.99788944)(201.47973633,128.47504194)(201.47973633,133.99788944)
\curveto(201.47973633,139.52073693)(205.95688883,143.99788944)(211.47973633,143.99788944)
\curveto(217.00258382,143.99788944)(221.47973633,139.52073693)(221.47973633,133.99788944)
\closepath
}
}
{
\newrgbcolor{curcolor}{0 0 0}
\pscustom[linestyle=none,fillstyle=solid,fillcolor=curcolor]
{
\newpath
\moveto(20.5,93.99788944)
\curveto(20.5,88.47504194)(16.0228475,83.99788944)(10.5,83.99788944)
\curveto(4.9771525,83.99788944)(0.5,88.47504194)(0.5,93.99788944)
\curveto(0.5,99.52073693)(4.9771525,103.99788944)(10.5,103.99788944)
\curveto(16.0228475,103.99788944)(20.5,99.52073693)(20.5,93.99788944)
\closepath
}
}
{
\newrgbcolor{curcolor}{0 0 0}
\pscustom[linewidth=1,linecolor=curcolor]
{
\newpath
\moveto(20.5,93.99788944)
\curveto(20.5,88.47504194)(16.0228475,83.99788944)(10.5,83.99788944)
\curveto(4.9771525,83.99788944)(0.5,88.47504194)(0.5,93.99788944)
\curveto(0.5,99.52073693)(4.9771525,103.99788944)(10.5,103.99788944)
\curveto(16.0228475,103.99788944)(20.5,99.52073693)(20.5,93.99788944)
\closepath
}
}
{
\newrgbcolor{curcolor}{0 0 0}
\pscustom[linestyle=none,fillstyle=solid,fillcolor=curcolor]
{
\newpath
\moveto(61.60121155,93.99788944)
\curveto(61.60121155,88.47504194)(57.12405904,83.99788944)(51.60121155,83.99788944)
\curveto(46.07836405,83.99788944)(41.60121155,88.47504194)(41.60121155,93.99788944)
\curveto(41.60121155,99.52073693)(46.07836405,103.99788944)(51.60121155,103.99788944)
\curveto(57.12405904,103.99788944)(61.60121155,99.52073693)(61.60121155,93.99788944)
\closepath
}
}
{
\newrgbcolor{curcolor}{0 0 0}
\pscustom[linewidth=1,linecolor=curcolor]
{
\newpath
\moveto(61.60121155,93.99788944)
\curveto(61.60121155,88.47504194)(57.12405904,83.99788944)(51.60121155,83.99788944)
\curveto(46.07836405,83.99788944)(41.60121155,88.47504194)(41.60121155,93.99788944)
\curveto(41.60121155,99.52073693)(46.07836405,103.99788944)(51.60121155,103.99788944)
\curveto(57.12405904,103.99788944)(61.60121155,99.52073693)(61.60121155,93.99788944)
\closepath
}
}
{
\newrgbcolor{curcolor}{0 0 0}
\pscustom[linestyle=none,fillstyle=solid,fillcolor=curcolor]
{
\newpath
\moveto(100.98986816,93.14159671)
\curveto(100.98986816,87.61874922)(96.51271566,83.14159671)(90.98986816,83.14159671)
\curveto(85.46702067,83.14159671)(80.98986816,87.61874922)(80.98986816,93.14159671)
\curveto(80.98986816,98.66444421)(85.46702067,103.14159671)(90.98986816,103.14159671)
\curveto(96.51271566,103.14159671)(100.98986816,98.66444421)(100.98986816,93.14159671)
\closepath
}
}
{
\newrgbcolor{curcolor}{0 0 0}
\pscustom[linewidth=1,linecolor=curcolor]
{
\newpath
\moveto(100.98986816,93.14159671)
\curveto(100.98986816,87.61874922)(96.51271566,83.14159671)(90.98986816,83.14159671)
\curveto(85.46702067,83.14159671)(80.98986816,87.61874922)(80.98986816,93.14159671)
\curveto(80.98986816,98.66444421)(85.46702067,103.14159671)(90.98986816,103.14159671)
\curveto(96.51271566,103.14159671)(100.98986816,98.66444421)(100.98986816,93.14159671)
\closepath
}
}
{
\newrgbcolor{curcolor}{0 0 0}
\pscustom[linestyle=none,fillstyle=solid,fillcolor=curcolor]
{
\newpath
\moveto(142.09107971,93.99788944)
\curveto(142.09107971,88.47504194)(137.61392721,83.99788944)(132.09107971,83.99788944)
\curveto(126.56823222,83.99788944)(122.09107971,88.47504194)(122.09107971,93.99788944)
\curveto(122.09107971,99.52073693)(126.56823222,103.99788944)(132.09107971,103.99788944)
\curveto(137.61392721,103.99788944)(142.09107971,99.52073693)(142.09107971,93.99788944)
\closepath
}
}
{
\newrgbcolor{curcolor}{0 0 0}
\pscustom[linewidth=1,linecolor=curcolor]
{
\newpath
\moveto(142.09107971,93.99788944)
\curveto(142.09107971,88.47504194)(137.61392721,83.99788944)(132.09107971,83.99788944)
\curveto(126.56823222,83.99788944)(122.09107971,88.47504194)(122.09107971,93.99788944)
\curveto(122.09107971,99.52073693)(126.56823222,103.99788944)(132.09107971,103.99788944)
\curveto(137.61392721,103.99788944)(142.09107971,99.52073693)(142.09107971,93.99788944)
\closepath
}
}
{
\newrgbcolor{curcolor}{0 0 0}
\pscustom[linestyle=none,fillstyle=solid,fillcolor=curcolor]
{
\newpath
\moveto(181.47973633,93.99788944)
\curveto(181.47973633,88.47504194)(177.00258382,83.99788944)(171.47973633,83.99788944)
\curveto(165.95688883,83.99788944)(161.47973633,88.47504194)(161.47973633,93.99788944)
\curveto(161.47973633,99.52073693)(165.95688883,103.99788944)(171.47973633,103.99788944)
\curveto(177.00258382,103.99788944)(181.47973633,99.52073693)(181.47973633,93.99788944)
\closepath
}
}
{
\newrgbcolor{curcolor}{0 0 0}
\pscustom[linewidth=1,linecolor=curcolor]
{
\newpath
\moveto(181.47973633,93.99788944)
\curveto(181.47973633,88.47504194)(177.00258382,83.99788944)(171.47973633,83.99788944)
\curveto(165.95688883,83.99788944)(161.47973633,88.47504194)(161.47973633,93.99788944)
\curveto(161.47973633,99.52073693)(165.95688883,103.99788944)(171.47973633,103.99788944)
\curveto(177.00258382,103.99788944)(181.47973633,99.52073693)(181.47973633,93.99788944)
\closepath
}
}
{
\newrgbcolor{curcolor}{0 0 0}
\pscustom[linestyle=none,fillstyle=solid,fillcolor=curcolor]
{
\newpath
\moveto(221.47973633,93.99788944)
\curveto(221.47973633,88.47504194)(217.00258382,83.99788944)(211.47973633,83.99788944)
\curveto(205.95688883,83.99788944)(201.47973633,88.47504194)(201.47973633,93.99788944)
\curveto(201.47973633,99.52073693)(205.95688883,103.99788944)(211.47973633,103.99788944)
\curveto(217.00258382,103.99788944)(221.47973633,99.52073693)(221.47973633,93.99788944)
\closepath
}
}
{
\newrgbcolor{curcolor}{0 0 0}
\pscustom[linewidth=1,linecolor=curcolor]
{
\newpath
\moveto(221.47973633,93.99788944)
\curveto(221.47973633,88.47504194)(217.00258382,83.99788944)(211.47973633,83.99788944)
\curveto(205.95688883,83.99788944)(201.47973633,88.47504194)(201.47973633,93.99788944)
\curveto(201.47973633,99.52073693)(205.95688883,103.99788944)(211.47973633,103.99788944)
\curveto(217.00258382,103.99788944)(221.47973633,99.52073693)(221.47973633,93.99788944)
\closepath
}
}
{
\newrgbcolor{curcolor}{0 0 0}
\pscustom[linestyle=none,fillstyle=solid,fillcolor=curcolor]
{
\newpath
\moveto(11.57163239,304.54847232)
\lineto(9.46225739,304.54847232)
\lineto(14.08530426,311.16663639)
\lineto(9.75229645,317.45960514)
\lineto(11.94077301,317.45960514)
\lineto(15.23667145,312.52015201)
\lineto(18.5062027,317.45960514)
\lineto(20.58921051,317.45960514)
\lineto(16.2562027,311.16663639)
\lineto(20.80014801,304.54847232)
\lineto(18.62924957,304.54847232)
\lineto(15.14878082,309.85706607)
\lineto(11.57163239,304.54847232)
\closepath
}
}
{
\newrgbcolor{curcolor}{0 0 0}
\pscustom[linestyle=none,fillstyle=solid,fillcolor=curcolor]
{
\newpath
\moveto(34.10678864,309.83948795)
\lineto(32.1468277,315.54358951)
\lineto(30.06381989,309.83948795)
\lineto(34.10678864,309.83948795)
\closepath
\moveto(31.2327652,317.45960514)
\lineto(33.21030426,317.45960514)
\lineto(37.89487457,304.54847232)
\lineto(35.97885895,304.54847232)
\lineto(34.66928864,308.41565982)
\lineto(29.56284332,308.41565982)
\lineto(28.16538239,304.54847232)
\lineto(26.37241364,304.54847232)
\lineto(31.2327652,317.45960514)
\closepath
\moveto(32.13803864,317.45960514)
\lineto(32.13803864,317.45960514)
\closepath
}
}
{
\newrgbcolor{curcolor}{0 0 0}
\pscustom[linestyle=none,fillstyle=solid,fillcolor=curcolor]
{
\newpath
\moveto(38.39585114,313.96155826)
\lineto(40.4437027,313.96155826)
\lineto(42.60581207,310.6480817)
\lineto(44.79428864,313.96155826)
\lineto(46.71909332,313.91761295)
\lineto(43.54624176,309.37366764)
\lineto(46.85971832,304.54847232)
\lineto(44.83823395,304.54847232)
\lineto(42.50034332,308.08167545)
\lineto(40.2327652,304.54847232)
\lineto(38.22885895,304.54847232)
\lineto(41.54233551,309.37366764)
\lineto(38.39585114,313.96155826)
\closepath
}
}
{
\newrgbcolor{curcolor}{0 0 0}
\pscustom[linestyle=none,fillstyle=solid,fillcolor=curcolor]
{
\newpath
\moveto(48.29233551,313.91761295)
\lineto(49.90073395,313.91761295)
\lineto(49.90073395,304.54847232)
\lineto(48.29233551,304.54847232)
\lineto(48.29233551,313.91761295)
\closepath
\moveto(48.29233551,317.45960514)
\lineto(49.90073395,317.45960514)
\lineto(49.90073395,315.66663639)
\lineto(48.29233551,315.66663639)
\lineto(48.29233551,317.45960514)
\closepath
}
}
{
\newrgbcolor{curcolor}{0 0 0}
\pscustom[linestyle=none,fillstyle=solid,fillcolor=curcolor]
{
\newpath
\moveto(53.2405777,307.50159732)
\curveto(53.2874527,306.97425357)(53.41928864,306.5699567)(53.63608551,306.2887067)
\curveto(54.03452301,305.77894107)(54.72592926,305.52405826)(55.71030426,305.52405826)
\curveto(56.29624176,305.52405826)(56.81186676,305.65003482)(57.25717926,305.90198795)
\curveto(57.70249176,306.15980045)(57.92514801,306.55530826)(57.92514801,307.08851139)
\curveto(57.92514801,307.49280826)(57.74643707,307.80042545)(57.3890152,308.01136295)
\curveto(57.16049957,308.1402692)(56.7093277,308.28968326)(56.03549957,308.45960514)
\lineto(54.77866364,308.77601139)
\curveto(53.97592926,308.97523014)(53.38413239,309.19788639)(53.00327301,309.44398014)
\curveto(52.32358551,309.87171451)(51.98374176,310.46351139)(51.98374176,311.21937076)
\curveto(51.98374176,312.10999576)(52.3030777,312.83069889)(52.94174957,313.38148014)
\curveto(53.58628082,313.93226139)(54.45053864,314.20765201)(55.53452301,314.20765201)
\curveto(56.95249176,314.20765201)(57.9749527,313.79163639)(58.60190582,312.95960514)
\curveto(58.99448395,312.43226139)(59.18491364,311.86390201)(59.17319489,311.25452701)
\lineto(57.67905426,311.25452701)
\curveto(57.64975739,311.61194889)(57.52378082,311.9371442)(57.30112457,312.23011295)
\curveto(56.93784332,312.64612857)(56.30796051,312.85413639)(55.41147614,312.85413639)
\curveto(54.81381989,312.85413639)(54.35971832,312.73987857)(54.04917145,312.51136295)
\curveto(53.74448395,312.28284732)(53.5921402,311.98108951)(53.5921402,311.60608951)
\curveto(53.5921402,311.19593326)(53.79428864,310.86780826)(54.19858551,310.62171451)
\curveto(54.43296051,310.47523014)(54.77866364,310.34632389)(55.23569489,310.23499576)
\lineto(56.28159332,309.98011295)
\curveto(57.41831207,309.70472232)(58.18003082,309.43812076)(58.56674957,309.18030826)
\curveto(59.18198395,308.77601139)(59.48960114,308.1402692)(59.48960114,307.2730817)
\curveto(59.48960114,306.43519107)(59.1702652,305.71155826)(58.53159332,305.10218326)
\curveto(57.89878082,304.49280826)(56.93198395,304.18812076)(55.6312027,304.18812076)
\curveto(54.23081207,304.18812076)(53.23764801,304.50452701)(52.65171051,305.13733951)
\curveto(52.07163239,305.77601139)(51.76108551,306.56409732)(51.72006989,307.50159732)
\lineto(53.2405777,307.50159732)
\closepath
\moveto(55.57846832,314.19007389)
\lineto(55.57846832,314.19007389)
\closepath
}
}
{
\newrgbcolor{curcolor}{0 0 0}
\pscustom[linestyle=none,fillstyle=solid,fillcolor=curcolor]
{
\newpath
\moveto(72.54135895,311.54456607)
\curveto(73.36167145,311.54456607)(74.00913239,311.70862857)(74.48374176,312.03675357)
\curveto(74.96421051,312.36487857)(75.20444489,312.95667545)(75.20444489,313.8121442)
\curveto(75.20444489,314.73206607)(74.87046051,315.3590192)(74.20249176,315.69300357)
\curveto(73.84506989,315.86878482)(73.36753082,315.95667545)(72.76987457,315.95667545)
\lineto(68.4983902,315.95667545)
\lineto(68.4983902,311.54456607)
\lineto(72.54135895,311.54456607)
\closepath
\moveto(66.74936676,317.45960514)
\lineto(72.72592926,317.45960514)
\curveto(73.71030426,317.45960514)(74.5218277,317.31605045)(75.16049957,317.02894107)
\curveto(76.3733902,316.47815982)(76.97983551,315.46155826)(76.97983551,313.97913639)
\curveto(76.97983551,313.20569889)(76.8187027,312.57288639)(76.49643707,312.08069889)
\curveto(76.18003082,311.58851139)(75.73471832,311.19300357)(75.16049957,310.89417545)
\curveto(75.66440582,310.68909732)(76.04233551,310.41956607)(76.29428864,310.0855817)
\curveto(76.55210114,309.75159732)(76.69565582,309.20960514)(76.7249527,308.45960514)
\lineto(76.78647614,306.72815982)
\curveto(76.80405426,306.23597232)(76.84506989,305.86976139)(76.90952301,305.62952701)
\curveto(77.01499176,305.21937076)(77.20249176,304.95569889)(77.47202301,304.83851139)
\lineto(77.47202301,304.54847232)
\lineto(75.32749176,304.54847232)
\curveto(75.26889801,304.65980045)(75.22202301,304.80335514)(75.18686676,304.97913639)
\curveto(75.15171051,305.15491764)(75.12241364,305.49476139)(75.09897614,305.99866764)
\lineto(74.99350739,308.15198795)
\curveto(74.95249176,308.99573795)(74.6390152,309.56116764)(74.0530777,309.84827701)
\curveto(73.71909332,310.00648014)(73.19467926,310.0855817)(72.47983551,310.0855817)
\lineto(68.4983902,310.0855817)
\lineto(68.4983902,304.54847232)
\lineto(66.74936676,304.54847232)
\lineto(66.74936676,317.45960514)
\closepath
}
}
{
\newrgbcolor{curcolor}{0 0 0}
\pscustom[linestyle=none,fillstyle=solid,fillcolor=curcolor]
{
\newpath
\moveto(83.25522614,314.17249576)
\curveto(83.92319489,314.17249576)(84.57065582,314.01429264)(85.19760895,313.69788639)
\curveto(85.82456207,313.38733951)(86.30210114,312.98304264)(86.63022614,312.48499576)
\curveto(86.94663239,312.01038639)(87.15756989,311.45667545)(87.26303864,310.82386295)
\curveto(87.35678864,310.3902692)(87.40366364,309.69886295)(87.40366364,308.7496442)
\lineto(80.50424957,308.7496442)
\curveto(80.53354645,307.79456607)(80.75913239,307.02698795)(81.18100739,306.44690982)
\curveto(81.60288239,305.87269107)(82.2562027,305.5855817)(83.14096832,305.5855817)
\curveto(83.9671402,305.5855817)(84.62631989,305.85804264)(85.11850739,306.40296451)
\curveto(85.39975739,306.71937076)(85.59897614,307.0855817)(85.71616364,307.50159732)
\lineto(87.2718277,307.50159732)
\curveto(87.23081207,307.1558942)(87.09311676,306.76917545)(86.85874176,306.34144107)
\curveto(86.63022614,305.91956607)(86.37241364,305.57386295)(86.08530426,305.3043317)
\curveto(85.60483551,304.8355817)(85.01010895,304.51917545)(84.30112457,304.35511295)
\curveto(83.9202652,304.26136295)(83.48960114,304.21448795)(83.00913239,304.21448795)
\curveto(81.83725739,304.21448795)(80.84409332,304.63929264)(80.0296402,305.48890201)
\curveto(79.21518707,306.34437076)(78.80796051,307.53968326)(78.80796051,309.07483951)
\curveto(78.80796051,310.58655826)(79.21811676,311.81409732)(80.03842926,312.7574567)
\curveto(80.85874176,313.70081607)(81.93100739,314.17249576)(83.25522614,314.17249576)
\closepath
\moveto(85.77768707,310.00648014)
\curveto(85.71323395,310.69202701)(85.56381989,311.23987857)(85.32944489,311.65003482)
\curveto(84.89585114,312.41175357)(84.17221832,312.79261295)(83.15854645,312.79261295)
\curveto(82.43198395,312.79261295)(81.82260895,312.52894107)(81.33042145,312.00159732)
\curveto(80.83823395,311.48011295)(80.57749176,310.81507389)(80.54819489,310.00648014)
\lineto(85.77768707,310.00648014)
\closepath
\moveto(83.10581207,314.19007389)
\lineto(83.10581207,314.19007389)
\closepath
}
}
{
\newrgbcolor{curcolor}{0 0 0}
\pscustom[linestyle=none,fillstyle=solid,fillcolor=curcolor]
{
\newpath
\moveto(90.2952652,307.50159732)
\curveto(90.3421402,306.97425357)(90.47397614,306.5699567)(90.69077301,306.2887067)
\curveto(91.08921051,305.77894107)(91.78061676,305.52405826)(92.76499176,305.52405826)
\curveto(93.35092926,305.52405826)(93.86655426,305.65003482)(94.31186676,305.90198795)
\curveto(94.75717926,306.15980045)(94.97983551,306.55530826)(94.97983551,307.08851139)
\curveto(94.97983551,307.49280826)(94.80112457,307.80042545)(94.4437027,308.01136295)
\curveto(94.21518707,308.1402692)(93.7640152,308.28968326)(93.09018707,308.45960514)
\lineto(91.83335114,308.77601139)
\curveto(91.03061676,308.97523014)(90.43881989,309.19788639)(90.05796051,309.44398014)
\curveto(89.37827301,309.87171451)(89.03842926,310.46351139)(89.03842926,311.21937076)
\curveto(89.03842926,312.10999576)(89.3577652,312.83069889)(89.99643707,313.38148014)
\curveto(90.64096832,313.93226139)(91.50522614,314.20765201)(92.58921051,314.20765201)
\curveto(94.00717926,314.20765201)(95.0296402,313.79163639)(95.65659332,312.95960514)
\curveto(96.04917145,312.43226139)(96.23960114,311.86390201)(96.22788239,311.25452701)
\lineto(94.73374176,311.25452701)
\curveto(94.70444489,311.61194889)(94.57846832,311.9371442)(94.35581207,312.23011295)
\curveto(93.99253082,312.64612857)(93.36264801,312.85413639)(92.46616364,312.85413639)
\curveto(91.86850739,312.85413639)(91.41440582,312.73987857)(91.10385895,312.51136295)
\curveto(90.79917145,312.28284732)(90.6468277,311.98108951)(90.6468277,311.60608951)
\curveto(90.6468277,311.19593326)(90.84897614,310.86780826)(91.25327301,310.62171451)
\curveto(91.48764801,310.47523014)(91.83335114,310.34632389)(92.29038239,310.23499576)
\lineto(93.33628082,309.98011295)
\curveto(94.47299957,309.70472232)(95.23471832,309.43812076)(95.62143707,309.18030826)
\curveto(96.23667145,308.77601139)(96.54428864,308.1402692)(96.54428864,307.2730817)
\curveto(96.54428864,306.43519107)(96.2249527,305.71155826)(95.58628082,305.10218326)
\curveto(94.95346832,304.49280826)(93.98667145,304.18812076)(92.6858902,304.18812076)
\curveto(91.28549957,304.18812076)(90.29233551,304.50452701)(89.70639801,305.13733951)
\curveto(89.12631989,305.77601139)(88.81577301,306.56409732)(88.77475739,307.50159732)
\lineto(90.2952652,307.50159732)
\closepath
\moveto(92.63315582,314.19007389)
\lineto(92.63315582,314.19007389)
\closepath
}
}
{
\newrgbcolor{curcolor}{0 0 0}
\pscustom[linestyle=none,fillstyle=solid,fillcolor=curcolor]
{
\newpath
\moveto(102.12534332,305.56800357)
\curveto(103.17417145,305.56800357)(103.89194489,305.96351139)(104.27866364,306.75452701)
\curveto(104.67124176,307.55140201)(104.86753082,308.43616764)(104.86753082,309.40882389)
\curveto(104.86753082,310.28773014)(104.72690582,311.00257389)(104.44565582,311.55335514)
\curveto(104.00034332,312.42054264)(103.2327652,312.85413639)(102.14292145,312.85413639)
\curveto(101.17612457,312.85413639)(100.47299957,312.48499576)(100.03354645,311.74671451)
\curveto(99.59409332,311.00843326)(99.37436676,310.11780826)(99.37436676,309.07483951)
\curveto(99.37436676,308.07288639)(99.59409332,307.23792545)(100.03354645,306.5699567)
\curveto(100.47299957,305.90198795)(101.1702652,305.56800357)(102.12534332,305.56800357)
\closepath
\moveto(102.18686676,314.2340192)
\curveto(103.39975739,314.2340192)(104.42514801,313.82972232)(105.26303864,313.02112857)
\curveto(106.10092926,312.21253482)(106.51987457,311.0230817)(106.51987457,309.4527692)
\curveto(106.51987457,307.93519107)(106.15073395,306.68128482)(105.4124527,305.69105045)
\curveto(104.67417145,304.70081607)(103.52866364,304.20569889)(101.97592926,304.20569889)
\curveto(100.68100739,304.20569889)(99.65268707,304.64222232)(98.89096832,305.5152692)
\curveto(98.12924957,306.39417545)(97.7483902,307.57190982)(97.7483902,309.04847232)
\curveto(97.7483902,310.63050357)(98.14975739,311.8902692)(98.95249176,312.8277692)
\curveto(99.75522614,313.7652692)(100.83335114,314.2340192)(102.18686676,314.2340192)
\closepath
\moveto(102.13413239,314.19007389)
\lineto(102.13413239,314.19007389)
\closepath
}
}
{
\newrgbcolor{curcolor}{0 0 0}
\pscustom[linestyle=none,fillstyle=solid,fillcolor=curcolor]
{
\newpath
\moveto(108.45346832,317.45960514)
\lineto(110.03549957,317.45960514)
\lineto(110.03549957,304.54847232)
\lineto(108.45346832,304.54847232)
\lineto(108.45346832,317.45960514)
\closepath
}
}
{
\newrgbcolor{curcolor}{0 0 0}
\pscustom[linestyle=none,fillstyle=solid,fillcolor=curcolor]
{
\newpath
\moveto(113.99936676,313.96155826)
\lineto(113.99936676,307.71253482)
\curveto(113.99936676,307.23206607)(114.07553864,306.83948795)(114.22788239,306.53480045)
\curveto(114.50913239,305.97230045)(115.03354645,305.69105045)(115.80112457,305.69105045)
\curveto(116.90268707,305.69105045)(117.65268707,306.18323795)(118.05112457,307.16761295)
\curveto(118.26792145,307.6949567)(118.37631989,308.41858951)(118.37631989,309.33851139)
\lineto(118.37631989,313.96155826)
\lineto(119.95835114,313.96155826)
\lineto(119.95835114,304.54847232)
\lineto(118.46421051,304.54847232)
\lineto(118.48178864,305.9371442)
\curveto(118.27671051,305.57972232)(118.0218277,305.27796451)(117.7171402,305.03187076)
\curveto(117.11362457,304.53968326)(116.3812027,304.29358951)(115.51987457,304.29358951)
\curveto(114.1780777,304.29358951)(113.2640152,304.7418317)(112.77768707,305.63831607)
\curveto(112.5140152,306.11878482)(112.38217926,306.76038639)(112.38217926,307.56312076)
\lineto(112.38217926,313.96155826)
\lineto(113.99936676,313.96155826)
\closepath
\moveto(116.1702652,314.19007389)
\lineto(116.1702652,314.19007389)
\closepath
}
}
{
\newrgbcolor{curcolor}{0 0 0}
\pscustom[linestyle=none,fillstyle=solid,fillcolor=curcolor]
{
\newpath
\moveto(122.75327301,316.58948795)
\lineto(124.35288239,316.58948795)
\lineto(124.35288239,313.96155826)
\lineto(125.85581207,313.96155826)
\lineto(125.85581207,312.66956607)
\lineto(124.35288239,312.66956607)
\lineto(124.35288239,306.52601139)
\curveto(124.35288239,306.19788639)(124.46421051,305.97815982)(124.68686676,305.8668317)
\curveto(124.80991364,305.80237857)(125.01499176,305.77015201)(125.30210114,305.77015201)
\lineto(125.54819489,305.77015201)
\curveto(125.63608551,305.77601139)(125.73862457,305.78480045)(125.85581207,305.7965192)
\lineto(125.85581207,304.54847232)
\curveto(125.67417145,304.49573795)(125.48374176,304.45765201)(125.28452301,304.43421451)
\curveto(125.09116364,304.41077701)(124.88022614,304.39905826)(124.65171051,304.39905826)
\curveto(123.91342926,304.39905826)(123.4124527,304.58655826)(123.14878082,304.96155826)
\curveto(122.88510895,305.34241764)(122.75327301,305.83460514)(122.75327301,306.43812076)
\lineto(122.75327301,312.66956607)
\lineto(121.47885895,312.66956607)
\lineto(121.47885895,313.96155826)
\lineto(122.75327301,313.96155826)
\lineto(122.75327301,316.58948795)
\closepath
}
}
{
\newrgbcolor{curcolor}{0 0 0}
\pscustom[linestyle=none,fillstyle=solid,fillcolor=curcolor]
{
\newpath
\moveto(127.46421051,313.91761295)
\lineto(129.07260895,313.91761295)
\lineto(129.07260895,304.54847232)
\lineto(127.46421051,304.54847232)
\lineto(127.46421051,313.91761295)
\closepath
\moveto(127.46421051,317.45960514)
\lineto(129.07260895,317.45960514)
\lineto(129.07260895,315.66663639)
\lineto(127.46421051,315.66663639)
\lineto(127.46421051,317.45960514)
\closepath
}
}
{
\newrgbcolor{curcolor}{0 0 0}
\pscustom[linestyle=none,fillstyle=solid,fillcolor=curcolor]
{
\newpath
\moveto(135.20737457,305.56800357)
\curveto(136.2562027,305.56800357)(136.97397614,305.96351139)(137.36069489,306.75452701)
\curveto(137.75327301,307.55140201)(137.94956207,308.43616764)(137.94956207,309.40882389)
\curveto(137.94956207,310.28773014)(137.80893707,311.00257389)(137.52768707,311.55335514)
\curveto(137.08237457,312.42054264)(136.31479645,312.85413639)(135.2249527,312.85413639)
\curveto(134.25815582,312.85413639)(133.55503082,312.48499576)(133.1155777,311.74671451)
\curveto(132.67612457,311.00843326)(132.45639801,310.11780826)(132.45639801,309.07483951)
\curveto(132.45639801,308.07288639)(132.67612457,307.23792545)(133.1155777,306.5699567)
\curveto(133.55503082,305.90198795)(134.25229645,305.56800357)(135.20737457,305.56800357)
\closepath
\moveto(135.26889801,314.2340192)
\curveto(136.48178864,314.2340192)(137.50717926,313.82972232)(138.34506989,313.02112857)
\curveto(139.18296051,312.21253482)(139.60190582,311.0230817)(139.60190582,309.4527692)
\curveto(139.60190582,307.93519107)(139.2327652,306.68128482)(138.49448395,305.69105045)
\curveto(137.7562027,304.70081607)(136.61069489,304.20569889)(135.05796051,304.20569889)
\curveto(133.76303864,304.20569889)(132.73471832,304.64222232)(131.97299957,305.5152692)
\curveto(131.21128082,306.39417545)(130.83042145,307.57190982)(130.83042145,309.04847232)
\curveto(130.83042145,310.63050357)(131.23178864,311.8902692)(132.03452301,312.8277692)
\curveto(132.83725739,313.7652692)(133.91538239,314.2340192)(135.26889801,314.2340192)
\closepath
\moveto(135.21616364,314.19007389)
\lineto(135.21616364,314.19007389)
\closepath
}
}
{
\newrgbcolor{curcolor}{0 0 0}
\pscustom[linestyle=none,fillstyle=solid,fillcolor=curcolor]
{
\newpath
\moveto(141.49155426,313.96155826)
\lineto(142.99448395,313.96155826)
\lineto(142.99448395,312.62562076)
\curveto(143.43979645,313.17640201)(143.91147614,313.57190982)(144.40952301,313.8121442)
\curveto(144.90756989,314.05237857)(145.46128082,314.17249576)(146.07065582,314.17249576)
\curveto(147.40659332,314.17249576)(148.30893707,313.70667545)(148.77768707,312.77503482)
\curveto(149.03549957,312.2652692)(149.16440582,311.53577701)(149.16440582,310.58655826)
\lineto(149.16440582,304.54847232)
\lineto(147.55600739,304.54847232)
\lineto(147.55600739,310.48108951)
\curveto(147.55600739,311.05530826)(147.47104645,311.51819889)(147.30112457,311.86976139)
\curveto(147.01987457,312.45569889)(146.51010895,312.74866764)(145.7718277,312.74866764)
\curveto(145.3968277,312.74866764)(145.08921051,312.7105817)(144.84897614,312.63440982)
\curveto(144.41538239,312.50550357)(144.03452301,312.24769107)(143.70639801,311.86097232)
\curveto(143.44272614,311.55042545)(143.26987457,311.22815982)(143.18784332,310.89417545)
\curveto(143.11167145,310.56605045)(143.07358551,310.09437076)(143.07358551,309.47913639)
\lineto(143.07358551,304.54847232)
\lineto(141.49155426,304.54847232)
\lineto(141.49155426,313.96155826)
\closepath
\moveto(145.2093277,314.19007389)
\lineto(145.2093277,314.19007389)
\closepath
}
}
{
\newrgbcolor{curcolor}{0 0 0}
\pscustom[linestyle=none,fillstyle=solid,fillcolor=curcolor]
{
\newpath
\moveto(295.34686279,201.43529789)
\lineto(297.38592529,201.43529789)
\lineto(301.09490967,195.23021976)
\lineto(304.80389404,201.43529789)
\lineto(306.85174561,201.43529789)
\lineto(301.97381592,193.72729007)
\lineto(301.97381592,188.52416507)
\lineto(300.22479248,188.52416507)
\lineto(300.22479248,193.72729007)
\lineto(295.34686279,201.43529789)
\closepath
\moveto(301.12127686,201.43529789)
\lineto(301.12127686,201.43529789)
\closepath
}
}
{
\newrgbcolor{curcolor}{0 0 0}
\pscustom[linestyle=none,fillstyle=solid,fillcolor=curcolor]
{
\newpath
\moveto(319.99139404,193.8151807)
\lineto(318.03143311,199.51928226)
\lineto(315.94842529,193.8151807)
\lineto(319.99139404,193.8151807)
\closepath
\moveto(317.11737061,201.43529789)
\lineto(319.09490967,201.43529789)
\lineto(323.77947998,188.52416507)
\lineto(321.86346436,188.52416507)
\lineto(320.55389404,192.39135257)
\lineto(315.44744873,192.39135257)
\lineto(314.04998779,188.52416507)
\lineto(312.25701904,188.52416507)
\lineto(317.11737061,201.43529789)
\closepath
\moveto(318.02264404,201.43529789)
\lineto(318.02264404,201.43529789)
\closepath
}
}
{
\newrgbcolor{curcolor}{0 0 0}
\pscustom[linestyle=none,fillstyle=solid,fillcolor=curcolor]
{
\newpath
\moveto(324.28045654,197.93725101)
\lineto(326.32830811,197.93725101)
\lineto(328.49041748,194.62377445)
\lineto(330.67889404,197.93725101)
\lineto(332.60369873,197.8933057)
\lineto(329.43084717,193.34936039)
\lineto(332.74432373,188.52416507)
\lineto(330.72283936,188.52416507)
\lineto(328.38494873,192.0573682)
\lineto(326.11737061,188.52416507)
\lineto(324.11346436,188.52416507)
\lineto(327.42694092,193.34936039)
\lineto(324.28045654,197.93725101)
\closepath
}
}
{
\newrgbcolor{curcolor}{0 0 0}
\pscustom[linestyle=none,fillstyle=solid,fillcolor=curcolor]
{
\newpath
\moveto(334.17694092,197.8933057)
\lineto(335.78533936,197.8933057)
\lineto(335.78533936,188.52416507)
\lineto(334.17694092,188.52416507)
\lineto(334.17694092,197.8933057)
\closepath
\moveto(334.17694092,201.43529789)
\lineto(335.78533936,201.43529789)
\lineto(335.78533936,199.64232914)
\lineto(334.17694092,199.64232914)
\lineto(334.17694092,201.43529789)
\closepath
}
}
{
\newrgbcolor{curcolor}{0 0 0}
\pscustom[linestyle=none,fillstyle=solid,fillcolor=curcolor]
{
\newpath
\moveto(339.12518311,191.47729007)
\curveto(339.17205811,190.94994632)(339.30389404,190.54564945)(339.52069092,190.26439945)
\curveto(339.91912842,189.75463382)(340.61053467,189.49975101)(341.59490967,189.49975101)
\curveto(342.18084717,189.49975101)(342.69647217,189.62572757)(343.14178467,189.8776807)
\curveto(343.58709717,190.1354932)(343.80975342,190.53100101)(343.80975342,191.06420414)
\curveto(343.80975342,191.46850101)(343.63104248,191.7761182)(343.27362061,191.9870557)
\curveto(343.04510498,192.11596195)(342.59393311,192.26537601)(341.92010498,192.43529789)
\lineto(340.66326904,192.75170414)
\curveto(339.86053467,192.95092289)(339.26873779,193.17357914)(338.88787842,193.41967289)
\curveto(338.20819092,193.84740726)(337.86834717,194.43920414)(337.86834717,195.19506351)
\curveto(337.86834717,196.08568851)(338.18768311,196.80639164)(338.82635498,197.35717289)
\curveto(339.47088623,197.90795414)(340.33514404,198.18334476)(341.41912842,198.18334476)
\curveto(342.83709717,198.18334476)(343.85955811,197.76732914)(344.48651123,196.93529789)
\curveto(344.87908936,196.40795414)(345.06951904,195.83959476)(345.05780029,195.23021976)
\lineto(343.56365967,195.23021976)
\curveto(343.53436279,195.58764164)(343.40838623,195.91283695)(343.18572998,196.2058057)
\curveto(342.82244873,196.62182132)(342.19256592,196.82982914)(341.29608154,196.82982914)
\curveto(340.69842529,196.82982914)(340.24432373,196.71557132)(339.93377686,196.4870557)
\curveto(339.62908936,196.25854007)(339.47674561,195.95678226)(339.47674561,195.58178226)
\curveto(339.47674561,195.17162601)(339.67889404,194.84350101)(340.08319092,194.59740726)
\curveto(340.31756592,194.45092289)(340.66326904,194.32201664)(341.12030029,194.21068851)
\lineto(342.16619873,193.9558057)
\curveto(343.30291748,193.68041507)(344.06463623,193.41381351)(344.45135498,193.15600101)
\curveto(345.06658936,192.75170414)(345.37420654,192.11596195)(345.37420654,191.24877445)
\curveto(345.37420654,190.41088382)(345.05487061,189.68725101)(344.41619873,189.07787601)
\curveto(343.78338623,188.46850101)(342.81658936,188.16381351)(341.51580811,188.16381351)
\curveto(340.11541748,188.16381351)(339.12225342,188.48021976)(338.53631592,189.11303226)
\curveto(337.95623779,189.75170414)(337.64569092,190.53979007)(337.60467529,191.47729007)
\lineto(339.12518311,191.47729007)
\closepath
\moveto(341.46307373,198.16576664)
\lineto(341.46307373,198.16576664)
\closepath
}
}
{
\newrgbcolor{curcolor}{0 0 0}
\pscustom[linestyle=none,fillstyle=solid,fillcolor=curcolor]
{
\newpath
\moveto(358.42596436,195.52025882)
\curveto(359.24627686,195.52025882)(359.89373779,195.68432132)(360.36834717,196.01244632)
\curveto(360.84881592,196.34057132)(361.08905029,196.9323682)(361.08905029,197.78783695)
\curveto(361.08905029,198.70775882)(360.75506592,199.33471195)(360.08709717,199.66869632)
\curveto(359.72967529,199.84447757)(359.25213623,199.9323682)(358.65447998,199.9323682)
\lineto(354.38299561,199.9323682)
\lineto(354.38299561,195.52025882)
\lineto(358.42596436,195.52025882)
\closepath
\moveto(352.63397217,201.43529789)
\lineto(358.61053467,201.43529789)
\curveto(359.59490967,201.43529789)(360.40643311,201.2917432)(361.04510498,201.00463382)
\curveto(362.25799561,200.45385257)(362.86444092,199.43725101)(362.86444092,197.95482914)
\curveto(362.86444092,197.18139164)(362.70330811,196.54857914)(362.38104248,196.05639164)
\curveto(362.06463623,195.56420414)(361.61932373,195.16869632)(361.04510498,194.8698682)
\curveto(361.54901123,194.66479007)(361.92694092,194.39525882)(362.17889404,194.06127445)
\curveto(362.43670654,193.72729007)(362.58026123,193.18529789)(362.60955811,192.43529789)
\lineto(362.67108154,190.70385257)
\curveto(362.68865967,190.21166507)(362.72967529,189.84545414)(362.79412842,189.60521976)
\curveto(362.89959717,189.19506351)(363.08709717,188.93139164)(363.35662842,188.81420414)
\lineto(363.35662842,188.52416507)
\lineto(361.21209717,188.52416507)
\curveto(361.15350342,188.6354932)(361.10662842,188.77904789)(361.07147217,188.95482914)
\curveto(361.03631592,189.13061039)(361.00701904,189.47045414)(360.98358154,189.97436039)
\lineto(360.87811279,192.1276807)
\curveto(360.83709717,192.9714307)(360.52362061,193.53686039)(359.93768311,193.82396976)
\curveto(359.60369873,193.98217289)(359.07928467,194.06127445)(358.36444092,194.06127445)
\lineto(354.38299561,194.06127445)
\lineto(354.38299561,188.52416507)
\lineto(352.63397217,188.52416507)
\lineto(352.63397217,201.43529789)
\closepath
}
}
{
\newrgbcolor{curcolor}{0 0 0}
\pscustom[linestyle=none,fillstyle=solid,fillcolor=curcolor]
{
\newpath
\moveto(369.13983154,198.14818851)
\curveto(369.80780029,198.14818851)(370.45526123,197.98998539)(371.08221436,197.67357914)
\curveto(371.70916748,197.36303226)(372.18670654,196.95873539)(372.51483154,196.46068851)
\curveto(372.83123779,195.98607914)(373.04217529,195.4323682)(373.14764404,194.7995557)
\curveto(373.24139404,194.36596195)(373.28826904,193.6745557)(373.28826904,192.72533695)
\lineto(366.38885498,192.72533695)
\curveto(366.41815186,191.77025882)(366.64373779,191.0026807)(367.06561279,190.42260257)
\curveto(367.48748779,189.84838382)(368.14080811,189.56127445)(369.02557373,189.56127445)
\curveto(369.85174561,189.56127445)(370.51092529,189.83373539)(371.00311279,190.37865726)
\curveto(371.28436279,190.69506351)(371.48358154,191.06127445)(371.60076904,191.47729007)
\lineto(373.15643311,191.47729007)
\curveto(373.11541748,191.13158695)(372.97772217,190.7448682)(372.74334717,190.31713382)
\curveto(372.51483154,189.89525882)(372.25701904,189.5495557)(371.96990967,189.28002445)
\curveto(371.48944092,188.81127445)(370.89471436,188.4948682)(370.18572998,188.3308057)
\curveto(369.80487061,188.2370557)(369.37420654,188.1901807)(368.89373779,188.1901807)
\curveto(367.72186279,188.1901807)(366.72869873,188.61498539)(365.91424561,189.46459476)
\curveto(365.09979248,190.32006351)(364.69256592,191.51537601)(364.69256592,193.05053226)
\curveto(364.69256592,194.56225101)(365.10272217,195.78979007)(365.92303467,196.73314945)
\curveto(366.74334717,197.67650882)(367.81561279,198.14818851)(369.13983154,198.14818851)
\closepath
\moveto(371.66229248,193.98217289)
\curveto(371.59783936,194.66771976)(371.44842529,195.21557132)(371.21405029,195.62572757)
\curveto(370.78045654,196.38744632)(370.05682373,196.7683057)(369.04315186,196.7683057)
\curveto(368.31658936,196.7683057)(367.70721436,196.50463382)(367.21502686,195.97729007)
\curveto(366.72283936,195.4558057)(366.46209717,194.79076664)(366.43280029,193.98217289)
\lineto(371.66229248,193.98217289)
\closepath
\moveto(368.99041748,198.16576664)
\lineto(368.99041748,198.16576664)
\closepath
}
}
{
\newrgbcolor{curcolor}{0 0 0}
\pscustom[linestyle=none,fillstyle=solid,fillcolor=curcolor]
{
\newpath
\moveto(376.17987061,191.47729007)
\curveto(376.22674561,190.94994632)(376.35858154,190.54564945)(376.57537842,190.26439945)
\curveto(376.97381592,189.75463382)(377.66522217,189.49975101)(378.64959717,189.49975101)
\curveto(379.23553467,189.49975101)(379.75115967,189.62572757)(380.19647217,189.8776807)
\curveto(380.64178467,190.1354932)(380.86444092,190.53100101)(380.86444092,191.06420414)
\curveto(380.86444092,191.46850101)(380.68572998,191.7761182)(380.32830811,191.9870557)
\curveto(380.09979248,192.11596195)(379.64862061,192.26537601)(378.97479248,192.43529789)
\lineto(377.71795654,192.75170414)
\curveto(376.91522217,192.95092289)(376.32342529,193.17357914)(375.94256592,193.41967289)
\curveto(375.26287842,193.84740726)(374.92303467,194.43920414)(374.92303467,195.19506351)
\curveto(374.92303467,196.08568851)(375.24237061,196.80639164)(375.88104248,197.35717289)
\curveto(376.52557373,197.90795414)(377.38983154,198.18334476)(378.47381592,198.18334476)
\curveto(379.89178467,198.18334476)(380.91424561,197.76732914)(381.54119873,196.93529789)
\curveto(381.93377686,196.40795414)(382.12420654,195.83959476)(382.11248779,195.23021976)
\lineto(380.61834717,195.23021976)
\curveto(380.58905029,195.58764164)(380.46307373,195.91283695)(380.24041748,196.2058057)
\curveto(379.87713623,196.62182132)(379.24725342,196.82982914)(378.35076904,196.82982914)
\curveto(377.75311279,196.82982914)(377.29901123,196.71557132)(376.98846436,196.4870557)
\curveto(376.68377686,196.25854007)(376.53143311,195.95678226)(376.53143311,195.58178226)
\curveto(376.53143311,195.17162601)(376.73358154,194.84350101)(377.13787842,194.59740726)
\curveto(377.37225342,194.45092289)(377.71795654,194.32201664)(378.17498779,194.21068851)
\lineto(379.22088623,193.9558057)
\curveto(380.35760498,193.68041507)(381.11932373,193.41381351)(381.50604248,193.15600101)
\curveto(382.12127686,192.75170414)(382.42889404,192.11596195)(382.42889404,191.24877445)
\curveto(382.42889404,190.41088382)(382.10955811,189.68725101)(381.47088623,189.07787601)
\curveto(380.83807373,188.46850101)(379.87127686,188.16381351)(378.57049561,188.16381351)
\curveto(377.17010498,188.16381351)(376.17694092,188.48021976)(375.59100342,189.11303226)
\curveto(375.01092529,189.75170414)(374.70037842,190.53979007)(374.65936279,191.47729007)
\lineto(376.17987061,191.47729007)
\closepath
\moveto(378.51776123,198.16576664)
\lineto(378.51776123,198.16576664)
\closepath
}
}
{
\newrgbcolor{curcolor}{0 0 0}
\pscustom[linestyle=none,fillstyle=solid,fillcolor=curcolor]
{
\newpath
\moveto(388.00994873,189.54369632)
\curveto(389.05877686,189.54369632)(389.77655029,189.93920414)(390.16326904,190.73021976)
\curveto(390.55584717,191.52709476)(390.75213623,192.41186039)(390.75213623,193.38451664)
\curveto(390.75213623,194.26342289)(390.61151123,194.97826664)(390.33026123,195.52904789)
\curveto(389.88494873,196.39623539)(389.11737061,196.82982914)(388.02752686,196.82982914)
\curveto(387.06072998,196.82982914)(386.35760498,196.46068851)(385.91815186,195.72240726)
\curveto(385.47869873,194.98412601)(385.25897217,194.09350101)(385.25897217,193.05053226)
\curveto(385.25897217,192.04857914)(385.47869873,191.2136182)(385.91815186,190.54564945)
\curveto(386.35760498,189.8776807)(387.05487061,189.54369632)(388.00994873,189.54369632)
\closepath
\moveto(388.07147217,198.20971195)
\curveto(389.28436279,198.20971195)(390.30975342,197.80541507)(391.14764404,196.99682132)
\curveto(391.98553467,196.18822757)(392.40447998,194.99877445)(392.40447998,193.42846195)
\curveto(392.40447998,191.91088382)(392.03533936,190.65697757)(391.29705811,189.6667432)
\curveto(390.55877686,188.67650882)(389.41326904,188.18139164)(387.86053467,188.18139164)
\curveto(386.56561279,188.18139164)(385.53729248,188.61791507)(384.77557373,189.49096195)
\curveto(384.01385498,190.3698682)(383.63299561,191.54760257)(383.63299561,193.02416507)
\curveto(383.63299561,194.60619632)(384.03436279,195.86596195)(384.83709717,196.80346195)
\curveto(385.63983154,197.74096195)(386.71795654,198.20971195)(388.07147217,198.20971195)
\closepath
\moveto(388.01873779,198.16576664)
\lineto(388.01873779,198.16576664)
\closepath
}
}
{
\newrgbcolor{curcolor}{0 0 0}
\pscustom[linestyle=none,fillstyle=solid,fillcolor=curcolor]
{
\newpath
\moveto(394.33807373,201.43529789)
\lineto(395.92010498,201.43529789)
\lineto(395.92010498,188.52416507)
\lineto(394.33807373,188.52416507)
\lineto(394.33807373,201.43529789)
\closepath
}
}
{
\newrgbcolor{curcolor}{0 0 0}
\pscustom[linestyle=none,fillstyle=solid,fillcolor=curcolor]
{
\newpath
\moveto(399.88397217,197.93725101)
\lineto(399.88397217,191.68822757)
\curveto(399.88397217,191.20775882)(399.96014404,190.8151807)(400.11248779,190.5104932)
\curveto(400.39373779,189.9479932)(400.91815186,189.6667432)(401.68572998,189.6667432)
\curveto(402.78729248,189.6667432)(403.53729248,190.1589307)(403.93572998,191.1433057)
\curveto(404.15252686,191.67064945)(404.26092529,192.39428226)(404.26092529,193.31420414)
\lineto(404.26092529,197.93725101)
\lineto(405.84295654,197.93725101)
\lineto(405.84295654,188.52416507)
\lineto(404.34881592,188.52416507)
\lineto(404.36639404,189.91283695)
\curveto(404.16131592,189.55541507)(403.90643311,189.25365726)(403.60174561,189.00756351)
\curveto(402.99822998,188.51537601)(402.26580811,188.26928226)(401.40447998,188.26928226)
\curveto(400.06268311,188.26928226)(399.14862061,188.71752445)(398.66229248,189.61400882)
\curveto(398.39862061,190.09447757)(398.26678467,190.73607914)(398.26678467,191.53881351)
\lineto(398.26678467,197.93725101)
\lineto(399.88397217,197.93725101)
\closepath
\moveto(402.05487061,198.16576664)
\lineto(402.05487061,198.16576664)
\closepath
}
}
{
\newrgbcolor{curcolor}{0 0 0}
\pscustom[linestyle=none,fillstyle=solid,fillcolor=curcolor]
{
\newpath
\moveto(408.63787842,200.5651807)
\lineto(410.23748779,200.5651807)
\lineto(410.23748779,197.93725101)
\lineto(411.74041748,197.93725101)
\lineto(411.74041748,196.64525882)
\lineto(410.23748779,196.64525882)
\lineto(410.23748779,190.50170414)
\curveto(410.23748779,190.17357914)(410.34881592,189.95385257)(410.57147217,189.84252445)
\curveto(410.69451904,189.77807132)(410.89959717,189.74584476)(411.18670654,189.74584476)
\lineto(411.43280029,189.74584476)
\curveto(411.52069092,189.75170414)(411.62322998,189.7604932)(411.74041748,189.77221195)
\lineto(411.74041748,188.52416507)
\curveto(411.55877686,188.4714307)(411.36834717,188.43334476)(411.16912842,188.40990726)
\curveto(410.97576904,188.38646976)(410.76483154,188.37475101)(410.53631592,188.37475101)
\curveto(409.79803467,188.37475101)(409.29705811,188.56225101)(409.03338623,188.93725101)
\curveto(408.76971436,189.31811039)(408.63787842,189.81029789)(408.63787842,190.41381351)
\lineto(408.63787842,196.64525882)
\lineto(407.36346436,196.64525882)
\lineto(407.36346436,197.93725101)
\lineto(408.63787842,197.93725101)
\lineto(408.63787842,200.5651807)
\closepath
}
}
{
\newrgbcolor{curcolor}{0 0 0}
\pscustom[linestyle=none,fillstyle=solid,fillcolor=curcolor]
{
\newpath
\moveto(413.34881592,197.8933057)
\lineto(414.95721436,197.8933057)
\lineto(414.95721436,188.52416507)
\lineto(413.34881592,188.52416507)
\lineto(413.34881592,197.8933057)
\closepath
\moveto(413.34881592,201.43529789)
\lineto(414.95721436,201.43529789)
\lineto(414.95721436,199.64232914)
\lineto(413.34881592,199.64232914)
\lineto(413.34881592,201.43529789)
\closepath
}
}
{
\newrgbcolor{curcolor}{0 0 0}
\pscustom[linestyle=none,fillstyle=solid,fillcolor=curcolor]
{
\newpath
\moveto(421.09197998,189.54369632)
\curveto(422.14080811,189.54369632)(422.85858154,189.93920414)(423.24530029,190.73021976)
\curveto(423.63787842,191.52709476)(423.83416748,192.41186039)(423.83416748,193.38451664)
\curveto(423.83416748,194.26342289)(423.69354248,194.97826664)(423.41229248,195.52904789)
\curveto(422.96697998,196.39623539)(422.19940186,196.82982914)(421.10955811,196.82982914)
\curveto(420.14276123,196.82982914)(419.43963623,196.46068851)(419.00018311,195.72240726)
\curveto(418.56072998,194.98412601)(418.34100342,194.09350101)(418.34100342,193.05053226)
\curveto(418.34100342,192.04857914)(418.56072998,191.2136182)(419.00018311,190.54564945)
\curveto(419.43963623,189.8776807)(420.13690186,189.54369632)(421.09197998,189.54369632)
\closepath
\moveto(421.15350342,198.20971195)
\curveto(422.36639404,198.20971195)(423.39178467,197.80541507)(424.22967529,196.99682132)
\curveto(425.06756592,196.18822757)(425.48651123,194.99877445)(425.48651123,193.42846195)
\curveto(425.48651123,191.91088382)(425.11737061,190.65697757)(424.37908936,189.6667432)
\curveto(423.64080811,188.67650882)(422.49530029,188.18139164)(420.94256592,188.18139164)
\curveto(419.64764404,188.18139164)(418.61932373,188.61791507)(417.85760498,189.49096195)
\curveto(417.09588623,190.3698682)(416.71502686,191.54760257)(416.71502686,193.02416507)
\curveto(416.71502686,194.60619632)(417.11639404,195.86596195)(417.91912842,196.80346195)
\curveto(418.72186279,197.74096195)(419.79998779,198.20971195)(421.15350342,198.20971195)
\closepath
\moveto(421.10076904,198.16576664)
\lineto(421.10076904,198.16576664)
\closepath
}
}
{
\newrgbcolor{curcolor}{0 0 0}
\pscustom[linestyle=none,fillstyle=solid,fillcolor=curcolor]
{
\newpath
\moveto(427.37615967,197.93725101)
\lineto(428.87908936,197.93725101)
\lineto(428.87908936,196.60131351)
\curveto(429.32440186,197.15209476)(429.79608154,197.54760257)(430.29412842,197.78783695)
\curveto(430.79217529,198.02807132)(431.34588623,198.14818851)(431.95526123,198.14818851)
\curveto(433.29119873,198.14818851)(434.19354248,197.6823682)(434.66229248,196.75072757)
\curveto(434.92010498,196.24096195)(435.04901123,195.51146976)(435.04901123,194.56225101)
\lineto(435.04901123,188.52416507)
\lineto(433.44061279,188.52416507)
\lineto(433.44061279,194.45678226)
\curveto(433.44061279,195.03100101)(433.35565186,195.49389164)(433.18572998,195.84545414)
\curveto(432.90447998,196.43139164)(432.39471436,196.72436039)(431.65643311,196.72436039)
\curveto(431.28143311,196.72436039)(430.97381592,196.68627445)(430.73358154,196.61010257)
\curveto(430.29998779,196.48119632)(429.91912842,196.22338382)(429.59100342,195.83666507)
\curveto(429.32733154,195.5261182)(429.15447998,195.20385257)(429.07244873,194.8698682)
\curveto(428.99627686,194.5417432)(428.95819092,194.07006351)(428.95819092,193.45482914)
\lineto(428.95819092,188.52416507)
\lineto(427.37615967,188.52416507)
\lineto(427.37615967,197.93725101)
\closepath
\moveto(431.09393311,198.16576664)
\lineto(431.09393311,198.16576664)
\closepath
}
}
{
\newrgbcolor{curcolor}{0 0 0}
\pscustom[linestyle=none,fillstyle=solid,fillcolor=curcolor]
{
\newpath
\moveto(43.01367188,7.91894046)
\curveto(43.0546875,7.18651859)(43.22753906,6.59179203)(43.53222656,6.13476078)
\curveto(44.11230469,5.27929203)(45.13476562,4.85155765)(46.59960938,4.85155765)
\curveto(47.25585938,4.85155765)(47.85351562,4.94530765)(48.39257812,5.13280765)
\curveto(49.43554688,5.4960889)(49.95703125,6.14647953)(49.95703125,7.08397953)
\curveto(49.95703125,7.78710453)(49.73730469,8.28808109)(49.29785156,8.58690921)
\curveto(48.85253906,8.87987796)(48.15527344,9.13476078)(47.20605469,9.35155765)
\lineto(45.45703125,9.74706546)
\curveto(44.31445312,10.00487796)(43.50585938,10.28905765)(43.03125,10.59960453)
\curveto(42.2109375,11.13866703)(41.80078125,11.94433109)(41.80078125,13.01659671)
\curveto(41.80078125,14.17675296)(42.20214844,15.1289014)(43.00488281,15.87304203)
\curveto(43.80761719,16.61718265)(44.94433594,16.98925296)(46.41503906,16.98925296)
\curveto(47.76855469,16.98925296)(48.91699219,16.66112796)(49.86035156,16.00487796)
\curveto(50.80957031,15.35448734)(51.28417969,14.31151859)(51.28417969,12.87597171)
\lineto(49.640625,12.87597171)
\curveto(49.55273438,13.56737796)(49.36523438,14.0976514)(49.078125,14.46679203)
\curveto(48.54492188,15.14062015)(47.63964844,15.47753421)(46.36230469,15.47753421)
\curveto(45.33105469,15.47753421)(44.58984375,15.26073734)(44.13867188,14.82714359)
\curveto(43.6875,14.39354984)(43.46191406,13.88964359)(43.46191406,13.31542484)
\curveto(43.46191406,12.68261234)(43.72558594,12.21972171)(44.25292969,11.92675296)
\curveto(44.59863281,11.73925296)(45.38085938,11.50487796)(46.59960938,11.22362796)
\lineto(48.41015625,10.81054203)
\curveto(49.28320312,10.61132328)(49.95703125,10.33886234)(50.43164062,9.99315921)
\curveto(51.25195312,9.38964359)(51.66210938,8.51366703)(51.66210938,7.36522953)
\curveto(51.66210938,5.93554203)(51.140625,4.91308109)(50.09765625,4.29784671)
\curveto(49.06054688,3.68261234)(47.85351562,3.37499515)(46.4765625,3.37499515)
\curveto(44.87109375,3.37499515)(43.61425781,3.7851514)(42.70605469,4.6054639)
\curveto(41.79785156,5.41991703)(41.35253906,6.52440921)(41.37011719,7.91894046)
\lineto(43.01367188,7.91894046)
\closepath
\moveto(46.546875,17.01562015)
\lineto(46.546875,17.01562015)
\closepath
}
}
{
\newrgbcolor{curcolor}{0 0 0}
\pscustom[linestyle=none,fillstyle=solid,fillcolor=curcolor]
{
\newpath
\moveto(53.6484375,13.16601078)
\lineto(55.21289062,13.16601078)
\lineto(55.21289062,11.83007328)
\curveto(55.58789062,12.2929639)(55.92773438,12.62987796)(56.23242188,12.84081546)
\curveto(56.75390625,13.19823734)(57.34570312,13.37694828)(58.0078125,13.37694828)
\curveto(58.7578125,13.37694828)(59.36132812,13.19237796)(59.81835938,12.82323734)
\curveto(60.07617188,12.61229984)(60.31054688,12.30175296)(60.52148438,11.89159671)
\curveto(60.87304688,12.39550296)(61.28613281,12.76757328)(61.76074219,13.00780765)
\curveto(62.23535156,13.2539014)(62.76855469,13.37694828)(63.36035156,13.37694828)
\curveto(64.62597656,13.37694828)(65.48730469,12.91991703)(65.94433594,12.00585453)
\curveto(66.19042969,11.51366703)(66.31347656,10.85155765)(66.31347656,10.0195264)
\lineto(66.31347656,3.75292484)
\lineto(64.66992188,3.75292484)
\lineto(64.66992188,10.29198734)
\curveto(64.66992188,10.91894046)(64.51171875,11.34960453)(64.1953125,11.58397953)
\curveto(63.88476562,11.81835453)(63.50390625,11.93554203)(63.05273438,11.93554203)
\curveto(62.43164062,11.93554203)(61.89550781,11.72753421)(61.44433594,11.31151859)
\curveto(60.99902344,10.89550296)(60.77636719,10.20116703)(60.77636719,9.22851078)
\lineto(60.77636719,3.75292484)
\lineto(59.16796875,3.75292484)
\lineto(59.16796875,9.89647953)
\curveto(59.16796875,10.5351514)(59.09179688,11.00097171)(58.93945312,11.29394046)
\curveto(58.69921875,11.73339359)(58.25097656,11.95312015)(57.59472656,11.95312015)
\curveto(56.99707031,11.95312015)(56.45214844,11.72167484)(55.95996094,11.25878421)
\curveto(55.47363281,10.79589359)(55.23046875,9.95800296)(55.23046875,8.74511234)
\lineto(55.23046875,3.75292484)
\lineto(53.6484375,3.75292484)
\lineto(53.6484375,13.16601078)
\closepath
}
}
{
\newrgbcolor{curcolor}{0 0 0}
\pscustom[linestyle=none,fillstyle=solid,fillcolor=curcolor]
{
\newpath
\moveto(69.83789062,6.25780765)
\curveto(69.83789062,5.8007764)(70.00488281,5.44042484)(70.33886719,5.17675296)
\curveto(70.67285156,4.91308109)(71.06835938,4.78124515)(71.52539062,4.78124515)
\curveto(72.08203125,4.78124515)(72.62109375,4.9101514)(73.14257812,5.1679639)
\curveto(74.02148438,5.59569828)(74.4609375,6.29589359)(74.4609375,7.26854984)
\lineto(74.4609375,8.5429639)
\curveto(74.26757812,8.41991703)(74.01855469,8.31737796)(73.71386719,8.23534671)
\curveto(73.40917969,8.15331546)(73.11035156,8.09472171)(72.81738281,8.05956546)
\lineto(71.859375,7.93651859)
\curveto(71.28515625,7.86034671)(70.85449219,7.74022953)(70.56738281,7.57616703)
\curveto(70.08105469,7.3007764)(69.83789062,6.86132328)(69.83789062,6.25780765)
\closepath
\moveto(73.66992188,9.4570264)
\curveto(74.03320312,9.5039014)(74.27636719,9.65624515)(74.39941406,9.91405765)
\curveto(74.46972656,10.05468265)(74.50488281,10.25683109)(74.50488281,10.52050296)
\curveto(74.50488281,11.05956546)(74.31152344,11.4492139)(73.92480469,11.68944828)
\curveto(73.54394531,11.93554203)(72.99609375,12.0585889)(72.28125,12.0585889)
\curveto(71.45507812,12.0585889)(70.86914062,11.83593265)(70.5234375,11.39062015)
\curveto(70.33007812,11.1445264)(70.20410156,10.77831546)(70.14550781,10.29198734)
\lineto(68.66894531,10.29198734)
\curveto(68.69824219,11.45214359)(69.07324219,12.25780765)(69.79394531,12.70897953)
\curveto(70.52050781,13.16601078)(71.36132812,13.3945264)(72.31640625,13.3945264)
\curveto(73.42382812,13.3945264)(74.32324219,13.1835889)(75.01464844,12.7617139)
\curveto(75.70019531,12.3398389)(76.04296875,11.6835889)(76.04296875,10.7929639)
\lineto(76.04296875,5.37011234)
\curveto(76.04296875,5.20604984)(76.07519531,5.0742139)(76.13964844,4.97460453)
\curveto(76.20996094,4.87499515)(76.35351562,4.82519046)(76.5703125,4.82519046)
\curveto(76.640625,4.82519046)(76.71972656,4.82812015)(76.80761719,4.83397953)
\curveto(76.89550781,4.84569828)(76.98925781,4.86034671)(77.08886719,4.87792484)
\lineto(77.08886719,3.70897953)
\curveto(76.84277344,3.63866703)(76.65527344,3.59472171)(76.52636719,3.57714359)
\curveto(76.39746094,3.55956546)(76.22167969,3.5507764)(75.99902344,3.5507764)
\curveto(75.45410156,3.5507764)(75.05859375,3.74413578)(74.8125,4.13085453)
\curveto(74.68359375,4.33593265)(74.59277344,4.62597171)(74.54003906,5.00097171)
\curveto(74.21777344,4.57909671)(73.75488281,4.21288578)(73.15136719,3.9023389)
\curveto(72.54785156,3.59179203)(71.8828125,3.43651859)(71.15625,3.43651859)
\curveto(70.28320312,3.43651859)(69.56835938,3.70019046)(69.01171875,4.22753421)
\curveto(68.4609375,4.76073734)(68.18554688,5.4257764)(68.18554688,6.2226514)
\curveto(68.18554688,7.09569828)(68.45800781,7.77245609)(69.00292969,8.25292484)
\curveto(69.54785156,8.73339359)(70.26269531,9.02929203)(71.14746094,9.14062015)
\lineto(73.66992188,9.4570264)
\closepath
\moveto(72.36035156,13.3945264)
\lineto(72.36035156,13.3945264)
\closepath
}
}
{
\newrgbcolor{curcolor}{0 0 0}
\pscustom[linestyle=none,fillstyle=solid,fillcolor=curcolor]
{
\newpath
\moveto(78.68847656,16.66405765)
\lineto(80.27050781,16.66405765)
\lineto(80.27050781,3.75292484)
\lineto(78.68847656,3.75292484)
\lineto(78.68847656,16.66405765)
\closepath
}
}
{
\newrgbcolor{curcolor}{0 0 0}
\pscustom[linestyle=none,fillstyle=solid,fillcolor=curcolor]
{
\newpath
\moveto(82.69628906,16.66405765)
\lineto(84.27832031,16.66405765)
\lineto(84.27832031,3.75292484)
\lineto(82.69628906,3.75292484)
\lineto(82.69628906,16.66405765)
\closepath
}
}
{
\newrgbcolor{curcolor}{0 0 0}
\pscustom[linestyle=none,fillstyle=solid,fillcolor=curcolor]
{
\newpath
\moveto(90.58007812,13.37694828)
\curveto(91.24804688,13.37694828)(91.89550781,13.21874515)(92.52246094,12.9023389)
\curveto(93.14941406,12.59179203)(93.62695312,12.18749515)(93.95507812,11.68944828)
\curveto(94.27148438,11.2148389)(94.48242188,10.66112796)(94.58789062,10.02831546)
\curveto(94.68164062,9.59472171)(94.72851562,8.90331546)(94.72851562,7.95409671)
\lineto(87.82910156,7.95409671)
\curveto(87.85839844,6.99901859)(88.08398438,6.23144046)(88.50585938,5.65136234)
\curveto(88.92773438,5.07714359)(89.58105469,4.79003421)(90.46582031,4.79003421)
\curveto(91.29199219,4.79003421)(91.95117188,5.06249515)(92.44335938,5.60741703)
\curveto(92.72460938,5.92382328)(92.92382812,6.29003421)(93.04101562,6.70604984)
\lineto(94.59667969,6.70604984)
\curveto(94.55566406,6.36034671)(94.41796875,5.97362796)(94.18359375,5.54589359)
\curveto(93.95507812,5.12401859)(93.69726562,4.77831546)(93.41015625,4.50878421)
\curveto(92.9296875,4.04003421)(92.33496094,3.72362796)(91.62597656,3.55956546)
\curveto(91.24511719,3.46581546)(90.81445312,3.41894046)(90.33398438,3.41894046)
\curveto(89.16210938,3.41894046)(88.16894531,3.84374515)(87.35449219,4.69335453)
\curveto(86.54003906,5.54882328)(86.1328125,6.74413578)(86.1328125,8.27929203)
\curveto(86.1328125,9.79101078)(86.54296875,11.01854984)(87.36328125,11.96190921)
\curveto(88.18359375,12.90526859)(89.25585938,13.37694828)(90.58007812,13.37694828)
\closepath
\moveto(93.10253906,9.21093265)
\curveto(93.03808594,9.89647953)(92.88867188,10.44433109)(92.65429688,10.85448734)
\curveto(92.22070312,11.61620609)(91.49707031,11.99706546)(90.48339844,11.99706546)
\curveto(89.75683594,11.99706546)(89.14746094,11.73339359)(88.65527344,11.20604984)
\curveto(88.16308594,10.68456546)(87.90234375,10.0195264)(87.87304688,9.21093265)
\lineto(93.10253906,9.21093265)
\closepath
\moveto(90.43066406,13.3945264)
\lineto(90.43066406,13.3945264)
\closepath
}
}
{
\newrgbcolor{curcolor}{0 0 0}
\pscustom[linestyle=none,fillstyle=solid,fillcolor=curcolor]
{
\newpath
\moveto(97.62011719,6.70604984)
\curveto(97.66699219,6.17870609)(97.79882812,5.77440921)(98.015625,5.49315921)
\curveto(98.4140625,4.98339359)(99.10546875,4.72851078)(100.08984375,4.72851078)
\curveto(100.67578125,4.72851078)(101.19140625,4.85448734)(101.63671875,5.10644046)
\curveto(102.08203125,5.36425296)(102.3046875,5.75976078)(102.3046875,6.2929639)
\curveto(102.3046875,6.69726078)(102.12597656,7.00487796)(101.76855469,7.21581546)
\curveto(101.54003906,7.34472171)(101.08886719,7.49413578)(100.41503906,7.66405765)
\lineto(99.15820312,7.9804639)
\curveto(98.35546875,8.17968265)(97.76367188,8.4023389)(97.3828125,8.64843265)
\curveto(96.703125,9.07616703)(96.36328125,9.6679639)(96.36328125,10.42382328)
\curveto(96.36328125,11.31444828)(96.68261719,12.0351514)(97.32128906,12.58593265)
\curveto(97.96582031,13.1367139)(98.83007812,13.41210453)(99.9140625,13.41210453)
\curveto(101.33203125,13.41210453)(102.35449219,12.9960889)(102.98144531,12.16405765)
\curveto(103.37402344,11.6367139)(103.56445312,11.06835453)(103.55273438,10.45897953)
\lineto(102.05859375,10.45897953)
\curveto(102.02929688,10.8164014)(101.90332031,11.14159671)(101.68066406,11.43456546)
\curveto(101.31738281,11.85058109)(100.6875,12.0585889)(99.79101562,12.0585889)
\curveto(99.19335938,12.0585889)(98.73925781,11.94433109)(98.42871094,11.71581546)
\curveto(98.12402344,11.48729984)(97.97167969,11.18554203)(97.97167969,10.81054203)
\curveto(97.97167969,10.40038578)(98.17382812,10.07226078)(98.578125,9.82616703)
\curveto(98.8125,9.67968265)(99.15820312,9.5507764)(99.61523438,9.43944828)
\lineto(100.66113281,9.18456546)
\curveto(101.79785156,8.90917484)(102.55957031,8.64257328)(102.94628906,8.38476078)
\curveto(103.56152344,7.9804639)(103.86914062,7.34472171)(103.86914062,6.47753421)
\curveto(103.86914062,5.63964359)(103.54980469,4.91601078)(102.91113281,4.30663578)
\curveto(102.27832031,3.69726078)(101.31152344,3.39257328)(100.01074219,3.39257328)
\curveto(98.61035156,3.39257328)(97.6171875,3.70897953)(97.03125,4.34179203)
\curveto(96.45117188,4.9804639)(96.140625,5.76854984)(96.09960938,6.70604984)
\lineto(97.62011719,6.70604984)
\closepath
\moveto(99.95800781,13.3945264)
\lineto(99.95800781,13.3945264)
\closepath
}
}
{
\newrgbcolor{curcolor}{0 0 0}
\pscustom[linestyle=none,fillstyle=solid,fillcolor=curcolor]
{
\newpath
\moveto(106.03125,15.79394046)
\lineto(107.63085938,15.79394046)
\lineto(107.63085938,13.16601078)
\lineto(109.13378906,13.16601078)
\lineto(109.13378906,11.87401859)
\lineto(107.63085938,11.87401859)
\lineto(107.63085938,5.7304639)
\curveto(107.63085938,5.4023389)(107.7421875,5.18261234)(107.96484375,5.07128421)
\curveto(108.08789062,5.00683109)(108.29296875,4.97460453)(108.58007812,4.97460453)
\lineto(108.82617188,4.97460453)
\curveto(108.9140625,4.9804639)(109.01660156,4.98925296)(109.13378906,5.00097171)
\lineto(109.13378906,3.75292484)
\curveto(108.95214844,3.70019046)(108.76171875,3.66210453)(108.5625,3.63866703)
\curveto(108.36914062,3.61522953)(108.15820312,3.60351078)(107.9296875,3.60351078)
\curveto(107.19140625,3.60351078)(106.69042969,3.79101078)(106.42675781,4.16601078)
\curveto(106.16308594,4.54687015)(106.03125,5.03905765)(106.03125,5.64257328)
\lineto(106.03125,11.87401859)
\lineto(104.75683594,11.87401859)
\lineto(104.75683594,13.16601078)
\lineto(106.03125,13.16601078)
\lineto(106.03125,15.79394046)
\closepath
}
}
{
\newrgbcolor{curcolor}{0 0 0}
\pscustom[linestyle=none,fillstyle=solid,fillcolor=curcolor]
{
\newpath
\moveto(119.70703125,4.8164014)
\curveto(120.4453125,4.8164014)(121.05761719,5.12401859)(121.54394531,5.73925296)
\curveto(122.03613281,6.36034671)(122.28222656,7.28612796)(122.28222656,8.51659671)
\curveto(122.28222656,9.26659671)(122.17382812,9.91112796)(121.95703125,10.45019046)
\curveto(121.546875,11.48729984)(120.796875,12.00585453)(119.70703125,12.00585453)
\curveto(118.61132812,12.00585453)(117.86132812,11.45800296)(117.45703125,10.36229984)
\curveto(117.24023438,9.77636234)(117.13183594,9.03222171)(117.13183594,8.12987796)
\curveto(117.13183594,7.40331546)(117.24023438,6.7851514)(117.45703125,6.27538578)
\curveto(117.8671875,5.30272953)(118.6171875,4.8164014)(119.70703125,4.8164014)
\closepath
\moveto(115.61132812,13.12206546)
\lineto(117.14941406,13.12206546)
\lineto(117.14941406,11.87401859)
\curveto(117.46582031,12.30175296)(117.81152344,12.63280765)(118.18652344,12.86718265)
\curveto(118.71972656,13.21874515)(119.34667969,13.3945264)(120.06738281,13.3945264)
\curveto(121.13378906,13.3945264)(122.0390625,12.98437015)(122.78320312,12.16405765)
\curveto(123.52734375,11.34960453)(123.89941406,10.1835889)(123.89941406,8.66601078)
\curveto(123.89941406,6.61522953)(123.36328125,5.15038578)(122.29101562,4.27147953)
\curveto(121.61132812,3.7148389)(120.8203125,3.43651859)(119.91796875,3.43651859)
\curveto(119.20898438,3.43651859)(118.61425781,3.59179203)(118.13378906,3.9023389)
\curveto(117.85253906,4.07812015)(117.5390625,4.37987796)(117.19335938,4.80761234)
\lineto(117.19335938,-0.00000485)
\lineto(115.61132812,-0.00000485)
\lineto(115.61132812,13.12206546)
\closepath
}
}
{
\newrgbcolor{curcolor}{0 0 0}
\pscustom[linestyle=none,fillstyle=solid,fillcolor=curcolor]
{
\newpath
\moveto(125.79785156,13.16601078)
\lineto(127.30078125,13.16601078)
\lineto(127.30078125,11.54003421)
\curveto(127.42382812,11.85644046)(127.72558594,12.24022953)(128.20605469,12.6914014)
\curveto(128.68652344,13.14843265)(129.24023438,13.37694828)(129.8671875,13.37694828)
\curveto(129.89648438,13.37694828)(129.94628906,13.37401859)(130.01660156,13.36815921)
\curveto(130.08691406,13.36229984)(130.20703125,13.35058109)(130.37695312,13.33300296)
\lineto(130.37695312,11.66308109)
\curveto(130.28320312,11.68065921)(130.1953125,11.69237796)(130.11328125,11.69823734)
\curveto(130.03710938,11.70409671)(129.95214844,11.7070264)(129.85839844,11.7070264)
\curveto(129.06152344,11.7070264)(128.44921875,11.4492139)(128.02148438,10.9335889)
\curveto(127.59375,10.42382328)(127.37988281,9.83495609)(127.37988281,9.16698734)
\lineto(127.37988281,3.75292484)
\lineto(125.79785156,3.75292484)
\lineto(125.79785156,13.16601078)
\closepath
}
}
{
\newrgbcolor{curcolor}{0 0 0}
\pscustom[linestyle=none,fillstyle=solid,fillcolor=curcolor]
{
\newpath
\moveto(131.765625,13.12206546)
\lineto(133.37402344,13.12206546)
\lineto(133.37402344,3.75292484)
\lineto(131.765625,3.75292484)
\lineto(131.765625,13.12206546)
\closepath
\moveto(131.765625,16.66405765)
\lineto(133.37402344,16.66405765)
\lineto(133.37402344,14.8710889)
\lineto(131.765625,14.8710889)
\lineto(131.765625,16.66405765)
\closepath
}
}
{
\newrgbcolor{curcolor}{0 0 0}
\pscustom[linestyle=none,fillstyle=solid,fillcolor=curcolor]
{
\newpath
\moveto(135.7734375,13.16601078)
\lineto(137.27636719,13.16601078)
\lineto(137.27636719,11.83007328)
\curveto(137.72167969,12.38085453)(138.19335938,12.77636234)(138.69140625,13.01659671)
\curveto(139.18945312,13.25683109)(139.74316406,13.37694828)(140.35253906,13.37694828)
\curveto(141.68847656,13.37694828)(142.59082031,12.91112796)(143.05957031,11.97948734)
\curveto(143.31738281,11.46972171)(143.44628906,10.74022953)(143.44628906,9.79101078)
\lineto(143.44628906,3.75292484)
\lineto(141.83789062,3.75292484)
\lineto(141.83789062,9.68554203)
\curveto(141.83789062,10.25976078)(141.75292969,10.7226514)(141.58300781,11.0742139)
\curveto(141.30175781,11.6601514)(140.79199219,11.95312015)(140.05371094,11.95312015)
\curveto(139.67871094,11.95312015)(139.37109375,11.91503421)(139.13085938,11.83886234)
\curveto(138.69726562,11.70995609)(138.31640625,11.45214359)(137.98828125,11.06542484)
\curveto(137.72460938,10.75487796)(137.55175781,10.43261234)(137.46972656,10.09862796)
\curveto(137.39355469,9.77050296)(137.35546875,9.29882328)(137.35546875,8.6835889)
\lineto(137.35546875,3.75292484)
\lineto(135.7734375,3.75292484)
\lineto(135.7734375,13.16601078)
\closepath
\moveto(139.49121094,13.3945264)
\lineto(139.49121094,13.3945264)
\closepath
}
}
{
\newrgbcolor{curcolor}{0 0 0}
\pscustom[linestyle=none,fillstyle=solid,fillcolor=curcolor]
{
\newpath
\moveto(146.109375,15.79394046)
\lineto(147.70898438,15.79394046)
\lineto(147.70898438,13.16601078)
\lineto(149.21191406,13.16601078)
\lineto(149.21191406,11.87401859)
\lineto(147.70898438,11.87401859)
\lineto(147.70898438,5.7304639)
\curveto(147.70898438,5.4023389)(147.8203125,5.18261234)(148.04296875,5.07128421)
\curveto(148.16601562,5.00683109)(148.37109375,4.97460453)(148.65820312,4.97460453)
\lineto(148.90429688,4.97460453)
\curveto(148.9921875,4.9804639)(149.09472656,4.98925296)(149.21191406,5.00097171)
\lineto(149.21191406,3.75292484)
\curveto(149.03027344,3.70019046)(148.83984375,3.66210453)(148.640625,3.63866703)
\curveto(148.44726562,3.61522953)(148.23632812,3.60351078)(148.0078125,3.60351078)
\curveto(147.26953125,3.60351078)(146.76855469,3.79101078)(146.50488281,4.16601078)
\curveto(146.24121094,4.54687015)(146.109375,5.03905765)(146.109375,5.64257328)
\lineto(146.109375,11.87401859)
\lineto(144.83496094,11.87401859)
\lineto(144.83496094,13.16601078)
\lineto(146.109375,13.16601078)
\lineto(146.109375,15.79394046)
\closepath
}
}
{
\newrgbcolor{curcolor}{0 0 0}
\pscustom[linestyle=none,fillstyle=solid,fillcolor=curcolor]
{
\newpath
\moveto(152.03320312,6.25780765)
\curveto(152.03320312,5.8007764)(152.20019531,5.44042484)(152.53417969,5.17675296)
\curveto(152.86816406,4.91308109)(153.26367188,4.78124515)(153.72070312,4.78124515)
\curveto(154.27734375,4.78124515)(154.81640625,4.9101514)(155.33789062,5.1679639)
\curveto(156.21679688,5.59569828)(156.65625,6.29589359)(156.65625,7.26854984)
\lineto(156.65625,8.5429639)
\curveto(156.46289062,8.41991703)(156.21386719,8.31737796)(155.90917969,8.23534671)
\curveto(155.60449219,8.15331546)(155.30566406,8.09472171)(155.01269531,8.05956546)
\lineto(154.0546875,7.93651859)
\curveto(153.48046875,7.86034671)(153.04980469,7.74022953)(152.76269531,7.57616703)
\curveto(152.27636719,7.3007764)(152.03320312,6.86132328)(152.03320312,6.25780765)
\closepath
\moveto(155.86523438,9.4570264)
\curveto(156.22851562,9.5039014)(156.47167969,9.65624515)(156.59472656,9.91405765)
\curveto(156.66503906,10.05468265)(156.70019531,10.25683109)(156.70019531,10.52050296)
\curveto(156.70019531,11.05956546)(156.50683594,11.4492139)(156.12011719,11.68944828)
\curveto(155.73925781,11.93554203)(155.19140625,12.0585889)(154.4765625,12.0585889)
\curveto(153.65039062,12.0585889)(153.06445312,11.83593265)(152.71875,11.39062015)
\curveto(152.52539062,11.1445264)(152.39941406,10.77831546)(152.34082031,10.29198734)
\lineto(150.86425781,10.29198734)
\curveto(150.89355469,11.45214359)(151.26855469,12.25780765)(151.98925781,12.70897953)
\curveto(152.71582031,13.16601078)(153.55664062,13.3945264)(154.51171875,13.3945264)
\curveto(155.61914062,13.3945264)(156.51855469,13.1835889)(157.20996094,12.7617139)
\curveto(157.89550781,12.3398389)(158.23828125,11.6835889)(158.23828125,10.7929639)
\lineto(158.23828125,5.37011234)
\curveto(158.23828125,5.20604984)(158.27050781,5.0742139)(158.33496094,4.97460453)
\curveto(158.40527344,4.87499515)(158.54882812,4.82519046)(158.765625,4.82519046)
\curveto(158.8359375,4.82519046)(158.91503906,4.82812015)(159.00292969,4.83397953)
\curveto(159.09082031,4.84569828)(159.18457031,4.86034671)(159.28417969,4.87792484)
\lineto(159.28417969,3.70897953)
\curveto(159.03808594,3.63866703)(158.85058594,3.59472171)(158.72167969,3.57714359)
\curveto(158.59277344,3.55956546)(158.41699219,3.5507764)(158.19433594,3.5507764)
\curveto(157.64941406,3.5507764)(157.25390625,3.74413578)(157.0078125,4.13085453)
\curveto(156.87890625,4.33593265)(156.78808594,4.62597171)(156.73535156,5.00097171)
\curveto(156.41308594,4.57909671)(155.95019531,4.21288578)(155.34667969,3.9023389)
\curveto(154.74316406,3.59179203)(154.078125,3.43651859)(153.3515625,3.43651859)
\curveto(152.47851562,3.43651859)(151.76367188,3.70019046)(151.20703125,4.22753421)
\curveto(150.65625,4.76073734)(150.38085938,5.4257764)(150.38085938,6.2226514)
\curveto(150.38085938,7.09569828)(150.65332031,7.77245609)(151.19824219,8.25292484)
\curveto(151.74316406,8.73339359)(152.45800781,9.02929203)(153.34277344,9.14062015)
\lineto(155.86523438,9.4570264)
\closepath
\moveto(154.55566406,13.3945264)
\lineto(154.55566406,13.3945264)
\closepath
}
}
{
\newrgbcolor{curcolor}{0 0 0}
\pscustom[linestyle=none,fillstyle=solid,fillcolor=curcolor]
{
\newpath
\moveto(160.71679688,16.70800296)
\lineto(162.25488281,16.70800296)
\lineto(162.25488281,12.02343265)
\curveto(162.60058594,12.47460453)(163.01367188,12.81737796)(163.49414062,13.05175296)
\curveto(163.97460938,13.29198734)(164.49609375,13.41210453)(165.05859375,13.41210453)
\curveto(166.23046875,13.41210453)(167.1796875,13.00780765)(167.90625,12.1992139)
\curveto(168.63867188,11.39647953)(169.00488281,10.20995609)(169.00488281,8.63964359)
\curveto(169.00488281,7.15136234)(168.64453125,5.91503421)(167.92382812,4.93065921)
\curveto(167.203125,3.94628421)(166.20410156,3.45409671)(164.92675781,3.45409671)
\curveto(164.21191406,3.45409671)(163.60839844,3.62694828)(163.11621094,3.9726514)
\curveto(162.82324219,4.17772953)(162.50976562,4.50585453)(162.17578125,4.9570264)
\lineto(162.17578125,3.75292484)
\lineto(160.71679688,3.75292484)
\lineto(160.71679688,16.70800296)
\closepath
\moveto(164.83007812,4.85155765)
\curveto(165.68554688,4.85155765)(166.32421875,5.1914014)(166.74609375,5.8710889)
\curveto(167.17382812,6.5507764)(167.38769531,7.44726078)(167.38769531,8.56054203)
\curveto(167.38769531,9.5507764)(167.17382812,10.3710889)(166.74609375,11.02147953)
\curveto(166.32421875,11.67187015)(165.70019531,11.99706546)(164.87402344,11.99706546)
\curveto(164.15332031,11.99706546)(163.52050781,11.7304639)(162.97558594,11.19726078)
\curveto(162.43652344,10.66405765)(162.16699219,9.7851514)(162.16699219,8.56054203)
\curveto(162.16699219,7.6757764)(162.27832031,6.95800296)(162.50097656,6.40722171)
\curveto(162.91699219,5.37011234)(163.69335938,4.85155765)(164.83007812,4.85155765)
\closepath
}
}
{
\newrgbcolor{curcolor}{0 0 0}
\pscustom[linestyle=none,fillstyle=solid,fillcolor=curcolor]
{
\newpath
\moveto(170.90332031,16.66405765)
\lineto(172.48535156,16.66405765)
\lineto(172.48535156,3.75292484)
\lineto(170.90332031,3.75292484)
\lineto(170.90332031,16.66405765)
\closepath
}
}
{
\newrgbcolor{curcolor}{0 0 0}
\pscustom[linestyle=none,fillstyle=solid,fillcolor=curcolor]
{
\newpath
\moveto(178.78710938,13.37694828)
\curveto(179.45507812,13.37694828)(180.10253906,13.21874515)(180.72949219,12.9023389)
\curveto(181.35644531,12.59179203)(181.83398438,12.18749515)(182.16210938,11.68944828)
\curveto(182.47851562,11.2148389)(182.68945312,10.66112796)(182.79492188,10.02831546)
\curveto(182.88867188,9.59472171)(182.93554688,8.90331546)(182.93554688,7.95409671)
\lineto(176.03613281,7.95409671)
\curveto(176.06542969,6.99901859)(176.29101562,6.23144046)(176.71289062,5.65136234)
\curveto(177.13476562,5.07714359)(177.78808594,4.79003421)(178.67285156,4.79003421)
\curveto(179.49902344,4.79003421)(180.15820312,5.06249515)(180.65039062,5.60741703)
\curveto(180.93164062,5.92382328)(181.13085938,6.29003421)(181.24804688,6.70604984)
\lineto(182.80371094,6.70604984)
\curveto(182.76269531,6.36034671)(182.625,5.97362796)(182.390625,5.54589359)
\curveto(182.16210938,5.12401859)(181.90429688,4.77831546)(181.6171875,4.50878421)
\curveto(181.13671875,4.04003421)(180.54199219,3.72362796)(179.83300781,3.55956546)
\curveto(179.45214844,3.46581546)(179.02148438,3.41894046)(178.54101562,3.41894046)
\curveto(177.36914062,3.41894046)(176.37597656,3.84374515)(175.56152344,4.69335453)
\curveto(174.74707031,5.54882328)(174.33984375,6.74413578)(174.33984375,8.27929203)
\curveto(174.33984375,9.79101078)(174.75,11.01854984)(175.5703125,11.96190921)
\curveto(176.390625,12.90526859)(177.46289062,13.37694828)(178.78710938,13.37694828)
\closepath
\moveto(181.30957031,9.21093265)
\curveto(181.24511719,9.89647953)(181.09570312,10.44433109)(180.86132812,10.85448734)
\curveto(180.42773438,11.61620609)(179.70410156,11.99706546)(178.69042969,11.99706546)
\curveto(177.96386719,11.99706546)(177.35449219,11.73339359)(176.86230469,11.20604984)
\curveto(176.37011719,10.68456546)(176.109375,10.0195264)(176.08007812,9.21093265)
\lineto(181.30957031,9.21093265)
\closepath
\moveto(178.63769531,13.3945264)
\lineto(178.63769531,13.3945264)
\closepath
}
}
{
\newrgbcolor{curcolor}{0 0 0}
\pscustom[linestyle=none,fillstyle=solid,fillcolor=curcolor]
{
\newpath
\moveto(193.79882812,13.37694828)
\curveto(194.46679688,13.37694828)(195.11425781,13.21874515)(195.74121094,12.9023389)
\curveto(196.36816406,12.59179203)(196.84570312,12.18749515)(197.17382812,11.68944828)
\curveto(197.49023438,11.2148389)(197.70117188,10.66112796)(197.80664062,10.02831546)
\curveto(197.90039062,9.59472171)(197.94726562,8.90331546)(197.94726562,7.95409671)
\lineto(191.04785156,7.95409671)
\curveto(191.07714844,6.99901859)(191.30273438,6.23144046)(191.72460938,5.65136234)
\curveto(192.14648438,5.07714359)(192.79980469,4.79003421)(193.68457031,4.79003421)
\curveto(194.51074219,4.79003421)(195.16992188,5.06249515)(195.66210938,5.60741703)
\curveto(195.94335938,5.92382328)(196.14257812,6.29003421)(196.25976562,6.70604984)
\lineto(197.81542969,6.70604984)
\curveto(197.77441406,6.36034671)(197.63671875,5.97362796)(197.40234375,5.54589359)
\curveto(197.17382812,5.12401859)(196.91601562,4.77831546)(196.62890625,4.50878421)
\curveto(196.1484375,4.04003421)(195.55371094,3.72362796)(194.84472656,3.55956546)
\curveto(194.46386719,3.46581546)(194.03320312,3.41894046)(193.55273438,3.41894046)
\curveto(192.38085938,3.41894046)(191.38769531,3.84374515)(190.57324219,4.69335453)
\curveto(189.75878906,5.54882328)(189.3515625,6.74413578)(189.3515625,8.27929203)
\curveto(189.3515625,9.79101078)(189.76171875,11.01854984)(190.58203125,11.96190921)
\curveto(191.40234375,12.90526859)(192.47460938,13.37694828)(193.79882812,13.37694828)
\closepath
\moveto(196.32128906,9.21093265)
\curveto(196.25683594,9.89647953)(196.10742188,10.44433109)(195.87304688,10.85448734)
\curveto(195.43945312,11.61620609)(194.71582031,11.99706546)(193.70214844,11.99706546)
\curveto(192.97558594,11.99706546)(192.36621094,11.73339359)(191.87402344,11.20604984)
\curveto(191.38183594,10.68456546)(191.12109375,10.0195264)(191.09179688,9.21093265)
\lineto(196.32128906,9.21093265)
\closepath
\moveto(193.64941406,13.3945264)
\lineto(193.64941406,13.3945264)
\closepath
}
}
{
\newrgbcolor{curcolor}{0 0 0}
\pscustom[linestyle=none,fillstyle=solid,fillcolor=curcolor]
{
\newpath
\moveto(199.94238281,16.66405765)
\lineto(201.52441406,16.66405765)
\lineto(201.52441406,3.75292484)
\lineto(199.94238281,3.75292484)
\lineto(199.94238281,16.66405765)
\closepath
}
}
{
\newrgbcolor{curcolor}{0 0 0}
\pscustom[linestyle=none,fillstyle=solid,fillcolor=curcolor]
{
\newpath
\moveto(207.82617188,13.37694828)
\curveto(208.49414062,13.37694828)(209.14160156,13.21874515)(209.76855469,12.9023389)
\curveto(210.39550781,12.59179203)(210.87304688,12.18749515)(211.20117188,11.68944828)
\curveto(211.51757812,11.2148389)(211.72851562,10.66112796)(211.83398438,10.02831546)
\curveto(211.92773438,9.59472171)(211.97460938,8.90331546)(211.97460938,7.95409671)
\lineto(205.07519531,7.95409671)
\curveto(205.10449219,6.99901859)(205.33007812,6.23144046)(205.75195312,5.65136234)
\curveto(206.17382812,5.07714359)(206.82714844,4.79003421)(207.71191406,4.79003421)
\curveto(208.53808594,4.79003421)(209.19726562,5.06249515)(209.68945312,5.60741703)
\curveto(209.97070312,5.92382328)(210.16992188,6.29003421)(210.28710938,6.70604984)
\lineto(211.84277344,6.70604984)
\curveto(211.80175781,6.36034671)(211.6640625,5.97362796)(211.4296875,5.54589359)
\curveto(211.20117188,5.12401859)(210.94335938,4.77831546)(210.65625,4.50878421)
\curveto(210.17578125,4.04003421)(209.58105469,3.72362796)(208.87207031,3.55956546)
\curveto(208.49121094,3.46581546)(208.06054688,3.41894046)(207.58007812,3.41894046)
\curveto(206.40820312,3.41894046)(205.41503906,3.84374515)(204.60058594,4.69335453)
\curveto(203.78613281,5.54882328)(203.37890625,6.74413578)(203.37890625,8.27929203)
\curveto(203.37890625,9.79101078)(203.7890625,11.01854984)(204.609375,11.96190921)
\curveto(205.4296875,12.90526859)(206.50195312,13.37694828)(207.82617188,13.37694828)
\closepath
\moveto(210.34863281,9.21093265)
\curveto(210.28417969,9.89647953)(210.13476562,10.44433109)(209.90039062,10.85448734)
\curveto(209.46679688,11.61620609)(208.74316406,11.99706546)(207.72949219,11.99706546)
\curveto(207.00292969,11.99706546)(206.39355469,11.73339359)(205.90136719,11.20604984)
\curveto(205.40917969,10.68456546)(205.1484375,10.0195264)(205.11914062,9.21093265)
\lineto(210.34863281,9.21093265)
\closepath
\moveto(207.67675781,13.3945264)
\lineto(207.67675781,13.3945264)
\closepath
}
}
{
\newrgbcolor{curcolor}{0 0 0}
\pscustom[linestyle=none,fillstyle=solid,fillcolor=curcolor]
{
\newpath
\moveto(213.92578125,13.16601078)
\lineto(215.49023438,13.16601078)
\lineto(215.49023438,11.83007328)
\curveto(215.86523438,12.2929639)(216.20507812,12.62987796)(216.50976562,12.84081546)
\curveto(217.03125,13.19823734)(217.62304688,13.37694828)(218.28515625,13.37694828)
\curveto(219.03515625,13.37694828)(219.63867188,13.19237796)(220.09570312,12.82323734)
\curveto(220.35351562,12.61229984)(220.58789062,12.30175296)(220.79882812,11.89159671)
\curveto(221.15039062,12.39550296)(221.56347656,12.76757328)(222.03808594,13.00780765)
\curveto(222.51269531,13.2539014)(223.04589844,13.37694828)(223.63769531,13.37694828)
\curveto(224.90332031,13.37694828)(225.76464844,12.91991703)(226.22167969,12.00585453)
\curveto(226.46777344,11.51366703)(226.59082031,10.85155765)(226.59082031,10.0195264)
\lineto(226.59082031,3.75292484)
\lineto(224.94726562,3.75292484)
\lineto(224.94726562,10.29198734)
\curveto(224.94726562,10.91894046)(224.7890625,11.34960453)(224.47265625,11.58397953)
\curveto(224.16210938,11.81835453)(223.78125,11.93554203)(223.33007812,11.93554203)
\curveto(222.70898438,11.93554203)(222.17285156,11.72753421)(221.72167969,11.31151859)
\curveto(221.27636719,10.89550296)(221.05371094,10.20116703)(221.05371094,9.22851078)
\lineto(221.05371094,3.75292484)
\lineto(219.4453125,3.75292484)
\lineto(219.4453125,9.89647953)
\curveto(219.4453125,10.5351514)(219.36914062,11.00097171)(219.21679688,11.29394046)
\curveto(218.9765625,11.73339359)(218.52832031,11.95312015)(217.87207031,11.95312015)
\curveto(217.27441406,11.95312015)(216.72949219,11.72167484)(216.23730469,11.25878421)
\curveto(215.75097656,10.79589359)(215.5078125,9.95800296)(215.5078125,8.74511234)
\lineto(215.5078125,3.75292484)
\lineto(213.92578125,3.75292484)
\lineto(213.92578125,13.16601078)
\closepath
}
}
{
\newrgbcolor{curcolor}{0 0 0}
\pscustom[linestyle=none,fillstyle=solid,fillcolor=curcolor]
{
\newpath
\moveto(232.82226562,13.37694828)
\curveto(233.49023438,13.37694828)(234.13769531,13.21874515)(234.76464844,12.9023389)
\curveto(235.39160156,12.59179203)(235.86914062,12.18749515)(236.19726562,11.68944828)
\curveto(236.51367188,11.2148389)(236.72460938,10.66112796)(236.83007812,10.02831546)
\curveto(236.92382812,9.59472171)(236.97070312,8.90331546)(236.97070312,7.95409671)
\lineto(230.07128906,7.95409671)
\curveto(230.10058594,6.99901859)(230.32617188,6.23144046)(230.74804688,5.65136234)
\curveto(231.16992188,5.07714359)(231.82324219,4.79003421)(232.70800781,4.79003421)
\curveto(233.53417969,4.79003421)(234.19335938,5.06249515)(234.68554688,5.60741703)
\curveto(234.96679688,5.92382328)(235.16601562,6.29003421)(235.28320312,6.70604984)
\lineto(236.83886719,6.70604984)
\curveto(236.79785156,6.36034671)(236.66015625,5.97362796)(236.42578125,5.54589359)
\curveto(236.19726562,5.12401859)(235.93945312,4.77831546)(235.65234375,4.50878421)
\curveto(235.171875,4.04003421)(234.57714844,3.72362796)(233.86816406,3.55956546)
\curveto(233.48730469,3.46581546)(233.05664062,3.41894046)(232.57617188,3.41894046)
\curveto(231.40429688,3.41894046)(230.41113281,3.84374515)(229.59667969,4.69335453)
\curveto(228.78222656,5.54882328)(228.375,6.74413578)(228.375,8.27929203)
\curveto(228.375,9.79101078)(228.78515625,11.01854984)(229.60546875,11.96190921)
\curveto(230.42578125,12.90526859)(231.49804688,13.37694828)(232.82226562,13.37694828)
\closepath
\moveto(235.34472656,9.21093265)
\curveto(235.28027344,9.89647953)(235.13085938,10.44433109)(234.89648438,10.85448734)
\curveto(234.46289062,11.61620609)(233.73925781,11.99706546)(232.72558594,11.99706546)
\curveto(231.99902344,11.99706546)(231.38964844,11.73339359)(230.89746094,11.20604984)
\curveto(230.40527344,10.68456546)(230.14453125,10.0195264)(230.11523438,9.21093265)
\lineto(235.34472656,9.21093265)
\closepath
\moveto(232.67285156,13.3945264)
\lineto(232.67285156,13.3945264)
\closepath
}
}
{
\newrgbcolor{curcolor}{0 0 0}
\pscustom[linestyle=none,fillstyle=solid,fillcolor=curcolor]
{
\newpath
\moveto(238.921875,13.16601078)
\lineto(240.42480469,13.16601078)
\lineto(240.42480469,11.83007328)
\curveto(240.87011719,12.38085453)(241.34179688,12.77636234)(241.83984375,13.01659671)
\curveto(242.33789062,13.25683109)(242.89160156,13.37694828)(243.50097656,13.37694828)
\curveto(244.83691406,13.37694828)(245.73925781,12.91112796)(246.20800781,11.97948734)
\curveto(246.46582031,11.46972171)(246.59472656,10.74022953)(246.59472656,9.79101078)
\lineto(246.59472656,3.75292484)
\lineto(244.98632812,3.75292484)
\lineto(244.98632812,9.68554203)
\curveto(244.98632812,10.25976078)(244.90136719,10.7226514)(244.73144531,11.0742139)
\curveto(244.45019531,11.6601514)(243.94042969,11.95312015)(243.20214844,11.95312015)
\curveto(242.82714844,11.95312015)(242.51953125,11.91503421)(242.27929688,11.83886234)
\curveto(241.84570312,11.70995609)(241.46484375,11.45214359)(241.13671875,11.06542484)
\curveto(240.87304688,10.75487796)(240.70019531,10.43261234)(240.61816406,10.09862796)
\curveto(240.54199219,9.77050296)(240.50390625,9.29882328)(240.50390625,8.6835889)
\lineto(240.50390625,3.75292484)
\lineto(238.921875,3.75292484)
\lineto(238.921875,13.16601078)
\closepath
\moveto(242.63964844,13.3945264)
\lineto(242.63964844,13.3945264)
\closepath
}
}
{
\newrgbcolor{curcolor}{0 0 0}
\pscustom[linestyle=none,fillstyle=solid,fillcolor=curcolor]
{
\newpath
\moveto(249.2578125,15.79394046)
\lineto(250.85742188,15.79394046)
\lineto(250.85742188,13.16601078)
\lineto(252.36035156,13.16601078)
\lineto(252.36035156,11.87401859)
\lineto(250.85742188,11.87401859)
\lineto(250.85742188,5.7304639)
\curveto(250.85742188,5.4023389)(250.96875,5.18261234)(251.19140625,5.07128421)
\curveto(251.31445312,5.00683109)(251.51953125,4.97460453)(251.80664062,4.97460453)
\lineto(252.05273438,4.97460453)
\curveto(252.140625,4.9804639)(252.24316406,4.98925296)(252.36035156,5.00097171)
\lineto(252.36035156,3.75292484)
\curveto(252.17871094,3.70019046)(251.98828125,3.66210453)(251.7890625,3.63866703)
\curveto(251.59570312,3.61522953)(251.38476562,3.60351078)(251.15625,3.60351078)
\curveto(250.41796875,3.60351078)(249.91699219,3.79101078)(249.65332031,4.16601078)
\curveto(249.38964844,4.54687015)(249.2578125,5.03905765)(249.2578125,5.64257328)
\lineto(249.2578125,11.87401859)
\lineto(247.98339844,11.87401859)
\lineto(247.98339844,13.16601078)
\lineto(249.2578125,13.16601078)
\lineto(249.2578125,15.79394046)
\closepath
}
}
{
\newrgbcolor{curcolor}{0 0 0}
\pscustom[linewidth=0.79306203,linecolor=curcolor]
{
\newpath
\moveto(10.928138,273.99787206)
\lineto(10.928138,292.86629206)
}
}
{
\newrgbcolor{curcolor}{0 0 0}
\pscustom[linewidth=0.79306203,linecolor=curcolor]
{
\newpath
\moveto(50.5,273.99787206)
\lineto(50.5,292.86629206)
}
}
{
\newrgbcolor{curcolor}{0 0 0}
\pscustom[linewidth=1,linecolor=curcolor]
{
\newpath
\moveto(10.5,283.99787206)
\lineto(50.5,283.99787206)
}
}
{
\newrgbcolor{curcolor}{0 0 0}
\pscustom[linestyle=none,fillstyle=solid,fillcolor=curcolor]
{
\newpath
\moveto(20.5,283.99787206)
\lineto(24.5,287.99787206)
\lineto(10.5,283.99787206)
\lineto(24.5,279.99787206)
\lineto(20.5,283.99787206)
\closepath
}
}
{
\newrgbcolor{curcolor}{0 0 0}
\pscustom[linewidth=1,linecolor=curcolor]
{
\newpath
\moveto(20.5,283.99787206)
\lineto(24.5,287.99787206)
\lineto(10.5,283.99787206)
\lineto(24.5,279.99787206)
\lineto(20.5,283.99787206)
\closepath
}
}
{
\newrgbcolor{curcolor}{0 0 0}
\pscustom[linestyle=none,fillstyle=solid,fillcolor=curcolor]
{
\newpath
\moveto(40.5,283.99787206)
\lineto(36.5,279.99787206)
\lineto(50.5,283.99787206)
\lineto(36.5,287.99787206)
\lineto(40.5,283.99787206)
\closepath
}
}
{
\newrgbcolor{curcolor}{0 0 0}
\pscustom[linewidth=1,linecolor=curcolor]
{
\newpath
\moveto(40.5,283.99787206)
\lineto(36.5,279.99787206)
\lineto(50.5,283.99787206)
\lineto(36.5,287.99787206)
\lineto(40.5,283.99787206)
\closepath
}
}
{
\newrgbcolor{curcolor}{0 0 0}
\pscustom[linewidth=1,linecolor=curcolor]
{
\newpath
\moveto(271.55262,213.99787206)
\lineto(291.55262,213.99787206)
}
}
{
\newrgbcolor{curcolor}{0 0 0}
\pscustom[linewidth=1,linecolor=curcolor]
{
\newpath
\moveto(271.55262,173.99787206)
\lineto(291.55262,173.99787206)
}
}
{
\newrgbcolor{curcolor}{0 0 0}
\pscustom[linewidth=1,linecolor=curcolor]
{
\newpath
\moveto(281.55262,213.99787206)
\lineto(281.55262,173.99787206)
}
}
{
\newrgbcolor{curcolor}{0 0 0}
\pscustom[linestyle=none,fillstyle=solid,fillcolor=curcolor]
{
\newpath
\moveto(281.55262,203.99787206)
\lineto(285.55262,199.99787206)
\lineto(281.55262,213.99787206)
\lineto(277.55262,199.99787206)
\lineto(281.55262,203.99787206)
\closepath
}
}
{
\newrgbcolor{curcolor}{0 0 0}
\pscustom[linewidth=1,linecolor=curcolor]
{
\newpath
\moveto(281.55262,203.99787206)
\lineto(285.55262,199.99787206)
\lineto(281.55262,213.99787206)
\lineto(277.55262,199.99787206)
\lineto(281.55262,203.99787206)
\closepath
}
}
{
\newrgbcolor{curcolor}{0 0 0}
\pscustom[linestyle=none,fillstyle=solid,fillcolor=curcolor]
{
\newpath
\moveto(281.55262,183.99787206)
\lineto(277.55262,187.99787206)
\lineto(281.55262,173.99787206)
\lineto(285.55262,187.99787206)
\lineto(281.55262,183.99787206)
\closepath
}
}
{
\newrgbcolor{curcolor}{0 0 0}
\pscustom[linewidth=1,linecolor=curcolor]
{
\newpath
\moveto(281.55262,183.99787206)
\lineto(277.55262,187.99787206)
\lineto(281.55262,173.99787206)
\lineto(285.55262,187.99787206)
\lineto(281.55262,183.99787206)
\closepath
}
}
{
\newrgbcolor{curcolor}{1 1 1}
\pscustom[linestyle=none,fillstyle=solid,fillcolor=curcolor,opacity=0]
{
\newpath
\moveto(211.11230469,173.754939)
\curveto(211.11230469,130.17981467)(175.59600265,94.85519687)(131.78442383,94.85519687)
\curveto(87.97284501,94.85519687)(52.45654297,130.17981467)(52.45654297,173.754939)
\curveto(52.45654297,217.33006332)(87.97284501,252.65468113)(131.78442383,252.65468113)
\curveto(175.59600265,252.65468113)(211.11230469,217.33006332)(211.11230469,173.754939)
\closepath
}
}
{
\newrgbcolor{curcolor}{0 0 0}
\pscustom[linewidth=1.97783792,linecolor=curcolor,linestyle=dashed,dash=3.95567594 1.97783797]
{
\newpath
\moveto(211.11230469,173.754939)
\curveto(211.11230469,130.17981467)(175.59600265,94.85519687)(131.78442383,94.85519687)
\curveto(87.97284501,94.85519687)(52.45654297,130.17981467)(52.45654297,173.754939)
\curveto(52.45654297,217.33006332)(87.97284501,252.65468113)(131.78442383,252.65468113)
\curveto(175.59600265,252.65468113)(211.11230469,217.33006332)(211.11230469,173.754939)
\closepath
}
}
{
\newrgbcolor{curcolor}{0 0 0}
\pscustom[linestyle=none,fillstyle=solid,fillcolor=curcolor]
{
\newpath
\moveto(261.47973633,253.99787418)
\curveto(261.47973633,248.47502668)(257.00258382,243.99787418)(251.47973633,243.99787418)
\curveto(245.95688883,243.99787418)(241.47973633,248.47502668)(241.47973633,253.99787418)
\curveto(241.47973633,259.52072168)(245.95688883,263.99787418)(251.47973633,263.99787418)
\curveto(257.00258382,263.99787418)(261.47973633,259.52072168)(261.47973633,253.99787418)
\closepath
}
}
{
\newrgbcolor{curcolor}{0 0 0}
\pscustom[linewidth=1,linecolor=curcolor]
{
\newpath
\moveto(261.47973633,253.99787418)
\curveto(261.47973633,248.47502668)(257.00258382,243.99787418)(251.47973633,243.99787418)
\curveto(245.95688883,243.99787418)(241.47973633,248.47502668)(241.47973633,253.99787418)
\curveto(241.47973633,259.52072168)(245.95688883,263.99787418)(251.47973633,263.99787418)
\curveto(257.00258382,263.99787418)(261.47973633,259.52072168)(261.47973633,253.99787418)
\closepath
}
}
{
\newrgbcolor{curcolor}{0 0 0}
\pscustom[linestyle=none,fillstyle=solid,fillcolor=curcolor]
{
\newpath
\moveto(261.47973633,213.99787418)
\curveto(261.47973633,208.47502668)(257.00258382,203.99787418)(251.47973633,203.99787418)
\curveto(245.95688883,203.99787418)(241.47973633,208.47502668)(241.47973633,213.99787418)
\curveto(241.47973633,219.52072168)(245.95688883,223.99787418)(251.47973633,223.99787418)
\curveto(257.00258382,223.99787418)(261.47973633,219.52072168)(261.47973633,213.99787418)
\closepath
}
}
{
\newrgbcolor{curcolor}{0 0 0}
\pscustom[linewidth=1,linecolor=curcolor]
{
\newpath
\moveto(261.47973633,213.99787418)
\curveto(261.47973633,208.47502668)(257.00258382,203.99787418)(251.47973633,203.99787418)
\curveto(245.95688883,203.99787418)(241.47973633,208.47502668)(241.47973633,213.99787418)
\curveto(241.47973633,219.52072168)(245.95688883,223.99787418)(251.47973633,223.99787418)
\curveto(257.00258382,223.99787418)(261.47973633,219.52072168)(261.47973633,213.99787418)
\closepath
}
}
{
\newrgbcolor{curcolor}{0 0 0}
\pscustom[linestyle=none,fillstyle=solid,fillcolor=curcolor]
{
\newpath
\moveto(261.47973633,173.99788944)
\curveto(261.47973633,168.47504194)(257.00258382,163.99788944)(251.47973633,163.99788944)
\curveto(245.95688883,163.99788944)(241.47973633,168.47504194)(241.47973633,173.99788944)
\curveto(241.47973633,179.52073693)(245.95688883,183.99788944)(251.47973633,183.99788944)
\curveto(257.00258382,183.99788944)(261.47973633,179.52073693)(261.47973633,173.99788944)
\closepath
}
}
{
\newrgbcolor{curcolor}{0 0 0}
\pscustom[linewidth=1,linecolor=curcolor]
{
\newpath
\moveto(261.47973633,173.99788944)
\curveto(261.47973633,168.47504194)(257.00258382,163.99788944)(251.47973633,163.99788944)
\curveto(245.95688883,163.99788944)(241.47973633,168.47504194)(241.47973633,173.99788944)
\curveto(241.47973633,179.52073693)(245.95688883,183.99788944)(251.47973633,183.99788944)
\curveto(257.00258382,183.99788944)(261.47973633,179.52073693)(261.47973633,173.99788944)
\closepath
}
}
{
\newrgbcolor{curcolor}{0 0 0}
\pscustom[linestyle=none,fillstyle=solid,fillcolor=curcolor]
{
\newpath
\moveto(261.47973633,133.99788944)
\curveto(261.47973633,128.47504194)(257.00258382,123.99788944)(251.47973633,123.99788944)
\curveto(245.95688883,123.99788944)(241.47973633,128.47504194)(241.47973633,133.99788944)
\curveto(241.47973633,139.52073693)(245.95688883,143.99788944)(251.47973633,143.99788944)
\curveto(257.00258382,143.99788944)(261.47973633,139.52073693)(261.47973633,133.99788944)
\closepath
}
}
{
\newrgbcolor{curcolor}{0 0 0}
\pscustom[linewidth=1,linecolor=curcolor]
{
\newpath
\moveto(261.47973633,133.99788944)
\curveto(261.47973633,128.47504194)(257.00258382,123.99788944)(251.47973633,123.99788944)
\curveto(245.95688883,123.99788944)(241.47973633,128.47504194)(241.47973633,133.99788944)
\curveto(241.47973633,139.52073693)(245.95688883,143.99788944)(251.47973633,143.99788944)
\curveto(257.00258382,143.99788944)(261.47973633,139.52073693)(261.47973633,133.99788944)
\closepath
}
}
{
\newrgbcolor{curcolor}{0 0 0}
\pscustom[linestyle=none,fillstyle=solid,fillcolor=curcolor]
{
\newpath
\moveto(261.47973633,93.99788944)
\curveto(261.47973633,88.47504194)(257.00258382,83.99788944)(251.47973633,83.99788944)
\curveto(245.95688883,83.99788944)(241.47973633,88.47504194)(241.47973633,93.99788944)
\curveto(241.47973633,99.52073693)(245.95688883,103.99788944)(251.47973633,103.99788944)
\curveto(257.00258382,103.99788944)(261.47973633,99.52073693)(261.47973633,93.99788944)
\closepath
}
}
{
\newrgbcolor{curcolor}{0 0 0}
\pscustom[linewidth=1,linecolor=curcolor]
{
\newpath
\moveto(261.47973633,93.99788944)
\curveto(261.47973633,88.47504194)(257.00258382,83.99788944)(251.47973633,83.99788944)
\curveto(245.95688883,83.99788944)(241.47973633,88.47504194)(241.47973633,93.99788944)
\curveto(241.47973633,99.52073693)(245.95688883,103.99788944)(251.47973633,103.99788944)
\curveto(257.00258382,103.99788944)(261.47973633,99.52073693)(261.47973633,93.99788944)
\closepath
}
}
{
\newrgbcolor{curcolor}{0 0 0}
\pscustom[linestyle=none,fillstyle=solid,fillcolor=curcolor]
{
\newpath
\moveto(20.5,53.99788944)
\curveto(20.5,48.47504194)(16.0228475,43.99788944)(10.5,43.99788944)
\curveto(4.9771525,43.99788944)(0.5,48.47504194)(0.5,53.99788944)
\curveto(0.5,59.52073693)(4.9771525,63.99788944)(10.5,63.99788944)
\curveto(16.0228475,63.99788944)(20.5,59.52073693)(20.5,53.99788944)
\closepath
}
}
{
\newrgbcolor{curcolor}{0 0 0}
\pscustom[linewidth=1,linecolor=curcolor]
{
\newpath
\moveto(20.5,53.99788944)
\curveto(20.5,48.47504194)(16.0228475,43.99788944)(10.5,43.99788944)
\curveto(4.9771525,43.99788944)(0.5,48.47504194)(0.5,53.99788944)
\curveto(0.5,59.52073693)(4.9771525,63.99788944)(10.5,63.99788944)
\curveto(16.0228475,63.99788944)(20.5,59.52073693)(20.5,53.99788944)
\closepath
}
}
{
\newrgbcolor{curcolor}{0 0 0}
\pscustom[linestyle=none,fillstyle=solid,fillcolor=curcolor]
{
\newpath
\moveto(61.60121155,53.99788944)
\curveto(61.60121155,48.47504194)(57.12405904,43.99788944)(51.60121155,43.99788944)
\curveto(46.07836405,43.99788944)(41.60121155,48.47504194)(41.60121155,53.99788944)
\curveto(41.60121155,59.52073693)(46.07836405,63.99788944)(51.60121155,63.99788944)
\curveto(57.12405904,63.99788944)(61.60121155,59.52073693)(61.60121155,53.99788944)
\closepath
}
}
{
\newrgbcolor{curcolor}{0 0 0}
\pscustom[linewidth=1,linecolor=curcolor]
{
\newpath
\moveto(61.60121155,53.99788944)
\curveto(61.60121155,48.47504194)(57.12405904,43.99788944)(51.60121155,43.99788944)
\curveto(46.07836405,43.99788944)(41.60121155,48.47504194)(41.60121155,53.99788944)
\curveto(41.60121155,59.52073693)(46.07836405,63.99788944)(51.60121155,63.99788944)
\curveto(57.12405904,63.99788944)(61.60121155,59.52073693)(61.60121155,53.99788944)
\closepath
}
}
{
\newrgbcolor{curcolor}{0 0 0}
\pscustom[linestyle=none,fillstyle=solid,fillcolor=curcolor]
{
\newpath
\moveto(100.98986816,53.14159671)
\curveto(100.98986816,47.61874922)(96.51271566,43.14159671)(90.98986816,43.14159671)
\curveto(85.46702067,43.14159671)(80.98986816,47.61874922)(80.98986816,53.14159671)
\curveto(80.98986816,58.66444421)(85.46702067,63.14159671)(90.98986816,63.14159671)
\curveto(96.51271566,63.14159671)(100.98986816,58.66444421)(100.98986816,53.14159671)
\closepath
}
}
{
\newrgbcolor{curcolor}{0 0 0}
\pscustom[linewidth=1,linecolor=curcolor]
{
\newpath
\moveto(100.98986816,53.14159671)
\curveto(100.98986816,47.61874922)(96.51271566,43.14159671)(90.98986816,43.14159671)
\curveto(85.46702067,43.14159671)(80.98986816,47.61874922)(80.98986816,53.14159671)
\curveto(80.98986816,58.66444421)(85.46702067,63.14159671)(90.98986816,63.14159671)
\curveto(96.51271566,63.14159671)(100.98986816,58.66444421)(100.98986816,53.14159671)
\closepath
}
}
{
\newrgbcolor{curcolor}{0 0 0}
\pscustom[linestyle=none,fillstyle=solid,fillcolor=curcolor]
{
\newpath
\moveto(142.09107971,53.99788944)
\curveto(142.09107971,48.47504194)(137.61392721,43.99788944)(132.09107971,43.99788944)
\curveto(126.56823222,43.99788944)(122.09107971,48.47504194)(122.09107971,53.99788944)
\curveto(122.09107971,59.52073693)(126.56823222,63.99788944)(132.09107971,63.99788944)
\curveto(137.61392721,63.99788944)(142.09107971,59.52073693)(142.09107971,53.99788944)
\closepath
}
}
{
\newrgbcolor{curcolor}{0 0 0}
\pscustom[linewidth=1,linecolor=curcolor]
{
\newpath
\moveto(142.09107971,53.99788944)
\curveto(142.09107971,48.47504194)(137.61392721,43.99788944)(132.09107971,43.99788944)
\curveto(126.56823222,43.99788944)(122.09107971,48.47504194)(122.09107971,53.99788944)
\curveto(122.09107971,59.52073693)(126.56823222,63.99788944)(132.09107971,63.99788944)
\curveto(137.61392721,63.99788944)(142.09107971,59.52073693)(142.09107971,53.99788944)
\closepath
}
}
{
\newrgbcolor{curcolor}{0 0 0}
\pscustom[linestyle=none,fillstyle=solid,fillcolor=curcolor]
{
\newpath
\moveto(181.47973633,53.99788944)
\curveto(181.47973633,48.47504194)(177.00258382,43.99788944)(171.47973633,43.99788944)
\curveto(165.95688883,43.99788944)(161.47973633,48.47504194)(161.47973633,53.99788944)
\curveto(161.47973633,59.52073693)(165.95688883,63.99788944)(171.47973633,63.99788944)
\curveto(177.00258382,63.99788944)(181.47973633,59.52073693)(181.47973633,53.99788944)
\closepath
}
}
{
\newrgbcolor{curcolor}{0 0 0}
\pscustom[linewidth=1,linecolor=curcolor]
{
\newpath
\moveto(181.47973633,53.99788944)
\curveto(181.47973633,48.47504194)(177.00258382,43.99788944)(171.47973633,43.99788944)
\curveto(165.95688883,43.99788944)(161.47973633,48.47504194)(161.47973633,53.99788944)
\curveto(161.47973633,59.52073693)(165.95688883,63.99788944)(171.47973633,63.99788944)
\curveto(177.00258382,63.99788944)(181.47973633,59.52073693)(181.47973633,53.99788944)
\closepath
}
}
{
\newrgbcolor{curcolor}{0 0 0}
\pscustom[linestyle=none,fillstyle=solid,fillcolor=curcolor]
{
\newpath
\moveto(221.47973633,53.99788944)
\curveto(221.47973633,48.47504194)(217.00258382,43.99788944)(211.47973633,43.99788944)
\curveto(205.95688883,43.99788944)(201.47973633,48.47504194)(201.47973633,53.99788944)
\curveto(201.47973633,59.52073693)(205.95688883,63.99788944)(211.47973633,63.99788944)
\curveto(217.00258382,63.99788944)(221.47973633,59.52073693)(221.47973633,53.99788944)
\closepath
}
}
{
\newrgbcolor{curcolor}{0 0 0}
\pscustom[linewidth=1,linecolor=curcolor]
{
\newpath
\moveto(221.47973633,53.99788944)
\curveto(221.47973633,48.47504194)(217.00258382,43.99788944)(211.47973633,43.99788944)
\curveto(205.95688883,43.99788944)(201.47973633,48.47504194)(201.47973633,53.99788944)
\curveto(201.47973633,59.52073693)(205.95688883,63.99788944)(211.47973633,63.99788944)
\curveto(217.00258382,63.99788944)(221.47973633,59.52073693)(221.47973633,53.99788944)
\closepath
}
}
{
\newrgbcolor{curcolor}{0 0 0}
\pscustom[linestyle=none,fillstyle=solid,fillcolor=curcolor]
{
\newpath
\moveto(261.47973633,53.99788944)
\curveto(261.47973633,48.47504194)(257.00258382,43.99788944)(251.47973633,43.99788944)
\curveto(245.95688883,43.99788944)(241.47973633,48.47504194)(241.47973633,53.99788944)
\curveto(241.47973633,59.52073693)(245.95688883,63.99788944)(251.47973633,63.99788944)
\curveto(257.00258382,63.99788944)(261.47973633,59.52073693)(261.47973633,53.99788944)
\closepath
}
}
{
\newrgbcolor{curcolor}{0 0 0}
\pscustom[linewidth=1,linecolor=curcolor]
{
\newpath
\moveto(261.47973633,53.99788944)
\curveto(261.47973633,48.47504194)(257.00258382,43.99788944)(251.47973633,43.99788944)
\curveto(245.95688883,43.99788944)(241.47973633,48.47504194)(241.47973633,53.99788944)
\curveto(241.47973633,59.52073693)(245.95688883,63.99788944)(251.47973633,63.99788944)
\curveto(257.00258382,63.99788944)(261.47973633,59.52073693)(261.47973633,53.99788944)
\closepath
}
}
{
\newrgbcolor{curcolor}{0 0 0}
\pscustom[linewidth=1,linecolor=curcolor]
{
\newpath
\moveto(150.5,93.99787206)
\lineto(150.5,23.99787206)
}
}
{
\newrgbcolor{curcolor}{0 0 0}
\pscustom[linestyle=none,fillstyle=solid,fillcolor=curcolor]
{
\newpath
\moveto(150.5,83.99787206)
\lineto(154.5,79.99787206)
\lineto(150.5,93.99787206)
\lineto(146.5,79.99787206)
\lineto(150.5,83.99787206)
\closepath
}
}
{
\newrgbcolor{curcolor}{0 0 0}
\pscustom[linewidth=1,linecolor=curcolor]
{
\newpath
\moveto(150.5,83.99787206)
\lineto(154.5,79.99787206)
\lineto(150.5,93.99787206)
\lineto(146.5,79.99787206)
\lineto(150.5,83.99787206)
\closepath
}
}
\end{pspicture}

    \caption{Measurements and Resolution}
  \end{figure}
\end{frame}

\section{First Layer}
\begin{frame}
  \frametitle{First Layer}
  Depending on the type of printer and its settings, the first layer printed may be thinner or thicker that the subsequent layers.  This is usually accompanied with the material extending beyond or within the subsequent layers.  This effect is sometimes called ``elephant foot''.  The slicer program often has a setting for ``elephant foot compensation''.
\end{frame}
\begin{frame}
  \frametitle{First Layer}
  \begin{figure}
    %LaTeX with PSTricks extensions
%%Creator: inkscape 0.92.2
%%Please note this file requires PSTricks extensions
\psset{xunit=.5pt,yunit=.5pt,runit=.5pt}
\begin{pspicture}(500.00001922,230.3632601)
{
\newrgbcolor{curcolor}{0.40000001 0.40000001 0.40000001}
\pscustom[linestyle=none,fillstyle=solid,fillcolor=curcolor]
{
\newpath
\moveto(0.00000006,10.00000341)
\lineto(500.00001928,10.00000341)
\lineto(500.00001928,0.00000371)
\lineto(0.00000006,0.00000371)
\closepath
}
}
{
\newrgbcolor{curcolor}{0 0 0}
\pscustom[linestyle=none,fillstyle=solid,fillcolor=curcolor]
{
\newpath
\moveto(64.15107667,191.52872306)
\lineto(73.107131,191.52872306)
\lineto(73.107131,189.94669187)
\lineto(65.90010003,189.94669187)
\lineto(65.90010003,186.02677015)
\lineto(72.23701384,186.02677015)
\lineto(72.23701384,184.48868428)
\lineto(65.90010003,184.48868428)
\lineto(65.90010003,178.61759076)
\lineto(64.15107667,178.61759076)
\closepath
}
}
{
\newrgbcolor{curcolor}{0 0 0}
\pscustom[linestyle=none,fillstyle=solid,fillcolor=curcolor]
{
\newpath
\moveto(74.77705306,187.98673101)
\lineto(76.38545144,187.98673101)
\lineto(76.38545144,178.61759076)
\lineto(74.77705306,178.61759076)
\closepath
\moveto(74.77705306,191.52872306)
\lineto(76.38545144,191.52872306)
\lineto(76.38545144,189.73575438)
\lineto(74.77705306,189.73575438)
\closepath
}
}
{
\newrgbcolor{curcolor}{0 0 0}
\pscustom[linestyle=none,fillstyle=solid,fillcolor=curcolor]
{
\newpath
\moveto(78.82881033,188.03067632)
\lineto(80.33173996,188.03067632)
\lineto(80.33173996,186.40469983)
\curveto(80.45478683,186.72110606)(80.75654463,187.10489511)(81.23701336,187.55606697)
\curveto(81.71748209,188.0130982)(82.27119301,188.24161382)(82.89814611,188.24161382)
\curveto(82.92744298,188.24161382)(82.97724767,188.23868413)(83.04756017,188.23282475)
\curveto(83.11787266,188.22696538)(83.23798984,188.21524663)(83.40791171,188.1976685)
\lineto(83.40791171,186.5277467)
\curveto(83.31416172,186.54532482)(83.2262711,186.55704357)(83.14423985,186.56290295)
\curveto(83.06806798,186.56876232)(82.98310704,186.57169201)(82.88935705,186.57169201)
\curveto(82.09248208,186.57169201)(81.48017742,186.31387952)(81.05244306,185.79825454)
\curveto(80.6247087,185.28848893)(80.41084152,184.69962177)(80.41084152,184.03165305)
\lineto(80.41084152,178.61759076)
\lineto(78.82881033,178.61759076)
\closepath
}
}
{
\newrgbcolor{curcolor}{0 0 0}
\pscustom[linestyle=none,fillstyle=solid,fillcolor=curcolor]
{
\newpath
\moveto(85.73701306,181.57071565)
\curveto(85.78388806,181.04337192)(85.91572399,180.63907506)(86.13252086,180.35782507)
\curveto(86.53095834,179.84805946)(87.22236457,179.59317666)(88.20673953,179.59317666)
\curveto(88.792677,179.59317666)(89.30830198,179.71915322)(89.75361446,179.97110633)
\curveto(90.19892695,180.22891882)(90.42158319,180.62442662)(90.42158319,181.15762972)
\curveto(90.42158319,181.56192658)(90.24287226,181.86954376)(89.8854504,182.08048125)
\curveto(89.65693478,182.20938749)(89.20576292,182.35880155)(88.53193483,182.52872342)
\lineto(87.27509894,182.84512966)
\curveto(86.4723646,183.0443484)(85.88056774,183.26700464)(85.49970838,183.51309838)
\curveto(84.82002091,183.94083274)(84.48017718,184.53262959)(84.48017718,185.28848893)
\curveto(84.48017718,186.1791139)(84.7995131,186.89981699)(85.43818495,187.45059822)
\curveto(86.08271617,188.00137945)(86.94697395,188.27677006)(88.03095828,188.27677006)
\curveto(89.44892698,188.27677006)(90.47138787,187.86075446)(91.09834097,187.02872324)
\curveto(91.49091908,186.50137951)(91.68134876,185.93302016)(91.66963001,185.32364518)
\lineto(90.17548945,185.32364518)
\curveto(90.14619257,185.68106704)(90.02021602,186.00626234)(89.79755978,186.29923108)
\curveto(89.43427854,186.71524669)(88.80439575,186.92325449)(87.90791141,186.92325449)
\curveto(87.31025519,186.92325449)(86.85615364,186.80899669)(86.54560678,186.58048107)
\curveto(86.24091929,186.35196545)(86.08857555,186.05020765)(86.08857555,185.67520767)
\curveto(86.08857555,185.26505143)(86.29072398,184.93692645)(86.69502084,184.69083271)
\curveto(86.92939583,184.54434834)(87.27509894,184.41544209)(87.73213017,184.30411397)
\lineto(88.77802857,184.04923117)
\curveto(89.91474727,183.77384056)(90.67646599,183.50723901)(91.06318472,183.24942652)
\curveto(91.67841907,182.84512966)(91.98603625,182.20938749)(91.98603625,181.34220003)
\curveto(91.98603625,180.50430944)(91.66670033,179.78067665)(91.02802848,179.17130168)
\curveto(90.395216,178.5619267)(89.42841916,178.25723922)(88.12763797,178.25723922)
\curveto(86.7272474,178.25723922)(85.73408338,178.57364545)(85.1481459,179.20645793)
\curveto(84.5680678,179.84512978)(84.25752093,180.63321568)(84.21650531,181.57071565)
\closepath
\moveto(88.07490359,188.25919194)
\closepath
}
}
{
\newrgbcolor{curcolor}{0 0 0}
\pscustom[linestyle=none,fillstyle=solid,fillcolor=curcolor]
{
\newpath
\moveto(94.11298929,190.65860591)
\lineto(95.7125986,190.65860591)
\lineto(95.7125986,188.03067632)
\lineto(97.21552823,188.03067632)
\lineto(97.21552823,186.73868419)
\lineto(95.7125986,186.73868419)
\lineto(95.7125986,180.59512975)
\curveto(95.7125986,180.26700476)(95.82392672,180.04727821)(96.04658296,179.93595009)
\curveto(96.16962983,179.87149696)(96.37470795,179.8392704)(96.66181731,179.8392704)
\lineto(96.90791105,179.8392704)
\curveto(96.99580167,179.84512978)(97.09834073,179.85391884)(97.21552823,179.86563759)
\lineto(97.21552823,178.61759076)
\curveto(97.03388761,178.56485639)(96.84345793,178.52677045)(96.64423919,178.50333296)
\curveto(96.45087982,178.47989546)(96.23994233,178.46817671)(96.01142671,178.46817671)
\curveto(95.27314549,178.46817671)(94.77216895,178.6556767)(94.50849709,179.03067668)
\curveto(94.24482522,179.41153604)(94.11298929,179.90372352)(94.11298929,180.50723913)
\lineto(94.11298929,186.73868419)
\lineto(92.83857528,186.73868419)
\lineto(92.83857528,188.03067632)
\lineto(94.11298929,188.03067632)
\closepath
}
}
{
\newrgbcolor{curcolor}{0 0 0}
\pscustom[linestyle=none,fillstyle=solid,fillcolor=curcolor]
{
\newpath
\moveto(103.82490192,191.52872306)
\lineto(105.40693311,191.52872306)
\lineto(105.40693311,178.61759076)
\lineto(103.82490192,178.61759076)
\closepath
}
}
{
\newrgbcolor{curcolor}{0 0 0}
\pscustom[linestyle=none,fillstyle=solid,fillcolor=curcolor]
{
\newpath
\moveto(109.00166185,181.12247348)
\curveto(109.00166185,180.66544224)(109.16865403,180.3050907)(109.50263839,180.04141883)
\curveto(109.83662275,179.77774697)(110.23213055,179.64591103)(110.68916178,179.64591103)
\curveto(111.24580238,179.64591103)(111.78486486,179.77481728)(112.30634922,180.03262977)
\curveto(113.18525543,180.46036413)(113.62470854,181.16055941)(113.62470854,182.13321562)
\lineto(113.62470854,183.40762963)
\curveto(113.43134917,183.28458276)(113.18232574,183.18204371)(112.87763826,183.10001246)
\curveto(112.57295077,183.01798121)(112.27412266,182.95938746)(111.98115392,182.92423122)
\lineto(111.02314614,182.80118435)
\curveto(110.44892742,182.72501247)(110.01826337,182.60489529)(109.73115401,182.4408328)
\curveto(109.2448259,182.16544218)(109.00166185,181.72598908)(109.00166185,181.12247348)
\closepath
\moveto(112.83369295,184.3216921)
\curveto(113.19697418,184.3685671)(113.44013823,184.52091084)(113.5631851,184.77872333)
\curveto(113.6334976,184.91934832)(113.66865385,185.12149675)(113.66865385,185.38516862)
\curveto(113.66865385,185.9242311)(113.47529448,186.31387952)(113.08857575,186.55411388)
\curveto(112.70771639,186.80020762)(112.15986485,186.92325449)(111.44502113,186.92325449)
\curveto(110.61884928,186.92325449)(110.03291181,186.70059825)(109.6872087,186.25528577)
\curveto(109.49384933,186.00919203)(109.36787277,185.64298111)(109.30927902,185.156653)
\lineto(107.83271658,185.156653)
\curveto(107.86201346,186.31680921)(108.23701344,187.12247324)(108.95771654,187.57364509)
\curveto(109.68427901,188.03067632)(110.52509929,188.25919194)(111.48017738,188.25919194)
\curveto(112.58759921,188.25919194)(113.48701323,188.04825445)(114.17841945,187.62637947)
\curveto(114.8639663,187.20450448)(115.20673973,186.54825451)(115.20673973,185.65762954)
\lineto(115.20673973,180.2347782)
\curveto(115.20673973,180.07071571)(115.23896629,179.93887977)(115.30341941,179.8392704)
\curveto(115.37373191,179.73966103)(115.51728659,179.68985635)(115.73408346,179.68985635)
\curveto(115.80439595,179.68985635)(115.88349751,179.69278603)(115.97138813,179.69864541)
\curveto(116.05927875,179.71036416)(116.15302875,179.72501259)(116.25263812,179.74259072)
\lineto(116.25263812,178.57364545)
\curveto(116.00654438,178.50333296)(115.81904439,178.45938764)(115.69013814,178.44180952)
\curveto(115.5612319,178.4242314)(115.38545066,178.41544233)(115.16279442,178.41544233)
\curveto(114.61787256,178.41544233)(114.22236477,178.6088017)(113.97627103,178.99552044)
\curveto(113.84736478,179.20059855)(113.75654447,179.4906376)(113.7038101,179.86563759)
\curveto(113.38154449,179.44376261)(112.91865388,179.07755168)(112.31513828,178.76700482)
\curveto(111.71162268,178.45645796)(111.04658364,178.30118453)(110.32002117,178.30118453)
\curveto(109.44697433,178.30118453)(108.73213061,178.56485639)(108.17549001,179.09220012)
\curveto(107.62470878,179.62540322)(107.34931817,180.29044226)(107.34931817,181.08731723)
\curveto(107.34931817,181.96036407)(107.62177909,182.63712185)(108.16670095,183.11759058)
\curveto(108.7116228,183.59805931)(109.42646652,183.89395774)(110.31123211,184.00528586)
\closepath
\moveto(111.52412269,188.25919194)
\closepath
}
}
{
\newrgbcolor{curcolor}{0 0 0}
\pscustom[linestyle=none,fillstyle=solid,fillcolor=curcolor]
{
\newpath
\moveto(123.6881851,188.03067632)
\lineto(125.43720847,188.03067632)
\curveto(125.21455223,187.42716072)(124.71943506,186.05020765)(123.95185696,183.89981711)
\curveto(123.37763824,182.28262968)(122.89716951,180.96427036)(122.51045077,179.94473915)
\curveto(121.59638831,177.54239549)(120.95185708,176.0775518)(120.5768571,175.55020807)
\curveto(120.20185711,175.02286434)(119.55732589,174.75919248)(118.64326343,174.75919248)
\curveto(118.42060719,174.75919248)(118.24775563,174.76798154)(118.12470876,174.78555967)
\curveto(118.00752127,174.80313779)(117.8610369,174.83536435)(117.68525565,174.88223935)
\lineto(117.68525565,176.32364554)
\curveto(117.96064627,176.24747367)(118.15986501,176.20059867)(118.28291188,176.18302055)
\curveto(118.40595875,176.16544242)(118.51435718,176.15665336)(118.60810718,176.15665336)
\curveto(118.90107592,176.15665336)(119.1149431,176.20645805)(119.24970872,176.30606742)
\curveto(119.39033371,176.39981741)(119.50752121,176.51700491)(119.6012712,176.6576299)
\curveto(119.63056808,176.7045049)(119.73603682,176.94473927)(119.91767744,177.378333)
\curveto(120.09931806,177.81192673)(120.23115399,178.13419235)(120.31318524,178.34512984)
\lineto(116.83271662,188.03067632)
\lineto(118.6256853,188.03067632)
\lineto(121.14814614,180.36661413)
\closepath
\moveto(121.13935708,188.25919194)
\closepath
}
}
{
\newrgbcolor{curcolor}{0 0 0}
\pscustom[linestyle=none,fillstyle=solid,fillcolor=curcolor]
{
\newpath
\moveto(130.72822208,188.24161382)
\curveto(131.3961908,188.24161382)(132.04365171,188.0834107)(132.67060481,187.76700446)
\curveto(133.29755791,187.4564576)(133.77509696,187.05216074)(134.10322194,186.55411388)
\curveto(134.41962818,186.07950453)(134.63056567,185.52579361)(134.73603442,184.89298114)
\curveto(134.82978441,184.4593874)(134.87665941,183.76798118)(134.87665941,182.81876247)
\lineto(127.97724563,182.81876247)
\curveto(128.0065425,181.86368438)(128.23212843,181.09610629)(128.65400341,180.51602819)
\curveto(129.07587839,179.94180946)(129.72919868,179.6547001)(130.61396427,179.6547001)
\curveto(131.44013611,179.6547001)(132.09931577,179.92716102)(132.59150325,180.47208288)
\curveto(132.87275324,180.78848911)(133.07197198,181.15470004)(133.18915948,181.57071565)
\lineto(134.74482348,181.57071565)
\curveto(134.70380786,181.22501253)(134.56611255,180.8382938)(134.33173756,180.41055944)
\curveto(134.10322194,179.98868446)(133.84540945,179.64298135)(133.55830009,179.37345011)
\curveto(133.07783136,178.90470013)(132.48310482,178.58829389)(131.77412047,178.4242314)
\curveto(131.39326111,178.3304814)(130.96259707,178.2836064)(130.48212834,178.2836064)
\curveto(129.31025339,178.2836064)(128.31708936,178.70841107)(127.50263627,179.55802041)
\curveto(126.68818318,180.41348913)(126.28095663,181.60880158)(126.28095663,183.14395777)
\curveto(126.28095663,184.65567646)(126.69111287,185.88321547)(127.51142533,186.82657481)
\curveto(128.3317378,187.76993415)(129.40400338,188.24161382)(130.72822208,188.24161382)
\closepath
\moveto(133.25068292,184.07559836)
\curveto(133.18622979,184.7611452)(133.03681574,185.30899675)(132.80244075,185.71915298)
\curveto(132.36884701,186.4808717)(131.64521423,186.86173106)(130.63154239,186.86173106)
\curveto(129.90497992,186.86173106)(129.29560495,186.59805919)(128.80341747,186.07071546)
\curveto(128.31122999,185.54923111)(128.05048781,184.88419207)(128.02119094,184.07559836)
\closepath
\moveto(130.57880802,188.25919194)
\closepath
}
}
{
\newrgbcolor{curcolor}{0 0 0}
\pscustom[linestyle=none,fillstyle=solid,fillcolor=curcolor]
{
\newpath
\moveto(136.87177692,188.03067632)
\lineto(138.37470655,188.03067632)
\lineto(138.37470655,186.40469983)
\curveto(138.49775342,186.72110606)(138.79951122,187.10489511)(139.27997995,187.55606697)
\curveto(139.76044868,188.0130982)(140.3141596,188.24161382)(140.9411127,188.24161382)
\curveto(140.97040957,188.24161382)(141.02021426,188.23868413)(141.09052675,188.23282475)
\curveto(141.16083925,188.22696538)(141.28095643,188.21524663)(141.4508783,188.1976685)
\lineto(141.4508783,186.5277467)
\curveto(141.3571283,186.54532482)(141.26923768,186.55704357)(141.18720644,186.56290295)
\curveto(141.11103456,186.56876232)(141.02607363,186.57169201)(140.93232363,186.57169201)
\curveto(140.13544867,186.57169201)(139.523144,186.31387952)(139.09540964,185.79825454)
\curveto(138.66767529,185.28848893)(138.45380811,184.69962177)(138.45380811,184.03165305)
\lineto(138.45380811,178.61759076)
\lineto(136.87177692,178.61759076)
\closepath
}
}
{
\newrgbcolor{curcolor}{0 0 0}
\pscustom[linestyle=none,fillstyle=solid,fillcolor=curcolor]
{
\newpath
\moveto(148.14814121,190.65860591)
\lineto(149.74775053,190.65860591)
\lineto(149.74775053,188.03067632)
\lineto(151.25068015,188.03067632)
\lineto(151.25068015,186.73868419)
\lineto(149.74775053,186.73868419)
\lineto(149.74775053,180.59512975)
\curveto(149.74775053,180.26700476)(149.85907865,180.04727821)(150.08173489,179.93595009)
\curveto(150.20478176,179.87149696)(150.40985987,179.8392704)(150.69696924,179.8392704)
\lineto(150.94306298,179.8392704)
\curveto(151.0309536,179.84512978)(151.13349266,179.85391884)(151.25068015,179.86563759)
\lineto(151.25068015,178.61759076)
\curveto(151.06903953,178.56485639)(150.87860986,178.52677045)(150.67939111,178.50333296)
\curveto(150.48603175,178.47989546)(150.27509425,178.46817671)(150.04657864,178.46817671)
\curveto(149.30829742,178.46817671)(148.80732088,178.6556767)(148.54364901,179.03067668)
\curveto(148.27997715,179.41153604)(148.14814121,179.90372352)(148.14814121,180.50723913)
\lineto(148.14814121,186.73868419)
\lineto(146.8737272,186.73868419)
\lineto(146.8737272,188.03067632)
\lineto(148.14814121,188.03067632)
\closepath
}
}
{
\newrgbcolor{curcolor}{0 0 0}
\pscustom[linestyle=none,fillstyle=solid,fillcolor=curcolor]
{
\newpath
\moveto(156.55927677,179.63712197)
\curveto(157.60810486,179.63712197)(158.32587827,180.03262977)(158.712597,180.82364536)
\curveto(159.10517511,181.62052033)(159.30146416,182.50528592)(159.30146416,183.47794213)
\curveto(159.30146416,184.35684835)(159.16083917,185.07169207)(158.87958918,185.6224733)
\curveto(158.4342767,186.48966076)(157.6666986,186.92325449)(156.5768549,186.92325449)
\curveto(155.61005806,186.92325449)(154.90693309,186.55411388)(154.46747998,185.81583266)
\curveto(154.02802687,185.07755144)(153.80830032,184.18692648)(153.80830032,183.14395777)
\curveto(153.80830032,182.14200468)(154.02802687,181.30704378)(154.46747998,180.63907506)
\curveto(154.90693309,179.97110633)(155.60419869,179.63712197)(156.55927677,179.63712197)
\closepath
\moveto(156.62080021,188.30313725)
\curveto(157.83369078,188.30313725)(158.85908137,187.89884039)(159.69697196,187.09024667)
\curveto(160.53486255,186.28165296)(160.95380785,185.09219988)(160.95380785,183.52188744)
\curveto(160.95380785,182.00430938)(160.58466724,180.75040318)(159.84638602,179.76016884)
\curveto(159.1081048,178.76993451)(157.96259703,178.27481734)(156.40986272,178.27481734)
\curveto(155.11494089,178.27481734)(154.08662062,178.71134076)(153.3249019,179.5843876)
\curveto(152.56318318,180.46329381)(152.18232382,181.64102814)(152.18232382,183.11759058)
\curveto(152.18232382,184.69962177)(152.583691,185.95938734)(153.38642534,186.89688731)
\curveto(154.18915968,187.83438727)(155.26728464,188.30313725)(156.62080021,188.30313725)
\closepath
\moveto(156.56806584,188.25919194)
\closepath
}
}
{
\newrgbcolor{curcolor}{0 0 0}
\pscustom[linestyle=none,fillstyle=solid,fillcolor=curcolor]
{
\newpath
\moveto(166.57880802,179.63712197)
\curveto(167.62763611,179.63712197)(168.34540952,180.03262977)(168.73212825,180.82364536)
\curveto(169.12470636,181.62052033)(169.32099541,182.50528592)(169.32099541,183.47794213)
\curveto(169.32099541,184.35684835)(169.18037042,185.07169207)(168.89912043,185.6224733)
\curveto(168.45380795,186.48966076)(167.68622985,186.92325449)(166.59638615,186.92325449)
\curveto(165.62958931,186.92325449)(164.92646434,186.55411388)(164.48701123,185.81583266)
\curveto(164.04755812,185.07755144)(163.82783157,184.18692648)(163.82783157,183.14395777)
\curveto(163.82783157,182.14200468)(164.04755812,181.30704378)(164.48701123,180.63907506)
\curveto(164.92646434,179.97110633)(165.62372994,179.63712197)(166.57880802,179.63712197)
\closepath
\moveto(166.64033146,188.30313725)
\curveto(167.85322203,188.30313725)(168.87861262,187.89884039)(169.71650321,187.09024667)
\curveto(170.5543938,186.28165296)(170.9733391,185.09219988)(170.9733391,183.52188744)
\curveto(170.9733391,182.00430938)(170.60419849,180.75040318)(169.86591727,179.76016884)
\curveto(169.12763605,178.76993451)(167.98212828,178.27481734)(166.42939397,178.27481734)
\curveto(165.13447214,178.27481734)(164.10615187,178.71134076)(163.34443315,179.5843876)
\curveto(162.58271443,180.46329381)(162.20185507,181.64102814)(162.20185507,183.11759058)
\curveto(162.20185507,184.69962177)(162.60322225,185.95938734)(163.40595659,186.89688731)
\curveto(164.20869093,187.83438727)(165.28681589,188.30313725)(166.64033146,188.30313725)
\closepath
\moveto(166.58759709,188.25919194)
\closepath
}
}
{
\newrgbcolor{curcolor}{0 0 0}
\pscustom[linestyle=none,fillstyle=solid,fillcolor=curcolor]
{
\newpath
\moveto(178.17158448,190.65860591)
\lineto(179.77119379,190.65860591)
\lineto(179.77119379,188.03067632)
\lineto(181.27412342,188.03067632)
\lineto(181.27412342,186.73868419)
\lineto(179.77119379,186.73868419)
\lineto(179.77119379,180.59512975)
\curveto(179.77119379,180.26700476)(179.88252191,180.04727821)(180.10517815,179.93595009)
\curveto(180.22822502,179.87149696)(180.43330314,179.8392704)(180.7204125,179.8392704)
\lineto(180.96650624,179.8392704)
\curveto(181.05439687,179.84512978)(181.15693592,179.85391884)(181.27412342,179.86563759)
\lineto(181.27412342,178.61759076)
\curveto(181.0924828,178.56485639)(180.90205312,178.52677045)(180.70283438,178.50333296)
\curveto(180.50947501,178.47989546)(180.29853752,178.46817671)(180.07002191,178.46817671)
\curveto(179.33174069,178.46817671)(178.83076414,178.6556767)(178.56709228,179.03067668)
\curveto(178.30342041,179.41153604)(178.17158448,179.90372352)(178.17158448,180.50723913)
\lineto(178.17158448,186.73868419)
\lineto(176.89717047,186.73868419)
\lineto(176.89717047,188.03067632)
\lineto(178.17158448,188.03067632)
\closepath
}
}
{
\newrgbcolor{curcolor}{0 0 0}
\pscustom[linestyle=none,fillstyle=solid,fillcolor=curcolor]
{
\newpath
\moveto(182.84736142,191.57266837)
\lineto(184.42939261,191.57266837)
\lineto(184.42939261,186.75626231)
\curveto(184.80439259,187.23087167)(185.14130664,187.56485603)(185.44013475,187.7582154)
\curveto(185.94990036,188.09219976)(186.58564252,188.25919194)(187.34736124,188.25919194)
\curveto(188.71259556,188.25919194)(189.63837677,187.7816529)(190.12470488,186.82657481)
\curveto(190.38837674,186.30509046)(190.52021267,185.58145767)(190.52021267,184.65567646)
\lineto(190.52021267,178.61759076)
\lineto(188.89423618,178.61759076)
\lineto(188.89423618,184.55020771)
\curveto(188.89423618,185.24161394)(188.80634556,185.74844985)(188.63056431,186.07071546)
\curveto(188.34345495,186.58634044)(187.80439247,186.84415293)(187.01337688,186.84415293)
\curveto(186.3571269,186.84415293)(185.76240036,186.61856701)(185.22919726,186.16739515)
\curveto(184.69599416,185.71622329)(184.42939261,184.86368426)(184.42939261,183.60977806)
\lineto(184.42939261,178.61759076)
\lineto(182.84736142,178.61759076)
\closepath
}
}
{
\newrgbcolor{curcolor}{0 0 0}
\pscustom[linestyle=none,fillstyle=solid,fillcolor=curcolor]
{
\newpath
\moveto(192.86689988,187.98673101)
\lineto(194.47529825,187.98673101)
\lineto(194.47529825,178.61759076)
\lineto(192.86689988,178.61759076)
\closepath
\moveto(192.86689988,191.52872306)
\lineto(194.47529825,191.52872306)
\lineto(194.47529825,189.73575438)
\lineto(192.86689988,189.73575438)
\closepath
}
}
{
\newrgbcolor{curcolor}{0 0 0}
\pscustom[linestyle=none,fillstyle=solid,fillcolor=curcolor]
{
\newpath
\moveto(196.87470373,188.03067632)
\lineto(198.37763335,188.03067632)
\lineto(198.37763335,186.69473888)
\curveto(198.82294584,187.24552011)(199.29462551,187.6410279)(199.79267236,187.88126227)
\curveto(200.29071922,188.12149663)(200.84443013,188.24161382)(201.45380511,188.24161382)
\curveto(202.78974255,188.24161382)(203.69208627,187.77579352)(204.16083625,186.84415293)
\curveto(204.41864874,186.33438733)(204.54755498,185.60489517)(204.54755498,184.65567646)
\lineto(204.54755498,178.61759076)
\lineto(202.93915661,178.61759076)
\lineto(202.93915661,184.55020771)
\curveto(202.93915661,185.12442644)(202.85419568,185.58731705)(202.68427381,185.93887953)
\curveto(202.40302382,186.52481701)(201.89325821,186.81778575)(201.15497699,186.81778575)
\curveto(200.77997701,186.81778575)(200.47235983,186.77969981)(200.23212547,186.70352794)
\curveto(199.79853173,186.57462169)(199.41767238,186.31680921)(199.08954739,185.93009047)
\curveto(198.82587552,185.61954361)(198.65302397,185.297278)(198.57099272,184.96329363)
\curveto(198.49482085,184.63516865)(198.45673491,184.16348898)(198.45673491,183.54825463)
\lineto(198.45673491,178.61759076)
\lineto(196.87470373,178.61759076)
\closepath
\moveto(200.59247702,188.25919194)
\closepath
}
}
{
\newrgbcolor{curcolor}{0 0 0}
\pscustom[linestyle=none,fillstyle=solid,fillcolor=curcolor]
{
\newpath
\moveto(316.91587072,230.31931384)
\lineto(325.87192505,230.31931384)
\lineto(325.87192505,228.73728265)
\lineto(318.66489409,228.73728265)
\lineto(318.66489409,224.81736094)
\lineto(325.0018079,224.81736094)
\lineto(325.0018079,223.27927506)
\lineto(318.66489409,223.27927506)
\lineto(318.66489409,217.40818155)
\lineto(316.91587072,217.40818155)
\closepath
}
}
{
\newrgbcolor{curcolor}{0 0 0}
\pscustom[linestyle=none,fillstyle=solid,fillcolor=curcolor]
{
\newpath
\moveto(327.54184712,226.7773218)
\lineto(329.15024549,226.7773218)
\lineto(329.15024549,217.40818155)
\lineto(327.54184712,217.40818155)
\closepath
\moveto(327.54184712,230.31931384)
\lineto(329.15024549,230.31931384)
\lineto(329.15024549,228.52634516)
\lineto(327.54184712,228.52634516)
\closepath
}
}
{
\newrgbcolor{curcolor}{0 0 0}
\pscustom[linestyle=none,fillstyle=solid,fillcolor=curcolor]
{
\newpath
\moveto(331.59360439,226.82126711)
\lineto(333.09653402,226.82126711)
\lineto(333.09653402,225.19529061)
\curveto(333.21958089,225.51169685)(333.52133869,225.89548589)(334.00180742,226.34665775)
\curveto(334.48227615,226.80368898)(335.03598707,227.0322046)(335.66294017,227.0322046)
\curveto(335.69223704,227.0322046)(335.74204172,227.02927491)(335.81235422,227.02341554)
\curveto(335.88266672,227.01755616)(336.0027839,227.00583741)(336.17270577,226.98825929)
\lineto(336.17270577,225.31833748)
\curveto(336.07895577,225.3359156)(335.99106515,225.34763435)(335.90903391,225.35349373)
\curveto(335.83286203,225.3593531)(335.7479011,225.36228279)(335.6541511,225.36228279)
\curveto(334.85727614,225.36228279)(334.24497147,225.1044703)(333.81723711,224.58884532)
\curveto(333.38950276,224.07907972)(333.17563558,223.49021255)(333.17563558,222.82224383)
\lineto(333.17563558,217.40818155)
\lineto(331.59360439,217.40818155)
\closepath
}
}
{
\newrgbcolor{curcolor}{0 0 0}
\pscustom[linestyle=none,fillstyle=solid,fillcolor=curcolor]
{
\newpath
\moveto(338.50180712,220.36130643)
\curveto(338.54868212,219.8339627)(338.68051805,219.42966584)(338.89731492,219.14841585)
\curveto(339.2957524,218.63865025)(339.98715862,218.38376744)(340.97153358,218.38376744)
\curveto(341.55747106,218.38376744)(342.07309604,218.509744)(342.51840852,218.76169712)
\curveto(342.963721,219.01950961)(343.18637724,219.4150174)(343.18637724,219.94822051)
\curveto(343.18637724,220.35251736)(343.00766631,220.66013454)(342.65024445,220.87107203)
\curveto(342.42172884,220.99997828)(341.97055698,221.14939233)(341.29672888,221.3193142)
\lineto(340.039893,221.63572044)
\curveto(339.23715865,221.83493918)(338.6453618,222.05759542)(338.26450244,222.30368916)
\curveto(337.58481497,222.73142352)(337.24497123,223.32322037)(337.24497123,224.07907972)
\curveto(337.24497123,224.96970468)(337.56430716,225.69040778)(338.20297901,226.241189)
\curveto(338.84751023,226.79197023)(339.71176801,227.06736085)(340.79575234,227.06736085)
\curveto(342.21372103,227.06736085)(343.23618193,226.65134524)(343.86313503,225.81931402)
\curveto(344.25571314,225.29197029)(344.44614282,224.72361094)(344.43442407,224.11423596)
\lineto(342.9402835,224.11423596)
\curveto(342.91098663,224.47165782)(342.78501007,224.79685312)(342.56235383,225.08982186)
\curveto(342.1990726,225.50583747)(341.56918981,225.71384528)(340.67270547,225.71384528)
\curveto(340.07504924,225.71384528)(339.6209477,225.59958747)(339.31040084,225.37107185)
\curveto(339.00571335,225.14255624)(338.85336961,224.84079844)(338.85336961,224.46579845)
\curveto(338.85336961,224.05564222)(339.05551803,223.72751723)(339.45981489,223.48142349)
\curveto(339.69418988,223.33493912)(340.039893,223.20603288)(340.49692423,223.09470476)
\lineto(341.54282262,222.83982195)
\curveto(342.67954133,222.56443134)(343.44126005,222.29782979)(343.82797878,222.0400173)
\curveto(344.44321313,221.63572044)(344.75083031,220.99997828)(344.75083031,220.13279081)
\curveto(344.75083031,219.29490022)(344.43149438,218.57126744)(343.79282253,217.96189246)
\curveto(343.16001006,217.35251749)(342.19321322,217.04783)(340.89243202,217.04783)
\curveto(339.49204145,217.04783)(338.49887743,217.36423623)(337.91293996,217.99704871)
\curveto(337.33286185,218.63572056)(337.02231499,219.42380646)(336.98129937,220.36130643)
\closepath
\moveto(340.83969765,227.04978272)
\closepath
}
}
{
\newrgbcolor{curcolor}{0 0 0}
\pscustom[linestyle=none,fillstyle=solid,fillcolor=curcolor]
{
\newpath
\moveto(346.87778335,229.44919669)
\lineto(348.47739266,229.44919669)
\lineto(348.47739266,226.82126711)
\lineto(349.98032228,226.82126711)
\lineto(349.98032228,225.52927497)
\lineto(348.47739266,225.52927497)
\lineto(348.47739266,219.38572053)
\curveto(348.47739266,219.05759554)(348.58872078,218.83786899)(348.81137702,218.72654087)
\curveto(348.93442389,218.66208775)(349.13950201,218.62986118)(349.42661137,218.62986118)
\lineto(349.67270511,218.62986118)
\curveto(349.76059573,218.63572056)(349.86313479,218.64450962)(349.98032228,218.65622837)
\lineto(349.98032228,217.40818155)
\curveto(349.79868167,217.35544717)(349.60825199,217.31736124)(349.40903325,217.29392374)
\curveto(349.21567388,217.27048624)(349.00473639,217.25876749)(348.77622077,217.25876749)
\curveto(348.03793955,217.25876749)(347.53696301,217.44626748)(347.27329114,217.82126747)
\curveto(347.00961928,218.20212683)(346.87778335,218.69431431)(346.87778335,219.29782991)
\lineto(346.87778335,225.52927497)
\lineto(345.60336934,225.52927497)
\lineto(345.60336934,226.82126711)
\lineto(346.87778335,226.82126711)
\closepath
}
}
{
\newrgbcolor{curcolor}{0 0 0}
\pscustom[linestyle=none,fillstyle=solid,fillcolor=curcolor]
{
\newpath
\moveto(356.58969598,230.31931384)
\lineto(358.17172716,230.31931384)
\lineto(358.17172716,217.40818155)
\lineto(356.58969598,217.40818155)
\closepath
}
}
{
\newrgbcolor{curcolor}{0 0 0}
\pscustom[linestyle=none,fillstyle=solid,fillcolor=curcolor]
{
\newpath
\moveto(361.76645591,219.91306426)
\curveto(361.76645591,219.45603303)(361.93344809,219.09568148)(362.26743245,218.83200961)
\curveto(362.60141681,218.56833775)(362.99692461,218.43650182)(363.45395584,218.43650182)
\curveto(364.01059644,218.43650182)(364.54965892,218.56540806)(365.07114327,218.82322055)
\curveto(365.95004949,219.25095491)(366.3895026,219.95115019)(366.3895026,220.9238064)
\lineto(366.3895026,222.19822042)
\curveto(366.19614323,222.07517355)(365.9471198,221.97263449)(365.64243231,221.89060324)
\curveto(365.33774483,221.80857199)(365.03891671,221.74997825)(364.74594797,221.714822)
\lineto(363.7879402,221.59177513)
\curveto(363.21372147,221.51560326)(362.78305743,221.39548607)(362.49594806,221.23142358)
\curveto(362.00961996,220.95603297)(361.76645591,220.51657986)(361.76645591,219.91306426)
\closepath
\moveto(365.598487,223.11228288)
\curveto(365.96176824,223.15915788)(366.20493229,223.31150162)(366.32797916,223.56931411)
\curveto(366.39829166,223.70993911)(366.43344791,223.91208753)(366.43344791,224.1757594)
\curveto(366.43344791,224.71482188)(366.24008854,225.1044703)(365.85336981,225.34470466)
\curveto(365.47251045,225.59079841)(364.92465891,225.71384528)(364.20981518,225.71384528)
\curveto(363.38364334,225.71384528)(362.79770587,225.49118903)(362.45200275,225.04587655)
\curveto(362.25864339,224.79978281)(362.13266683,224.43357189)(362.07407308,223.94724378)
\lineto(360.59751064,223.94724378)
\curveto(360.62680751,225.10739999)(361.0018075,225.91306402)(361.7225106,226.36423587)
\curveto(362.44907307,226.82126711)(363.28989335,227.04978272)(364.24497143,227.04978272)
\curveto(365.35239326,227.04978272)(366.25180729,226.83884523)(366.94321351,226.41697025)
\curveto(367.62876036,225.99509526)(367.97153378,225.33884529)(367.97153378,224.44822033)
\lineto(367.97153378,219.02536898)
\curveto(367.97153378,218.86130649)(368.00376034,218.72947055)(368.06821347,218.62986118)
\curveto(368.13852596,218.53025181)(368.28208065,218.48044713)(368.49887751,218.48044713)
\curveto(368.56919001,218.48044713)(368.64829157,218.48337681)(368.73618219,218.48923619)
\curveto(368.82407281,218.50095494)(368.91782281,218.51560338)(369.01743218,218.5331815)
\lineto(369.01743218,217.36423623)
\curveto(368.77133844,217.29392374)(368.58383845,217.24997843)(368.4549322,217.2324003)
\curveto(368.32602596,217.21482218)(368.15024471,217.20603312)(367.92758847,217.20603312)
\curveto(367.38266662,217.20603312)(366.98715882,217.39939248)(366.74106508,217.78611122)
\curveto(366.61215884,217.99118933)(366.52133853,218.28122839)(366.46860416,218.65622837)
\curveto(366.14633854,218.23435339)(365.68344794,217.86814246)(365.07993234,217.5575956)
\curveto(364.47641674,217.24704874)(363.8113777,217.09177531)(363.08481523,217.09177531)
\curveto(362.21176839,217.09177531)(361.49692467,217.35544717)(360.94028406,217.8827909)
\curveto(360.38950284,218.415994)(360.11411222,219.08103304)(360.11411222,219.87790801)
\curveto(360.11411222,220.75095485)(360.38657315,221.42771263)(360.931495,221.90818137)
\curveto(361.47641686,222.3886501)(362.19126058,222.68454852)(363.07602617,222.79587664)
\closepath
\moveto(364.28891674,227.04978272)
\closepath
}
}
{
\newrgbcolor{curcolor}{0 0 0}
\pscustom[linestyle=none,fillstyle=solid,fillcolor=curcolor]
{
\newpath
\moveto(376.45297916,226.82126711)
\lineto(378.20200252,226.82126711)
\curveto(377.97934628,226.2177515)(377.48422912,224.84079844)(376.71665102,222.6904079)
\curveto(376.14243229,221.07322046)(375.66196356,219.75486114)(375.27524483,218.73532993)
\curveto(374.36118237,216.33298628)(373.71665114,214.86814258)(373.34165116,214.34079886)
\curveto(372.96665117,213.81345513)(372.32211995,213.54978326)(371.40805748,213.54978326)
\curveto(371.18540124,213.54978326)(371.01254969,213.55857232)(370.88950282,213.57615045)
\curveto(370.77231532,213.59372857)(370.62583095,213.62595513)(370.45004971,213.67283013)
\lineto(370.45004971,215.11423632)
\curveto(370.72544032,215.03806445)(370.92465907,214.99118945)(371.04770594,214.97361133)
\curveto(371.17075281,214.95603321)(371.27915124,214.94724414)(371.37290124,214.94724414)
\curveto(371.66586997,214.94724414)(371.87973715,214.99704883)(372.01450277,215.0966582)
\curveto(372.15512777,215.1904082)(372.27231526,215.30759569)(372.36606526,215.44822069)
\curveto(372.39536213,215.49509568)(372.50083088,215.73533005)(372.6824715,216.16892378)
\curveto(372.86411211,216.60251752)(372.99594805,216.92478313)(373.07797929,217.13572062)
\lineto(369.59751068,226.82126711)
\lineto(371.39047936,226.82126711)
\lineto(373.9129402,219.15720491)
\closepath
\moveto(373.90415113,227.04978272)
\closepath
}
}
{
\newrgbcolor{curcolor}{0 0 0}
\pscustom[linestyle=none,fillstyle=solid,fillcolor=curcolor]
{
\newpath
\moveto(383.49301614,227.0322046)
\curveto(384.16098486,227.0322046)(384.80844577,226.87400148)(385.43539887,226.55759524)
\curveto(386.06235197,226.24704838)(386.53989101,225.84275152)(386.868016,225.34470466)
\curveto(387.18442224,224.87009531)(387.39535973,224.31638439)(387.50082847,223.68357192)
\curveto(387.59457847,223.24997819)(387.64145347,222.55857196)(387.64145347,221.60935325)
\lineto(380.74203968,221.60935325)
\curveto(380.77133656,220.65427517)(380.99692249,219.88669707)(381.41879747,219.30661897)
\curveto(381.84067245,218.73240024)(382.49399274,218.44529088)(383.37875833,218.44529088)
\curveto(384.20493017,218.44529088)(384.86410983,218.71775181)(385.35629731,219.26267366)
\curveto(385.6375473,219.5790799)(385.83676604,219.94529082)(385.95395354,220.36130643)
\lineto(387.50961754,220.36130643)
\curveto(387.46860191,220.01560332)(387.33090661,219.62888458)(387.09653162,219.20115022)
\curveto(386.868016,218.77927524)(386.61020351,218.43357213)(386.32309415,218.16404089)
\curveto(385.84262542,217.69529091)(385.24789888,217.37888467)(384.53891453,217.21482218)
\curveto(384.15805517,217.12107218)(383.72739113,217.07419718)(383.2469224,217.07419718)
\curveto(382.07504744,217.07419718)(381.08188342,217.49900185)(380.26743033,218.3486112)
\curveto(379.45297723,219.20407991)(379.04575069,220.39939236)(379.04575069,221.93454855)
\curveto(379.04575069,223.44626724)(379.45590692,224.67380625)(380.27621939,225.61716559)
\curveto(381.09653186,226.56052493)(382.16879744,227.0322046)(383.49301614,227.0322046)
\closepath
\moveto(386.01547697,222.86618914)
\curveto(385.95102385,223.55173599)(385.80160979,224.09958753)(385.5672348,224.50974376)
\curveto(385.13364107,225.27146248)(384.41000829,225.65232184)(383.39633645,225.65232184)
\curveto(382.66977398,225.65232184)(382.06039901,225.38864998)(381.56821153,224.86130625)
\curveto(381.07602404,224.33982189)(380.81528187,223.67478286)(380.78598499,222.86618914)
\closepath
\moveto(383.34360208,227.04978272)
\closepath
}
}
{
\newrgbcolor{curcolor}{0 0 0}
\pscustom[linestyle=none,fillstyle=solid,fillcolor=curcolor]
{
\newpath
\moveto(389.63657098,226.82126711)
\lineto(391.13950061,226.82126711)
\lineto(391.13950061,225.19529061)
\curveto(391.26254748,225.51169685)(391.56430528,225.89548589)(392.04477401,226.34665775)
\curveto(392.52524274,226.80368898)(393.07895365,227.0322046)(393.70590675,227.0322046)
\curveto(393.73520363,227.0322046)(393.78500831,227.02927491)(393.85532081,227.02341554)
\curveto(393.92563331,227.01755616)(394.04575049,227.00583741)(394.21567236,226.98825929)
\lineto(394.21567236,225.31833748)
\curveto(394.12192236,225.3359156)(394.03403174,225.34763435)(393.95200049,225.35349373)
\curveto(393.87582862,225.3593531)(393.79086769,225.36228279)(393.69711769,225.36228279)
\curveto(392.90024272,225.36228279)(392.28793806,225.1044703)(391.8602037,224.58884532)
\curveto(391.43246934,224.07907972)(391.21860216,223.49021255)(391.21860216,222.82224383)
\lineto(391.21860216,217.40818155)
\lineto(389.63657098,217.40818155)
\closepath
}
}
{
\newrgbcolor{curcolor}{0 0 0}
\pscustom[linestyle=none,fillstyle=solid,fillcolor=curcolor]
{
\newpath
\moveto(400.91293527,229.44919669)
\lineto(402.51254458,229.44919669)
\lineto(402.51254458,226.82126711)
\lineto(404.01547421,226.82126711)
\lineto(404.01547421,225.52927497)
\lineto(402.51254458,225.52927497)
\lineto(402.51254458,219.38572053)
\curveto(402.51254458,219.05759554)(402.6238727,218.83786899)(402.84652894,218.72654087)
\curveto(402.96957581,218.66208775)(403.17465393,218.62986118)(403.46176329,218.62986118)
\lineto(403.70785703,218.62986118)
\curveto(403.79574766,218.63572056)(403.89828671,218.64450962)(404.01547421,218.65622837)
\lineto(404.01547421,217.40818155)
\curveto(403.83383359,217.35544717)(403.64340391,217.31736124)(403.44418517,217.29392374)
\curveto(403.2508258,217.27048624)(403.03988831,217.25876749)(402.8113727,217.25876749)
\curveto(402.07309147,217.25876749)(401.57211493,217.44626748)(401.30844307,217.82126747)
\curveto(401.0447712,218.20212683)(400.91293527,218.69431431)(400.91293527,219.29782991)
\lineto(400.91293527,225.52927497)
\lineto(399.63852126,225.52927497)
\lineto(399.63852126,226.82126711)
\lineto(400.91293527,226.82126711)
\closepath
}
}
{
\newrgbcolor{curcolor}{0 0 0}
\pscustom[linestyle=none,fillstyle=solid,fillcolor=curcolor]
{
\newpath
\moveto(409.32407083,218.42771275)
\curveto(410.37289891,218.42771275)(411.09067232,218.82322055)(411.47739106,219.61423614)
\curveto(411.86996917,220.41111111)(412.06625822,221.2958767)(412.06625822,222.26853291)
\curveto(412.06625822,223.14743913)(411.92563323,223.86228285)(411.64438324,224.41306408)
\curveto(411.19907076,225.28025154)(410.43149266,225.71384528)(409.34164895,225.71384528)
\curveto(408.37485212,225.71384528)(407.67172715,225.34470466)(407.23227404,224.60642344)
\curveto(406.79282093,223.86814222)(406.57309438,222.97751726)(406.57309438,221.93454855)
\curveto(406.57309438,220.93259547)(406.79282093,220.09763456)(407.23227404,219.42966584)
\curveto(407.67172715,218.76169712)(408.36899274,218.42771275)(409.32407083,218.42771275)
\closepath
\moveto(409.38559427,227.09372803)
\curveto(410.59848484,227.09372803)(411.62387543,226.68943117)(412.46176602,225.88083746)
\curveto(413.29965661,225.07224374)(413.7186019,223.88279066)(413.7186019,222.31247822)
\curveto(413.7186019,220.79490016)(413.34946129,219.54099396)(412.61118007,218.55075962)
\curveto(411.87289885,217.56052529)(410.72739109,217.06540812)(409.17465677,217.06540812)
\curveto(407.87973495,217.06540812)(406.85141468,217.50193154)(406.08969596,218.37497838)
\curveto(405.32797724,219.2538846)(404.94711788,220.43161892)(404.94711788,221.90818137)
\curveto(404.94711788,223.49021255)(405.34848505,224.74997813)(406.15121939,225.68747809)
\curveto(406.95395374,226.62497805)(408.03207869,227.09372803)(409.38559427,227.09372803)
\closepath
\moveto(409.33285989,227.04978272)
\closepath
}
}
{
\newrgbcolor{curcolor}{0 0 0}
\pscustom[linestyle=none,fillstyle=solid,fillcolor=curcolor]
{
\newpath
\moveto(419.34360208,218.42771275)
\curveto(420.39243016,218.42771275)(421.11020357,218.82322055)(421.49692231,219.61423614)
\curveto(421.88950042,220.41111111)(422.08578947,221.2958767)(422.08578947,222.26853291)
\curveto(422.08578947,223.14743913)(421.94516448,223.86228285)(421.66391449,224.41306408)
\curveto(421.21860201,225.28025154)(420.45102391,225.71384528)(419.3611802,225.71384528)
\curveto(418.39438337,225.71384528)(417.6912584,225.34470466)(417.25180529,224.60642344)
\curveto(416.81235218,223.86814222)(416.59262563,222.97751726)(416.59262563,221.93454855)
\curveto(416.59262563,220.93259547)(416.81235218,220.09763456)(417.25180529,219.42966584)
\curveto(417.6912584,218.76169712)(418.38852399,218.42771275)(419.34360208,218.42771275)
\closepath
\moveto(419.40512552,227.09372803)
\curveto(420.61801609,227.09372803)(421.64340668,226.68943117)(422.48129727,225.88083746)
\curveto(423.31918786,225.07224374)(423.73813315,223.88279066)(423.73813315,222.31247822)
\curveto(423.73813315,220.79490016)(423.36899254,219.54099396)(422.63071132,218.55075962)
\curveto(421.8924301,217.56052529)(420.74692234,217.06540812)(419.19418802,217.06540812)
\curveto(417.8992662,217.06540812)(416.87094593,217.50193154)(416.10922721,218.37497838)
\curveto(415.34750849,219.2538846)(414.96664913,220.43161892)(414.96664913,221.90818137)
\curveto(414.96664913,223.49021255)(415.3680163,224.74997813)(416.17075064,225.68747809)
\curveto(416.97348499,226.62497805)(418.05160994,227.09372803)(419.40512552,227.09372803)
\closepath
\moveto(419.35239114,227.04978272)
\closepath
}
}
{
\newrgbcolor{curcolor}{0 0 0}
\pscustom[linestyle=none,fillstyle=solid,fillcolor=curcolor]
{
\newpath
\moveto(430.93637854,229.44919669)
\lineto(432.53598785,229.44919669)
\lineto(432.53598785,226.82126711)
\lineto(434.03891748,226.82126711)
\lineto(434.03891748,225.52927497)
\lineto(432.53598785,225.52927497)
\lineto(432.53598785,219.38572053)
\curveto(432.53598785,219.05759554)(432.64731597,218.83786899)(432.86997221,218.72654087)
\curveto(432.99301908,218.66208775)(433.1980972,218.62986118)(433.48520656,218.62986118)
\lineto(433.7313003,218.62986118)
\curveto(433.81919092,218.63572056)(433.92172998,218.64450962)(434.03891748,218.65622837)
\lineto(434.03891748,217.40818155)
\curveto(433.85727686,217.35544717)(433.66684718,217.31736124)(433.46762844,217.29392374)
\curveto(433.27426907,217.27048624)(433.06333158,217.25876749)(432.83481596,217.25876749)
\curveto(432.09653474,217.25876749)(431.5955582,217.44626748)(431.33188633,217.82126747)
\curveto(431.06821447,218.20212683)(430.93637854,218.69431431)(430.93637854,219.29782991)
\lineto(430.93637854,225.52927497)
\lineto(429.66196453,225.52927497)
\lineto(429.66196453,226.82126711)
\lineto(430.93637854,226.82126711)
\closepath
}
}
{
\newrgbcolor{curcolor}{0 0 0}
\pscustom[linestyle=none,fillstyle=solid,fillcolor=curcolor]
{
\newpath
\moveto(435.61215548,230.36325915)
\lineto(437.19418666,230.36325915)
\lineto(437.19418666,225.54685309)
\curveto(437.56918665,226.02146245)(437.9061007,226.35544681)(438.20492881,226.54880618)
\curveto(438.71469441,226.88279054)(439.35043658,227.04978272)(440.1121553,227.04978272)
\curveto(441.47738962,227.04978272)(442.40317083,226.57224368)(442.88949893,225.61716559)
\curveto(443.1531708,225.09568124)(443.28500673,224.37204845)(443.28500673,223.44626724)
\lineto(443.28500673,217.40818155)
\lineto(441.65903023,217.40818155)
\lineto(441.65903023,223.3407985)
\curveto(441.65903023,224.03220472)(441.57113961,224.53904063)(441.39535837,224.86130625)
\curveto(441.10824901,225.37693123)(440.56918653,225.63474372)(439.77817093,225.63474372)
\curveto(439.12192096,225.63474372)(438.52719442,225.40915779)(437.99399132,224.95798593)
\curveto(437.46078821,224.50681407)(437.19418666,223.65427505)(437.19418666,222.40036885)
\lineto(437.19418666,217.40818155)
\lineto(435.61215548,217.40818155)
\closepath
}
}
{
\newrgbcolor{curcolor}{0 0 0}
\pscustom[linestyle=none,fillstyle=solid,fillcolor=curcolor]
{
\newpath
\moveto(445.63169393,226.7773218)
\lineto(447.24009231,226.7773218)
\lineto(447.24009231,217.40818155)
\lineto(445.63169393,217.40818155)
\closepath
\moveto(445.63169393,230.31931384)
\lineto(447.24009231,230.31931384)
\lineto(447.24009231,228.52634516)
\lineto(445.63169393,228.52634516)
\closepath
}
}
{
\newrgbcolor{curcolor}{0 0 0}
\pscustom[linestyle=none,fillstyle=solid,fillcolor=curcolor]
{
\newpath
\moveto(453.26938045,227.09372803)
\curveto(454.32992728,227.09372803)(455.19125537,226.83591554)(455.85336472,226.32029056)
\curveto(456.52133345,225.80466558)(456.92270062,224.91697031)(457.05746624,223.65720473)
\lineto(455.51938036,223.65720473)
\curveto(455.42563036,224.23728283)(455.21176319,224.71775157)(454.87777882,225.09861092)
\curveto(454.54379446,225.48532966)(454.00766167,225.67868903)(453.26938045,225.67868903)
\curveto(452.26156799,225.67868903)(451.5408649,225.18650155)(451.10727116,224.20212659)
\curveto(450.82602117,223.56345474)(450.68539618,222.77536883)(450.68539618,221.83786887)
\curveto(450.68539618,220.89450953)(450.88461492,220.10056425)(451.28305241,219.45603303)
\curveto(451.68148989,218.8115018)(452.30844299,218.48923619)(453.16391171,218.48923619)
\curveto(453.82016168,218.48923619)(454.33871635,218.68845493)(454.71957571,219.08689242)
\curveto(455.10629444,219.49118927)(455.37289599,220.0419705)(455.51938036,220.7392361)
\lineto(457.05746624,220.7392361)
\curveto(456.88168499,219.49118927)(456.44223189,218.57712681)(455.73910691,217.99704871)
\curveto(455.03598194,217.42282998)(454.13656792,217.13572062)(453.04086484,217.13572062)
\curveto(451.81039613,217.13572062)(450.82895086,217.58396279)(450.09652902,218.48044713)
\curveto(449.36410717,219.38279084)(448.99789625,220.5077908)(448.99789625,221.85544699)
\curveto(448.99789625,223.50779068)(449.39926342,224.79392344)(450.20199776,225.71384528)
\curveto(451.0047321,226.63376711)(452.027193,227.09372803)(453.26938045,227.09372803)
\closepath
\moveto(453.02328671,227.04978272)
\closepath
}
}
{
\newrgbcolor{curcolor}{0 0 0}
\pscustom[linestyle=none,fillstyle=solid,fillcolor=curcolor]
{
\newpath
\moveto(458.60435019,230.31931384)
\lineto(460.12485794,230.31931384)
\lineto(460.12485794,222.82224383)
\lineto(464.18540465,226.82126711)
\lineto(466.20688894,226.82126711)
\lineto(462.60337346,223.29685318)
\lineto(466.40903737,217.40818155)
\lineto(464.38755308,217.40818155)
\lineto(461.45200632,222.15427511)
\lineto(460.12485794,220.94138453)
\lineto(460.12485794,217.40818155)
\lineto(458.60435019,217.40818155)
\closepath
}
}
{
\newrgbcolor{curcolor}{1 0 0}
\pscustom[linestyle=none,fillstyle=solid,fillcolor=curcolor]
{
\newpath
\moveto(418.20882848,49.99231394)
\curveto(418.20882848,27.86986211)(400.68528683,9.93607149)(379.06890274,9.93607149)
\curveto(357.45251866,9.93607149)(339.928977,27.86986211)(339.928977,49.99231394)
\curveto(339.928977,72.11476578)(357.45251866,90.0485564)(379.06890274,90.0485564)
\curveto(400.68528683,90.0485564)(418.20882848,72.11476578)(418.20882848,49.99231394)
\closepath
}
}
{
\newrgbcolor{curcolor}{1 0 0}
\pscustom[linestyle=none,fillstyle=solid,fillcolor=curcolor]
{
\newpath
\moveto(380.00000126,89.99996496)
\lineto(419.99999645,89.99996496)
\lineto(419.99999645,9.99996737)
\lineto(380.00000126,9.99996737)
\closepath
}
}
{
\newrgbcolor{curcolor}{1 0 0}
\pscustom[linestyle=none,fillstyle=solid,fillcolor=curcolor]
{
\newpath
\moveto(460.15270449,49.3443801)
\curveto(460.15270449,27.5797822)(442.04896698,9.93608951)(419.71687503,9.93608951)
\curveto(397.38478308,9.93608951)(379.28104557,27.5797822)(379.28104557,49.3443801)
\curveto(379.28104557,71.10897801)(397.38478308,88.7526707)(419.71687503,88.7526707)
\curveto(442.04896698,88.7526707)(460.15270449,71.10897801)(460.15270449,49.3443801)
\closepath
}
}
{
\newrgbcolor{curcolor}{1 0 0}
\pscustom[linestyle=none,fillstyle=solid,fillcolor=curcolor]
{
\newpath
\moveto(69.86273824,29.25791203)
\curveto(69.86273824,17.87108014)(61.04213611,8.64022904)(50.16137197,8.64022904)
\curveto(39.28060783,8.64022904)(30.46000569,17.87108014)(30.46000569,29.25791203)
\curveto(30.46000569,40.64474392)(39.28060783,49.87559502)(50.16137197,49.87559502)
\curveto(61.04213611,49.87559502)(69.86273824,40.64474392)(69.86273824,29.25791203)
\closepath
}
}
{
\newrgbcolor{curcolor}{1 0 0}
\pscustom[linestyle=none,fillstyle=solid,fillcolor=curcolor]
{
\newpath
\moveto(49.9999508,50.69082526)
\lineto(209.99994599,50.69082526)
\lineto(209.99994599,10.00000341)
\lineto(49.9999508,10.00000341)
\closepath
}
}
{
\newrgbcolor{curcolor}{1 0 0}
\pscustom[linestyle=none,fillstyle=solid,fillcolor=curcolor]
{
\newpath
\moveto(229.86288482,29.25791203)
\curveto(229.86288482,17.87108014)(221.04228269,8.64022904)(210.16151855,8.64022904)
\curveto(199.28075441,8.64022904)(190.46015227,17.87108014)(190.46015227,29.25791203)
\curveto(190.46015227,40.64474392)(199.28075441,49.87559502)(210.16151855,49.87559502)
\curveto(221.04228269,49.87559502)(229.86288482,40.64474392)(229.86288482,29.25791203)
\closepath
}
}
{
\newrgbcolor{curcolor}{0 0 1}
\pscustom[linestyle=none,fillstyle=solid,fillcolor=curcolor]
{
\newpath
\moveto(123.06614439,79.53858724)
\curveto(123.06614439,62.42609045)(109.16882767,48.55367085)(92.02561872,48.55367085)
\curveto(74.88240978,48.55367085)(60.98509306,62.42609045)(60.98509306,79.53858724)
\curveto(60.98509306,96.65108403)(74.88240978,110.52350363)(92.02561872,110.52350363)
\curveto(109.16882767,110.52350363)(123.06614439,96.65108403)(123.06614439,79.53858724)
\closepath
}
}
{
\newrgbcolor{curcolor}{0 0 1}
\pscustom[linestyle=none,fillstyle=solid,fillcolor=curcolor]
{
\newpath
\moveto(90.99996858,110.69082526)
\lineto(170.99996618,110.69082526)
\lineto(170.99996618,50.00000581)
\lineto(90.99996858,50.00000581)
\closepath
}
}
{
\newrgbcolor{curcolor}{0 0 1}
\pscustom[linestyle=none,fillstyle=solid,fillcolor=curcolor]
{
\newpath
\moveto(201.06621936,79.53858724)
\curveto(201.06621936,62.42609045)(187.16890264,48.55367085)(170.0256937,48.55367085)
\curveto(152.88248475,48.55367085)(138.98516803,62.42609045)(138.98516803,79.53858724)
\curveto(138.98516803,96.65108403)(152.88248475,110.52350363)(170.0256937,110.52350363)
\curveto(187.16890264,110.52350363)(201.06621936,96.65108403)(201.06621936,79.53858724)
\closepath
}
}
{
\newrgbcolor{curcolor}{0 0 1}
\pscustom[linestyle=none,fillstyle=solid,fillcolor=curcolor]
{
\newpath
\moveto(123.06614439,139.53860236)
\curveto(123.06614439,122.42610557)(109.16882767,108.55368597)(92.02561872,108.55368597)
\curveto(74.88240978,108.55368597)(60.98509306,122.42610557)(60.98509306,139.53860236)
\curveto(60.98509306,156.65109915)(74.88240978,170.52351875)(92.02561872,170.52351875)
\curveto(109.16882767,170.52351875)(123.06614439,156.65109915)(123.06614439,139.53860236)
\closepath
}
}
{
\newrgbcolor{curcolor}{0 0 1}
\pscustom[linestyle=none,fillstyle=solid,fillcolor=curcolor]
{
\newpath
\moveto(90.99996858,170.69084038)
\lineto(170.99996618,170.69084038)
\lineto(170.99996618,110.00002093)
\lineto(90.99996858,110.00002093)
\closepath
}
}
{
\newrgbcolor{curcolor}{0 0 1}
\pscustom[linestyle=none,fillstyle=solid,fillcolor=curcolor]
{
\newpath
\moveto(201.06621936,139.53860236)
\curveto(201.06621936,122.42610557)(187.16890264,108.55368597)(170.0256937,108.55368597)
\curveto(152.88248475,108.55368597)(138.98516803,122.42610557)(138.98516803,139.53860236)
\curveto(138.98516803,156.65109915)(152.88248475,170.52351875)(170.0256937,170.52351875)
\curveto(187.16890264,170.52351875)(201.06621936,156.65109915)(201.06621936,139.53860236)
\closepath
}
}
{
\newrgbcolor{curcolor}{0 0 1}
\pscustom[linestyle=none,fillstyle=solid,fillcolor=curcolor]
{
\newpath
\moveto(391.06629053,119.53858976)
\curveto(391.06629053,102.42609297)(377.16897381,88.55367337)(360.02576487,88.55367337)
\curveto(342.88255592,88.55367337)(328.9852392,102.42609297)(328.9852392,119.53858976)
\curveto(328.9852392,136.65108655)(342.88255592,150.52350615)(360.02576487,150.52350615)
\curveto(377.16897381,150.52350615)(391.06629053,136.65108655)(391.06629053,119.53858976)
\closepath
}
}
{
\newrgbcolor{curcolor}{0 0 1}
\pscustom[linestyle=none,fillstyle=solid,fillcolor=curcolor]
{
\newpath
\moveto(359.00011472,150.69082778)
\lineto(439.00011232,150.69082778)
\lineto(439.00011232,90.00000833)
\lineto(359.00011472,90.00000833)
\closepath
}
}
{
\newrgbcolor{curcolor}{0 0 1}
\pscustom[linestyle=none,fillstyle=solid,fillcolor=curcolor]
{
\newpath
\moveto(469.0663655,119.53858976)
\curveto(469.0663655,102.42609297)(455.16904878,88.55367337)(438.02583984,88.55367337)
\curveto(420.8826309,88.55367337)(406.98531418,102.42609297)(406.98531418,119.53858976)
\curveto(406.98531418,136.65108655)(420.8826309,150.52350615)(438.02583984,150.52350615)
\curveto(455.16904878,150.52350615)(469.0663655,136.65108655)(469.0663655,119.53858976)
\closepath
}
}
{
\newrgbcolor{curcolor}{0 0 1}
\pscustom[linestyle=none,fillstyle=solid,fillcolor=curcolor]
{
\newpath
\moveto(391.06629053,179.53862756)
\curveto(391.06629053,162.42613077)(377.16897381,148.55371117)(360.02576487,148.55371117)
\curveto(342.88255592,148.55371117)(328.9852392,162.42613077)(328.9852392,179.53862756)
\curveto(328.9852392,196.65112435)(342.88255592,210.52354394)(360.02576487,210.52354394)
\curveto(377.16897381,210.52354394)(391.06629053,196.65112435)(391.06629053,179.53862756)
\closepath
}
}
{
\newrgbcolor{curcolor}{0 0 1}
\pscustom[linestyle=none,fillstyle=solid,fillcolor=curcolor]
{
\newpath
\moveto(359.00011472,210.69086558)
\lineto(439.00011232,210.69086558)
\lineto(439.00011232,150.00004613)
\lineto(359.00011472,150.00004613)
\closepath
}
}
{
\newrgbcolor{curcolor}{0 0 1}
\pscustom[linestyle=none,fillstyle=solid,fillcolor=curcolor]
{
\newpath
\moveto(469.0663655,179.53862756)
\curveto(469.0663655,162.42613077)(455.16904878,148.55371117)(438.02583984,148.55371117)
\curveto(420.8826309,148.55371117)(406.98531418,162.42613077)(406.98531418,179.53862756)
\curveto(406.98531418,196.65112435)(420.8826309,210.52354394)(438.02583984,210.52354394)
\curveto(455.16904878,210.52354394)(469.0663655,196.65112435)(469.0663655,179.53862756)
\closepath
}
}
\end{pspicture}

    \caption{First Layer Problems}
  \end{figure}
\end{frame}

\section{Holes and Voids}
\begin{frame}
  \frametitle{Holes and Voids}
  The size of holes in a 3D printed object are generally less than specified because holes are printed as polygons.  The effect is more pronounced when the polygon has a small number of sides.
\end{frame}

\begin{frame}
  \frametitle{Holes and Voids}
  \begin{figure}
    %LaTeX with PSTricks extensions
%%Creator: inkscape 0.91
%%Please note this file requires PSTricks extensions
\psset{xunit=.5pt,yunit=.5pt,runit=.5pt}
\begin{pspicture}(340.01193309,346.67858514)
{
\newrgbcolor{curcolor}{0.50196081 0.50196081 0.50196081}
\pscustom[linestyle=none,fillstyle=solid,fillcolor=curcolor]
{
\newpath
\moveto(0,290.57926201)
\lineto(340.01193237,290.57926201)
\lineto(340.01193237,0.00000786)
\lineto(0,0.00000786)
\closepath
}
}
{
\newrgbcolor{curcolor}{1 1 1}
\pscustom[linestyle=none,fillstyle=solid,fillcolor=curcolor]
{
\newpath
\moveto(269.72229004,135.8629229)
\curveto(269.72229004,88.91871918)(231.66649376,50.8629229)(184.72229004,50.8629229)
\curveto(137.77808632,50.8629229)(99.72229004,88.91871918)(99.72229004,135.8629229)
\curveto(99.72229004,182.80712661)(137.77808632,220.8629229)(184.72229004,220.8629229)
\curveto(231.66649376,220.8629229)(269.72229004,182.80712661)(269.72229004,135.8629229)
\closepath
}
}
{
\newrgbcolor{curcolor}{0 0 0}
\pscustom[linewidth=2.5,linecolor=curcolor]
{
\newpath
\moveto(269.72229004,135.8629229)
\curveto(269.72229004,88.91871918)(231.66649376,50.8629229)(184.72229004,50.8629229)
\curveto(137.77808632,50.8629229)(99.72229004,88.91871918)(99.72229004,135.8629229)
\curveto(99.72229004,182.80712661)(137.77808632,220.8629229)(184.72229004,220.8629229)
\curveto(231.66649376,220.8629229)(269.72229004,182.80712661)(269.72229004,135.8629229)
\closepath
}
}
{
\newrgbcolor{curcolor}{1 1 1}
\pscustom[linestyle=none,fillstyle=solid,fillcolor=curcolor]
{
\newpath
\moveto(257.34315061,93.74599668)
\lineto(184.33911529,50.27747053)
\lineto(111.95017831,91.88561141)
\lineto(112.56527666,176.96227845)
\lineto(185.56931198,220.43080461)
\lineto(257.95824896,178.82266373)
\closepath
}
}
{
\newrgbcolor{curcolor}{0 0 0}
\pscustom[linewidth=2.74393634,linecolor=curcolor,linestyle=dashed,dash=19.33600873 6.44533624]
{
\newpath
\moveto(257.34315061,93.74599668)
\lineto(184.33911529,50.27747053)
\lineto(111.95017831,91.88561141)
\lineto(112.56527666,176.96227845)
\lineto(185.56931198,220.43080461)
\lineto(257.95824896,178.82266373)
\closepath
}
}
{
\newrgbcolor{curcolor}{0 0 0}
\pscustom[linewidth=2.5,linecolor=curcolor]
{
\newpath
\moveto(0,340.57927514)
\lineto(30,340.57927514)
}
}
{
\newrgbcolor{curcolor}{0 0 0}
\pscustom[linestyle=none,fillstyle=solid,fillcolor=curcolor]
{
\newpath
\moveto(46.328125,335.21764397)
\curveto(46.91992188,335.21764397)(47.40625,335.2791674)(47.78710938,335.40221428)
\curveto(48.46679688,335.6307299)(49.0234375,336.07018303)(49.45703125,336.72057365)
\curveto(49.80273438,337.24205803)(50.05175781,337.91002678)(50.20410156,338.7244799)
\curveto(50.29199219,339.21080803)(50.3359375,339.6619799)(50.3359375,340.07799553)
\curveto(50.3359375,341.6776049)(50.01660156,342.9197924)(49.37792969,343.80455803)
\curveto(48.74511719,344.68932365)(47.72265625,345.13170647)(46.31054688,345.13170647)
\lineto(43.20800781,345.13170647)
\lineto(43.20800781,335.21764397)
\lineto(46.328125,335.21764397)
\closepath
\moveto(41.45019531,346.63463615)
\lineto(46.6796875,346.63463615)
\curveto(48.45507812,346.63463615)(49.83203125,346.00475334)(50.81054688,344.74498772)
\curveto(51.68359375,343.60826897)(52.12011719,342.15221428)(52.12011719,340.37682365)
\curveto(52.12011719,339.0057299)(51.86230469,337.76647209)(51.34667969,336.65905022)
\curveto(50.43847656,334.70201897)(48.87695312,333.72350334)(46.66210938,333.72350334)
\lineto(41.45019531,333.72350334)
\lineto(41.45019531,346.63463615)
\closepath
}
}
{
\newrgbcolor{curcolor}{0 0 0}
\pscustom[linestyle=none,fillstyle=solid,fillcolor=curcolor]
{
\newpath
\moveto(58.08789062,343.34752678)
\curveto(58.75585938,343.34752678)(59.40332031,343.18932365)(60.03027344,342.8729174)
\curveto(60.65722656,342.56237053)(61.13476562,342.15807365)(61.46289062,341.66002678)
\curveto(61.77929688,341.1854174)(61.99023438,340.63170647)(62.09570312,339.99889397)
\curveto(62.18945312,339.56530022)(62.23632812,338.87389397)(62.23632812,337.92467522)
\lineto(55.33691406,337.92467522)
\curveto(55.36621094,336.96959709)(55.59179688,336.20201897)(56.01367188,335.62194084)
\curveto(56.43554688,335.04772209)(57.08886719,334.76061272)(57.97363281,334.76061272)
\curveto(58.79980469,334.76061272)(59.45898438,335.03307365)(59.95117188,335.57799553)
\curveto(60.23242188,335.89440178)(60.43164062,336.26061272)(60.54882812,336.67662834)
\lineto(62.10449219,336.67662834)
\curveto(62.06347656,336.33092522)(61.92578125,335.94420647)(61.69140625,335.51647209)
\curveto(61.46289062,335.09459709)(61.20507812,334.74889397)(60.91796875,334.47936272)
\curveto(60.4375,334.01061272)(59.84277344,333.69420647)(59.13378906,333.53014397)
\curveto(58.75292969,333.43639397)(58.32226562,333.38951897)(57.84179688,333.38951897)
\curveto(56.66992188,333.38951897)(55.67675781,333.81432365)(54.86230469,334.66393303)
\curveto(54.04785156,335.51940178)(53.640625,336.71471428)(53.640625,338.24987053)
\curveto(53.640625,339.76158928)(54.05078125,340.98912834)(54.87109375,341.93248772)
\curveto(55.69140625,342.87584709)(56.76367188,343.34752678)(58.08789062,343.34752678)
\closepath
\moveto(60.61035156,339.18151115)
\curveto(60.54589844,339.86705803)(60.39648438,340.41490959)(60.16210938,340.82506584)
\curveto(59.72851562,341.58678459)(59.00488281,341.96764397)(57.99121094,341.96764397)
\curveto(57.26464844,341.96764397)(56.65527344,341.70397209)(56.16308594,341.17662834)
\curveto(55.67089844,340.65514397)(55.41015625,339.9901049)(55.38085938,339.18151115)
\lineto(60.61035156,339.18151115)
\closepath
\moveto(57.93847656,343.3651049)
\lineto(57.93847656,343.3651049)
\closepath
}
}
{
\newrgbcolor{curcolor}{0 0 0}
\pscustom[linestyle=none,fillstyle=solid,fillcolor=curcolor]
{
\newpath
\moveto(65.12792969,336.67662834)
\curveto(65.17480469,336.14928459)(65.30664062,335.74498772)(65.5234375,335.46373772)
\curveto(65.921875,334.95397209)(66.61328125,334.69908928)(67.59765625,334.69908928)
\curveto(68.18359375,334.69908928)(68.69921875,334.82506584)(69.14453125,335.07701897)
\curveto(69.58984375,335.33483147)(69.8125,335.73033928)(69.8125,336.2635424)
\curveto(69.8125,336.66783928)(69.63378906,336.97545647)(69.27636719,337.18639397)
\curveto(69.04785156,337.31530022)(68.59667969,337.46471428)(67.92285156,337.63463615)
\lineto(66.66601562,337.9510424)
\curveto(65.86328125,338.15026115)(65.27148438,338.3729174)(64.890625,338.61901115)
\curveto(64.2109375,339.04674553)(63.87109375,339.6385424)(63.87109375,340.39440178)
\curveto(63.87109375,341.28502678)(64.19042969,342.0057299)(64.82910156,342.55651115)
\curveto(65.47363281,343.1072924)(66.33789062,343.38268303)(67.421875,343.38268303)
\curveto(68.83984375,343.38268303)(69.86230469,342.9666674)(70.48925781,342.13463615)
\curveto(70.88183594,341.6072924)(71.07226562,341.03893303)(71.06054688,340.42955803)
\lineto(69.56640625,340.42955803)
\curveto(69.53710938,340.7869799)(69.41113281,341.11217522)(69.18847656,341.40514397)
\curveto(68.82519531,341.82115959)(68.1953125,342.0291674)(67.29882812,342.0291674)
\curveto(66.70117188,342.0291674)(66.24707031,341.91490959)(65.93652344,341.68639397)
\curveto(65.63183594,341.45787834)(65.47949219,341.15612053)(65.47949219,340.78112053)
\curveto(65.47949219,340.37096428)(65.68164062,340.04283928)(66.0859375,339.79674553)
\curveto(66.3203125,339.65026115)(66.66601562,339.5213549)(67.12304688,339.41002678)
\lineto(68.16894531,339.15514397)
\curveto(69.30566406,338.87975334)(70.06738281,338.61315178)(70.45410156,338.35533928)
\curveto(71.06933594,337.9510424)(71.37695312,337.31530022)(71.37695312,336.44811272)
\curveto(71.37695312,335.61022209)(71.05761719,334.88658928)(70.41894531,334.27721428)
\curveto(69.78613281,333.66783928)(68.81933594,333.36315178)(67.51855469,333.36315178)
\curveto(66.11816406,333.36315178)(65.125,333.67955803)(64.5390625,334.31237053)
\curveto(63.95898438,334.9510424)(63.6484375,335.73912834)(63.60742188,336.67662834)
\lineto(65.12792969,336.67662834)
\closepath
\moveto(67.46582031,343.3651049)
\lineto(67.46582031,343.3651049)
\closepath
}
}
{
\newrgbcolor{curcolor}{0 0 0}
\pscustom[linestyle=none,fillstyle=solid,fillcolor=curcolor]
{
\newpath
\moveto(73.22265625,343.09264397)
\lineto(74.83105469,343.09264397)
\lineto(74.83105469,333.72350334)
\lineto(73.22265625,333.72350334)
\lineto(73.22265625,343.09264397)
\closepath
\moveto(73.22265625,346.63463615)
\lineto(74.83105469,346.63463615)
\lineto(74.83105469,344.8416674)
\lineto(73.22265625,344.8416674)
\lineto(73.22265625,346.63463615)
\closepath
}
}
{
\newrgbcolor{curcolor}{0 0 0}
\pscustom[linestyle=none,fillstyle=solid,fillcolor=curcolor]
{
\newpath
\moveto(77.27441406,343.13658928)
\lineto(78.77734375,343.13658928)
\lineto(78.77734375,341.51061272)
\curveto(78.90039062,341.82701897)(79.20214844,342.21080803)(79.68261719,342.6619799)
\curveto(80.16308594,343.11901115)(80.71679688,343.34752678)(81.34375,343.34752678)
\curveto(81.37304688,343.34752678)(81.42285156,343.34459709)(81.49316406,343.33873772)
\curveto(81.56347656,343.33287834)(81.68359375,343.32115959)(81.85351562,343.30358147)
\lineto(81.85351562,341.63365959)
\curveto(81.75976562,341.65123772)(81.671875,341.66295647)(81.58984375,341.66881584)
\curveto(81.51367188,341.67467522)(81.42871094,341.6776049)(81.33496094,341.6776049)
\curveto(80.53808594,341.6776049)(79.92578125,341.4197924)(79.49804688,340.9041674)
\curveto(79.0703125,340.39440178)(78.85644531,339.80553459)(78.85644531,339.13756584)
\lineto(78.85644531,333.72350334)
\lineto(77.27441406,333.72350334)
\lineto(77.27441406,343.13658928)
\closepath
}
}
{
\newrgbcolor{curcolor}{0 0 0}
\pscustom[linestyle=none,fillstyle=solid,fillcolor=curcolor]
{
\newpath
\moveto(87.16210938,343.34752678)
\curveto(87.83007812,343.34752678)(88.47753906,343.18932365)(89.10449219,342.8729174)
\curveto(89.73144531,342.56237053)(90.20898438,342.15807365)(90.53710938,341.66002678)
\curveto(90.85351562,341.1854174)(91.06445312,340.63170647)(91.16992188,339.99889397)
\curveto(91.26367188,339.56530022)(91.31054688,338.87389397)(91.31054688,337.92467522)
\lineto(84.41113281,337.92467522)
\curveto(84.44042969,336.96959709)(84.66601562,336.20201897)(85.08789062,335.62194084)
\curveto(85.50976562,335.04772209)(86.16308594,334.76061272)(87.04785156,334.76061272)
\curveto(87.87402344,334.76061272)(88.53320312,335.03307365)(89.02539062,335.57799553)
\curveto(89.30664062,335.89440178)(89.50585938,336.26061272)(89.62304688,336.67662834)
\lineto(91.17871094,336.67662834)
\curveto(91.13769531,336.33092522)(91,335.94420647)(90.765625,335.51647209)
\curveto(90.53710938,335.09459709)(90.27929688,334.74889397)(89.9921875,334.47936272)
\curveto(89.51171875,334.01061272)(88.91699219,333.69420647)(88.20800781,333.53014397)
\curveto(87.82714844,333.43639397)(87.39648438,333.38951897)(86.91601562,333.38951897)
\curveto(85.74414062,333.38951897)(84.75097656,333.81432365)(83.93652344,334.66393303)
\curveto(83.12207031,335.51940178)(82.71484375,336.71471428)(82.71484375,338.24987053)
\curveto(82.71484375,339.76158928)(83.125,340.98912834)(83.9453125,341.93248772)
\curveto(84.765625,342.87584709)(85.83789062,343.34752678)(87.16210938,343.34752678)
\closepath
\moveto(89.68457031,339.18151115)
\curveto(89.62011719,339.86705803)(89.47070312,340.41490959)(89.23632812,340.82506584)
\curveto(88.80273438,341.58678459)(88.07910156,341.96764397)(87.06542969,341.96764397)
\curveto(86.33886719,341.96764397)(85.72949219,341.70397209)(85.23730469,341.17662834)
\curveto(84.74511719,340.65514397)(84.484375,339.9901049)(84.45507812,339.18151115)
\lineto(89.68457031,339.18151115)
\closepath
\moveto(87.01269531,343.3651049)
\lineto(87.01269531,343.3651049)
\closepath
}
}
{
\newrgbcolor{curcolor}{0 0 0}
\pscustom[linestyle=none,fillstyle=solid,fillcolor=curcolor]
{
\newpath
\moveto(94.26367188,338.32018303)
\curveto(94.26367188,337.31237053)(94.47753906,336.46862053)(94.90527344,335.78893303)
\curveto(95.33300781,335.10924553)(96.01855469,334.76940178)(96.96191406,334.76940178)
\curveto(97.69433594,334.76940178)(98.29492188,335.08287834)(98.76367188,335.70983147)
\curveto(99.23828125,336.34264397)(99.47558594,337.2479174)(99.47558594,338.42565178)
\curveto(99.47558594,339.6151049)(99.23242188,340.49401115)(98.74609375,341.06237053)
\curveto(98.25976562,341.63658928)(97.65917969,341.92369865)(96.94433594,341.92369865)
\curveto(96.14746094,341.92369865)(95.5,341.61901115)(95.00195312,341.00963615)
\curveto(94.50976562,340.40026115)(94.26367188,339.50377678)(94.26367188,338.32018303)
\closepath
\moveto(96.64550781,343.30358147)
\curveto(97.36621094,343.30358147)(97.96972656,343.15123772)(98.45605469,342.84655022)
\curveto(98.73730469,342.67076897)(99.05664062,342.36315178)(99.4140625,341.92369865)
\lineto(99.4140625,346.67858147)
\lineto(100.93457031,346.67858147)
\lineto(100.93457031,333.72350334)
\lineto(99.51074219,333.72350334)
\lineto(99.51074219,335.03307365)
\curveto(99.14160156,334.45299553)(98.70507812,334.03405022)(98.20117188,333.77623772)
\curveto(97.69726562,333.51842522)(97.12011719,333.38951897)(96.46972656,333.38951897)
\curveto(95.42089844,333.38951897)(94.51269531,333.82897209)(93.74511719,334.70787834)
\curveto(92.97753906,335.59264397)(92.59375,336.76744865)(92.59375,338.2322924)
\curveto(92.59375,339.60338615)(92.94238281,340.78990959)(93.63964844,341.79186272)
\curveto(94.34277344,342.79967522)(95.34472656,343.30358147)(96.64550781,343.30358147)
\closepath
}
}
{
\newrgbcolor{curcolor}{0 0 0}
\pscustom[linestyle=none,fillstyle=solid,fillcolor=curcolor]
{
\newpath
\moveto(108.2734375,346.67858147)
\lineto(109.85546875,346.67858147)
\lineto(109.85546875,341.86217522)
\curveto(110.23046875,342.33678459)(110.56738281,342.67076897)(110.86621094,342.86412834)
\curveto(111.37597656,343.19811272)(112.01171875,343.3651049)(112.7734375,343.3651049)
\curveto(114.13867188,343.3651049)(115.06445312,342.88756584)(115.55078125,341.93248772)
\curveto(115.81445312,341.41100334)(115.94628906,340.68737053)(115.94628906,339.76158928)
\lineto(115.94628906,333.72350334)
\lineto(114.3203125,333.72350334)
\lineto(114.3203125,339.65612053)
\curveto(114.3203125,340.34752678)(114.23242188,340.85436272)(114.05664062,341.17662834)
\curveto(113.76953125,341.69225334)(113.23046875,341.95006584)(112.43945312,341.95006584)
\curveto(111.78320312,341.95006584)(111.18847656,341.7244799)(110.65527344,341.27330803)
\curveto(110.12207031,340.82213615)(109.85546875,339.96959709)(109.85546875,338.71569084)
\lineto(109.85546875,333.72350334)
\lineto(108.2734375,333.72350334)
\lineto(108.2734375,346.67858147)
\closepath
}
}
{
\newrgbcolor{curcolor}{0 0 0}
\pscustom[linestyle=none,fillstyle=solid,fillcolor=curcolor]
{
\newpath
\moveto(122.02832031,334.74303459)
\curveto(123.07714844,334.74303459)(123.79492188,335.1385424)(124.18164062,335.92955803)
\curveto(124.57421875,336.72643303)(124.77050781,337.61119865)(124.77050781,338.5838549)
\curveto(124.77050781,339.46276115)(124.62988281,340.1776049)(124.34863281,340.72838615)
\curveto(123.90332031,341.59557365)(123.13574219,342.0291674)(122.04589844,342.0291674)
\curveto(121.07910156,342.0291674)(120.37597656,341.66002678)(119.93652344,340.92174553)
\curveto(119.49707031,340.18346428)(119.27734375,339.29283928)(119.27734375,338.24987053)
\curveto(119.27734375,337.2479174)(119.49707031,336.41295647)(119.93652344,335.74498772)
\curveto(120.37597656,335.07701897)(121.07324219,334.74303459)(122.02832031,334.74303459)
\closepath
\moveto(122.08984375,343.40905022)
\curveto(123.30273438,343.40905022)(124.328125,343.00475334)(125.16601562,342.19615959)
\curveto(126.00390625,341.38756584)(126.42285156,340.19811272)(126.42285156,338.62780022)
\curveto(126.42285156,337.11022209)(126.05371094,335.85631584)(125.31542969,334.86608147)
\curveto(124.57714844,333.87584709)(123.43164062,333.3807299)(121.87890625,333.3807299)
\curveto(120.58398438,333.3807299)(119.55566406,333.81725334)(118.79394531,334.69030022)
\curveto(118.03222656,335.56920647)(117.65136719,336.74694084)(117.65136719,338.22350334)
\curveto(117.65136719,339.80553459)(118.05273438,341.06530022)(118.85546875,342.00280022)
\curveto(119.65820312,342.94030022)(120.73632812,343.40905022)(122.08984375,343.40905022)
\closepath
\moveto(122.03710938,343.3651049)
\lineto(122.03710938,343.3651049)
\closepath
}
}
{
\newrgbcolor{curcolor}{0 0 0}
\pscustom[linestyle=none,fillstyle=solid,fillcolor=curcolor]
{
\newpath
\moveto(128.35644531,346.63463615)
\lineto(129.93847656,346.63463615)
\lineto(129.93847656,333.72350334)
\lineto(128.35644531,333.72350334)
\lineto(128.35644531,346.63463615)
\closepath
}
}
{
\newrgbcolor{curcolor}{0 0 0}
\pscustom[linestyle=none,fillstyle=solid,fillcolor=curcolor]
{
\newpath
\moveto(136.24023438,343.34752678)
\curveto(136.90820312,343.34752678)(137.55566406,343.18932365)(138.18261719,342.8729174)
\curveto(138.80957031,342.56237053)(139.28710938,342.15807365)(139.61523438,341.66002678)
\curveto(139.93164062,341.1854174)(140.14257812,340.63170647)(140.24804688,339.99889397)
\curveto(140.34179688,339.56530022)(140.38867188,338.87389397)(140.38867188,337.92467522)
\lineto(133.48925781,337.92467522)
\curveto(133.51855469,336.96959709)(133.74414062,336.20201897)(134.16601562,335.62194084)
\curveto(134.58789062,335.04772209)(135.24121094,334.76061272)(136.12597656,334.76061272)
\curveto(136.95214844,334.76061272)(137.61132812,335.03307365)(138.10351562,335.57799553)
\curveto(138.38476562,335.89440178)(138.58398438,336.26061272)(138.70117188,336.67662834)
\lineto(140.25683594,336.67662834)
\curveto(140.21582031,336.33092522)(140.078125,335.94420647)(139.84375,335.51647209)
\curveto(139.61523438,335.09459709)(139.35742188,334.74889397)(139.0703125,334.47936272)
\curveto(138.58984375,334.01061272)(137.99511719,333.69420647)(137.28613281,333.53014397)
\curveto(136.90527344,333.43639397)(136.47460938,333.38951897)(135.99414062,333.38951897)
\curveto(134.82226562,333.38951897)(133.82910156,333.81432365)(133.01464844,334.66393303)
\curveto(132.20019531,335.51940178)(131.79296875,336.71471428)(131.79296875,338.24987053)
\curveto(131.79296875,339.76158928)(132.203125,340.98912834)(133.0234375,341.93248772)
\curveto(133.84375,342.87584709)(134.91601562,343.34752678)(136.24023438,343.34752678)
\closepath
\moveto(138.76269531,339.18151115)
\curveto(138.69824219,339.86705803)(138.54882812,340.41490959)(138.31445312,340.82506584)
\curveto(137.88085938,341.58678459)(137.15722656,341.96764397)(136.14355469,341.96764397)
\curveto(135.41699219,341.96764397)(134.80761719,341.70397209)(134.31542969,341.17662834)
\curveto(133.82324219,340.65514397)(133.5625,339.9901049)(133.53320312,339.18151115)
\lineto(138.76269531,339.18151115)
\closepath
\moveto(136.09082031,343.3651049)
\lineto(136.09082031,343.3651049)
\closepath
}
}
{
\newrgbcolor{curcolor}{0 0 0}
\pscustom[linewidth=2.5,linecolor=curcolor,linestyle=dashed,dash=7.5 2.5]
{
\newpath
\moveto(0,310.57927514)
\lineto(30,310.57927514)
}
}
{
\newrgbcolor{curcolor}{0 0 0}
\pscustom[linestyle=none,fillstyle=solid,fillcolor=curcolor]
{
\newpath
\moveto(47.99804688,310.72846245)
\lineto(46.03808594,316.43256401)
\lineto(43.95507812,310.72846245)
\lineto(47.99804688,310.72846245)
\closepath
\moveto(45.12402344,318.34857963)
\lineto(47.1015625,318.34857963)
\lineto(51.78613281,305.43744682)
\lineto(49.87011719,305.43744682)
\lineto(48.56054688,309.30463432)
\lineto(43.45410156,309.30463432)
\lineto(42.05664062,305.43744682)
\lineto(40.26367188,305.43744682)
\lineto(45.12402344,318.34857963)
\closepath
\moveto(46.02929688,318.34857963)
\lineto(46.02929688,318.34857963)
\closepath
}
}
{
\newrgbcolor{curcolor}{0 0 0}
\pscustom[linestyle=none,fillstyle=solid,fillcolor=curcolor]
{
\newpath
\moveto(56.81347656,315.1229937)
\curveto(57.87402344,315.1229937)(58.73535156,314.8651812)(59.39746094,314.3495562)
\curveto(60.06542969,313.8339312)(60.46679688,312.94623588)(60.6015625,311.68647026)
\lineto(59.06347656,311.68647026)
\curveto(58.96972656,312.26654838)(58.75585938,312.74701713)(58.421875,313.12787651)
\curveto(58.08789062,313.51459526)(57.55175781,313.70795463)(56.81347656,313.70795463)
\curveto(55.80566406,313.70795463)(55.08496094,313.21576713)(54.65136719,312.23139213)
\curveto(54.37011719,311.59272026)(54.22949219,310.80463432)(54.22949219,309.86713432)
\curveto(54.22949219,308.92377495)(54.42871094,308.12982963)(54.82714844,307.48529838)
\curveto(55.22558594,306.84076713)(55.85253906,306.51850151)(56.70800781,306.51850151)
\curveto(57.36425781,306.51850151)(57.8828125,306.71772026)(58.26367188,307.11615776)
\curveto(58.65039062,307.52045463)(58.91699219,308.07123588)(59.06347656,308.76850151)
\lineto(60.6015625,308.76850151)
\curveto(60.42578125,307.52045463)(59.98632812,306.60639213)(59.28320312,306.02631401)
\curveto(58.58007812,305.45209526)(57.68066406,305.16498588)(56.58496094,305.16498588)
\curveto(55.35449219,305.16498588)(54.37304688,305.61322807)(53.640625,306.50971245)
\curveto(52.90820312,307.4120562)(52.54199219,308.5370562)(52.54199219,309.88471245)
\curveto(52.54199219,311.5370562)(52.94335938,312.82318901)(53.74609375,313.74311088)
\curveto(54.54882812,314.66303276)(55.57128906,315.1229937)(56.81347656,315.1229937)
\closepath
\moveto(56.56738281,315.07904838)
\lineto(56.56738281,315.07904838)
\closepath
}
}
{
\newrgbcolor{curcolor}{0 0 0}
\pscustom[linestyle=none,fillstyle=solid,fillcolor=curcolor]
{
\newpath
\moveto(62.5,317.47846245)
\lineto(64.09960938,317.47846245)
\lineto(64.09960938,314.85053276)
\lineto(65.60253906,314.85053276)
\lineto(65.60253906,313.55854057)
\lineto(64.09960938,313.55854057)
\lineto(64.09960938,307.41498588)
\curveto(64.09960938,307.08686088)(64.2109375,306.86713432)(64.43359375,306.7558062)
\curveto(64.55664062,306.69135307)(64.76171875,306.65912651)(65.04882812,306.65912651)
\lineto(65.29492188,306.65912651)
\curveto(65.3828125,306.66498588)(65.48535156,306.67377495)(65.60253906,306.6854937)
\lineto(65.60253906,305.43744682)
\curveto(65.42089844,305.38471245)(65.23046875,305.34662651)(65.03125,305.32318901)
\curveto(64.83789062,305.29975151)(64.62695312,305.28803276)(64.3984375,305.28803276)
\curveto(63.66015625,305.28803276)(63.15917969,305.47553276)(62.89550781,305.85053276)
\curveto(62.63183594,306.23139213)(62.5,306.72357963)(62.5,307.32709526)
\lineto(62.5,313.55854057)
\lineto(61.22558594,313.55854057)
\lineto(61.22558594,314.85053276)
\lineto(62.5,314.85053276)
\lineto(62.5,317.47846245)
\closepath
}
}
{
\newrgbcolor{curcolor}{0 0 0}
\pscustom[linestyle=none,fillstyle=solid,fillcolor=curcolor]
{
\newpath
\moveto(68.79296875,314.85053276)
\lineto(68.79296875,308.60150932)
\curveto(68.79296875,308.12104057)(68.86914062,307.72846245)(69.02148438,307.42377495)
\curveto(69.30273438,306.86127495)(69.82714844,306.58002495)(70.59472656,306.58002495)
\curveto(71.69628906,306.58002495)(72.44628906,307.07221245)(72.84472656,308.05658745)
\curveto(73.06152344,308.5839312)(73.16992188,309.30756401)(73.16992188,310.22748588)
\lineto(73.16992188,314.85053276)
\lineto(74.75195312,314.85053276)
\lineto(74.75195312,305.43744682)
\lineto(73.2578125,305.43744682)
\lineto(73.27539062,306.8261187)
\curveto(73.0703125,306.46869682)(72.81542969,306.16693901)(72.51074219,305.92084526)
\curveto(71.90722656,305.42865776)(71.17480469,305.18256401)(70.31347656,305.18256401)
\curveto(68.97167969,305.18256401)(68.05761719,305.6308062)(67.57128906,306.52729057)
\curveto(67.30761719,307.00775932)(67.17578125,307.64936088)(67.17578125,308.45209526)
\lineto(67.17578125,314.85053276)
\lineto(68.79296875,314.85053276)
\closepath
\moveto(70.96386719,315.07904838)
\lineto(70.96386719,315.07904838)
\closepath
}
}
{
\newrgbcolor{curcolor}{0 0 0}
\pscustom[linestyle=none,fillstyle=solid,fillcolor=curcolor]
{
\newpath
\moveto(78.44335938,307.94232963)
\curveto(78.44335938,307.48529838)(78.61035156,307.12494682)(78.94433594,306.86127495)
\curveto(79.27832031,306.59760307)(79.67382812,306.46576713)(80.13085938,306.46576713)
\curveto(80.6875,306.46576713)(81.2265625,306.59467338)(81.74804688,306.85248588)
\curveto(82.62695312,307.28022026)(83.06640625,307.98041557)(83.06640625,308.95307182)
\lineto(83.06640625,310.22748588)
\curveto(82.87304688,310.10443901)(82.62402344,310.00189995)(82.31933594,309.9198687)
\curveto(82.01464844,309.83783745)(81.71582031,309.7792437)(81.42285156,309.74408745)
\lineto(80.46484375,309.62104057)
\curveto(79.890625,309.5448687)(79.45996094,309.42475151)(79.17285156,309.26068901)
\curveto(78.68652344,308.98529838)(78.44335938,308.54584526)(78.44335938,307.94232963)
\closepath
\moveto(82.27539062,311.14154838)
\curveto(82.63867188,311.18842338)(82.88183594,311.34076713)(83.00488281,311.59857963)
\curveto(83.07519531,311.73920463)(83.11035156,311.94135307)(83.11035156,312.20502495)
\curveto(83.11035156,312.74408745)(82.91699219,313.13373588)(82.53027344,313.37397026)
\curveto(82.14941406,313.62006401)(81.6015625,313.74311088)(80.88671875,313.74311088)
\curveto(80.06054688,313.74311088)(79.47460938,313.52045463)(79.12890625,313.07514213)
\curveto(78.93554688,312.82904838)(78.80957031,312.46283745)(78.75097656,311.97650932)
\lineto(77.27441406,311.97650932)
\curveto(77.30371094,313.13666557)(77.67871094,313.94232963)(78.39941406,314.39350151)
\curveto(79.12597656,314.85053276)(79.96679688,315.07904838)(80.921875,315.07904838)
\curveto(82.02929688,315.07904838)(82.92871094,314.86811088)(83.62011719,314.44623588)
\curveto(84.30566406,314.02436088)(84.6484375,313.36811088)(84.6484375,312.47748588)
\lineto(84.6484375,307.05463432)
\curveto(84.6484375,306.89057182)(84.68066406,306.75873588)(84.74511719,306.65912651)
\curveto(84.81542969,306.55951713)(84.95898438,306.50971245)(85.17578125,306.50971245)
\curveto(85.24609375,306.50971245)(85.32519531,306.51264213)(85.41308594,306.51850151)
\curveto(85.50097656,306.53022026)(85.59472656,306.5448687)(85.69433594,306.56244682)
\lineto(85.69433594,305.39350151)
\curveto(85.44824219,305.32318901)(85.26074219,305.2792437)(85.13183594,305.26166557)
\curveto(85.00292969,305.24408745)(84.82714844,305.23529838)(84.60449219,305.23529838)
\curveto(84.05957031,305.23529838)(83.6640625,305.42865776)(83.41796875,305.81537651)
\curveto(83.2890625,306.02045463)(83.19824219,306.3104937)(83.14550781,306.6854937)
\curveto(82.82324219,306.2636187)(82.36035156,305.89740776)(81.75683594,305.58686088)
\curveto(81.15332031,305.27631401)(80.48828125,305.12104057)(79.76171875,305.12104057)
\curveto(78.88867188,305.12104057)(78.17382812,305.38471245)(77.6171875,305.9120562)
\curveto(77.06640625,306.44525932)(76.79101562,307.11029838)(76.79101562,307.90717338)
\curveto(76.79101562,308.78022026)(77.06347656,309.45697807)(77.60839844,309.93744682)
\curveto(78.15332031,310.41791557)(78.86816406,310.71381401)(79.75292969,310.82514213)
\lineto(82.27539062,311.14154838)
\closepath
\moveto(80.96582031,315.07904838)
\lineto(80.96582031,315.07904838)
\closepath
}
}
{
\newrgbcolor{curcolor}{0 0 0}
\pscustom[linestyle=none,fillstyle=solid,fillcolor=curcolor]
{
\newpath
\moveto(87.29394531,318.34857963)
\lineto(88.87597656,318.34857963)
\lineto(88.87597656,305.43744682)
\lineto(87.29394531,305.43744682)
\lineto(87.29394531,318.34857963)
\closepath
}
}
{
\newrgbcolor{curcolor}{0 0 0}
\pscustom[linestyle=none,fillstyle=solid,fillcolor=curcolor]
{
\newpath
\moveto(96.25,318.39252495)
\lineto(97.83203125,318.39252495)
\lineto(97.83203125,313.5761187)
\curveto(98.20703125,314.05072807)(98.54394531,314.38471245)(98.84277344,314.57807182)
\curveto(99.35253906,314.9120562)(99.98828125,315.07904838)(100.75,315.07904838)
\curveto(102.11523438,315.07904838)(103.04101562,314.60150932)(103.52734375,313.6464312)
\curveto(103.79101562,313.12494682)(103.92285156,312.40131401)(103.92285156,311.47553276)
\lineto(103.92285156,305.43744682)
\lineto(102.296875,305.43744682)
\lineto(102.296875,311.37006401)
\curveto(102.296875,312.06147026)(102.20898438,312.5683062)(102.03320312,312.89057182)
\curveto(101.74609375,313.40619682)(101.20703125,313.66400932)(100.41601562,313.66400932)
\curveto(99.75976562,313.66400932)(99.16503906,313.43842338)(98.63183594,312.98725151)
\curveto(98.09863281,312.53607963)(97.83203125,311.68354057)(97.83203125,310.42963432)
\lineto(97.83203125,305.43744682)
\lineto(96.25,305.43744682)
\lineto(96.25,318.39252495)
\closepath
}
}
{
\newrgbcolor{curcolor}{0 0 0}
\pscustom[linestyle=none,fillstyle=solid,fillcolor=curcolor]
{
\newpath
\moveto(110.00488281,306.45697807)
\curveto(111.05371094,306.45697807)(111.77148438,306.85248588)(112.15820312,307.64350151)
\curveto(112.55078125,308.44037651)(112.74707031,309.32514213)(112.74707031,310.29779838)
\curveto(112.74707031,311.17670463)(112.60644531,311.89154838)(112.32519531,312.44232963)
\curveto(111.87988281,313.30951713)(111.11230469,313.74311088)(110.02246094,313.74311088)
\curveto(109.05566406,313.74311088)(108.35253906,313.37397026)(107.91308594,312.63568901)
\curveto(107.47363281,311.89740776)(107.25390625,311.00678276)(107.25390625,309.96381401)
\curveto(107.25390625,308.96186088)(107.47363281,308.12689995)(107.91308594,307.4589312)
\curveto(108.35253906,306.79096245)(109.04980469,306.45697807)(110.00488281,306.45697807)
\closepath
\moveto(110.06640625,315.1229937)
\curveto(111.27929688,315.1229937)(112.3046875,314.71869682)(113.14257812,313.91010307)
\curveto(113.98046875,313.10150932)(114.39941406,311.9120562)(114.39941406,310.3417437)
\curveto(114.39941406,308.82416557)(114.03027344,307.57025932)(113.29199219,306.58002495)
\curveto(112.55371094,305.58979057)(111.40820312,305.09467338)(109.85546875,305.09467338)
\curveto(108.56054688,305.09467338)(107.53222656,305.53119682)(106.77050781,306.4042437)
\curveto(106.00878906,307.28314995)(105.62792969,308.46088432)(105.62792969,309.93744682)
\curveto(105.62792969,311.51947807)(106.02929688,312.7792437)(106.83203125,313.7167437)
\curveto(107.63476562,314.6542437)(108.71289062,315.1229937)(110.06640625,315.1229937)
\closepath
\moveto(110.01367188,315.07904838)
\lineto(110.01367188,315.07904838)
\closepath
}
}
{
\newrgbcolor{curcolor}{0 0 0}
\pscustom[linestyle=none,fillstyle=solid,fillcolor=curcolor]
{
\newpath
\moveto(116.33300781,318.34857963)
\lineto(117.91503906,318.34857963)
\lineto(117.91503906,305.43744682)
\lineto(116.33300781,305.43744682)
\lineto(116.33300781,318.34857963)
\closepath
}
}
{
\newrgbcolor{curcolor}{0 0 0}
\pscustom[linestyle=none,fillstyle=solid,fillcolor=curcolor]
{
\newpath
\moveto(124.21679688,315.06147026)
\curveto(124.88476562,315.06147026)(125.53222656,314.90326713)(126.15917969,314.58686088)
\curveto(126.78613281,314.27631401)(127.26367188,313.87201713)(127.59179688,313.37397026)
\curveto(127.90820312,312.89936088)(128.11914062,312.34564995)(128.22460938,311.71283745)
\curveto(128.31835938,311.2792437)(128.36523438,310.58783745)(128.36523438,309.6386187)
\lineto(121.46582031,309.6386187)
\curveto(121.49511719,308.68354057)(121.72070312,307.91596245)(122.14257812,307.33588432)
\curveto(122.56445312,306.76166557)(123.21777344,306.4745562)(124.10253906,306.4745562)
\curveto(124.92871094,306.4745562)(125.58789062,306.74701713)(126.08007812,307.29193901)
\curveto(126.36132812,307.60834526)(126.56054688,307.9745562)(126.67773438,308.39057182)
\lineto(128.23339844,308.39057182)
\curveto(128.19238281,308.0448687)(128.0546875,307.65814995)(127.8203125,307.23041557)
\curveto(127.59179688,306.80854057)(127.33398438,306.46283745)(127.046875,306.1933062)
\curveto(126.56640625,305.7245562)(125.97167969,305.40814995)(125.26269531,305.24408745)
\curveto(124.88183594,305.15033745)(124.45117188,305.10346245)(123.97070312,305.10346245)
\curveto(122.79882812,305.10346245)(121.80566406,305.52826713)(120.99121094,306.37787651)
\curveto(120.17675781,307.23334526)(119.76953125,308.42865776)(119.76953125,309.96381401)
\curveto(119.76953125,311.47553276)(120.1796875,312.70307182)(121,313.6464312)
\curveto(121.8203125,314.58979057)(122.89257812,315.06147026)(124.21679688,315.06147026)
\closepath
\moveto(126.73925781,310.89545463)
\curveto(126.67480469,311.58100151)(126.52539062,312.12885307)(126.29101562,312.53900932)
\curveto(125.85742188,313.30072807)(125.13378906,313.68158745)(124.12011719,313.68158745)
\curveto(123.39355469,313.68158745)(122.78417969,313.41791557)(122.29199219,312.89057182)
\curveto(121.79980469,312.36908745)(121.5390625,311.70404838)(121.50976562,310.89545463)
\lineto(126.73925781,310.89545463)
\closepath
\moveto(124.06738281,315.07904838)
\lineto(124.06738281,315.07904838)
\closepath
}
}
\end{pspicture}

    \caption{Desired Hole vs Actual Hole}
  \end{figure}
\end{frame}

\section{Overhangs}
\begin{frame}
  \frametitle{Overhangs}
  \begin{itemize}
    \item For some types of printers (typically those that fuse a powder), overhangs and unsupported sections are not a problem.  For the rest, something may need to be done.
    \item A rule of thumb is that overhangs with an overhang angle less that 45\degree\ can print without too much trouble while larger angles will require support.
    \item There are test objects available online with varying overhang angles that you can use to see what your printer can do.
  \end{itemize}
\end{frame}

\begin{frame}
  \frametitle{Angle of Overhangs}
  The angle of overhang is important in determining if your print will succeed or not.
  \begin{figure}
    %LaTeX with PSTricks extensions
%%Creator: inkscape 0.91
%%Please note this file requires PSTricks extensions
\psset{xunit=.5pt,yunit=.5pt,runit=.5pt}
\begin{pspicture}(296.88089025,240.80906665)
{
\newrgbcolor{curcolor}{1 1 1}
\pscustom[linestyle=none,fillstyle=solid,fillcolor=curcolor,opacity=0]
{
\newpath
\moveto(91.76619538,202.64898195)
\curveto(108.63986036,202.78352934)(124.28808196,189.92578294)(132.84250993,168.89761715)
}
}
{
\newrgbcolor{curcolor}{0 0 0}
\pscustom[linewidth=1.75704265,linecolor=curcolor]
{
\newpath
\moveto(91.76619538,202.64898195)
\curveto(108.63986036,202.78352934)(124.28808196,189.92578294)(132.84250993,168.89761715)
}
}
{
\newrgbcolor{curcolor}{0 0 0}
\pscustom[linewidth=0.52859384,linecolor=curcolor]
{
\newpath
\moveto(0,23.55328665)
\lineto(181.61748,23.55328665)
}
}
{
\newrgbcolor{curcolor}{0 0 0}
\pscustom[linewidth=10,linecolor=curcolor]
{
\newpath
\moveto(91.8644,23.98142665)
\lineto(91.8644,133.98142665)
\lineto(161.8644,203.98142665)
}
}
{
\newrgbcolor{curcolor}{0 0 0}
\pscustom[linewidth=1,linecolor=curcolor]
{
\newpath
\moveto(91.8644,133.98142665)
\lineto(91.8644,223.98142665)
}
}
{
\newrgbcolor{curcolor}{0 0 0}
\pscustom[linestyle=none,fillstyle=solid,fillcolor=curcolor]
{
\newpath
\moveto(53.40248594,16.89257685)
\lineto(59.21205625,16.89257685)
\curveto(60.36049375,16.89257685)(61.286275,16.56738154)(61.9894,15.91699091)
\curveto(62.692525,15.27245966)(63.0440875,14.36425654)(63.0440875,13.19238154)
\curveto(63.0440875,12.18456904)(62.73061094,11.30566279)(62.10365781,10.55566279)
\curveto(61.47670469,9.81152216)(60.5128375,9.43945185)(59.21205625,9.43945185)
\lineto(55.15150937,9.43945185)
\lineto(55.15150937,3.98144404)
\lineto(53.40248594,3.98144404)
\lineto(53.40248594,16.89257685)
\closepath
\moveto(61.27748594,13.18359247)
\curveto(61.27748594,14.13281122)(60.92592344,14.77734247)(60.22279844,15.11718622)
\curveto(59.83607969,15.29882685)(59.30580625,15.38964716)(58.63197812,15.38964716)
\lineto(55.15150937,15.38964716)
\lineto(55.15150937,10.91601435)
\lineto(58.63197812,10.91601435)
\curveto(59.41713437,10.91601435)(60.05287656,11.08300654)(60.53920469,11.41699091)
\curveto(61.03139219,11.75097529)(61.27748594,12.33984247)(61.27748594,13.18359247)
\closepath
}
}
{
\newrgbcolor{curcolor}{0 0 0}
\pscustom[linestyle=none,fillstyle=solid,fillcolor=curcolor]
{
\newpath
\moveto(65.09193906,13.39452997)
\lineto(66.59486875,13.39452997)
\lineto(66.59486875,11.76855341)
\curveto(66.71791562,12.08495966)(67.01967344,12.46874872)(67.50014219,12.9199206)
\curveto(67.98061094,13.37695185)(68.53432187,13.60546747)(69.161275,13.60546747)
\curveto(69.19057187,13.60546747)(69.24037656,13.60253779)(69.31068906,13.59667841)
\curveto(69.38100156,13.59081904)(69.50111875,13.57910029)(69.67104062,13.56152216)
\lineto(69.67104062,11.89160029)
\curveto(69.57729062,11.90917841)(69.4894,11.92089716)(69.40736875,11.92675654)
\curveto(69.33119687,11.93261591)(69.24623594,11.9355456)(69.15248594,11.9355456)
\curveto(68.35561094,11.9355456)(67.74330625,11.6777331)(67.31557187,11.1621081)
\curveto(66.8878375,10.65234247)(66.67397031,10.06347529)(66.67397031,9.39550654)
\lineto(66.67397031,3.98144404)
\lineto(65.09193906,3.98144404)
\lineto(65.09193906,13.39452997)
\closepath
}
}
{
\newrgbcolor{curcolor}{0 0 0}
\pscustom[linestyle=none,fillstyle=solid,fillcolor=curcolor]
{
\newpath
\moveto(71.0597125,13.35058466)
\lineto(72.66811094,13.35058466)
\lineto(72.66811094,3.98144404)
\lineto(71.0597125,3.98144404)
\lineto(71.0597125,13.35058466)
\closepath
\moveto(71.0597125,16.89257685)
\lineto(72.66811094,16.89257685)
\lineto(72.66811094,15.0996081)
\lineto(71.0597125,15.0996081)
\lineto(71.0597125,16.89257685)
\closepath
}
}
{
\newrgbcolor{curcolor}{0 0 0}
\pscustom[linestyle=none,fillstyle=solid,fillcolor=curcolor]
{
\newpath
\moveto(75.067525,13.39452997)
\lineto(76.57045469,13.39452997)
\lineto(76.57045469,12.05859247)
\curveto(77.01576719,12.60937372)(77.48744687,13.00488154)(77.98549375,13.24511591)
\curveto(78.48354062,13.48535029)(79.03725156,13.60546747)(79.64662656,13.60546747)
\curveto(80.98256406,13.60546747)(81.88490781,13.13964716)(82.35365781,12.20800654)
\curveto(82.61147031,11.69824091)(82.74037656,10.96874872)(82.74037656,10.01952997)
\lineto(82.74037656,3.98144404)
\lineto(81.13197812,3.98144404)
\lineto(81.13197812,9.91406122)
\curveto(81.13197812,10.48827997)(81.04701719,10.9511706)(80.87709531,11.3027331)
\curveto(80.59584531,11.8886706)(80.08607969,12.18163935)(79.34779844,12.18163935)
\curveto(78.97279844,12.18163935)(78.66518125,12.14355341)(78.42494687,12.06738154)
\curveto(77.99135312,11.93847529)(77.61049375,11.68066279)(77.28236875,11.29394404)
\curveto(77.01869687,10.98339716)(76.84584531,10.66113154)(76.76381406,10.32714716)
\curveto(76.68764219,9.99902216)(76.64955625,9.52734247)(76.64955625,8.9121081)
\lineto(76.64955625,3.98144404)
\lineto(75.067525,3.98144404)
\lineto(75.067525,13.39452997)
\closepath
\moveto(78.78529844,13.6230456)
\lineto(78.78529844,13.6230456)
\closepath
}
}
{
\newrgbcolor{curcolor}{0 0 0}
\pscustom[linestyle=none,fillstyle=solid,fillcolor=curcolor]
{
\newpath
\moveto(85.4034625,16.02245966)
\lineto(87.00307187,16.02245966)
\lineto(87.00307187,13.39452997)
\lineto(88.50600156,13.39452997)
\lineto(88.50600156,12.10253779)
\lineto(87.00307187,12.10253779)
\lineto(87.00307187,5.9589831)
\curveto(87.00307187,5.6308581)(87.1144,5.41113154)(87.33705625,5.29980341)
\curveto(87.46010312,5.23535029)(87.66518125,5.20312372)(87.95229062,5.20312372)
\lineto(88.19838437,5.20312372)
\curveto(88.286275,5.2089831)(88.38881406,5.21777216)(88.50600156,5.22949091)
\lineto(88.50600156,3.98144404)
\curveto(88.32436094,3.92870966)(88.13393125,3.89062372)(87.9347125,3.86718622)
\curveto(87.74135312,3.84374872)(87.53041562,3.83202997)(87.3019,3.83202997)
\curveto(86.56361875,3.83202997)(86.06264219,4.01952997)(85.79897031,4.39452997)
\curveto(85.53529844,4.77538935)(85.4034625,5.26757685)(85.4034625,5.87109247)
\lineto(85.4034625,12.10253779)
\lineto(84.12904844,12.10253779)
\lineto(84.12904844,13.39452997)
\lineto(85.4034625,13.39452997)
\lineto(85.4034625,16.02245966)
\closepath
}
}
{
\newrgbcolor{curcolor}{0 0 0}
\pscustom[linestyle=none,fillstyle=solid,fillcolor=curcolor]
{
\newpath
\moveto(90.1144,13.35058466)
\lineto(91.72279844,13.35058466)
\lineto(91.72279844,3.98144404)
\lineto(90.1144,3.98144404)
\lineto(90.1144,13.35058466)
\closepath
\moveto(90.1144,16.89257685)
\lineto(91.72279844,16.89257685)
\lineto(91.72279844,15.0996081)
\lineto(90.1144,15.0996081)
\lineto(90.1144,16.89257685)
\closepath
}
}
{
\newrgbcolor{curcolor}{0 0 0}
\pscustom[linestyle=none,fillstyle=solid,fillcolor=curcolor]
{
\newpath
\moveto(94.1222125,13.39452997)
\lineto(95.62514219,13.39452997)
\lineto(95.62514219,12.05859247)
\curveto(96.07045469,12.60937372)(96.54213437,13.00488154)(97.04018125,13.24511591)
\curveto(97.53822812,13.48535029)(98.09193906,13.60546747)(98.70131406,13.60546747)
\curveto(100.03725156,13.60546747)(100.93959531,13.13964716)(101.40834531,12.20800654)
\curveto(101.66615781,11.69824091)(101.79506406,10.96874872)(101.79506406,10.01952997)
\lineto(101.79506406,3.98144404)
\lineto(100.18666562,3.98144404)
\lineto(100.18666562,9.91406122)
\curveto(100.18666562,10.48827997)(100.10170469,10.9511706)(99.93178281,11.3027331)
\curveto(99.65053281,11.8886706)(99.14076719,12.18163935)(98.40248594,12.18163935)
\curveto(98.02748594,12.18163935)(97.71986875,12.14355341)(97.47963437,12.06738154)
\curveto(97.04604062,11.93847529)(96.66518125,11.68066279)(96.33705625,11.29394404)
\curveto(96.07338437,10.98339716)(95.90053281,10.66113154)(95.81850156,10.32714716)
\curveto(95.74232969,9.99902216)(95.70424375,9.52734247)(95.70424375,8.9121081)
\lineto(95.70424375,3.98144404)
\lineto(94.1222125,3.98144404)
\lineto(94.1222125,13.39452997)
\closepath
\moveto(97.83998594,13.6230456)
\lineto(97.83998594,13.6230456)
\closepath
}
}
{
\newrgbcolor{curcolor}{0 0 0}
\pscustom[linestyle=none,fillstyle=solid,fillcolor=curcolor]
{
\newpath
\moveto(107.46400937,13.56152216)
\curveto(108.20229062,13.56152216)(108.84682187,13.37988154)(109.39760312,13.01660029)
\curveto(109.69643125,12.81152216)(110.00111875,12.51269404)(110.31166562,12.12011591)
\lineto(110.31166562,13.30663935)
\lineto(111.77065,13.30663935)
\lineto(111.77065,4.74609247)
\curveto(111.77065,3.55077997)(111.59486875,2.6074206)(111.24330625,1.91601435)
\curveto(110.58705625,0.6386706)(109.34779844,-0.00000128)(107.52553281,-0.00000128)
\curveto(106.51186094,-0.00000128)(105.65932187,0.22851435)(104.96791562,0.6855456)
\curveto(104.27650937,1.13671747)(103.88979062,1.84570185)(103.80775937,2.81249872)
\lineto(105.41615781,2.81249872)
\curveto(105.49232969,2.39062372)(105.64467344,2.06542841)(105.87318906,1.83691279)
\curveto(106.23061094,1.48535029)(106.79311094,1.30956904)(107.56068906,1.30956904)
\curveto(108.77357969,1.30956904)(109.567525,1.73730341)(109.942525,2.59277216)
\curveto(110.16518125,3.09667841)(110.26772031,3.99609247)(110.25014219,5.29101435)
\curveto(109.93373594,4.8105456)(109.55287656,4.45312372)(109.10756406,4.21874872)
\curveto(108.66225156,3.98437372)(108.07338437,3.86718622)(107.3409625,3.86718622)
\curveto(106.32143125,3.86718622)(105.42787656,4.22753779)(104.66029844,4.94824091)
\curveto(103.89857969,5.67480341)(103.51772031,6.8730456)(103.51772031,8.54296747)
\curveto(103.51772031,10.11913935)(103.90150937,11.3496081)(104.6690875,12.23437372)
\curveto(105.442525,13.11913935)(106.37416562,13.56152216)(107.46400937,13.56152216)
\closepath
\moveto(110.31166562,8.72753779)
\curveto(110.31166562,9.89355341)(110.07143125,10.75781122)(109.5909625,11.32031122)
\curveto(109.11049375,11.88281122)(108.49818906,12.16406122)(107.75404844,12.16406122)
\curveto(106.64076719,12.16406122)(105.87904844,11.64257685)(105.46889219,10.5996081)
\curveto(105.25209531,10.04296747)(105.14369687,9.31347529)(105.14369687,8.41113154)
\curveto(105.14369687,7.35058466)(105.35756406,6.54199091)(105.78529844,5.98535029)
\curveto(106.21889219,5.43456904)(106.79897031,5.15917841)(107.52553281,5.15917841)
\curveto(108.66225156,5.15917841)(109.46205625,5.67187372)(109.92494687,6.69726435)
\curveto(110.18275937,7.27734247)(110.31166562,7.95410029)(110.31166562,8.72753779)
\closepath
\moveto(107.64857969,13.6230456)
\lineto(107.64857969,13.6230456)
\closepath
}
}
{
\newrgbcolor{curcolor}{0 0 0}
\pscustom[linestyle=none,fillstyle=solid,fillcolor=curcolor]
{
\newpath
\moveto(123.12611875,5.0449206)
\curveto(123.8644,5.0449206)(124.47670469,5.35253779)(124.96303281,5.96777216)
\curveto(125.45522031,6.58886591)(125.70131406,7.51464716)(125.70131406,8.74511591)
\curveto(125.70131406,9.49511591)(125.59291562,10.13964716)(125.37611875,10.67870966)
\curveto(124.9659625,11.71581904)(124.2159625,12.23437372)(123.12611875,12.23437372)
\curveto(122.03041562,12.23437372)(121.28041562,11.68652216)(120.87611875,10.59081904)
\curveto(120.65932187,10.00488154)(120.55092344,9.26074091)(120.55092344,8.35839716)
\curveto(120.55092344,7.63183466)(120.65932187,7.0136706)(120.87611875,6.50390497)
\curveto(121.286275,5.53124872)(122.036275,5.0449206)(123.12611875,5.0449206)
\closepath
\moveto(119.03041562,13.35058466)
\lineto(120.56850156,13.35058466)
\lineto(120.56850156,12.10253779)
\curveto(120.88490781,12.53027216)(121.23061094,12.86132685)(121.60561094,13.09570185)
\curveto(122.13881406,13.44726435)(122.76576719,13.6230456)(123.48647031,13.6230456)
\curveto(124.55287656,13.6230456)(125.45815,13.21288935)(126.20229062,12.39257685)
\curveto(126.94643125,11.57812372)(127.31850156,10.4121081)(127.31850156,8.89452997)
\curveto(127.31850156,6.84374872)(126.78236875,5.37890497)(125.71010312,4.49999872)
\curveto(125.03041562,3.9433581)(124.2394,3.66503779)(123.33705625,3.66503779)
\curveto(122.62807187,3.66503779)(122.03334531,3.82031122)(121.55287656,4.1308581)
\curveto(121.27162656,4.30663935)(120.95815,4.60839716)(120.61244687,5.03613154)
\lineto(120.61244687,0.22851435)
\lineto(119.03041562,0.22851435)
\lineto(119.03041562,13.35058466)
\closepath
}
}
{
\newrgbcolor{curcolor}{0 0 0}
\pscustom[linestyle=none,fillstyle=solid,fillcolor=curcolor]
{
\newpath
\moveto(129.21693906,16.89257685)
\lineto(130.79897031,16.89257685)
\lineto(130.79897031,3.98144404)
\lineto(129.21693906,3.98144404)
\lineto(129.21693906,16.89257685)
\closepath
}
}
{
\newrgbcolor{curcolor}{0 0 0}
\pscustom[linestyle=none,fillstyle=solid,fillcolor=curcolor]
{
\newpath
\moveto(134.39369687,6.48632685)
\curveto(134.39369687,6.0292956)(134.56068906,5.66894404)(134.89467344,5.40527216)
\curveto(135.22865781,5.14160029)(135.62416562,5.00976435)(136.08119687,5.00976435)
\curveto(136.6378375,5.00976435)(137.1769,5.1386706)(137.69838437,5.3964831)
\curveto(138.57729062,5.82421747)(139.01674375,6.52441279)(139.01674375,7.49706904)
\lineto(139.01674375,8.7714831)
\curveto(138.82338437,8.64843622)(138.57436094,8.54589716)(138.26967344,8.46386591)
\curveto(137.96498594,8.38183466)(137.66615781,8.32324091)(137.37318906,8.28808466)
\lineto(136.41518125,8.16503779)
\curveto(135.8409625,8.08886591)(135.41029844,7.96874872)(135.12318906,7.80468622)
\curveto(134.63686094,7.5292956)(134.39369687,7.08984247)(134.39369687,6.48632685)
\closepath
\moveto(138.22572812,9.6855456)
\curveto(138.58900937,9.7324206)(138.83217344,9.88476435)(138.95522031,10.14257685)
\curveto(139.02553281,10.28320185)(139.06068906,10.48535029)(139.06068906,10.74902216)
\curveto(139.06068906,11.28808466)(138.86732969,11.6777331)(138.48061094,11.91796747)
\curveto(138.09975156,12.16406122)(137.5519,12.2871081)(136.83705625,12.2871081)
\curveto(136.01088437,12.2871081)(135.42494687,12.06445185)(135.07924375,11.61913935)
\curveto(134.88588437,11.3730456)(134.75990781,11.00683466)(134.70131406,10.52050654)
\lineto(133.22475156,10.52050654)
\curveto(133.25404844,11.68066279)(133.62904844,12.48632685)(134.34975156,12.93749872)
\curveto(135.07631406,13.39452997)(135.91713437,13.6230456)(136.8722125,13.6230456)
\curveto(137.97963437,13.6230456)(138.87904844,13.4121081)(139.57045469,12.9902331)
\curveto(140.25600156,12.5683581)(140.598775,11.9121081)(140.598775,11.0214831)
\lineto(140.598775,5.59863154)
\curveto(140.598775,5.43456904)(140.63100156,5.3027331)(140.69545469,5.20312372)
\curveto(140.76576719,5.10351435)(140.90932187,5.05370966)(141.12611875,5.05370966)
\curveto(141.19643125,5.05370966)(141.27553281,5.05663935)(141.36342344,5.06249872)
\curveto(141.45131406,5.07421747)(141.54506406,5.08886591)(141.64467344,5.10644404)
\lineto(141.64467344,3.93749872)
\curveto(141.39857969,3.86718622)(141.21107969,3.82324091)(141.08217344,3.80566279)
\curveto(140.95326719,3.78808466)(140.77748594,3.7792956)(140.55482969,3.7792956)
\curveto(140.00990781,3.7792956)(139.6144,3.97265497)(139.36830625,4.35937372)
\curveto(139.2394,4.56445185)(139.14857969,4.85449091)(139.09584531,5.22949091)
\curveto(138.77357969,4.80761591)(138.31068906,4.44140497)(137.70717344,4.1308581)
\curveto(137.10365781,3.82031122)(136.43861875,3.66503779)(135.71205625,3.66503779)
\curveto(134.83900937,3.66503779)(134.12416562,3.92870966)(133.567525,4.45605341)
\curveto(133.01674375,4.98925654)(132.74135312,5.6542956)(132.74135312,6.4511706)
\curveto(132.74135312,7.32421747)(133.01381406,8.00097529)(133.55873594,8.48144404)
\curveto(134.10365781,8.96191279)(134.81850156,9.25781122)(135.70326719,9.36913935)
\lineto(138.22572812,9.6855456)
\closepath
\moveto(136.91615781,13.6230456)
\lineto(136.91615781,13.6230456)
\closepath
}
}
{
\newrgbcolor{curcolor}{0 0 0}
\pscustom[linestyle=none,fillstyle=solid,fillcolor=curcolor]
{
\newpath
\moveto(143.51674375,16.02245966)
\lineto(145.11635312,16.02245966)
\lineto(145.11635312,13.39452997)
\lineto(146.61928281,13.39452997)
\lineto(146.61928281,12.10253779)
\lineto(145.11635312,12.10253779)
\lineto(145.11635312,5.9589831)
\curveto(145.11635312,5.6308581)(145.22768125,5.41113154)(145.4503375,5.29980341)
\curveto(145.57338437,5.23535029)(145.7784625,5.20312372)(146.06557187,5.20312372)
\lineto(146.31166562,5.20312372)
\curveto(146.39955625,5.2089831)(146.50209531,5.21777216)(146.61928281,5.22949091)
\lineto(146.61928281,3.98144404)
\curveto(146.43764219,3.92870966)(146.2472125,3.89062372)(146.04799375,3.86718622)
\curveto(145.85463437,3.84374872)(145.64369687,3.83202997)(145.41518125,3.83202997)
\curveto(144.6769,3.83202997)(144.17592344,4.01952997)(143.91225156,4.39452997)
\curveto(143.64857969,4.77538935)(143.51674375,5.26757685)(143.51674375,5.87109247)
\lineto(143.51674375,12.10253779)
\lineto(142.24232969,12.10253779)
\lineto(142.24232969,13.39452997)
\lineto(143.51674375,13.39452997)
\lineto(143.51674375,16.02245966)
\closepath
}
}
{
\newrgbcolor{curcolor}{0 0 0}
\pscustom[linestyle=none,fillstyle=solid,fillcolor=curcolor]
{
\newpath
\moveto(148.62318906,14.82714716)
\curveto(148.64662656,15.48339716)(148.76088437,15.96386591)(148.9659625,16.26855341)
\curveto(149.33510312,16.80761591)(150.04701719,17.07714716)(151.10170469,17.07714716)
\curveto(151.20131406,17.07714716)(151.30385312,17.07421747)(151.40932187,17.0683581)
\curveto(151.51479062,17.06249872)(151.63490781,17.05370966)(151.76967344,17.04199091)
\lineto(151.76967344,15.60058466)
\curveto(151.60561094,15.61230341)(151.48549375,15.61816279)(151.40932187,15.61816279)
\curveto(151.33900937,15.62402216)(151.27162656,15.62695185)(151.20717344,15.62695185)
\curveto(150.72670469,15.62695185)(150.43959531,15.50097529)(150.34584531,15.24902216)
\curveto(150.25209531,15.00292841)(150.20522031,14.37011591)(150.20522031,13.35058466)
\lineto(151.76967344,13.35058466)
\lineto(151.76967344,12.10253779)
\lineto(150.18764219,12.10253779)
\lineto(150.18764219,3.98144404)
\lineto(148.62318906,3.98144404)
\lineto(148.62318906,12.10253779)
\lineto(147.31361875,12.10253779)
\lineto(147.31361875,13.35058466)
\lineto(148.62318906,13.35058466)
\lineto(148.62318906,14.82714716)
\closepath
}
}
{
\newrgbcolor{curcolor}{0 0 0}
\pscustom[linestyle=none,fillstyle=solid,fillcolor=curcolor]
{
\newpath
\moveto(156.99037656,5.00097529)
\curveto(158.03920469,5.00097529)(158.75697812,5.3964831)(159.14369687,6.18749872)
\curveto(159.536275,6.98437372)(159.73256406,7.86913935)(159.73256406,8.8417956)
\curveto(159.73256406,9.72070185)(159.59193906,10.4355456)(159.31068906,10.98632685)
\curveto(158.86537656,11.85351435)(158.09779844,12.2871081)(157.00795469,12.2871081)
\curveto(156.04115781,12.2871081)(155.33803281,11.91796747)(154.89857969,11.17968622)
\curveto(154.45912656,10.44140497)(154.2394,9.55077997)(154.2394,8.50781122)
\curveto(154.2394,7.5058581)(154.45912656,6.67089716)(154.89857969,6.00292841)
\curveto(155.33803281,5.33495966)(156.03529844,5.00097529)(156.99037656,5.00097529)
\closepath
\moveto(157.0519,13.66699091)
\curveto(158.26479062,13.66699091)(159.29018125,13.26269404)(160.12807187,12.45410029)
\curveto(160.9659625,11.64550654)(161.38490781,10.45605341)(161.38490781,8.88574091)
\curveto(161.38490781,7.36816279)(161.01576719,6.11425654)(160.27748594,5.12402216)
\curveto(159.53920469,4.13378779)(158.39369687,3.6386706)(156.8409625,3.6386706)
\curveto(155.54604062,3.6386706)(154.51772031,4.07519404)(153.75600156,4.94824091)
\curveto(152.99428281,5.82714716)(152.61342344,7.00488154)(152.61342344,8.48144404)
\curveto(152.61342344,10.06347529)(153.01479062,11.32324091)(153.817525,12.26074091)
\curveto(154.62025937,13.19824091)(155.69838437,13.66699091)(157.0519,13.66699091)
\closepath
\moveto(156.99916562,13.6230456)
\lineto(156.99916562,13.6230456)
\closepath
}
}
{
\newrgbcolor{curcolor}{0 0 0}
\pscustom[linestyle=none,fillstyle=solid,fillcolor=curcolor]
{
\newpath
\moveto(163.31850156,13.39452997)
\lineto(164.82143125,13.39452997)
\lineto(164.82143125,11.76855341)
\curveto(164.94447812,12.08495966)(165.24623594,12.46874872)(165.72670469,12.9199206)
\curveto(166.20717344,13.37695185)(166.76088437,13.60546747)(167.3878375,13.60546747)
\curveto(167.41713437,13.60546747)(167.46693906,13.60253779)(167.53725156,13.59667841)
\curveto(167.60756406,13.59081904)(167.72768125,13.57910029)(167.89760312,13.56152216)
\lineto(167.89760312,11.89160029)
\curveto(167.80385312,11.90917841)(167.7159625,11.92089716)(167.63393125,11.92675654)
\curveto(167.55775937,11.93261591)(167.47279844,11.9355456)(167.37904844,11.9355456)
\curveto(166.58217344,11.9355456)(165.96986875,11.6777331)(165.54213437,11.1621081)
\curveto(165.1144,10.65234247)(164.90053281,10.06347529)(164.90053281,9.39550654)
\lineto(164.90053281,3.98144404)
\lineto(163.31850156,3.98144404)
\lineto(163.31850156,13.39452997)
\closepath
}
}
{
\newrgbcolor{curcolor}{0 0 0}
\pscustom[linestyle=none,fillstyle=solid,fillcolor=curcolor]
{
\newpath
\moveto(169.286275,13.39452997)
\lineto(170.85072812,13.39452997)
\lineto(170.85072812,12.05859247)
\curveto(171.22572812,12.5214831)(171.56557187,12.85839716)(171.87025937,13.06933466)
\curveto(172.39174375,13.42675654)(172.98354062,13.60546747)(173.64565,13.60546747)
\curveto(174.39565,13.60546747)(174.99916562,13.42089716)(175.45619687,13.05175654)
\curveto(175.71400937,12.84081904)(175.94838437,12.53027216)(176.15932187,12.12011591)
\curveto(176.51088437,12.62402216)(176.92397031,12.99609247)(177.39857969,13.23632685)
\curveto(177.87318906,13.4824206)(178.40639219,13.60546747)(178.99818906,13.60546747)
\curveto(180.26381406,13.60546747)(181.12514219,13.14843622)(181.58217344,12.23437372)
\curveto(181.82826719,11.74218622)(181.95131406,11.08007685)(181.95131406,10.2480456)
\lineto(181.95131406,3.98144404)
\lineto(180.30775937,3.98144404)
\lineto(180.30775937,10.52050654)
\curveto(180.30775937,11.14745966)(180.14955625,11.57812372)(179.83315,11.81249872)
\curveto(179.52260312,12.04687372)(179.14174375,12.16406122)(178.69057187,12.16406122)
\curveto(178.06947812,12.16406122)(177.53334531,11.95605341)(177.08217344,11.54003779)
\curveto(176.63686094,11.12402216)(176.41420469,10.42968622)(176.41420469,9.45702997)
\lineto(176.41420469,3.98144404)
\lineto(174.80580625,3.98144404)
\lineto(174.80580625,10.12499872)
\curveto(174.80580625,10.7636706)(174.72963437,11.22949091)(174.57729062,11.52245966)
\curveto(174.33705625,11.96191279)(173.88881406,12.18163935)(173.23256406,12.18163935)
\curveto(172.63490781,12.18163935)(172.08998594,11.95019404)(171.59779844,11.48730341)
\curveto(171.11147031,11.02441279)(170.86830625,10.18652216)(170.86830625,8.97363154)
\lineto(170.86830625,3.98144404)
\lineto(169.286275,3.98144404)
\lineto(169.286275,13.39452997)
\closepath
}
}
{
\newrgbcolor{curcolor}{0 0 0}
\pscustom[linestyle=none,fillstyle=solid,fillcolor=curcolor]
{
\newpath
\moveto(138.4647163,229.95591588)
\lineto(138.4647163,230.28544343)
\lineto(138.45408638,230.64686075)
\lineto(138.43282654,231.0295379)
\lineto(138.41156669,231.42284498)
\lineto(138.37967693,231.8374119)
\lineto(138.33715725,232.25197882)
\lineto(138.29463756,232.68780558)
\lineto(138.23085804,233.12363234)
\lineto(138.15644859,233.57008902)
\lineto(138.08203914,234.02717563)
\lineto(137.98636985,234.47363231)
\lineto(137.88007064,234.93071891)
\lineto(137.75251159,235.37717559)
\lineto(137.62495254,235.82363228)
\lineto(137.47613365,236.25945904)
\lineto(137.30605491,236.6952858)
\lineto(137.12534625,237.12048264)
\lineto(136.93400767,237.52441964)
\lineto(136.72140925,237.91772671)
\lineto(136.48755099,238.30040387)
\lineto(136.24306281,238.66182118)
\lineto(135.97731478,239.00197865)
\lineto(135.679677,239.32087628)
\lineto(135.37140929,239.60788415)
\lineto(135.04188174,239.87363218)
\lineto(134.871803,240.00119123)
\lineto(134.69109434,240.10749044)
\lineto(134.51038569,240.21378965)
\lineto(134.31904711,240.32008886)
\lineto(134.12770853,240.40512823)
\lineto(133.92574003,240.49016759)
\lineto(133.72377153,240.56457704)
\lineto(133.51117311,240.62835657)
\lineto(133.28794477,240.68150617)
\lineto(133.06471643,240.73465578)
\lineto(132.83085817,240.76654554)
\lineto(132.59699991,240.78780538)
\lineto(132.35251172,240.80906522)
\lineto(132.10802354,240.80906522)
\lineto(132.05487394,239.65040384)
\curveto(134.64857466,239.65040384)(134.95684237,236.25945904)(134.95684237,233.24056147)
\curveto(134.95684237,230.64686075)(134.54227545,227.93623089)(133.26668493,222.69567984)
\lineto(122.20093717,222.69567984)
\curveto(123.10448045,226.09725456)(124.16747255,230.32796312)(126.27219691,234.09095515)
\curveto(127.71786617,236.67402596)(129.66314171,239.65040384)(132.05487394,239.65040384)
\lineto(132.10802354,240.80906522)
\curveto(124.75211821,240.80906522)(116.58833888,225.88465614)(116.58833888,213.75591628)
\curveto(116.58833888,208.77048333)(118.1190475,202.89213702)(122.95566156,202.89213702)
\lineto(122.95566156,204.06142833)
\curveto(121.19109467,204.06142833)(120.07495297,205.60276687)(120.07495297,210.51379037)
\curveto(120.07495297,212.74607378)(120.3938506,215.81812095)(121.77574033,220.99489248)
\lineto(132.79896841,220.99489248)
\curveto(132.22495267,218.57127049)(131.10881097,214.07481391)(129.07849606,210.24804235)
\curveto(127.23951972,206.64449913)(125.23046465,204.06142833)(122.95566156,204.06142833)
\lineto(122.95566156,202.89213702)
\curveto(130.40723618,202.89213702)(138.4647163,218.09292405)(138.4647163,229.95591588)
\closepath
}
}
{
\newrgbcolor{curcolor}{0 0 0}
\pscustom[linewidth=0,linecolor=curcolor]
{
\newpath
\moveto(138.4647163,229.95591588)
\lineto(138.4647163,230.28544343)
\lineto(138.45408638,230.64686075)
\lineto(138.43282654,231.0295379)
\lineto(138.41156669,231.42284498)
\lineto(138.37967693,231.8374119)
\lineto(138.33715725,232.25197882)
\lineto(138.29463756,232.68780558)
\lineto(138.23085804,233.12363234)
\lineto(138.15644859,233.57008902)
\lineto(138.08203914,234.02717563)
\lineto(137.98636985,234.47363231)
\lineto(137.88007064,234.93071891)
\lineto(137.75251159,235.37717559)
\lineto(137.62495254,235.82363228)
\lineto(137.47613365,236.25945904)
\lineto(137.30605491,236.6952858)
\lineto(137.12534625,237.12048264)
\lineto(136.93400767,237.52441964)
\lineto(136.72140925,237.91772671)
\lineto(136.48755099,238.30040387)
\lineto(136.24306281,238.66182118)
\lineto(135.97731478,239.00197865)
\lineto(135.679677,239.32087628)
\lineto(135.37140929,239.60788415)
\lineto(135.04188174,239.87363218)
\lineto(134.871803,240.00119123)
\lineto(134.69109434,240.10749044)
\lineto(134.51038569,240.21378965)
\lineto(134.31904711,240.32008886)
\lineto(134.12770853,240.40512823)
\lineto(133.92574003,240.49016759)
\lineto(133.72377153,240.56457704)
\lineto(133.51117311,240.62835657)
\lineto(133.28794477,240.68150617)
\lineto(133.06471643,240.73465578)
\lineto(132.83085817,240.76654554)
\lineto(132.59699991,240.78780538)
\lineto(132.35251172,240.80906522)
\lineto(132.10802354,240.80906522)
\lineto(132.05487394,239.65040384)
\curveto(134.64857466,239.65040384)(134.95684237,236.25945904)(134.95684237,233.24056147)
\curveto(134.95684237,230.64686075)(134.54227545,227.93623089)(133.26668493,222.69567984)
\lineto(122.20093717,222.69567984)
\curveto(123.10448045,226.09725456)(124.16747255,230.32796312)(126.27219691,234.09095515)
\curveto(127.71786617,236.67402596)(129.66314171,239.65040384)(132.05487394,239.65040384)
\lineto(132.10802354,240.80906522)
\curveto(124.75211821,240.80906522)(116.58833888,225.88465614)(116.58833888,213.75591628)
\curveto(116.58833888,208.77048333)(118.1190475,202.89213702)(122.95566156,202.89213702)
\lineto(122.95566156,204.06142833)
\curveto(121.19109467,204.06142833)(120.07495297,205.60276687)(120.07495297,210.51379037)
\curveto(120.07495297,212.74607378)(120.3938506,215.81812095)(121.77574033,220.99489248)
\lineto(132.79896841,220.99489248)
\curveto(132.22495267,218.57127049)(131.10881097,214.07481391)(129.07849606,210.24804235)
\curveto(127.23951972,206.64449913)(125.23046465,204.06142833)(122.95566156,204.06142833)
\lineto(122.95566156,202.89213702)
\curveto(130.40723618,202.89213702)(138.4647163,218.09292405)(138.4647163,229.95591588)
\closepath
}
}
{
\newrgbcolor{curcolor}{0 0 0}
\pscustom[linestyle=none,fillstyle=solid,fillcolor=curcolor]
{
\newpath
\moveto(155.88673887,231.29067865)
\lineto(153.92677793,236.99478022)
\lineto(151.84377012,231.29067865)
\lineto(155.88673887,231.29067865)
\closepath
\moveto(153.01271543,238.91079584)
\lineto(154.99025449,238.91079584)
\lineto(159.6748248,225.99966303)
\lineto(157.75880918,225.99966303)
\lineto(156.44923887,229.86685053)
\lineto(151.34279355,229.86685053)
\lineto(149.94533262,225.99966303)
\lineto(148.15236387,225.99966303)
\lineto(153.01271543,238.91079584)
\closepath
\moveto(153.91798887,238.91079584)
\lineto(153.91798887,238.91079584)
\closepath
}
}
{
\newrgbcolor{curcolor}{0 0 0}
\pscustom[linestyle=none,fillstyle=solid,fillcolor=curcolor]
{
\newpath
\moveto(161.07228574,235.41274897)
\lineto(162.57521543,235.41274897)
\lineto(162.57521543,234.07681147)
\curveto(163.02052793,234.62759272)(163.49220762,235.02310053)(163.99025449,235.2633349)
\curveto(164.48830137,235.50356928)(165.0420123,235.62368647)(165.6513873,235.62368647)
\curveto(166.9873248,235.62368647)(167.88966855,235.15786615)(168.35841855,234.22622553)
\curveto(168.61623105,233.7164599)(168.7451373,232.98696772)(168.7451373,232.03774897)
\lineto(168.7451373,225.99966303)
\lineto(167.13673887,225.99966303)
\lineto(167.13673887,231.93228022)
\curveto(167.13673887,232.50649897)(167.05177793,232.96938959)(166.88185605,233.32095209)
\curveto(166.60060605,233.90688959)(166.09084043,234.19985834)(165.35255918,234.19985834)
\curveto(164.97755918,234.19985834)(164.66994199,234.1617724)(164.42970762,234.08560053)
\curveto(163.99611387,233.95669428)(163.61525449,233.69888178)(163.28712949,233.31216303)
\curveto(163.02345762,233.00161615)(162.85060605,232.67935053)(162.7685748,232.34536615)
\curveto(162.69240293,232.01724115)(162.65431699,231.54556147)(162.65431699,230.93032709)
\lineto(162.65431699,225.99966303)
\lineto(161.07228574,225.99966303)
\lineto(161.07228574,235.41274897)
\closepath
\moveto(164.79005918,235.64126459)
\lineto(164.79005918,235.64126459)
\closepath
}
}
{
\newrgbcolor{curcolor}{0 0 0}
\pscustom[linestyle=none,fillstyle=solid,fillcolor=curcolor]
{
\newpath
\moveto(174.41408262,235.57974115)
\curveto(175.15236387,235.57974115)(175.79689512,235.39810053)(176.34767637,235.03481928)
\curveto(176.64650449,234.82974115)(176.95119199,234.53091303)(177.26173887,234.1383349)
\lineto(177.26173887,235.32485834)
\lineto(178.72072324,235.32485834)
\lineto(178.72072324,226.76431147)
\curveto(178.72072324,225.56899897)(178.54494199,224.62563959)(178.19337949,223.93423334)
\curveto(177.53712949,222.65688959)(176.29787168,222.01821772)(174.47560605,222.01821772)
\curveto(173.46193418,222.01821772)(172.60939512,222.24673334)(171.91798887,222.70376459)
\curveto(171.22658262,223.15493647)(170.83986387,223.86392084)(170.75783262,224.83071772)
\lineto(172.36623105,224.83071772)
\curveto(172.44240293,224.40884272)(172.59474668,224.0836474)(172.8232623,223.85513178)
\curveto(173.18068418,223.50356928)(173.74318418,223.32778803)(174.5107623,223.32778803)
\curveto(175.72365293,223.32778803)(176.51759824,223.7555224)(176.89259824,224.61099115)
\curveto(177.11525449,225.1148974)(177.21779355,226.01431147)(177.20021543,227.30923334)
\curveto(176.88380918,226.82876459)(176.5029498,226.47134272)(176.0576373,226.23696772)
\curveto(175.6123248,226.00259272)(175.02345762,225.88540522)(174.29103574,225.88540522)
\curveto(173.27150449,225.88540522)(172.3779498,226.24575678)(171.61037168,226.9664599)
\curveto(170.84865293,227.6930224)(170.46779355,228.89126459)(170.46779355,230.56118647)
\curveto(170.46779355,232.13735834)(170.85158262,233.36782709)(171.61916074,234.25259272)
\curveto(172.39259824,235.13735834)(173.32423887,235.57974115)(174.41408262,235.57974115)
\closepath
\moveto(177.26173887,230.74575678)
\curveto(177.26173887,231.9117724)(177.02150449,232.77603022)(176.54103574,233.33853022)
\curveto(176.06056699,233.90103022)(175.4482623,234.18228022)(174.70412168,234.18228022)
\curveto(173.59084043,234.18228022)(172.82912168,233.66079584)(172.41896543,232.61782709)
\curveto(172.20216855,232.06118647)(172.09377012,231.33169428)(172.09377012,230.42935053)
\curveto(172.09377012,229.36880365)(172.3076373,228.5602099)(172.73537168,228.00356928)
\curveto(173.16896543,227.45278803)(173.74904355,227.1773974)(174.47560605,227.1773974)
\curveto(175.6123248,227.1773974)(176.41212949,227.69009272)(176.87502012,228.71548334)
\curveto(177.13283262,229.29556147)(177.26173887,229.97231928)(177.26173887,230.74575678)
\closepath
\moveto(174.59865293,235.64126459)
\lineto(174.59865293,235.64126459)
\closepath
}
}
{
\newrgbcolor{curcolor}{0 0 0}
\pscustom[linestyle=none,fillstyle=solid,fillcolor=curcolor]
{
\newpath
\moveto(181.15529355,238.91079584)
\lineto(182.7373248,238.91079584)
\lineto(182.7373248,225.99966303)
\lineto(181.15529355,225.99966303)
\lineto(181.15529355,238.91079584)
\closepath
}
}
{
\newrgbcolor{curcolor}{0 0 0}
\pscustom[linestyle=none,fillstyle=solid,fillcolor=curcolor]
{
\newpath
\moveto(189.03908262,235.62368647)
\curveto(189.70705137,235.62368647)(190.3545123,235.46548334)(190.98146543,235.14907709)
\curveto(191.60841855,234.83853022)(192.08595762,234.43423334)(192.41408262,233.93618647)
\curveto(192.73048887,233.46157709)(192.94142637,232.90786615)(193.04689512,232.27505365)
\curveto(193.14064512,231.8414599)(193.18752012,231.15005365)(193.18752012,230.2008349)
\lineto(186.28810605,230.2008349)
\curveto(186.31740293,229.24575678)(186.54298887,228.47817865)(186.96486387,227.89810053)
\curveto(187.38673887,227.32388178)(188.04005918,227.0367724)(188.9248248,227.0367724)
\curveto(189.75099668,227.0367724)(190.41017637,227.30923334)(190.90236387,227.85415522)
\curveto(191.18361387,228.17056147)(191.38283262,228.5367724)(191.50002012,228.95278803)
\lineto(193.05568418,228.95278803)
\curveto(193.01466855,228.6070849)(192.87697324,228.22036615)(192.64259824,227.79263178)
\curveto(192.41408262,227.37075678)(192.15627012,227.02505365)(191.86916074,226.7555224)
\curveto(191.38869199,226.2867724)(190.79396543,225.97036615)(190.08498105,225.80630365)
\curveto(189.70412168,225.71255365)(189.27345762,225.66567865)(188.79298887,225.66567865)
\curveto(187.62111387,225.66567865)(186.6279498,226.09048334)(185.81349668,226.94009272)
\curveto(184.99904355,227.79556147)(184.59181699,228.99087397)(184.59181699,230.52603022)
\curveto(184.59181699,232.03774897)(185.00197324,233.26528803)(185.82228574,234.2086474)
\curveto(186.64259824,235.15200678)(187.71486387,235.62368647)(189.03908262,235.62368647)
\closepath
\moveto(191.56154355,231.45767084)
\curveto(191.49709043,232.14321772)(191.34767637,232.69106928)(191.11330137,233.10122553)
\curveto(190.67970762,233.86294428)(189.9560748,234.24380365)(188.94240293,234.24380365)
\curveto(188.21584043,234.24380365)(187.60646543,233.98013178)(187.11427793,233.45278803)
\curveto(186.62209043,232.93130365)(186.36134824,232.26626459)(186.33205137,231.45767084)
\lineto(191.56154355,231.45767084)
\closepath
\moveto(188.88966855,235.64126459)
\lineto(188.88966855,235.64126459)
\closepath
}
}
{
\newrgbcolor{curcolor}{0 0 0}
\pscustom[linestyle=none,fillstyle=solid,fillcolor=curcolor]
{
\newpath
\moveto(203.86623105,227.01919428)
\curveto(204.91505918,227.01919428)(205.63283262,227.41470209)(206.01955137,228.20571772)
\curveto(206.41212949,229.00259272)(206.60841855,229.88735834)(206.60841855,230.86001459)
\curveto(206.60841855,231.73892084)(206.46779355,232.45376459)(206.18654355,233.00454584)
\curveto(205.74123105,233.87173334)(204.97365293,234.30532709)(203.88380918,234.30532709)
\curveto(202.9170123,234.30532709)(202.2138873,233.93618647)(201.77443418,233.19790522)
\curveto(201.33498105,232.45962397)(201.11525449,231.56899897)(201.11525449,230.52603022)
\curveto(201.11525449,229.52407709)(201.33498105,228.68911615)(201.77443418,228.0211474)
\curveto(202.2138873,227.35317865)(202.91115293,227.01919428)(203.86623105,227.01919428)
\closepath
\moveto(203.92775449,235.6852099)
\curveto(205.14064512,235.6852099)(206.16603574,235.28091303)(207.00392637,234.47231928)
\curveto(207.84181699,233.66372553)(208.2607623,232.4742724)(208.2607623,230.9039599)
\curveto(208.2607623,229.38638178)(207.89162168,228.13247553)(207.15334043,227.14224115)
\curveto(206.41505918,226.15200678)(205.26955137,225.65688959)(203.71681699,225.65688959)
\curveto(202.42189512,225.65688959)(201.3935748,226.09341303)(200.63185605,226.9664599)
\curveto(199.8701373,227.84536615)(199.48927793,229.02310053)(199.48927793,230.49966303)
\curveto(199.48927793,232.08169428)(199.89064512,233.3414599)(200.69337949,234.2789599)
\curveto(201.49611387,235.2164599)(202.57423887,235.6852099)(203.92775449,235.6852099)
\closepath
\moveto(203.87502012,235.64126459)
\lineto(203.87502012,235.64126459)
\closepath
}
}
{
\newrgbcolor{curcolor}{0 0 0}
\pscustom[linestyle=none,fillstyle=solid,fillcolor=curcolor]
{
\newpath
\moveto(210.54591855,236.84536615)
\curveto(210.56935605,237.50161615)(210.68361387,237.9820849)(210.88869199,238.2867724)
\curveto(211.25783262,238.8258349)(211.96974668,239.09536615)(213.02443418,239.09536615)
\curveto(213.12404355,239.09536615)(213.22658262,239.09243647)(213.33205137,239.08657709)
\curveto(213.43752012,239.08071772)(213.5576373,239.07192865)(213.69240293,239.0602099)
\lineto(213.69240293,237.61880365)
\curveto(213.52834043,237.6305224)(213.40822324,237.63638178)(213.33205137,237.63638178)
\curveto(213.26173887,237.64224115)(213.19435605,237.64517084)(213.12990293,237.64517084)
\curveto(212.64943418,237.64517084)(212.3623248,237.51919428)(212.2685748,237.26724115)
\curveto(212.1748248,237.0211474)(212.1279498,236.3883349)(212.1279498,235.36880365)
\lineto(213.69240293,235.36880365)
\lineto(213.69240293,234.12075678)
\lineto(212.11037168,234.12075678)
\lineto(212.11037168,225.99966303)
\lineto(210.54591855,225.99966303)
\lineto(210.54591855,234.12075678)
\lineto(209.23634824,234.12075678)
\lineto(209.23634824,235.36880365)
\lineto(210.54591855,235.36880365)
\lineto(210.54591855,236.84536615)
\closepath
}
}
{
\newrgbcolor{curcolor}{0 0 0}
\pscustom[linestyle=none,fillstyle=solid,fillcolor=curcolor]
{
\newpath
\moveto(225.95314512,239.26235834)
\curveto(228.23244199,239.26235834)(229.91994199,238.52993647)(231.01564512,237.06509272)
\curveto(231.87111387,235.92251459)(232.29884824,234.46060053)(232.29884824,232.67935053)
\curveto(232.29884824,230.75161615)(231.80959043,229.14907709)(230.8310748,227.87173334)
\curveto(229.6826373,226.37173334)(228.04494199,225.62173334)(225.91798887,225.62173334)
\curveto(223.93166074,225.62173334)(222.3701373,226.27798334)(221.23341855,227.59048334)
\curveto(220.21974668,228.85610834)(219.71291074,230.45571772)(219.71291074,232.38931147)
\curveto(219.71291074,234.13540522)(220.14650449,235.62954584)(221.01369199,236.87173334)
\curveto(222.12697324,238.46548334)(223.77345762,239.26235834)(225.95314512,239.26235834)
\closepath
\moveto(226.12892637,227.18618647)
\curveto(227.66994199,227.18618647)(228.78322324,227.73696772)(229.46877012,228.83853022)
\curveto(230.16017637,229.94595209)(230.50587949,231.21743647)(230.50587949,232.65298334)
\curveto(230.50587949,234.17056147)(230.10744199,235.39224115)(229.31056699,236.3180224)
\curveto(228.51955137,237.24380365)(227.43556699,237.70669428)(226.05861387,237.70669428)
\curveto(224.72267637,237.70669428)(223.63283262,237.24673334)(222.78908262,236.32681147)
\curveto(221.94533262,235.41274897)(221.52345762,234.06216303)(221.52345762,232.27505365)
\curveto(221.52345762,230.84536615)(221.88380918,229.6383349)(222.6045123,228.6539599)
\curveto(223.3310748,227.67544428)(224.50587949,227.18618647)(226.12892637,227.18618647)
\closepath
\moveto(226.00587949,239.26235834)
\lineto(226.00587949,239.26235834)
\closepath
}
}
{
\newrgbcolor{curcolor}{0 0 0}
\pscustom[linestyle=none,fillstyle=solid,fillcolor=curcolor]
{
\newpath
\moveto(234.93556699,235.41274897)
\lineto(237.44923887,227.74868647)
\lineto(240.07716855,235.41274897)
\lineto(241.80861387,235.41274897)
\lineto(238.25783262,225.99966303)
\lineto(236.57033262,225.99966303)
\lineto(233.09865293,235.41274897)
\lineto(234.93556699,235.41274897)
\closepath
}
}
{
\newrgbcolor{curcolor}{0 0 0}
\pscustom[linestyle=none,fillstyle=solid,fillcolor=curcolor]
{
\newpath
\moveto(247.08205137,235.62368647)
\curveto(247.75002012,235.62368647)(248.39748105,235.46548334)(249.02443418,235.14907709)
\curveto(249.6513873,234.83853022)(250.12892637,234.43423334)(250.45705137,233.93618647)
\curveto(250.77345762,233.46157709)(250.98439512,232.90786615)(251.08986387,232.27505365)
\curveto(251.18361387,231.8414599)(251.23048887,231.15005365)(251.23048887,230.2008349)
\lineto(244.3310748,230.2008349)
\curveto(244.36037168,229.24575678)(244.58595762,228.47817865)(245.00783262,227.89810053)
\curveto(245.42970762,227.32388178)(246.08302793,227.0367724)(246.96779355,227.0367724)
\curveto(247.79396543,227.0367724)(248.45314512,227.30923334)(248.94533262,227.85415522)
\curveto(249.22658262,228.17056147)(249.42580137,228.5367724)(249.54298887,228.95278803)
\lineto(251.09865293,228.95278803)
\curveto(251.0576373,228.6070849)(250.91994199,228.22036615)(250.68556699,227.79263178)
\curveto(250.45705137,227.37075678)(250.19923887,227.02505365)(249.91212949,226.7555224)
\curveto(249.43166074,226.2867724)(248.83693418,225.97036615)(248.1279498,225.80630365)
\curveto(247.74709043,225.71255365)(247.31642637,225.66567865)(246.83595762,225.66567865)
\curveto(245.66408262,225.66567865)(244.67091855,226.09048334)(243.85646543,226.94009272)
\curveto(243.0420123,227.79556147)(242.63478574,228.99087397)(242.63478574,230.52603022)
\curveto(242.63478574,232.03774897)(243.04494199,233.26528803)(243.86525449,234.2086474)
\curveto(244.68556699,235.15200678)(245.75783262,235.62368647)(247.08205137,235.62368647)
\closepath
\moveto(249.6045123,231.45767084)
\curveto(249.54005918,232.14321772)(249.39064512,232.69106928)(249.15627012,233.10122553)
\curveto(248.72267637,233.86294428)(247.99904355,234.24380365)(246.98537168,234.24380365)
\curveto(246.25880918,234.24380365)(245.64943418,233.98013178)(245.15724668,233.45278803)
\curveto(244.66505918,232.93130365)(244.40431699,232.26626459)(244.37502012,231.45767084)
\lineto(249.6045123,231.45767084)
\closepath
\moveto(246.9326373,235.64126459)
\lineto(246.9326373,235.64126459)
\closepath
}
}
{
\newrgbcolor{curcolor}{0 0 0}
\pscustom[linestyle=none,fillstyle=solid,fillcolor=curcolor]
{
\newpath
\moveto(253.22560605,235.41274897)
\lineto(254.72853574,235.41274897)
\lineto(254.72853574,233.7867724)
\curveto(254.85158262,234.10317865)(255.15334043,234.48696772)(255.63380918,234.93813959)
\curveto(256.11427793,235.39517084)(256.66798887,235.62368647)(257.29494199,235.62368647)
\curveto(257.32423887,235.62368647)(257.37404355,235.62075678)(257.44435605,235.6148974)
\curveto(257.51466855,235.60903803)(257.63478574,235.59731928)(257.80470762,235.57974115)
\lineto(257.80470762,233.90981928)
\curveto(257.71095762,233.9273974)(257.62306699,233.93911615)(257.54103574,233.94497553)
\curveto(257.46486387,233.9508349)(257.37990293,233.95376459)(257.28615293,233.95376459)
\curveto(256.48927793,233.95376459)(255.87697324,233.69595209)(255.44923887,233.18032709)
\curveto(255.02150449,232.67056147)(254.8076373,232.08169428)(254.8076373,231.41372553)
\lineto(254.8076373,225.99966303)
\lineto(253.22560605,225.99966303)
\lineto(253.22560605,235.41274897)
\closepath
}
}
{
\newrgbcolor{curcolor}{0 0 0}
\pscustom[linestyle=none,fillstyle=solid,fillcolor=curcolor]
{
\newpath
\moveto(259.19337949,238.95474115)
\lineto(260.77541074,238.95474115)
\lineto(260.77541074,234.1383349)
\curveto(261.15041074,234.61294428)(261.4873248,234.94692865)(261.78615293,235.14028803)
\curveto(262.29591855,235.4742724)(262.93166074,235.64126459)(263.69337949,235.64126459)
\curveto(265.05861387,235.64126459)(265.98439512,235.16372553)(266.47072324,234.2086474)
\curveto(266.73439512,233.68716303)(266.86623105,232.96353022)(266.86623105,232.03774897)
\lineto(266.86623105,225.99966303)
\lineto(265.24025449,225.99966303)
\lineto(265.24025449,231.93228022)
\curveto(265.24025449,232.62368647)(265.15236387,233.1305224)(264.97658262,233.45278803)
\curveto(264.68947324,233.96841303)(264.15041074,234.22622553)(263.35939512,234.22622553)
\curveto(262.70314512,234.22622553)(262.10841855,234.00063959)(261.57521543,233.54946772)
\curveto(261.0420123,233.09829584)(260.77541074,232.24575678)(260.77541074,230.99185053)
\lineto(260.77541074,225.99966303)
\lineto(259.19337949,225.99966303)
\lineto(259.19337949,238.95474115)
\closepath
}
}
{
\newrgbcolor{curcolor}{0 0 0}
\pscustom[linestyle=none,fillstyle=solid,fillcolor=curcolor]
{
\newpath
\moveto(270.42580137,228.50454584)
\curveto(270.42580137,228.04751459)(270.59279355,227.68716303)(270.92677793,227.42349115)
\curveto(271.2607623,227.15981928)(271.65627012,227.02798334)(272.11330137,227.02798334)
\curveto(272.66994199,227.02798334)(273.20900449,227.15688959)(273.73048887,227.41470209)
\curveto(274.60939512,227.84243647)(275.04884824,228.54263178)(275.04884824,229.51528803)
\lineto(275.04884824,230.78970209)
\curveto(274.85548887,230.66665522)(274.60646543,230.56411615)(274.30177793,230.4820849)
\curveto(273.99709043,230.40005365)(273.6982623,230.3414599)(273.40529355,230.30630365)
\lineto(272.44728574,230.18325678)
\curveto(271.87306699,230.1070849)(271.44240293,229.98696772)(271.15529355,229.82290522)
\curveto(270.66896543,229.54751459)(270.42580137,229.10806147)(270.42580137,228.50454584)
\closepath
\moveto(274.25783262,231.70376459)
\curveto(274.62111387,231.75063959)(274.86427793,231.90298334)(274.9873248,232.16079584)
\curveto(275.0576373,232.30142084)(275.09279355,232.50356928)(275.09279355,232.76724115)
\curveto(275.09279355,233.30630365)(274.89943418,233.69595209)(274.51271543,233.93618647)
\curveto(274.13185605,234.18228022)(273.58400449,234.30532709)(272.86916074,234.30532709)
\curveto(272.04298887,234.30532709)(271.45705137,234.08267084)(271.11134824,233.63735834)
\curveto(270.91798887,233.39126459)(270.7920123,233.02505365)(270.73341855,232.53872553)
\lineto(269.25685605,232.53872553)
\curveto(269.28615293,233.69888178)(269.66115293,234.50454584)(270.38185605,234.95571772)
\curveto(271.10841855,235.41274897)(271.94923887,235.64126459)(272.90431699,235.64126459)
\curveto(274.01173887,235.64126459)(274.91115293,235.43032709)(275.60255918,235.00845209)
\curveto(276.28810605,234.58657709)(276.63087949,233.93032709)(276.63087949,233.03970209)
\lineto(276.63087949,227.61685053)
\curveto(276.63087949,227.45278803)(276.66310605,227.32095209)(276.72755918,227.22134272)
\curveto(276.79787168,227.12173334)(276.94142637,227.07192865)(277.15822324,227.07192865)
\curveto(277.22853574,227.07192865)(277.3076373,227.07485834)(277.39552793,227.08071772)
\curveto(277.48341855,227.09243647)(277.57716855,227.1070849)(277.67677793,227.12466303)
\lineto(277.67677793,225.95571772)
\curveto(277.43068418,225.88540522)(277.24318418,225.8414599)(277.11427793,225.82388178)
\curveto(276.98537168,225.80630365)(276.80959043,225.79751459)(276.58693418,225.79751459)
\curveto(276.0420123,225.79751459)(275.64650449,225.99087397)(275.40041074,226.37759272)
\curveto(275.27150449,226.58267084)(275.18068418,226.8727099)(275.1279498,227.2477099)
\curveto(274.80568418,226.8258349)(274.34279355,226.45962397)(273.73927793,226.14907709)
\curveto(273.1357623,225.83853022)(272.47072324,225.68325678)(271.74416074,225.68325678)
\curveto(270.87111387,225.68325678)(270.15627012,225.94692865)(269.59962949,226.4742724)
\curveto(269.04884824,227.00747553)(268.77345762,227.67251459)(268.77345762,228.46938959)
\curveto(268.77345762,229.34243647)(269.04591855,230.01919428)(269.59084043,230.49966303)
\curveto(270.1357623,230.98013178)(270.85060605,231.27603022)(271.73537168,231.38735834)
\lineto(274.25783262,231.70376459)
\closepath
\moveto(272.9482623,235.64126459)
\lineto(272.9482623,235.64126459)
\closepath
}
}
{
\newrgbcolor{curcolor}{0 0 0}
\pscustom[linestyle=none,fillstyle=solid,fillcolor=curcolor]
{
\newpath
\moveto(279.23244199,235.41274897)
\lineto(280.73537168,235.41274897)
\lineto(280.73537168,234.07681147)
\curveto(281.18068418,234.62759272)(281.65236387,235.02310053)(282.15041074,235.2633349)
\curveto(282.64845762,235.50356928)(283.20216855,235.62368647)(283.81154355,235.62368647)
\curveto(285.14748105,235.62368647)(286.0498248,235.15786615)(286.5185748,234.22622553)
\curveto(286.7763873,233.7164599)(286.90529355,232.98696772)(286.90529355,232.03774897)
\lineto(286.90529355,225.99966303)
\lineto(285.29689512,225.99966303)
\lineto(285.29689512,231.93228022)
\curveto(285.29689512,232.50649897)(285.21193418,232.96938959)(285.0420123,233.32095209)
\curveto(284.7607623,233.90688959)(284.25099668,234.19985834)(283.51271543,234.19985834)
\curveto(283.13771543,234.19985834)(282.83009824,234.1617724)(282.58986387,234.08560053)
\curveto(282.15627012,233.95669428)(281.77541074,233.69888178)(281.44728574,233.31216303)
\curveto(281.18361387,233.00161615)(281.0107623,232.67935053)(280.92873105,232.34536615)
\curveto(280.85255918,232.01724115)(280.81447324,231.54556147)(280.81447324,230.93032709)
\lineto(280.81447324,225.99966303)
\lineto(279.23244199,225.99966303)
\lineto(279.23244199,235.41274897)
\closepath
\moveto(282.95021543,235.64126459)
\lineto(282.95021543,235.64126459)
\closepath
}
}
{
\newrgbcolor{curcolor}{0 0 0}
\pscustom[linestyle=none,fillstyle=solid,fillcolor=curcolor]
{
\newpath
\moveto(292.57423887,235.57974115)
\curveto(293.31252012,235.57974115)(293.95705137,235.39810053)(294.50783262,235.03481928)
\curveto(294.80666074,234.82974115)(295.11134824,234.53091303)(295.42189512,234.1383349)
\lineto(295.42189512,235.32485834)
\lineto(296.88087949,235.32485834)
\lineto(296.88087949,226.76431147)
\curveto(296.88087949,225.56899897)(296.70509824,224.62563959)(296.35353574,223.93423334)
\curveto(295.69728574,222.65688959)(294.45802793,222.01821772)(292.6357623,222.01821772)
\curveto(291.62209043,222.01821772)(290.76955137,222.24673334)(290.07814512,222.70376459)
\curveto(289.38673887,223.15493647)(289.00002012,223.86392084)(288.91798887,224.83071772)
\lineto(290.5263873,224.83071772)
\curveto(290.60255918,224.40884272)(290.75490293,224.0836474)(290.98341855,223.85513178)
\curveto(291.34084043,223.50356928)(291.90334043,223.32778803)(292.67091855,223.32778803)
\curveto(293.88380918,223.32778803)(294.67775449,223.7555224)(295.05275449,224.61099115)
\curveto(295.27541074,225.1148974)(295.3779498,226.01431147)(295.36037168,227.30923334)
\curveto(295.04396543,226.82876459)(294.66310605,226.47134272)(294.21779355,226.23696772)
\curveto(293.77248105,226.00259272)(293.18361387,225.88540522)(292.45119199,225.88540522)
\curveto(291.43166074,225.88540522)(290.53810605,226.24575678)(289.77052793,226.9664599)
\curveto(289.00880918,227.6930224)(288.6279498,228.89126459)(288.6279498,230.56118647)
\curveto(288.6279498,232.13735834)(289.01173887,233.36782709)(289.77931699,234.25259272)
\curveto(290.55275449,235.13735834)(291.48439512,235.57974115)(292.57423887,235.57974115)
\closepath
\moveto(295.42189512,230.74575678)
\curveto(295.42189512,231.9117724)(295.18166074,232.77603022)(294.70119199,233.33853022)
\curveto(294.22072324,233.90103022)(293.60841855,234.18228022)(292.86427793,234.18228022)
\curveto(291.75099668,234.18228022)(290.98927793,233.66079584)(290.57912168,232.61782709)
\curveto(290.3623248,232.06118647)(290.25392637,231.33169428)(290.25392637,230.42935053)
\curveto(290.25392637,229.36880365)(290.46779355,228.5602099)(290.89552793,228.00356928)
\curveto(291.32912168,227.45278803)(291.9091998,227.1773974)(292.6357623,227.1773974)
\curveto(293.77248105,227.1773974)(294.57228574,227.69009272)(295.03517637,228.71548334)
\curveto(295.29298887,229.29556147)(295.42189512,229.97231928)(295.42189512,230.74575678)
\closepath
\moveto(292.75880918,235.64126459)
\lineto(292.75880918,235.64126459)
\closepath
}
}
\end{pspicture}

    \caption{The Angle of Overhang}
  \end{figure}
\end{frame}

\begin{frame}
  \frametitle{Unsupported Sections}
  If the Angle of Overhang exceeds 90\degree\ the printer will try to start printing a section on thin air (depending on the printer technology).  Thus usually does not end well.
  \begin{itemize}
    \item You can choose a printer technology, such as powder fusing methods, where the section would be supported by unfused powder.
    \item You may be able to reorient your part so there is no longer an unsupported section.
    \item You may be able to use the slicer program to automatically generate a support structure.
    \item You may be able to add a support structure to your design.
    \item Some combination of the above.
  \end{itemize}
\end{frame}

\begin{frame}
  \frametitle{Unsupported Sections}
  \begin{figure}
    %LaTeX with PSTricks extensions
%%Creator: inkscape 0.91
%%Please note this file requires PSTricks extensions
\psset{xunit=.5pt,yunit=.5pt,runit=.5pt}
\begin{pspicture}(469.66331061,193.64062933)
{
\newrgbcolor{curcolor}{0 0 0}
\pscustom[linewidth=1.05554247,linecolor=curcolor]
{
\newpath
\moveto(0,20.37791933)
\lineto(467.95141,20.37791933)
}
}
{
\newrgbcolor{curcolor}{0.5411765 0.5411765 0.5411765}
\pscustom[linestyle=none,fillstyle=solid,fillcolor=curcolor]
{
\newpath
\moveto(27.95141,20.37791933)
\lineto(127.95141,20.37791933)
\lineto(127.95141,110.37791933)
\lineto(167.95141,80.37791933)
\lineto(187.95141,80.37791933)
\lineto(117.95141,170.37791933)
\lineto(27.95141,170.37791933)
\closepath
}
}
{
\newrgbcolor{curcolor}{0 0 0}
\pscustom[linewidth=1,linecolor=curcolor]
{
\newpath
\moveto(27.95141,20.37791933)
\lineto(127.95141,20.37791933)
\lineto(127.95141,110.37791933)
\lineto(167.95141,80.37791933)
\lineto(187.95141,80.37791933)
\lineto(117.95141,170.37791933)
\lineto(27.95141,170.37791933)
\closepath
}
}
{
\newrgbcolor{curcolor}{0.5411765 0.5411765 0.5411765}
\pscustom[linestyle=none,fillstyle=solid,fillcolor=curcolor]
{
\newpath
\moveto(237.95141,20.37791933)
\lineto(337.95141,20.37791933)
\lineto(337.95141,110.37791933)
\lineto(357.95141,130.37791933)
\lineto(327.95141,170.37791933)
\lineto(237.95141,170.37791933)
\closepath
}
}
{
\newrgbcolor{curcolor}{0 0 0}
\pscustom[linewidth=1,linecolor=curcolor]
{
\newpath
\moveto(237.95141,20.37791933)
\lineto(337.95141,20.37791933)
\lineto(337.95141,110.37791933)
\lineto(357.95141,130.37791933)
\lineto(327.95141,170.37791933)
\lineto(237.95141,170.37791933)
\closepath
}
}
{
\newrgbcolor{curcolor}{0 0 0}
\pscustom[linewidth=1,linecolor=curcolor]
{
\newpath
\moveto(351.95141,26.37791933)
\curveto(353.03304,26.81716933)(354.03619,27.56513933)(355.19629,27.69567933)
\curveto(355.59741,27.74077933)(355.69152,27.01991933)(356.05256,26.83939933)
\curveto(356.85987,26.43574933)(357.81408,26.38677933)(358.62139,25.98312933)
\curveto(358.98242,25.80260933)(359.11662,25.30736933)(359.47766,25.12684933)
\curveto(363.38164,23.17485933)(360.82509,25.19984933)(365.47159,24.27056933)
\curveto(367.24172,23.91654933)(368.82861,22.85479933)(370.60924,22.55801933)
\curveto(372.29849,22.27647933)(374.03938,22.68936933)(375.74689,22.55801933)
\curveto(377.75921,22.40322933)(379.73085,21.88446933)(381.74082,21.70174933)
\curveto(383.73059,21.52085933)(385.73677,21.70174933)(387.73474,21.70174933)
\lineto(391.15984,21.70174933)
\curveto(392.30154,21.70174933)(393.49139,21.37367933)(394.58494,21.70174933)
\curveto(396.41888,22.25192933)(397.90616,23.66509933)(399.7226,24.27056933)
\curveto(401.36967,24.81959933)(403.13469,24.93511933)(404.86025,25.12684933)
\curveto(406.50184,25.30924933)(408.3301,25.12684933)(409.9979,25.12684933)
\curveto(410.28332,25.12684933)(410.65235,24.92501933)(410.85417,25.12684933)
\curveto(411.056,25.32867933)(410.72653,25.72782933)(410.85417,25.98312933)
\curveto(411.36794,27.01065933)(412.90923,27.52441933)(413.423,28.55194933)
\curveto(413.55064,28.80723933)(413.423,29.12279933)(413.423,29.40822933)
\lineto(413.423,31.12077933)
\curveto(413.423,31.69162933)(413.56145,32.27951933)(413.423,32.83332933)
\curveto(413.26821,33.45249933)(412.85215,33.97502933)(412.56672,34.54587933)
\curveto(412.2813,35.11672933)(412.16175,35.80712933)(411.71045,36.25842933)
\curveto(410.98275,36.98611933)(409.78451,37.16736933)(409.14162,37.97097933)
\curveto(408.57778,38.67577933)(409.00742,39.99824933)(408.28535,40.53979933)
\curveto(407.60033,41.05356933)(406.55617,40.37186933)(405.71652,40.53979933)
\curveto(404.16672,40.02369933)(404.124,41.92691933)(403.1477,42.25234933)
\curveto(401.76699,42.71258933)(400.26033,42.69042933)(398.86632,43.10862933)
\curveto(397.39408,43.55029933)(396.05718,44.37950933)(394.58494,44.82117933)
\curveto(390.95262,45.91087933)(387.99707,45.67744933)(384.30964,45.67744933)
\lineto(382.59709,45.67744933)
\curveto(382.02624,45.67744933)(381.43835,45.81590933)(380.88454,45.67744933)
\curveto(379.83133,45.41414933)(377.15566,43.47695933)(376.60317,43.10862933)
\curveto(375.74689,42.53777933)(374.9168,41.92554933)(374.03434,41.39607933)
\curveto(372.93979,40.73934933)(371.67132,40.39157933)(370.60924,39.68352933)
\curveto(368.888,38.53602933)(370.40536,39.07189933)(369.75296,37.11469933)
\curveto(369.62532,36.73175933)(369.12059,36.59428933)(368.89669,36.25842933)
\curveto(367.27454,36.35592933)(368.32584,34.26044933)(368.04041,33.68959933)
\curveto(367.8599,33.32855933)(367.31178,33.21625933)(367.18414,32.83332933)
\curveto(367.00362,32.29176933)(367.18414,31.69162933)(367.18414,31.12077933)
\curveto(367.18414,28.97083933)(367.21484,28.52126933)(368.89669,26.83939933)
\curveto(369.18211,26.55397933)(369.39193,26.16363933)(369.75296,25.98312933)
\curveto(370.00826,25.85547933)(370.34423,26.08912933)(370.60924,25.98312933)
\curveto(377.61605,22.47971933)(370.46427,25.74601933)(377.45944,23.41429933)
\curveto(378.06492,23.21246933)(378.57941,22.79505933)(379.17199,22.55801933)
\curveto(380.01003,22.22280933)(380.85575,21.87875933)(381.74082,21.70174933)
\curveto(382.02076,21.64574933)(385.74225,21.64574933)(386.02219,21.70174933)
\curveto(390.05013,22.50733933)(388.08274,22.92382933)(392.01612,25.98312933)
\curveto(393.34476,27.01650933)(397.91303,29.48896933)(399.7226,30.26449933)
\curveto(400.55221,30.62004933)(401.40636,30.94375933)(402.29142,31.12077933)
\curveto(403.56151,31.37478933)(404.96698,30.47795933)(405.71652,31.97704933)
\curveto(405.84417,32.23233933)(405.88778,32.60498933)(405.71652,32.83332933)
\curveto(404.98995,33.80208933)(404.15527,34.73042933)(403.1477,35.40214933)
\curveto(402.39669,35.90281933)(401.4434,35.99906933)(400.57887,36.25842933)
\curveto(398.58858,36.85551933)(396.6346,37.62936933)(394.58494,37.97097933)
\curveto(393.17724,38.20559933)(391.73069,37.97097933)(390.30357,37.97097933)
\lineto(377.45944,37.97097933)
\lineto(364.61531,37.97097933)
\lineto(362.04649,37.97097933)
\lineto(361.19021,37.97097933)
\curveto(360.90479,37.97097933)(360.58923,38.09861933)(360.33394,37.97097933)
\curveto(359.9729,37.79045933)(359.81352,37.33860933)(359.47766,37.11469933)
\curveto(358.94662,36.76067933)(358.21641,36.70971933)(357.76511,36.25842933)
\curveto(357.56329,36.05659933)(357.96694,35.60397933)(357.76511,35.40214933)
\curveto(357.56329,35.20032933)(357.11066,35.60397933)(356.90884,35.40214933)
\curveto(356.70701,35.20032933)(356.90884,34.83129933)(356.90884,34.54587933)
\curveto(356.90884,34.26044933)(356.81854,33.96037933)(356.90884,33.68959933)
\curveto(357.11066,33.08411933)(357.47969,32.54789933)(357.76511,31.97704933)
\curveto(358.64029,30.22669933)(358.60286,29.85025933)(360.33394,28.55194933)
\curveto(360.84452,28.16900933)(361.59519,28.14696933)(362.04649,27.69567933)
\curveto(363.08942,26.65274933)(362.44759,25.78257933)(363.75904,25.12684933)
\curveto(364.56634,24.72319933)(365.47159,24.55599933)(366.32786,24.27056933)
\curveto(369.12392,23.33855933)(369.87154,22.97133933)(373.17807,22.55801933)
\curveto(374.02773,22.45181933)(374.89062,22.55801933)(375.74689,22.55801933)
\curveto(376.88859,22.55801933)(378.07422,22.24437933)(379.17199,22.55801933)
\curveto(380.16151,22.84073933)(380.79491,23.86518933)(381.74082,24.27056933)
\curveto(382.8225,24.73414933)(384.06401,24.71362933)(385.16592,25.12684933)
\curveto(386.3611,25.57504933)(387.34495,26.56249933)(388.59102,26.83939933)
\curveto(389.98416,27.14898933)(391.44527,26.83939933)(392.87239,26.83939933)
\lineto(398.01005,26.83939933)
\curveto(399.46687,26.83939933)(401.11758,26.52183933)(402.29142,27.69567933)
\curveto(402.49325,27.89749933)(402.16378,28.29665933)(402.29142,28.55194933)
\curveto(402.47194,28.91298933)(402.96718,29.04718933)(403.1477,29.40822933)
\curveto(403.27799,29.66881933)(403.1477,31.59141933)(403.1477,31.97704933)
\curveto(403.1477,32.54789933)(403.32822,33.14804933)(403.1477,33.68959933)
\curveto(401.30967,39.20368933)(395.13066,39.39049933)(390.30357,40.53979933)
\curveto(387.47195,41.21399933)(384.58228,41.62091933)(381.74082,42.25234933)
\curveto(372.16525,44.38025933)(378.01843,43.96489933)(369.75296,43.96489933)
\lineto(368.04041,43.96489933)
\curveto(367.75499,43.96489933)(367.38596,44.16672933)(367.18414,43.96489933)
\curveto(366.31772,43.09848933)(365.97318,40.51951933)(365.47159,39.68352933)
\curveto(365.05623,38.99126933)(364.26336,38.60137933)(363.75904,37.97097933)
\curveto(363.11616,37.16736933)(362.77418,36.12984933)(362.04649,35.40214933)
\curveto(361.00356,34.35921933)(360.13339,35.00104933)(359.47766,33.68959933)
\curveto(359.35002,33.43430933)(359.47766,33.11874933)(359.47766,32.83332933)
\lineto(359.47766,31.97704933)
\lineto(359.47766,29.40822933)
\curveto(359.47766,28.83737933)(359.29714,28.23722933)(359.47766,27.69567933)
\curveto(359.76866,26.82266933)(361.58024,26.33280933)(362.04649,25.98312933)
\curveto(362.69233,25.49873933)(363.06678,24.68592933)(363.75904,24.27056933)
\curveto(364.50735,23.82158933)(369.28154,22.61694933)(369.75296,22.55801933)
\curveto(370.88585,22.41640933)(372.04203,22.67162933)(373.17807,22.55801933)
\curveto(374.90562,22.38526933)(376.58466,21.83490933)(378.31572,21.70174933)
\curveto(381.73073,21.43905933)(385.16592,21.70174933)(388.59102,21.70174933)
\lineto(399.7226,21.70174933)
\lineto(401.43515,21.70174933)
\curveto(402.29142,21.70174933)(403.17326,21.49406933)(404.00397,21.70174933)
\curveto(405.62678,22.10744933)(404.46094,22.61568933)(404.86025,23.41429933)
\curveto(405.04077,23.77533933)(405.4311,23.98514933)(405.71652,24.27056933)
\curveto(405.71652,24.84142933)(405.536,25.44156933)(405.71652,25.98312933)
\curveto(405.94141,26.65779933)(408.72998,28.58492933)(409.14162,29.40822933)
\curveto(409.26927,29.66351933)(409.14162,29.97907933)(409.14162,30.26449933)
\lineto(409.14162,31.12077933)
\curveto(409.14162,31.69162933)(409.45827,32.35834933)(409.14162,32.83332933)
\curveto(408.7876,33.36436933)(407.99992,33.40417933)(407.42907,33.68959933)
\curveto(403.85361,35.47732933)(407.24335,34.10157933)(400.57887,34.54587933)
\curveto(379.69716,35.93798933)(411.13782,35.40214933)(378.31572,35.40214933)
\curveto(355.71894,35.40214933)(375.48959,34.96086933)(358.62139,36.25842933)
\curveto(357.48305,36.34602933)(356.31582,36.03451933)(355.19629,36.25842933)
\curveto(354.80047,36.33762933)(354.46766,37.49763933)(354.34001,37.11469933)
\curveto(353.97897,36.03158933)(354.34001,34.83129933)(354.34001,33.68959933)
\lineto(354.34001,31.12077933)
\curveto(354.34001,30.54992933)(354.59678,29.91806933)(354.34001,29.40822933)
\curveto(353.76149,28.25950933)(351.37622,27.52829933)(351.95141,26.37791933)
\curveto(352.04991,26.18099933)(354.86281,28.09225933)(356.05256,27.69567933)
\curveto(356.65804,27.49384933)(357.17253,27.07642933)(357.76511,26.83939933)
\curveto(359.81734,26.01850933)(362.50818,25.36097933)(364.61531,25.12684933)
\curveto(365.75003,25.00076933)(366.9057,25.25292933)(368.04041,25.12684933)
\curveto(369.4869,24.96612933)(370.87238,24.40233933)(372.32179,24.27056933)
\curveto(374.02731,24.11552933)(375.75281,24.41278933)(377.45944,24.27056933)
\curveto(379.18962,24.12638933)(380.86692,23.55847933)(382.59709,23.41429933)
\curveto(384.30373,23.27207933)(386.02219,23.41429933)(387.73474,23.41429933)
\lineto(393.72867,23.41429933)
\curveto(400.10652,23.41429933)(399.94824,23.16647933)(406.5728,24.27056933)
\curveto(411.3778,25.07140933)(408.4947,24.96511933)(412.56672,25.98312933)
\curveto(412.84363,26.05232933)(413.16771,25.85547933)(413.423,25.98312933)
\curveto(413.78404,26.16363933)(413.94342,26.61548933)(414.27927,26.83939933)
\curveto(414.81031,27.19342933)(415.46079,27.34164933)(415.99182,27.69567933)
\curveto(416.32768,27.91957933)(416.56267,28.26652933)(416.8481,28.55194933)
\curveto(417.13352,28.83737933)(417.52386,29.04718933)(417.70437,29.40822933)
\curveto(417.83202,29.66351933)(417.77357,29.98759933)(417.70437,30.26449933)
\curveto(416.76531,34.02075933)(416.80428,33.09075933)(412.56672,35.40214933)
\curveto(410.88583,36.31900933)(409.10997,37.05411933)(407.42907,37.97097933)
\curveto(405.96799,38.76792933)(404.63629,39.79550933)(403.1477,40.53979933)
\curveto(400.99114,41.61807933)(397.89404,42.69981933)(395.44122,43.10862933)
\curveto(394.87814,43.20242933)(394.29952,43.10862933)(393.72867,43.10862933)
\curveto(389.56315,43.10862933)(387.26328,43.23470933)(382.59709,42.25234933)
\curveto(379.68109,41.63845933)(376.94016,40.34393933)(374.03434,39.68352933)
\curveto(370.64834,38.91397933)(367.17537,38.59212933)(363.75904,37.97097933)
\curveto(360.89521,37.45027933)(358.04245,36.86831933)(355.19629,36.25842933)
\curveto(354.04557,36.01184933)(352.9362,35.56857933)(351.77119,35.40214933)
\curveto(350.92352,35.28105933)(349.80784,36.00762933)(349.20236,35.40214933)
\curveto(348.59688,34.79666933)(349.20236,33.68959933)(349.20236,32.83332933)
\curveto(349.20236,31.97704933)(348.86687,31.05231933)(349.20236,30.26449933)
\curveto(351.18176,25.61640933)(352.2964,26.08968933)(356.90884,25.12684933)
\curveto(362.80904,23.89518933)(362.26966,24.27056933)(368.89669,24.27056933)
\lineto(390.30357,24.27056933)
\lineto(398.01005,24.27056933)
\curveto(398.86632,24.27056933)(399.72783,24.17596933)(400.57887,24.27056933)
\curveto(402.30443,24.46229933)(404.00054,24.86284933)(405.71652,25.12684933)
\curveto(407.71131,25.43373933)(409.72474,25.62208933)(411.71045,25.98312933)
\curveto(411.73765,25.98812933)(417.29679,27.28808933)(417.70437,27.69567933)
\curveto(418.10803,28.09932933)(417.70437,28.83737933)(417.70437,29.40822933)
\curveto(417.70437,29.97907933)(417.99807,30.63127933)(417.70437,31.12077933)
\curveto(416.42275,33.25681933)(414.4308,34.00409933)(412.56672,35.40214933)
\curveto(412.2438,35.64433933)(412.07149,36.07790933)(411.71045,36.25842933)
\curveto(410.90314,36.66207933)(409.97124,36.75914933)(409.14162,37.11469933)
\curveto(407.96837,37.61751933)(406.95487,38.51766933)(405.71652,38.82724933)
\curveto(403.48406,39.38536933)(401.13618,39.30521933)(398.86632,39.68352933)
\curveto(395.99515,40.16205933)(393.17873,40.94210933)(390.30357,41.39607933)
\curveto(387.75056,41.79917933)(385.16402,41.95035933)(382.59709,42.25234933)
\curveto(380.31168,42.52122933)(378.04296,42.95555933)(375.74689,43.10862933)
\curveto(373.75334,43.24152933)(371.75094,43.10862933)(369.75296,43.10862933)
\lineto(366.32786,43.10862933)
\curveto(365.75701,43.10862933)(365.17508,43.22057933)(364.61531,43.10862933)
\curveto(363.73025,42.93161933)(362.92213,42.47125933)(362.04649,42.25234933)
\curveto(361.76958,42.18314933)(361.39204,42.45417933)(361.19021,42.25234933)
\curveto(360.98839,42.05052933)(361.31786,41.65136933)(361.19021,41.39607933)
\curveto(361.00969,41.03503933)(360.51446,40.90083933)(360.33394,40.53979933)
\curveto(360.20629,40.28450933)(360.46158,39.93881933)(360.33394,39.68352933)
\curveto(360.22247,39.46059933)(358.80829,38.64034933)(359.47766,37.97097933)
\curveto(359.92896,37.51967933)(360.67963,37.49763933)(361.19021,37.11469933)
\curveto(364.37597,34.72537933)(361.28071,35.78504933)(365.47159,33.68959933)
\curveto(366.27889,33.28594933)(367.23311,33.23697933)(368.04041,32.83332933)
\curveto(369.52901,32.08902933)(370.80093,30.94043933)(372.32179,30.26449933)
\curveto(373.31118,29.82476933)(379.35126,28.68734933)(380.02827,28.55194933)
\lineto(384.30964,27.69567933)
\curveto(385.73677,27.41024933)(387.14453,27.00011933)(388.59102,26.83939933)
\curveto(389.72574,26.71331933)(390.87442,26.83939933)(392.01612,26.83939933)
\lineto(404.86025,26.83939933)
\lineto(411.71045,26.83939933)
\lineto(414.27927,26.83939933)
\curveto(415.13555,26.83939933)(416.01739,26.63171933)(416.8481,26.83939933)
\curveto(417.2397,26.93729933)(417.60647,27.30407933)(417.70437,27.69567933)
\curveto(417.91205,28.52637933)(417.70437,29.40822933)(417.70437,30.26449933)
\lineto(417.70437,31.97704933)
\curveto(417.70437,32.83332933)(417.97515,33.73353933)(417.70437,34.54587933)
\curveto(417.04413,36.52660933)(416.97042,34.74974933)(415.99182,35.40214933)
\curveto(415.32011,35.84995933)(414.95099,36.66688933)(414.27927,37.11469933)
\curveto(410.89318,39.37209933)(405.76348,40.29963933)(402.29142,41.39607933)
\curveto(399.70933,42.21147933)(397.2013,43.26720933)(394.58494,43.96489933)
\curveto(390.4941,45.05579933)(389.16587,44.82117933)(385.16592,44.82117933)
\curveto(382.40384,44.82117933)(380.15989,45.04508933)(377.45944,43.96489933)
\curveto(375.91418,43.34679933)(374.63915,42.19302933)(373.17807,41.39607933)
\curveto(369.26696,39.26274933)(365.05915,37.61296933)(361.19021,35.40214933)
\curveto(358.5594,33.89882933)(358.30139,33.01208933)(355.19629,31.97704933)
\curveto(354.65473,31.79652933)(353.88739,32.38069933)(353.48374,31.97704933)
\curveto(352.92641,31.41971933)(353.7624,30.11502933)(353.48374,29.40822933)
\curveto(353.06857,28.35520933)(351.27227,27.28343933)(351.95141,26.37791933)
\curveto(352.77543,25.27921933)(354.70273,26.23630933)(356.05256,25.98312933)
\curveto(363.0238,24.67554933)(352.89865,25.12684933)(362.90276,25.12684933)
\lineto(369.75296,25.12684933)
\lineto(376.60317,25.12684933)
\curveto(378.60114,25.12684933)(380.60419,25.26919933)(382.59709,25.12684933)
\curveto(396.28028,24.14947933)(383.67012,24.27056933)(391.15984,24.27056933)
\curveto(392.04441,24.27056933)(395.63174,24.08035933)(396.2975,24.27056933)
\curveto(405.42355,27.92099933)(395.76389,24.35856933)(404.86025,26.83939933)
\curveto(406.60183,27.31437933)(408.26217,28.05602933)(409.9979,28.55194933)
\curveto(410.27234,28.63034933)(410.72653,28.29665933)(410.85417,28.55194933)
\curveto(411.10947,29.06252933)(410.85417,29.69364933)(410.85417,30.26449933)
\curveto(410.85417,30.83534933)(411.03469,31.43549933)(410.85417,31.97704933)
\curveto(410.72653,32.35998933)(410.2218,32.49746933)(409.9979,32.83332933)
\curveto(409.64387,33.36436933)(409.52456,34.03528933)(409.14162,34.54587933)
\curveto(408.65724,35.19171933)(407.99992,35.68757933)(407.42907,36.25842933)
\curveto(406.85822,36.82927933)(406.40878,37.55561933)(405.71652,37.97097933)
\curveto(404.94255,38.43535933)(403.97731,38.47169933)(403.1477,38.82724933)
\curveto(401.97445,39.33007933)(400.9222,40.10357933)(399.7226,40.53979933)
\curveto(395.48072,42.08230933)(392.19909,42.66218933)(387.73474,43.10862933)
\curveto(386.03069,43.27902933)(384.30964,43.10862933)(382.59709,43.10862933)
\curveto(380.88454,43.10862933)(379.15877,43.32103933)(377.45944,43.10862933)
\curveto(373.40433,42.60173933)(369.48826,41.29292933)(365.47159,40.53979933)
\curveto(363.4879,40.16785933)(361.48358,39.90640933)(359.47766,39.68352933)
\curveto(358.9103,39.62052933)(358.30667,39.86404933)(357.76511,39.68352933)
\curveto(357.38217,39.55587933)(357.03648,39.21018933)(356.90884,38.82724933)
\curveto(356.63806,38.01491933)(356.90884,37.11469933)(356.90884,36.25842933)
\lineto(356.90884,33.68959933)
\curveto(356.90884,33.40417933)(356.81854,33.10409933)(356.90884,32.83332933)
\curveto(357.11066,32.22784933)(357.31382,31.57206933)(357.76511,31.12077933)
\curveto(358.21641,30.66947933)(358.90681,30.54992933)(359.47766,30.26449933)
\curveto(362.14764,28.92950933)(361.34953,29.14081933)(365.47159,28.55194933)
\curveto(372.14824,27.59813933)(375.635,27.39297933)(382.59709,27.69567933)
\curveto(391.31977,28.07491933)(400.24149,28.95553933)(407.42907,34.54587933)
\curveto(408.24141,35.17768933)(408.52415,36.29140933)(409.14162,37.11469933)
\curveto(409.38381,37.43761933)(409.81738,37.60993933)(409.9979,37.97097933)
\curveto(410.25319,38.48155933)(409.74261,39.17293933)(409.9979,39.68352933)
\curveto(410.17842,40.04456933)(410.67365,40.17876933)(410.85417,40.53979933)
\curveto(411.80071,42.43286933)(411.71045,42.50535933)(411.71045,43.96489933)
\lineto(411.71045,45.67744933)
\curveto(411.71045,45.96287933)(411.93879,46.36246933)(411.71045,46.53372933)
\curveto(410.68928,47.29960933)(409.40595,47.63503933)(408.28535,48.24627933)
\curveto(406.26516,49.34819933)(404.39426,50.73678933)(402.29142,51.67137933)
\curveto(399.23855,53.02820933)(396.01572,53.96487933)(392.87239,55.09647933)
\curveto(388.88007,56.53371933)(384.92582,58.08464933)(380.88454,59.37785933)
\curveto(377.78502,60.36969933)(374.63335,61.20130933)(371.46551,61.94667933)
\curveto(370.82877,62.09650933)(364.62292,63.23299933)(362.90276,62.80295933)
\curveto(360.19371,62.12569933)(360.23536,60.08625933)(358.62139,57.66530933)
\curveto(357.25942,55.62235933)(355.70198,53.71432933)(354.34001,51.67137933)
\curveto(353.41682,50.28659933)(352.83664,48.66855933)(351.77119,47.38999933)
\curveto(351.3626,46.89969933)(350.58967,46.88775933)(350.05864,46.53372933)
\curveto(349.72278,46.30981933)(349.38288,46.03848933)(349.20236,45.67744933)
\curveto(349.17996,45.63264933)(349.17996,43.15342933)(349.20236,43.10862933)
\curveto(349.38288,42.74758933)(349.77321,42.53777933)(350.05864,42.25234933)
\curveto(351.85339,40.45759933)(352.90241,40.66661933)(355.19629,39.68352933)
\curveto(356.36954,39.18070933)(357.36972,38.22130933)(358.62139,37.97097933)
\curveto(360.30068,37.63511933)(362.0524,38.11319933)(363.75904,37.97097933)
\curveto(365.48921,37.82679933)(367.1807,37.37869933)(368.89669,37.11469933)
\curveto(370.89148,36.80780933)(372.88064,36.44114933)(374.89062,36.25842933)
\curveto(376.02763,36.15505933)(377.18549,36.41988933)(378.31572,36.25842933)
\curveto(379.20924,36.13077933)(379.99423,35.55053933)(380.88454,35.40214933)
\curveto(382.01071,35.21445933)(383.16794,35.40214933)(384.30964,35.40214933)
\curveto(384.88049,35.40214933)(385.46839,35.26369933)(386.02219,35.40214933)
\curveto(387.86272,35.86227933)(391.55088,39.33459933)(392.01612,39.68352933)
\lineto(394.58494,41.39607933)
\curveto(395.15579,41.39607933)(395.78691,41.14078933)(396.2975,41.39607933)
\curveto(396.55279,41.52371933)(396.09567,42.05052933)(396.2975,42.25234933)
\curveto(396.49932,42.45417933)(396.95194,42.05052933)(397.15377,42.25234933)
\curveto(397.3556,42.45417933)(397.15377,42.82319933)(397.15377,43.10862933)
\curveto(397.15377,43.67947933)(397.26572,44.26140933)(397.15377,44.82117933)
\curveto(396.97676,45.70623933)(396.51641,46.51435933)(396.2975,47.38999933)
\curveto(396.2283,47.66690933)(396.42514,47.99098933)(396.2975,48.24627933)
\curveto(396.11698,48.60731933)(395.80226,48.92203933)(395.44122,49.10255933)
\curveto(395.18593,49.23019933)(394.84024,48.97490933)(394.58494,49.10255933)
\curveto(391.12979,50.83012933)(398.83514,49.36878933)(391.15984,51.67137933)
\curveto(387.63284,52.72947933)(384.44081,52.52765933)(380.88454,52.52765933)
\lineto(377.45944,52.52765933)
\curveto(376.31774,52.52765933)(375.16457,52.68911933)(374.03434,52.52765933)
\curveto(372.27044,52.27566933)(369.42193,50.64958933)(368.04041,49.95882933)
\curveto(365.40917,48.64320933)(361.15066,47.23180933)(360.33394,43.96489933)
\curveto(360.12626,43.13419933)(360.33394,42.25234933)(360.33394,41.39607933)
\curveto(360.33394,40.82522933)(360.51446,40.22507933)(360.33394,39.68352933)
\curveto(359.87726,38.31348933)(358.3417,39.38662933)(359.47766,37.11469933)
\curveto(359.60531,36.85940933)(360.06316,37.20499933)(360.33394,37.11469933)
\curveto(360.93942,36.91287933)(361.47564,36.54384933)(362.04649,36.25842933)
\curveto(363.18819,35.68757933)(364.26063,34.94952933)(365.47159,34.54587933)
\curveto(368.88006,33.40971933)(370.52576,33.68959933)(374.03434,33.68959933)
\lineto(377.45944,33.68959933)
\curveto(378.60114,33.68959933)(379.77003,33.44192933)(380.88454,33.68959933)
\curveto(382.385,34.02303933)(383.73879,34.83129933)(385.16592,35.40214933)
\curveto(386.59304,35.97299933)(388.03451,36.50921933)(389.44729,37.11469933)
\curveto(392.03111,38.22204933)(394.6014,39.36177933)(397.15377,40.53979933)
\curveto(398.31274,41.07470933)(399.49644,41.57582933)(400.57887,42.25234933)
\curveto(401.78907,43.00872933)(402.72751,44.18294933)(404.00397,44.82117933)
\curveto(404.51456,45.07646933)(405.20594,44.56588933)(405.71652,44.82117933)
\curveto(405.97181,44.94882933)(405.71652,45.39202933)(405.71652,45.67744933)
\lineto(405.71652,50.81510933)
\curveto(405.71652,52.24222933)(405.9964,53.69706933)(405.71652,55.09647933)
\curveto(405.33968,56.98070933)(404.24843,57.05393933)(403.1477,58.52157933)
\curveto(402.76476,59.03216933)(402.82246,59.88010933)(402.29142,60.23412933)
\curveto(401.81645,60.55077933)(401.13268,60.09567933)(400.57887,60.23412933)
\curveto(399.9597,60.38892933)(399.4718,60.88857933)(398.86632,61.09040933)
\curveto(398.42388,61.23788933)(395.03926,61.15530933)(394.58494,61.09040933)
\curveto(388.66625,60.24487933)(391.79227,60.44493933)(386.02219,58.52157933)
\curveto(384.90575,58.14942933)(383.64969,58.19160933)(382.59709,57.66530933)
\curveto(381.15453,56.94401933)(380.67536,55.53437933)(380.02827,54.24020933)
\curveto(379.74284,53.66935933)(379.52602,53.05869933)(379.17199,52.52765933)
\curveto(378.94809,52.19179933)(378.44336,52.05431933)(378.31572,51.67137933)
\curveto(378.1352,51.12981933)(378.31572,50.52967933)(378.31572,49.95882933)
\lineto(378.31572,48.24627933)
\curveto(378.31572,47.25156933)(378.12165,44.74115933)(378.31572,43.96489933)
\curveto(378.47051,43.34572933)(378.97017,42.85782933)(379.17199,42.25234933)
\curveto(379.26229,41.98157933)(379.04435,41.65136933)(379.17199,41.39607933)
\curveto(379.35251,41.03503933)(379.74284,40.82522933)(380.02827,40.53979933)
\curveto(380.59912,39.96894933)(381.0691,39.27505933)(381.74082,38.82724933)
\curveto(383.21807,37.84241933)(388.39029,35.63703933)(389.44729,35.40214933)
\curveto(390.56181,35.15447933)(391.78928,35.76318933)(392.87239,35.40214933)
\curveto(393.63827,35.14685933)(393.81907,33.94488933)(394.58494,33.68959933)
\curveto(395.66806,33.32855933)(396.87982,33.85105933)(398.01005,33.68959933)
\curveto(398.90357,33.56195933)(399.69381,33.01033933)(400.57887,32.83332933)
\curveto(401.13864,32.72136933)(401.72057,32.83332933)(402.29142,32.83332933)
\lineto(404.86025,32.83332933)
\curveto(405.4311,32.83332933)(406.04278,32.62131933)(406.5728,32.83332933)
\curveto(407.52831,33.21552933)(408.22115,34.08563933)(409.14162,34.54587933)
\curveto(410.51641,35.23326933)(412.02371,35.62238933)(413.423,36.25842933)
\curveto(415.16607,37.05072933)(416.81098,38.04961933)(418.56065,38.82724933)
\curveto(419.38545,39.19382933)(420.37847,39.18285933)(421.12948,39.68352933)
\curveto(421.36696,39.84184933)(421.00183,40.28450933)(421.12948,40.53979933)
\curveto(421.71426,41.70936933)(422.32044,40.91360933)(421.98575,42.25234933)
\curveto(421.06074,45.95238933)(421.34616,44.20656933)(418.56065,47.38999933)
\curveto(417.35716,48.76541933)(416.53956,50.50136933)(415.13555,51.67137933)
\curveto(414.69701,52.03682933)(413.95302,51.45936933)(413.423,51.67137933)
\curveto(412.46749,52.05357933)(411.73663,52.85445933)(410.85417,53.38392933)
\curveto(408.44434,54.82982933)(404.20494,55.60007933)(401.43515,55.09647933)
\curveto(392.05652,53.39127933)(398.72219,53.82175933)(390.30357,50.81510933)
\curveto(390.28177,50.80710933)(376.66832,47.40861933)(376.60317,47.38999933)
\curveto(374.86744,46.89407933)(373.2168,46.11527933)(371.46551,45.67744933)
\curveto(370.91171,45.53899933)(370.22794,45.99410933)(369.75296,45.67744933)
\curveto(369.22192,45.32342933)(369.25072,44.49593933)(368.89669,43.96489933)
\curveto(368.67278,43.62904933)(368.22093,43.46966933)(368.04041,43.10862933)
\curveto(367.91277,42.85333933)(368.13071,42.52312933)(368.04041,42.25234933)
\curveto(367.83859,41.64687933)(367.33893,41.15897933)(367.18414,40.53979933)
\curveto(367.04569,39.98599933)(367.18414,39.39809933)(367.18414,38.82724933)
\lineto(367.18414,35.40214933)
\curveto(367.18414,35.19447933)(367.15664,32.86085933)(367.18414,32.83332933)
\curveto(368.09336,31.92410933)(371.69967,31.25901933)(372.32179,31.12077933)
\curveto(373.74252,30.80505933)(375.151,30.36130933)(376.60317,30.26449933)
\curveto(382.98684,29.83891933)(385.9458,30.25359933)(392.01612,31.12077933)
\curveto(393.73484,31.36630933)(395.45894,31.60041933)(397.15377,31.97704933)
\curveto(398.03487,32.17284933)(398.84695,32.61441933)(399.7226,32.83332933)
\curveto(399.9995,32.90252933)(400.45123,32.57802933)(400.57887,32.83332933)
\curveto(400.83416,33.34390933)(400.57887,33.97502933)(400.57887,34.54587933)
\curveto(400.57887,35.40214933)(400.71964,36.27007933)(400.57887,37.11469933)
\curveto(400.31435,38.70184933)(399.5651,39.99850933)(398.86632,41.39607933)
\curveto(397.76315,43.60240933)(396.93322,46.08540933)(394.58494,47.38999933)
\curveto(393.00693,48.26667933)(391.10652,48.39145933)(389.44729,49.10255933)
\curveto(387.10079,50.10819933)(384.98746,51.63126933)(382.59709,52.52765933)
\curveto(380.39328,53.35408933)(377.97978,53.49590933)(375.74689,54.24020933)
\curveto(373.68471,54.92759933)(371.81515,56.12163933)(369.75296,56.80902933)
\curveto(366.82179,57.78608933)(365.46604,57.23994933)(362.90276,58.52157933)
\curveto(362.67983,58.63304933)(361.85958,60.04722933)(361.19021,59.37785933)
\curveto(360.98839,59.17602933)(361.31786,58.77686933)(361.19021,58.52157933)
\curveto(360.72998,57.60110933)(360.02309,56.82543933)(359.47766,55.95275933)
\curveto(357.71351,53.13010933)(355.99146,50.28003933)(354.34001,47.38999933)
\curveto(354.02336,46.83586933)(353.93503,46.12874933)(353.48374,45.67744933)
\curveto(353.03244,45.22615933)(352.22248,45.27247933)(351.77119,44.82117933)
\curveto(351.56936,44.61934933)(351.77119,44.25032933)(351.77119,43.96489933)
\lineto(351.77119,42.25234933)
\lineto(351.77119,41.39607933)
\curveto(351.77119,41.11064933)(351.61286,40.77728933)(351.77119,40.53979933)
\curveto(353.29006,38.26148933)(353.64796,39.00151933)(356.05256,37.97097933)
\curveto(357.22581,37.46815933)(358.26671,36.66207933)(359.47766,36.25842933)
\curveto(360.85836,35.79818933)(362.33831,35.71786933)(363.75904,35.40214933)
\curveto(367.70099,34.52615933)(367.74569,34.14720933)(372.32179,33.68959933)
\curveto(373.45782,33.57599933)(374.60519,33.68959933)(375.74689,33.68959933)
\lineto(380.02827,33.68959933)
\curveto(380.88454,33.68959933)(381.75745,33.52166933)(382.59709,33.68959933)
\curveto(384.69924,34.11002933)(387.52167,37.24886933)(389.44729,37.97097933)
\curveto(390.83455,38.49119933)(395.25233,38.44947933)(396.2975,40.53979933)
\curveto(396.42514,40.79509933)(396.2975,41.11064933)(396.2975,41.39607933)
\curveto(396.2975,42.25234933)(396.61551,43.16986933)(396.2975,43.96489933)
\curveto(396.247,44.09124933)(393.08865,47.10166933)(392.87239,47.38999933)
\curveto(392.10652,48.41116933)(392.06244,49.91250933)(391.15984,50.81510933)
\curveto(390.70855,51.26639933)(390.01814,51.38595933)(389.44729,51.67137933)
\curveto(383.03979,54.87513933)(388.42214,52.38577933)(378.31572,55.95275933)
\curveto(369.33365,59.12289933)(374.27522,58.52157933)(366.32786,58.52157933)
\lineto(362.90276,58.52157933)
\curveto(362.33191,58.52157933)(361.73177,58.70209933)(361.19021,58.52157933)
\curveto(360.80727,58.39393933)(360.6698,57.88920933)(360.33394,57.66530933)
\curveto(359.8029,57.31127933)(359.00432,57.31961933)(358.62139,56.80902933)
\curveto(357.88739,55.83036933)(357.00285,50.33489933)(356.90884,49.95882933)
\curveto(356.47101,48.20753933)(355.63411,46.57246933)(355.19629,44.82117933)
\curveto(355.11909,44.51229933)(355.11909,41.70495933)(355.19629,41.39607933)
\curveto(355.4152,40.52042933)(355.51101,39.54932933)(356.05256,38.82724933)
\curveto(356.4355,38.31666933)(357.19426,38.25639933)(357.76511,37.97097933)
\curveto(359.73938,36.98383933)(361.59159,35.94400933)(363.75904,35.40214933)
\curveto(364.31284,35.26369933)(364.90074,35.40214933)(365.47159,35.40214933)
\curveto(365.75701,35.40214933)(366.60292,35.32594933)(366.32786,35.40214933)
\curveto(364.88754,35.80139933)(363.41023,36.05265933)(361.95141,36.37791933)
}
}
{
\newrgbcolor{curcolor}{0 0 0}
\pscustom[linestyle=none,fillstyle=solid,fillcolor=curcolor]
{
\newpath
\moveto(34.279535,181.87207733)
\curveto(34.87133187,181.87207733)(35.35766,181.93360077)(35.73851937,182.05664765)
\curveto(36.41820687,182.28516327)(36.9748475,182.7246164)(37.40844125,183.37500702)
\curveto(37.75414437,183.8964914)(38.00316781,184.56446015)(38.15551156,185.37891327)
\curveto(38.24340219,185.8652414)(38.2873475,186.31641327)(38.2873475,186.7324289)
\curveto(38.2873475,188.33203827)(37.96801156,189.57422577)(37.32933969,190.4589914)
\curveto(36.69652719,191.34375702)(35.67406625,191.78613983)(34.26195687,191.78613983)
\lineto(31.15941781,191.78613983)
\lineto(31.15941781,181.87207733)
\lineto(34.279535,181.87207733)
\closepath
\moveto(29.40160531,193.28906952)
\lineto(34.6310975,193.28906952)
\curveto(36.40648812,193.28906952)(37.78344125,192.65918671)(38.76195687,191.39942108)
\curveto(39.63500375,190.26270233)(40.07152719,188.80664765)(40.07152719,187.03125702)
\curveto(40.07152719,185.66016327)(39.81371469,184.42090546)(39.29808969,183.31348358)
\curveto(38.38988656,181.35645233)(36.82836312,180.37793671)(34.61351937,180.37793671)
\lineto(29.40160531,180.37793671)
\lineto(29.40160531,193.28906952)
\closepath
}
}
{
\newrgbcolor{curcolor}{0 0 0}
\pscustom[linestyle=none,fillstyle=solid,fillcolor=curcolor]
{
\newpath
\moveto(46.03930062,190.00196015)
\curveto(46.70726937,190.00196015)(47.35473031,189.84375702)(47.98168344,189.52735077)
\curveto(48.60863656,189.2168039)(49.08617562,188.81250702)(49.41430062,188.31446015)
\curveto(49.73070687,187.83985077)(49.94164437,187.28613983)(50.04711312,186.65332733)
\curveto(50.14086312,186.21973358)(50.18773812,185.52832733)(50.18773812,184.57910858)
\lineto(43.28832406,184.57910858)
\curveto(43.31762094,183.62403046)(43.54320687,182.85645233)(43.96508187,182.27637421)
\curveto(44.38695687,181.70215546)(45.04027719,181.41504608)(45.92504281,181.41504608)
\curveto(46.75121469,181.41504608)(47.41039437,181.68750702)(47.90258187,182.2324289)
\curveto(48.18383187,182.54883515)(48.38305062,182.91504608)(48.50023812,183.33106171)
\lineto(50.05590219,183.33106171)
\curveto(50.01488656,182.98535858)(49.87719125,182.59863983)(49.64281625,182.17090546)
\curveto(49.41430062,181.74903046)(49.15648812,181.40332733)(48.86937875,181.13379608)
\curveto(48.38891,180.66504608)(47.79418344,180.34863983)(47.08519906,180.18457733)
\curveto(46.70433969,180.09082733)(46.27367562,180.04395233)(45.79320687,180.04395233)
\curveto(44.62133187,180.04395233)(43.62816781,180.46875702)(42.81371469,181.3183664)
\curveto(41.99926156,182.17383515)(41.592035,183.36914765)(41.592035,184.9043039)
\curveto(41.592035,186.41602265)(42.00219125,187.64356171)(42.82250375,188.58692108)
\curveto(43.64281625,189.53028046)(44.71508187,190.00196015)(46.03930062,190.00196015)
\closepath
\moveto(48.56176156,185.83594452)
\curveto(48.49730844,186.5214914)(48.34789437,187.06934296)(48.11351937,187.47949921)
\curveto(47.67992562,188.24121796)(46.95629281,188.62207733)(45.94262094,188.62207733)
\curveto(45.21605844,188.62207733)(44.60668344,188.35840546)(44.11449594,187.83106171)
\curveto(43.62230844,187.30957733)(43.36156625,186.64453827)(43.33226937,185.83594452)
\lineto(48.56176156,185.83594452)
\closepath
\moveto(45.88988656,190.01953827)
\lineto(45.88988656,190.01953827)
\closepath
}
}
{
\newrgbcolor{curcolor}{0 0 0}
\pscustom[linestyle=none,fillstyle=solid,fillcolor=curcolor]
{
\newpath
\moveto(53.07933969,183.33106171)
\curveto(53.12621469,182.80371796)(53.25805062,182.39942108)(53.4748475,182.11817108)
\curveto(53.873285,181.60840546)(54.56469125,181.35352265)(55.54906625,181.35352265)
\curveto(56.13500375,181.35352265)(56.65062875,181.47949921)(57.09594125,181.73145233)
\curveto(57.54125375,181.98926483)(57.76391,182.38477265)(57.76391,182.91797577)
\curveto(57.76391,183.32227265)(57.58519906,183.62988983)(57.22777719,183.84082733)
\curveto(56.99926156,183.96973358)(56.54808969,184.11914765)(55.87426156,184.28906952)
\lineto(54.61742562,184.60547577)
\curveto(53.81469125,184.80469452)(53.22289437,185.02735077)(52.842035,185.27344452)
\curveto(52.1623475,185.7011789)(51.82250375,186.29297577)(51.82250375,187.04883515)
\curveto(51.82250375,187.93946015)(52.14183969,188.66016327)(52.78051156,189.21094452)
\curveto(53.42504281,189.76172577)(54.28930062,190.0371164)(55.373285,190.0371164)
\curveto(56.79125375,190.0371164)(57.81371469,189.62110077)(58.44066781,188.78906952)
\curveto(58.83324594,188.26172577)(59.02367562,187.6933664)(59.01195687,187.0839914)
\lineto(57.51781625,187.0839914)
\curveto(57.48851937,187.44141327)(57.36254281,187.76660858)(57.13988656,188.05957733)
\curveto(56.77660531,188.47559296)(56.1467225,188.68360077)(55.25023812,188.68360077)
\curveto(54.65258187,188.68360077)(54.19848031,188.56934296)(53.88793344,188.34082733)
\curveto(53.58324594,188.11231171)(53.43090219,187.8105539)(53.43090219,187.4355539)
\curveto(53.43090219,187.02539765)(53.63305062,186.69727265)(54.0373475,186.4511789)
\curveto(54.2717225,186.30469452)(54.61742562,186.17578827)(55.07445687,186.06446015)
\lineto(56.12035531,185.80957733)
\curveto(57.25707406,185.53418671)(58.01879281,185.26758515)(58.40551156,185.00977265)
\curveto(59.02074594,184.60547577)(59.32836312,183.96973358)(59.32836312,183.10254608)
\curveto(59.32836312,182.26465546)(59.00902719,181.54102265)(58.37035531,180.93164765)
\curveto(57.73754281,180.32227265)(56.77074594,180.01758515)(55.46996469,180.01758515)
\curveto(54.06957406,180.01758515)(53.07641,180.3339914)(52.4904725,180.9668039)
\curveto(51.91039437,181.60547577)(51.5998475,182.39356171)(51.55883187,183.33106171)
\lineto(53.07933969,183.33106171)
\closepath
\moveto(55.41723031,190.01953827)
\lineto(55.41723031,190.01953827)
\closepath
}
}
{
\newrgbcolor{curcolor}{0 0 0}
\pscustom[linestyle=none,fillstyle=solid,fillcolor=curcolor]
{
\newpath
\moveto(61.17406625,189.74707733)
\lineto(62.78246469,189.74707733)
\lineto(62.78246469,180.37793671)
\lineto(61.17406625,180.37793671)
\lineto(61.17406625,189.74707733)
\closepath
\moveto(61.17406625,193.28906952)
\lineto(62.78246469,193.28906952)
\lineto(62.78246469,191.49610077)
\lineto(61.17406625,191.49610077)
\lineto(61.17406625,193.28906952)
\closepath
}
}
{
\newrgbcolor{curcolor}{0 0 0}
\pscustom[linestyle=none,fillstyle=solid,fillcolor=curcolor]
{
\newpath
\moveto(65.22582406,189.79102265)
\lineto(66.72875375,189.79102265)
\lineto(66.72875375,188.16504608)
\curveto(66.85180062,188.48145233)(67.15355844,188.8652414)(67.63402719,189.31641327)
\curveto(68.11449594,189.77344452)(68.66820687,190.00196015)(69.29516,190.00196015)
\curveto(69.32445687,190.00196015)(69.37426156,189.99903046)(69.44457406,189.99317108)
\curveto(69.51488656,189.98731171)(69.63500375,189.97559296)(69.80492562,189.95801483)
\lineto(69.80492562,188.28809296)
\curveto(69.71117562,188.30567108)(69.623285,188.31738983)(69.54125375,188.32324921)
\curveto(69.46508187,188.32910858)(69.38012094,188.33203827)(69.28637094,188.33203827)
\curveto(68.48949594,188.33203827)(67.87719125,188.07422577)(67.44945687,187.55860077)
\curveto(67.0217225,187.04883515)(66.80785531,186.45996796)(66.80785531,185.79199921)
\lineto(66.80785531,180.37793671)
\lineto(65.22582406,180.37793671)
\lineto(65.22582406,189.79102265)
\closepath
}
}
{
\newrgbcolor{curcolor}{0 0 0}
\pscustom[linestyle=none,fillstyle=solid,fillcolor=curcolor]
{
\newpath
\moveto(75.11351937,190.00196015)
\curveto(75.78148812,190.00196015)(76.42894906,189.84375702)(77.05590219,189.52735077)
\curveto(77.68285531,189.2168039)(78.16039437,188.81250702)(78.48851937,188.31446015)
\curveto(78.80492562,187.83985077)(79.01586312,187.28613983)(79.12133187,186.65332733)
\curveto(79.21508187,186.21973358)(79.26195687,185.52832733)(79.26195687,184.57910858)
\lineto(72.36254281,184.57910858)
\curveto(72.39183969,183.62403046)(72.61742562,182.85645233)(73.03930062,182.27637421)
\curveto(73.46117562,181.70215546)(74.11449594,181.41504608)(74.99926156,181.41504608)
\curveto(75.82543344,181.41504608)(76.48461312,181.68750702)(76.97680062,182.2324289)
\curveto(77.25805062,182.54883515)(77.45726937,182.91504608)(77.57445687,183.33106171)
\lineto(79.13012094,183.33106171)
\curveto(79.08910531,182.98535858)(78.95141,182.59863983)(78.717035,182.17090546)
\curveto(78.48851937,181.74903046)(78.23070687,181.40332733)(77.9435975,181.13379608)
\curveto(77.46312875,180.66504608)(76.86840219,180.34863983)(76.15941781,180.18457733)
\curveto(75.77855844,180.09082733)(75.34789437,180.04395233)(74.86742562,180.04395233)
\curveto(73.69555062,180.04395233)(72.70238656,180.46875702)(71.88793344,181.3183664)
\curveto(71.07348031,182.17383515)(70.66625375,183.36914765)(70.66625375,184.9043039)
\curveto(70.66625375,186.41602265)(71.07641,187.64356171)(71.8967225,188.58692108)
\curveto(72.717035,189.53028046)(73.78930062,190.00196015)(75.11351937,190.00196015)
\closepath
\moveto(77.63598031,185.83594452)
\curveto(77.57152719,186.5214914)(77.42211312,187.06934296)(77.18773812,187.47949921)
\curveto(76.75414437,188.24121796)(76.03051156,188.62207733)(75.01683969,188.62207733)
\curveto(74.29027719,188.62207733)(73.68090219,188.35840546)(73.18871469,187.83106171)
\curveto(72.69652719,187.30957733)(72.435785,186.64453827)(72.40648812,185.83594452)
\lineto(77.63598031,185.83594452)
\closepath
\moveto(74.96410531,190.01953827)
\lineto(74.96410531,190.01953827)
\closepath
}
}
{
\newrgbcolor{curcolor}{0 0 0}
\pscustom[linestyle=none,fillstyle=solid,fillcolor=curcolor]
{
\newpath
\moveto(82.21508187,184.9746164)
\curveto(82.21508187,183.9668039)(82.42894906,183.1230539)(82.85668344,182.4433664)
\curveto(83.28441781,181.7636789)(83.96996469,181.42383515)(84.91332406,181.42383515)
\curveto(85.64574594,181.42383515)(86.24633187,181.73731171)(86.71508187,182.36426483)
\curveto(87.18969125,182.99707733)(87.42699594,183.90235077)(87.42699594,185.08008515)
\curveto(87.42699594,186.26953827)(87.18383187,187.14844452)(86.69750375,187.7168039)
\curveto(86.21117562,188.29102265)(85.61058969,188.57813202)(84.89574594,188.57813202)
\curveto(84.09887094,188.57813202)(83.45141,188.27344452)(82.95336312,187.66406952)
\curveto(82.46117562,187.05469452)(82.21508187,186.15821015)(82.21508187,184.9746164)
\closepath
\moveto(84.59691781,189.95801483)
\curveto(85.31762094,189.95801483)(85.92113656,189.80567108)(86.40746469,189.50098358)
\curveto(86.68871469,189.32520233)(87.00805062,189.01758515)(87.3654725,188.57813202)
\lineto(87.3654725,193.33301483)
\lineto(88.88598031,193.33301483)
\lineto(88.88598031,180.37793671)
\lineto(87.46215219,180.37793671)
\lineto(87.46215219,181.68750702)
\curveto(87.09301156,181.1074289)(86.65648812,180.68848358)(86.15258187,180.43067108)
\curveto(85.64867562,180.17285858)(85.07152719,180.04395233)(84.42113656,180.04395233)
\curveto(83.37230844,180.04395233)(82.46410531,180.48340546)(81.69652719,181.36231171)
\curveto(80.92894906,182.24707733)(80.54516,183.42188202)(80.54516,184.88672577)
\curveto(80.54516,186.25781952)(80.89379281,187.44434296)(81.59105844,188.44629608)
\curveto(82.29418344,189.45410858)(83.29613656,189.95801483)(84.59691781,189.95801483)
\closepath
}
}
{
\newrgbcolor{curcolor}{0 0 0}
\pscustom[linestyle=none,fillstyle=solid,fillcolor=curcolor]
{
\newpath
\moveto(102.00805062,193.64063202)
\curveto(104.2873475,193.64063202)(105.9748475,192.90821015)(107.07055062,191.4433664)
\curveto(107.92601937,190.30078827)(108.35375375,188.83887421)(108.35375375,187.05762421)
\curveto(108.35375375,185.12988983)(107.86449594,183.52735077)(106.88598031,182.25000702)
\curveto(105.73754281,180.75000702)(104.0998475,180.00000702)(101.97289437,180.00000702)
\curveto(99.98656625,180.00000702)(98.42504281,180.65625702)(97.28832406,181.96875702)
\curveto(96.27465219,183.23438202)(95.76781625,184.8339914)(95.76781625,186.76758515)
\curveto(95.76781625,188.5136789)(96.20141,190.00781952)(97.0685975,191.25000702)
\curveto(98.18187875,192.84375702)(99.82836312,193.64063202)(102.00805062,193.64063202)
\closepath
\moveto(102.18383187,181.56446015)
\curveto(103.7248475,181.56446015)(104.83812875,182.1152414)(105.52367562,183.2168039)
\curveto(106.21508187,184.32422577)(106.560785,185.59571015)(106.560785,187.03125702)
\curveto(106.560785,188.54883515)(106.1623475,189.77051483)(105.3654725,190.69629608)
\curveto(104.57445687,191.62207733)(103.4904725,192.08496796)(102.11351937,192.08496796)
\curveto(100.77758187,192.08496796)(99.68773812,191.62500702)(98.84398812,190.70508515)
\curveto(98.00023812,189.79102265)(97.57836312,188.44043671)(97.57836312,186.65332733)
\curveto(97.57836312,185.22363983)(97.93871469,184.01660858)(98.65941781,183.03223358)
\curveto(99.38598031,182.05371796)(100.560785,181.56446015)(102.18383187,181.56446015)
\closepath
\moveto(102.060785,193.64063202)
\lineto(102.060785,193.64063202)
\closepath
}
}
{
\newrgbcolor{curcolor}{0 0 0}
\pscustom[linestyle=none,fillstyle=solid,fillcolor=curcolor]
{
\newpath
\moveto(110.09398812,193.33301483)
\lineto(111.63207406,193.33301483)
\lineto(111.63207406,188.64844452)
\curveto(111.97777719,189.0996164)(112.39086312,189.44238983)(112.87133187,189.67676483)
\curveto(113.35180062,189.91699921)(113.873285,190.0371164)(114.435785,190.0371164)
\curveto(115.60766,190.0371164)(116.55687875,189.63281952)(117.28344125,188.82422577)
\curveto(118.01586312,188.0214914)(118.38207406,186.83496796)(118.38207406,185.26465546)
\curveto(118.38207406,183.77637421)(118.0217225,182.54004608)(117.30101937,181.55567108)
\curveto(116.58031625,180.57129608)(115.58129281,180.07910858)(114.30394906,180.07910858)
\curveto(113.58910531,180.07910858)(112.98558969,180.25196015)(112.49340219,180.59766327)
\curveto(112.20043344,180.8027414)(111.88695687,181.1308664)(111.5529725,181.58203827)
\lineto(111.5529725,180.37793671)
\lineto(110.09398812,180.37793671)
\lineto(110.09398812,193.33301483)
\closepath
\moveto(114.20726937,181.47656952)
\curveto(115.06273812,181.47656952)(115.70141,181.81641327)(116.123285,182.49610077)
\curveto(116.55101937,183.17578827)(116.76488656,184.07227265)(116.76488656,185.1855539)
\curveto(116.76488656,186.17578827)(116.55101937,186.99610077)(116.123285,187.6464914)
\curveto(115.70141,188.29688202)(115.07738656,188.62207733)(114.25121469,188.62207733)
\curveto(113.53051156,188.62207733)(112.89769906,188.35547577)(112.35277719,187.82227265)
\curveto(111.81371469,187.28906952)(111.54418344,186.41016327)(111.54418344,185.1855539)
\curveto(111.54418344,184.30078827)(111.65551156,183.58301483)(111.87816781,183.03223358)
\curveto(112.29418344,181.99512421)(113.07055062,181.47656952)(114.20726937,181.47656952)
\closepath
}
}
{
\newrgbcolor{curcolor}{0 0 0}
\pscustom[linestyle=none,fillstyle=solid,fillcolor=curcolor]
{
\newpath
\moveto(121.8185975,191.46094452)
\lineto(120.23656625,191.46094452)
\lineto(120.23656625,193.28906952)
\lineto(121.8185975,193.28906952)
\lineto(121.8185975,191.46094452)
\closepath
\moveto(118.74242562,177.99610077)
\curveto(119.45141,178.01953827)(119.87621469,178.0839914)(120.01683969,178.18946015)
\curveto(120.16332406,178.28906952)(120.23656625,178.60840546)(120.23656625,179.14746796)
\lineto(120.23656625,189.74707733)
\lineto(121.8185975,189.74707733)
\lineto(121.8185975,178.98047577)
\curveto(121.8185975,178.2949289)(121.70726937,177.78223358)(121.48461312,177.44238983)
\curveto(121.1154725,176.86817108)(120.41527719,176.58106171)(119.38402719,176.58106171)
\curveto(119.30785531,176.58106171)(119.22582406,176.5839914)(119.13793344,176.58985077)
\curveto(119.05590219,176.59571015)(118.92406625,176.6074289)(118.74242562,176.62500702)
\lineto(118.74242562,177.99610077)
\closepath
}
}
{
\newrgbcolor{curcolor}{0 0 0}
\pscustom[linestyle=none,fillstyle=solid,fillcolor=curcolor]
{
\newpath
\moveto(128.16430063,190.00196015)
\curveto(128.83226938,190.00196015)(129.47973031,189.84375702)(130.10668344,189.52735077)
\curveto(130.73363656,189.2168039)(131.21117563,188.81250702)(131.53930063,188.31446015)
\curveto(131.85570688,187.83985077)(132.06664438,187.28613983)(132.17211313,186.65332733)
\curveto(132.26586313,186.21973358)(132.31273813,185.52832733)(132.31273813,184.57910858)
\lineto(125.41332406,184.57910858)
\curveto(125.44262094,183.62403046)(125.66820687,182.85645233)(126.09008187,182.27637421)
\curveto(126.51195687,181.70215546)(127.16527719,181.41504608)(128.05004281,181.41504608)
\curveto(128.87621469,181.41504608)(129.53539438,181.68750702)(130.02758188,182.2324289)
\curveto(130.30883188,182.54883515)(130.50805063,182.91504608)(130.62523813,183.33106171)
\lineto(132.18090219,183.33106171)
\curveto(132.13988656,182.98535858)(132.00219125,182.59863983)(131.76781625,182.17090546)
\curveto(131.53930063,181.74903046)(131.28148813,181.40332733)(130.99437875,181.13379608)
\curveto(130.51391,180.66504608)(129.91918344,180.34863983)(129.21019906,180.18457733)
\curveto(128.82933969,180.09082733)(128.39867563,180.04395233)(127.91820687,180.04395233)
\curveto(126.74633187,180.04395233)(125.75316781,180.46875702)(124.93871469,181.3183664)
\curveto(124.12426156,182.17383515)(123.717035,183.36914765)(123.717035,184.9043039)
\curveto(123.717035,186.41602265)(124.12719125,187.64356171)(124.94750375,188.58692108)
\curveto(125.76781625,189.53028046)(126.84008188,190.00196015)(128.16430063,190.00196015)
\closepath
\moveto(130.68676156,185.83594452)
\curveto(130.62230844,186.5214914)(130.47289438,187.06934296)(130.23851938,187.47949921)
\curveto(129.80492563,188.24121796)(129.08129281,188.62207733)(128.06762094,188.62207733)
\curveto(127.34105844,188.62207733)(126.73168344,188.35840546)(126.23949594,187.83106171)
\curveto(125.74730844,187.30957733)(125.48656625,186.64453827)(125.45726937,185.83594452)
\lineto(130.68676156,185.83594452)
\closepath
\moveto(128.01488656,190.01953827)
\lineto(128.01488656,190.01953827)
\closepath
}
}
{
\newrgbcolor{curcolor}{0 0 0}
\pscustom[linestyle=none,fillstyle=solid,fillcolor=curcolor]
{
\newpath
\moveto(137.89379281,190.06348358)
\curveto(138.95433969,190.06348358)(139.81566781,189.80567108)(140.47777719,189.29004608)
\curveto(141.14574594,188.77442108)(141.54711313,187.88672577)(141.68187875,186.62696015)
\lineto(140.14379281,186.62696015)
\curveto(140.05004281,187.20703827)(139.83617563,187.68750702)(139.50219125,188.0683664)
\curveto(139.16820688,188.45508515)(138.63207406,188.64844452)(137.89379281,188.64844452)
\curveto(136.88598031,188.64844452)(136.16527719,188.15625702)(135.73168344,187.17188202)
\curveto(135.45043344,186.53321015)(135.30980844,185.74512421)(135.30980844,184.80762421)
\curveto(135.30980844,183.86426483)(135.50902719,183.07031952)(135.90746469,182.42578827)
\curveto(136.30590219,181.78125702)(136.93285531,181.4589914)(137.78832406,181.4589914)
\curveto(138.44457406,181.4589914)(138.96312875,181.65821015)(139.34398813,182.05664765)
\curveto(139.73070688,182.46094452)(139.99730844,183.01172577)(140.14379281,183.7089914)
\lineto(141.68187875,183.7089914)
\curveto(141.5060975,182.46094452)(141.06664438,181.54688202)(140.36351938,180.9668039)
\curveto(139.66039438,180.39258515)(138.76098031,180.10547577)(137.66527719,180.10547577)
\curveto(136.43480844,180.10547577)(135.45336313,180.55371796)(134.72094125,181.45020233)
\curveto(133.98851938,182.35254608)(133.62230844,183.47754608)(133.62230844,184.82520233)
\curveto(133.62230844,186.47754608)(134.02367563,187.7636789)(134.82641,188.68360077)
\curveto(135.62914438,189.60352265)(136.65160531,190.06348358)(137.89379281,190.06348358)
\closepath
\moveto(137.64769906,190.01953827)
\lineto(137.64769906,190.01953827)
\closepath
}
}
{
\newrgbcolor{curcolor}{0 0 0}
\pscustom[linestyle=none,fillstyle=solid,fillcolor=curcolor]
{
\newpath
\moveto(143.58031625,192.41895233)
\lineto(145.17992563,192.41895233)
\lineto(145.17992563,189.79102265)
\lineto(146.68285531,189.79102265)
\lineto(146.68285531,188.49903046)
\lineto(145.17992563,188.49903046)
\lineto(145.17992563,182.35547577)
\curveto(145.17992563,182.02735077)(145.29125375,181.80762421)(145.51391,181.69629608)
\curveto(145.63695688,181.63184296)(145.842035,181.5996164)(146.12914438,181.5996164)
\lineto(146.37523813,181.5996164)
\curveto(146.46312875,181.60547577)(146.56566781,181.61426483)(146.68285531,181.62598358)
\lineto(146.68285531,180.37793671)
\curveto(146.50121469,180.32520233)(146.310785,180.2871164)(146.11156625,180.2636789)
\curveto(145.91820688,180.2402414)(145.70726938,180.22852265)(145.47875375,180.22852265)
\curveto(144.7404725,180.22852265)(144.23949594,180.41602265)(143.97582406,180.79102265)
\curveto(143.71215219,181.17188202)(143.58031625,181.66406952)(143.58031625,182.26758515)
\lineto(143.58031625,188.49903046)
\lineto(142.30590219,188.49903046)
\lineto(142.30590219,189.79102265)
\lineto(143.58031625,189.79102265)
\lineto(143.58031625,192.41895233)
\closepath
}
}
{
\newrgbcolor{curcolor}{0 0 0}
\pscustom[linestyle=none,fillstyle=solid,fillcolor=curcolor]
{
\newpath
\moveto(245.94945688,185.66895233)
\lineto(243.98949594,191.3730539)
\lineto(241.90648813,185.66895233)
\lineto(245.94945688,185.66895233)
\closepath
\moveto(243.07543344,193.28906952)
\lineto(245.0529725,193.28906952)
\lineto(249.73754281,180.37793671)
\lineto(247.82152719,180.37793671)
\lineto(246.51195688,184.24512421)
\lineto(241.40551156,184.24512421)
\lineto(240.00805063,180.37793671)
\lineto(238.21508188,180.37793671)
\lineto(243.07543344,193.28906952)
\closepath
\moveto(243.98070688,193.28906952)
\lineto(243.98070688,193.28906952)
\closepath
}
}
{
\newrgbcolor{curcolor}{0 0 0}
\pscustom[linestyle=none,fillstyle=solid,fillcolor=curcolor]
{
\newpath
\moveto(254.76488656,190.06348358)
\curveto(255.82543344,190.06348358)(256.68676156,189.80567108)(257.34887094,189.29004608)
\curveto(258.01683969,188.77442108)(258.41820688,187.88672577)(258.5529725,186.62696015)
\lineto(257.01488656,186.62696015)
\curveto(256.92113656,187.20703827)(256.70726938,187.68750702)(256.373285,188.0683664)
\curveto(256.03930063,188.45508515)(255.50316781,188.64844452)(254.76488656,188.64844452)
\curveto(253.75707406,188.64844452)(253.03637094,188.15625702)(252.60277719,187.17188202)
\curveto(252.32152719,186.53321015)(252.18090219,185.74512421)(252.18090219,184.80762421)
\curveto(252.18090219,183.86426483)(252.38012094,183.07031952)(252.77855844,182.42578827)
\curveto(253.17699594,181.78125702)(253.80394906,181.4589914)(254.65941781,181.4589914)
\curveto(255.31566781,181.4589914)(255.8342225,181.65821015)(256.21508188,182.05664765)
\curveto(256.60180063,182.46094452)(256.86840219,183.01172577)(257.01488656,183.7089914)
\lineto(258.5529725,183.7089914)
\curveto(258.37719125,182.46094452)(257.93773813,181.54688202)(257.23461313,180.9668039)
\curveto(256.53148813,180.39258515)(255.63207406,180.10547577)(254.53637094,180.10547577)
\curveto(253.30590219,180.10547577)(252.32445688,180.55371796)(251.592035,181.45020233)
\curveto(250.85961313,182.35254608)(250.49340219,183.47754608)(250.49340219,184.82520233)
\curveto(250.49340219,186.47754608)(250.89476938,187.7636789)(251.69750375,188.68360077)
\curveto(252.50023813,189.60352265)(253.52269906,190.06348358)(254.76488656,190.06348358)
\closepath
\moveto(254.51879281,190.01953827)
\lineto(254.51879281,190.01953827)
\closepath
}
}
{
\newrgbcolor{curcolor}{0 0 0}
\pscustom[linestyle=none,fillstyle=solid,fillcolor=curcolor]
{
\newpath
\moveto(260.45141,192.41895233)
\lineto(262.05101938,192.41895233)
\lineto(262.05101938,189.79102265)
\lineto(263.55394906,189.79102265)
\lineto(263.55394906,188.49903046)
\lineto(262.05101938,188.49903046)
\lineto(262.05101938,182.35547577)
\curveto(262.05101938,182.02735077)(262.1623475,181.80762421)(262.38500375,181.69629608)
\curveto(262.50805063,181.63184296)(262.71312875,181.5996164)(263.00023813,181.5996164)
\lineto(263.24633188,181.5996164)
\curveto(263.3342225,181.60547577)(263.43676156,181.61426483)(263.55394906,181.62598358)
\lineto(263.55394906,180.37793671)
\curveto(263.37230844,180.32520233)(263.18187875,180.2871164)(262.98266,180.2636789)
\curveto(262.78930063,180.2402414)(262.57836313,180.22852265)(262.3498475,180.22852265)
\curveto(261.61156625,180.22852265)(261.11058969,180.41602265)(260.84691781,180.79102265)
\curveto(260.58324594,181.17188202)(260.45141,181.66406952)(260.45141,182.26758515)
\lineto(260.45141,188.49903046)
\lineto(259.17699594,188.49903046)
\lineto(259.17699594,189.79102265)
\lineto(260.45141,189.79102265)
\lineto(260.45141,192.41895233)
\closepath
}
}
{
\newrgbcolor{curcolor}{0 0 0}
\pscustom[linestyle=none,fillstyle=solid,fillcolor=curcolor]
{
\newpath
\moveto(266.74437875,189.79102265)
\lineto(266.74437875,183.54199921)
\curveto(266.74437875,183.06153046)(266.82055063,182.66895233)(266.97289438,182.36426483)
\curveto(267.25414438,181.80176483)(267.77855844,181.52051483)(268.54613656,181.52051483)
\curveto(269.64769906,181.52051483)(270.39769906,182.01270233)(270.79613656,182.99707733)
\curveto(271.01293344,183.52442108)(271.12133188,184.2480539)(271.12133188,185.16797577)
\lineto(271.12133188,189.79102265)
\lineto(272.70336313,189.79102265)
\lineto(272.70336313,180.37793671)
\lineto(271.2092225,180.37793671)
\lineto(271.22680063,181.76660858)
\curveto(271.0217225,181.40918671)(270.76683969,181.1074289)(270.46215219,180.86133515)
\curveto(269.85863656,180.36914765)(269.12621469,180.1230539)(268.26488656,180.1230539)
\curveto(266.92308969,180.1230539)(266.00902719,180.57129608)(265.52269906,181.46778046)
\curveto(265.25902719,181.94824921)(265.12719125,182.58985077)(265.12719125,183.39258515)
\lineto(265.12719125,189.79102265)
\lineto(266.74437875,189.79102265)
\closepath
\moveto(268.91527719,190.01953827)
\lineto(268.91527719,190.01953827)
\closepath
}
}
{
\newrgbcolor{curcolor}{0 0 0}
\pscustom[linestyle=none,fillstyle=solid,fillcolor=curcolor]
{
\newpath
\moveto(276.39476938,182.88281952)
\curveto(276.39476938,182.42578827)(276.56176156,182.06543671)(276.89574594,181.80176483)
\curveto(277.22973031,181.53809296)(277.62523813,181.40625702)(278.08226938,181.40625702)
\curveto(278.63891,181.40625702)(279.1779725,181.53516327)(279.69945688,181.79297577)
\curveto(280.57836313,182.22071015)(281.01781625,182.92090546)(281.01781625,183.89356171)
\lineto(281.01781625,185.16797577)
\curveto(280.82445688,185.0449289)(280.57543344,184.94238983)(280.27074594,184.86035858)
\curveto(279.96605844,184.77832733)(279.66723031,184.71973358)(279.37426156,184.68457733)
\lineto(278.41625375,184.56153046)
\curveto(277.842035,184.48535858)(277.41137094,184.3652414)(277.12426156,184.2011789)
\curveto(276.63793344,183.92578827)(276.39476938,183.48633515)(276.39476938,182.88281952)
\closepath
\moveto(280.22680063,186.08203827)
\curveto(280.59008188,186.12891327)(280.83324594,186.28125702)(280.95629281,186.53906952)
\curveto(281.02660531,186.67969452)(281.06176156,186.88184296)(281.06176156,187.14551483)
\curveto(281.06176156,187.68457733)(280.86840219,188.07422577)(280.48168344,188.31446015)
\curveto(280.10082406,188.5605539)(279.5529725,188.68360077)(278.83812875,188.68360077)
\curveto(278.01195688,188.68360077)(277.42601938,188.46094452)(277.08031625,188.01563202)
\curveto(276.88695688,187.76953827)(276.76098031,187.40332733)(276.70238656,186.91699921)
\lineto(275.22582406,186.91699921)
\curveto(275.25512094,188.07715546)(275.63012094,188.88281952)(276.35082406,189.3339914)
\curveto(277.07738656,189.79102265)(277.91820688,190.01953827)(278.873285,190.01953827)
\curveto(279.98070688,190.01953827)(280.88012094,189.80860077)(281.57152719,189.38672577)
\curveto(282.25707406,188.96485077)(282.5998475,188.30860077)(282.5998475,187.41797577)
\lineto(282.5998475,181.99512421)
\curveto(282.5998475,181.83106171)(282.63207406,181.69922577)(282.69652719,181.5996164)
\curveto(282.76683969,181.50000702)(282.91039438,181.45020233)(283.12719125,181.45020233)
\curveto(283.19750375,181.45020233)(283.27660531,181.45313202)(283.36449594,181.4589914)
\curveto(283.45238656,181.47071015)(283.54613656,181.48535858)(283.64574594,181.50293671)
\lineto(283.64574594,180.3339914)
\curveto(283.39965219,180.2636789)(283.21215219,180.21973358)(283.08324594,180.20215546)
\curveto(282.95433969,180.18457733)(282.77855844,180.17578827)(282.55590219,180.17578827)
\curveto(282.01098031,180.17578827)(281.6154725,180.36914765)(281.36937875,180.7558664)
\curveto(281.2404725,180.96094452)(281.14965219,181.25098358)(281.09691781,181.62598358)
\curveto(280.77465219,181.20410858)(280.31176156,180.83789765)(279.70824594,180.52735077)
\curveto(279.10473031,180.2168039)(278.43969125,180.06153046)(277.71312875,180.06153046)
\curveto(276.84008188,180.06153046)(276.12523813,180.32520233)(275.5685975,180.85254608)
\curveto(275.01781625,181.38574921)(274.74242563,182.05078827)(274.74242563,182.84766327)
\curveto(274.74242563,183.72071015)(275.01488656,184.39746796)(275.55980844,184.87793671)
\curveto(276.10473031,185.35840546)(276.81957406,185.6543039)(277.70433969,185.76563202)
\lineto(280.22680063,186.08203827)
\closepath
\moveto(278.91723031,190.01953827)
\lineto(278.91723031,190.01953827)
\closepath
}
}
{
\newrgbcolor{curcolor}{0 0 0}
\pscustom[linestyle=none,fillstyle=solid,fillcolor=curcolor]
{
\newpath
\moveto(285.24535531,193.28906952)
\lineto(286.82738656,193.28906952)
\lineto(286.82738656,180.37793671)
\lineto(285.24535531,180.37793671)
\lineto(285.24535531,193.28906952)
\closepath
}
}
{
\newrgbcolor{curcolor}{0 0 0}
\pscustom[linestyle=none,fillstyle=solid,fillcolor=curcolor]
{
\newpath
\moveto(299.98461313,193.64063202)
\curveto(302.26391,193.64063202)(303.95141,192.90821015)(305.04711313,191.4433664)
\curveto(305.90258188,190.30078827)(306.33031625,188.83887421)(306.33031625,187.05762421)
\curveto(306.33031625,185.12988983)(305.84105844,183.52735077)(304.86254281,182.25000702)
\curveto(303.71410531,180.75000702)(302.07641,180.00000702)(299.94945688,180.00000702)
\curveto(297.96312875,180.00000702)(296.40160531,180.65625702)(295.26488656,181.96875702)
\curveto(294.25121469,183.23438202)(293.74437875,184.8339914)(293.74437875,186.76758515)
\curveto(293.74437875,188.5136789)(294.1779725,190.00781952)(295.04516,191.25000702)
\curveto(296.15844125,192.84375702)(297.80492563,193.64063202)(299.98461313,193.64063202)
\closepath
\moveto(300.16039438,181.56446015)
\curveto(301.70141,181.56446015)(302.81469125,182.1152414)(303.50023813,183.2168039)
\curveto(304.19164438,184.32422577)(304.5373475,185.59571015)(304.5373475,187.03125702)
\curveto(304.5373475,188.54883515)(304.13891,189.77051483)(303.342035,190.69629608)
\curveto(302.55101938,191.62207733)(301.467035,192.08496796)(300.09008188,192.08496796)
\curveto(298.75414438,192.08496796)(297.66430063,191.62500702)(296.82055063,190.70508515)
\curveto(295.97680063,189.79102265)(295.55492563,188.44043671)(295.55492563,186.65332733)
\curveto(295.55492563,185.22363983)(295.91527719,184.01660858)(296.63598031,183.03223358)
\curveto(297.36254281,182.05371796)(298.5373475,181.56446015)(300.16039438,181.56446015)
\closepath
\moveto(300.0373475,193.64063202)
\lineto(300.0373475,193.64063202)
\closepath
}
}
{
\newrgbcolor{curcolor}{0 0 0}
\pscustom[linestyle=none,fillstyle=solid,fillcolor=curcolor]
{
\newpath
\moveto(308.07055063,193.33301483)
\lineto(309.60863656,193.33301483)
\lineto(309.60863656,188.64844452)
\curveto(309.95433969,189.0996164)(310.36742563,189.44238983)(310.84789438,189.67676483)
\curveto(311.32836313,189.91699921)(311.8498475,190.0371164)(312.4123475,190.0371164)
\curveto(313.5842225,190.0371164)(314.53344125,189.63281952)(315.26000375,188.82422577)
\curveto(315.99242563,188.0214914)(316.35863656,186.83496796)(316.35863656,185.26465546)
\curveto(316.35863656,183.77637421)(315.998285,182.54004608)(315.27758188,181.55567108)
\curveto(314.55687875,180.57129608)(313.55785531,180.07910858)(312.28051156,180.07910858)
\curveto(311.56566781,180.07910858)(310.96215219,180.25196015)(310.46996469,180.59766327)
\curveto(310.17699594,180.8027414)(309.86351938,181.1308664)(309.529535,181.58203827)
\lineto(309.529535,180.37793671)
\lineto(308.07055063,180.37793671)
\lineto(308.07055063,193.33301483)
\closepath
\moveto(312.18383188,181.47656952)
\curveto(313.03930063,181.47656952)(313.6779725,181.81641327)(314.0998475,182.49610077)
\curveto(314.52758188,183.17578827)(314.74144906,184.07227265)(314.74144906,185.1855539)
\curveto(314.74144906,186.17578827)(314.52758188,186.99610077)(314.0998475,187.6464914)
\curveto(313.6779725,188.29688202)(313.05394906,188.62207733)(312.22777719,188.62207733)
\curveto(311.50707406,188.62207733)(310.87426156,188.35547577)(310.32933969,187.82227265)
\curveto(309.79027719,187.28906952)(309.52074594,186.41016327)(309.52074594,185.1855539)
\curveto(309.52074594,184.30078827)(309.63207406,183.58301483)(309.85473031,183.03223358)
\curveto(310.27074594,181.99512421)(311.04711313,181.47656952)(312.18383188,181.47656952)
\closepath
}
}
{
\newrgbcolor{curcolor}{0 0 0}
\pscustom[linestyle=none,fillstyle=solid,fillcolor=curcolor]
{
\newpath
\moveto(319.79516,191.46094452)
\lineto(318.21312875,191.46094452)
\lineto(318.21312875,193.28906952)
\lineto(319.79516,193.28906952)
\lineto(319.79516,191.46094452)
\closepath
\moveto(316.71898813,177.99610077)
\curveto(317.4279725,178.01953827)(317.85277719,178.0839914)(317.99340219,178.18946015)
\curveto(318.13988656,178.28906952)(318.21312875,178.60840546)(318.21312875,179.14746796)
\lineto(318.21312875,189.74707733)
\lineto(319.79516,189.74707733)
\lineto(319.79516,178.98047577)
\curveto(319.79516,178.2949289)(319.68383188,177.78223358)(319.46117563,177.44238983)
\curveto(319.092035,176.86817108)(318.39183969,176.58106171)(317.36058969,176.58106171)
\curveto(317.28441781,176.58106171)(317.20238656,176.5839914)(317.11449594,176.58985077)
\curveto(317.03246469,176.59571015)(316.90062875,176.6074289)(316.71898813,176.62500702)
\lineto(316.71898813,177.99610077)
\closepath
}
}
{
\newrgbcolor{curcolor}{0 0 0}
\pscustom[linestyle=none,fillstyle=solid,fillcolor=curcolor]
{
\newpath
\moveto(326.14086313,190.00196015)
\curveto(326.80883188,190.00196015)(327.45629281,189.84375702)(328.08324594,189.52735077)
\curveto(328.71019906,189.2168039)(329.18773813,188.81250702)(329.51586313,188.31446015)
\curveto(329.83226938,187.83985077)(330.04320688,187.28613983)(330.14867563,186.65332733)
\curveto(330.24242563,186.21973358)(330.28930063,185.52832733)(330.28930063,184.57910858)
\lineto(323.38988656,184.57910858)
\curveto(323.41918344,183.62403046)(323.64476938,182.85645233)(324.06664438,182.27637421)
\curveto(324.48851938,181.70215546)(325.14183969,181.41504608)(326.02660531,181.41504608)
\curveto(326.85277719,181.41504608)(327.51195688,181.68750702)(328.00414438,182.2324289)
\curveto(328.28539438,182.54883515)(328.48461313,182.91504608)(328.60180063,183.33106171)
\lineto(330.15746469,183.33106171)
\curveto(330.11644906,182.98535858)(329.97875375,182.59863983)(329.74437875,182.17090546)
\curveto(329.51586313,181.74903046)(329.25805063,181.40332733)(328.97094125,181.13379608)
\curveto(328.4904725,180.66504608)(327.89574594,180.34863983)(327.18676156,180.18457733)
\curveto(326.80590219,180.09082733)(326.37523813,180.04395233)(325.89476938,180.04395233)
\curveto(324.72289438,180.04395233)(323.72973031,180.46875702)(322.91527719,181.3183664)
\curveto(322.10082406,182.17383515)(321.6935975,183.36914765)(321.6935975,184.9043039)
\curveto(321.6935975,186.41602265)(322.10375375,187.64356171)(322.92406625,188.58692108)
\curveto(323.74437875,189.53028046)(324.81664438,190.00196015)(326.14086313,190.00196015)
\closepath
\moveto(328.66332406,185.83594452)
\curveto(328.59887094,186.5214914)(328.44945688,187.06934296)(328.21508188,187.47949921)
\curveto(327.78148813,188.24121796)(327.05785531,188.62207733)(326.04418344,188.62207733)
\curveto(325.31762094,188.62207733)(324.70824594,188.35840546)(324.21605844,187.83106171)
\curveto(323.72387094,187.30957733)(323.46312875,186.64453827)(323.43383188,185.83594452)
\lineto(328.66332406,185.83594452)
\closepath
\moveto(325.99144906,190.01953827)
\lineto(325.99144906,190.01953827)
\closepath
}
}
{
\newrgbcolor{curcolor}{0 0 0}
\pscustom[linestyle=none,fillstyle=solid,fillcolor=curcolor]
{
\newpath
\moveto(335.87035531,190.06348358)
\curveto(336.93090219,190.06348358)(337.79223031,189.80567108)(338.45433969,189.29004608)
\curveto(339.12230844,188.77442108)(339.52367563,187.88672577)(339.65844125,186.62696015)
\lineto(338.12035531,186.62696015)
\curveto(338.02660531,187.20703827)(337.81273813,187.68750702)(337.47875375,188.0683664)
\curveto(337.14476938,188.45508515)(336.60863656,188.64844452)(335.87035531,188.64844452)
\curveto(334.86254281,188.64844452)(334.14183969,188.15625702)(333.70824594,187.17188202)
\curveto(333.42699594,186.53321015)(333.28637094,185.74512421)(333.28637094,184.80762421)
\curveto(333.28637094,183.86426483)(333.48558969,183.07031952)(333.88402719,182.42578827)
\curveto(334.28246469,181.78125702)(334.90941781,181.4589914)(335.76488656,181.4589914)
\curveto(336.42113656,181.4589914)(336.93969125,181.65821015)(337.32055063,182.05664765)
\curveto(337.70726938,182.46094452)(337.97387094,183.01172577)(338.12035531,183.7089914)
\lineto(339.65844125,183.7089914)
\curveto(339.48266,182.46094452)(339.04320688,181.54688202)(338.34008188,180.9668039)
\curveto(337.63695688,180.39258515)(336.73754281,180.10547577)(335.64183969,180.10547577)
\curveto(334.41137094,180.10547577)(333.42992563,180.55371796)(332.69750375,181.45020233)
\curveto(331.96508188,182.35254608)(331.59887094,183.47754608)(331.59887094,184.82520233)
\curveto(331.59887094,186.47754608)(332.00023813,187.7636789)(332.8029725,188.68360077)
\curveto(333.60570688,189.60352265)(334.62816781,190.06348358)(335.87035531,190.06348358)
\closepath
\moveto(335.62426156,190.01953827)
\lineto(335.62426156,190.01953827)
\closepath
}
}
{
\newrgbcolor{curcolor}{0 0 0}
\pscustom[linestyle=none,fillstyle=solid,fillcolor=curcolor]
{
\newpath
\moveto(341.55687875,192.41895233)
\lineto(343.15648813,192.41895233)
\lineto(343.15648813,189.79102265)
\lineto(344.65941781,189.79102265)
\lineto(344.65941781,188.49903046)
\lineto(343.15648813,188.49903046)
\lineto(343.15648813,182.35547577)
\curveto(343.15648813,182.02735077)(343.26781625,181.80762421)(343.4904725,181.69629608)
\curveto(343.61351938,181.63184296)(343.8185975,181.5996164)(344.10570688,181.5996164)
\lineto(344.35180063,181.5996164)
\curveto(344.43969125,181.60547577)(344.54223031,181.61426483)(344.65941781,181.62598358)
\lineto(344.65941781,180.37793671)
\curveto(344.47777719,180.32520233)(344.2873475,180.2871164)(344.08812875,180.2636789)
\curveto(343.89476938,180.2402414)(343.68383188,180.22852265)(343.45531625,180.22852265)
\curveto(342.717035,180.22852265)(342.21605844,180.41602265)(341.95238656,180.79102265)
\curveto(341.68871469,181.17188202)(341.55687875,181.66406952)(341.55687875,182.26758515)
\lineto(341.55687875,188.49903046)
\lineto(340.28246469,188.49903046)
\lineto(340.28246469,189.79102265)
\lineto(341.55687875,189.79102265)
\lineto(341.55687875,192.41895233)
\closepath
}
}
{
\newrgbcolor{curcolor}{0 0 0}
\pscustom[linestyle=none,fillstyle=solid,fillcolor=curcolor]
{
\newpath
\moveto(353.48949594,83.28906952)
\lineto(359.29906625,83.28906952)
\curveto(360.44750375,83.28906952)(361.373285,82.96387421)(362.07641,82.31348358)
\curveto(362.779535,81.66895233)(363.1310975,80.76074921)(363.1310975,79.58887421)
\curveto(363.1310975,78.58106171)(362.81762094,77.70215546)(362.19066781,76.95215546)
\curveto(361.56371469,76.20801483)(360.5998475,75.83594452)(359.29906625,75.83594452)
\lineto(355.23851938,75.83594452)
\lineto(355.23851938,70.37793671)
\lineto(353.48949594,70.37793671)
\lineto(353.48949594,83.28906952)
\closepath
\moveto(361.36449594,79.58008515)
\curveto(361.36449594,80.5293039)(361.01293344,81.17383515)(360.30980844,81.5136789)
\curveto(359.92308969,81.69531952)(359.39281625,81.78613983)(358.71898813,81.78613983)
\lineto(355.23851938,81.78613983)
\lineto(355.23851938,77.31250702)
\lineto(358.71898813,77.31250702)
\curveto(359.50414438,77.31250702)(360.13988656,77.47949921)(360.62621469,77.81348358)
\curveto(361.11840219,78.14746796)(361.36449594,78.73633515)(361.36449594,79.58008515)
\closepath
}
}
{
\newrgbcolor{curcolor}{0 0 0}
\pscustom[linestyle=none,fillstyle=solid,fillcolor=curcolor]
{
\newpath
\moveto(365.13500375,79.74707733)
\lineto(366.74340219,79.74707733)
\lineto(366.74340219,70.37793671)
\lineto(365.13500375,70.37793671)
\lineto(365.13500375,79.74707733)
\closepath
\moveto(365.13500375,83.28906952)
\lineto(366.74340219,83.28906952)
\lineto(366.74340219,81.49610077)
\lineto(365.13500375,81.49610077)
\lineto(365.13500375,83.28906952)
\closepath
}
}
{
\newrgbcolor{curcolor}{0 0 0}
\pscustom[linestyle=none,fillstyle=solid,fillcolor=curcolor]
{
\newpath
\moveto(369.18676156,83.28906952)
\lineto(370.76879281,83.28906952)
\lineto(370.76879281,70.37793671)
\lineto(369.18676156,70.37793671)
\lineto(369.18676156,83.28906952)
\closepath
}
}
{
\newrgbcolor{curcolor}{0 0 0}
\pscustom[linestyle=none,fillstyle=solid,fillcolor=curcolor]
{
\newpath
\moveto(377.07055063,80.00196015)
\curveto(377.73851938,80.00196015)(378.38598031,79.84375702)(379.01293344,79.52735077)
\curveto(379.63988656,79.2168039)(380.11742563,78.81250702)(380.44555063,78.31446015)
\curveto(380.76195688,77.83985077)(380.97289438,77.28613983)(381.07836313,76.65332733)
\curveto(381.17211313,76.21973358)(381.21898813,75.52832733)(381.21898813,74.57910858)
\lineto(374.31957406,74.57910858)
\curveto(374.34887094,73.62403046)(374.57445688,72.85645233)(374.99633188,72.27637421)
\curveto(375.41820688,71.70215546)(376.07152719,71.41504608)(376.95629281,71.41504608)
\curveto(377.78246469,71.41504608)(378.44164438,71.68750702)(378.93383188,72.2324289)
\curveto(379.21508188,72.54883515)(379.41430063,72.91504608)(379.53148813,73.33106171)
\lineto(381.08715219,73.33106171)
\curveto(381.04613656,72.98535858)(380.90844125,72.59863983)(380.67406625,72.17090546)
\curveto(380.44555063,71.74903046)(380.18773813,71.40332733)(379.90062875,71.13379608)
\curveto(379.42016,70.66504608)(378.82543344,70.34863983)(378.11644906,70.18457733)
\curveto(377.73558969,70.09082733)(377.30492563,70.04395233)(376.82445688,70.04395233)
\curveto(375.65258188,70.04395233)(374.65941781,70.46875702)(373.84496469,71.3183664)
\curveto(373.03051156,72.17383515)(372.623285,73.36914765)(372.623285,74.9043039)
\curveto(372.623285,76.41602265)(373.03344125,77.64356171)(373.85375375,78.58692108)
\curveto(374.67406625,79.53028046)(375.74633188,80.00196015)(377.07055063,80.00196015)
\closepath
\moveto(379.59301156,75.83594452)
\curveto(379.52855844,76.5214914)(379.37914438,77.06934296)(379.14476938,77.47949921)
\curveto(378.71117563,78.24121796)(377.98754281,78.62207733)(376.97387094,78.62207733)
\curveto(376.24730844,78.62207733)(375.63793344,78.35840546)(375.14574594,77.83106171)
\curveto(374.65355844,77.30957733)(374.39281625,76.64453827)(374.36351938,75.83594452)
\lineto(379.59301156,75.83594452)
\closepath
\moveto(376.92113656,80.01953827)
\lineto(376.92113656,80.01953827)
\closepath
}
}
{
\newrgbcolor{curcolor}{0 0 0}
\pscustom[linestyle=none,fillstyle=solid,fillcolor=curcolor]
{
\newpath
\moveto(391.89769906,71.39746796)
\curveto(392.94652719,71.39746796)(393.66430063,71.79297577)(394.05101938,72.5839914)
\curveto(394.4435975,73.3808664)(394.63988656,74.26563202)(394.63988656,75.23828827)
\curveto(394.63988656,76.11719452)(394.49926156,76.83203827)(394.21801156,77.38281952)
\curveto(393.77269906,78.25000702)(393.00512094,78.68360077)(391.91527719,78.68360077)
\curveto(390.94848031,78.68360077)(390.24535531,78.31446015)(389.80590219,77.5761789)
\curveto(389.36644906,76.83789765)(389.1467225,75.94727265)(389.1467225,74.9043039)
\curveto(389.1467225,73.90235077)(389.36644906,73.06738983)(389.80590219,72.39942108)
\curveto(390.24535531,71.73145233)(390.94262094,71.39746796)(391.89769906,71.39746796)
\closepath
\moveto(391.9592225,80.06348358)
\curveto(393.17211313,80.06348358)(394.19750375,79.65918671)(395.03539438,78.85059296)
\curveto(395.873285,78.04199921)(396.29223031,76.85254608)(396.29223031,75.28223358)
\curveto(396.29223031,73.76465546)(395.92308969,72.51074921)(395.18480844,71.52051483)
\curveto(394.44652719,70.53028046)(393.30101938,70.03516327)(391.748285,70.03516327)
\curveto(390.45336313,70.03516327)(389.42504281,70.47168671)(388.66332406,71.34473358)
\curveto(387.90160531,72.22363983)(387.52074594,73.40137421)(387.52074594,74.87793671)
\curveto(387.52074594,76.45996796)(387.92211313,77.71973358)(388.7248475,78.65723358)
\curveto(389.52758188,79.59473358)(390.60570688,80.06348358)(391.9592225,80.06348358)
\closepath
\moveto(391.90648813,80.01953827)
\lineto(391.90648813,80.01953827)
\closepath
}
}
{
\newrgbcolor{curcolor}{0 0 0}
\pscustom[linestyle=none,fillstyle=solid,fillcolor=curcolor]
{
\newpath
\moveto(398.57738656,81.22363983)
\curveto(398.60082406,81.87988983)(398.71508188,82.36035858)(398.92016,82.66504608)
\curveto(399.28930063,83.20410858)(400.00121469,83.47363983)(401.05590219,83.47363983)
\curveto(401.15551156,83.47363983)(401.25805063,83.47071015)(401.36351938,83.46485077)
\curveto(401.46898813,83.4589914)(401.58910531,83.45020233)(401.72387094,83.43848358)
\lineto(401.72387094,81.99707733)
\curveto(401.55980844,82.00879608)(401.43969125,82.01465546)(401.36351938,82.01465546)
\curveto(401.29320688,82.02051483)(401.22582406,82.02344452)(401.16137094,82.02344452)
\curveto(400.68090219,82.02344452)(400.39379281,81.89746796)(400.30004281,81.64551483)
\curveto(400.20629281,81.39942108)(400.15941781,80.76660858)(400.15941781,79.74707733)
\lineto(401.72387094,79.74707733)
\lineto(401.72387094,78.49903046)
\lineto(400.14183969,78.49903046)
\lineto(400.14183969,70.37793671)
\lineto(398.57738656,70.37793671)
\lineto(398.57738656,78.49903046)
\lineto(397.26781625,78.49903046)
\lineto(397.26781625,79.74707733)
\lineto(398.57738656,79.74707733)
\lineto(398.57738656,81.22363983)
\closepath
}
}
{
\newrgbcolor{curcolor}{0 0 0}
\pscustom[linestyle=none,fillstyle=solid,fillcolor=curcolor]
{
\newpath
\moveto(408.57933969,83.28906952)
\lineto(417.53539438,83.28906952)
\lineto(417.53539438,81.70703827)
\lineto(410.32836313,81.70703827)
\lineto(410.32836313,77.7871164)
\lineto(416.66527719,77.7871164)
\lineto(416.66527719,76.24903046)
\lineto(410.32836313,76.24903046)
\lineto(410.32836313,70.37793671)
\lineto(408.57933969,70.37793671)
\lineto(408.57933969,83.28906952)
\closepath
}
}
{
\newrgbcolor{curcolor}{0 0 0}
\pscustom[linestyle=none,fillstyle=solid,fillcolor=curcolor]
{
\newpath
\moveto(419.20531625,79.74707733)
\lineto(420.81371469,79.74707733)
\lineto(420.81371469,70.37793671)
\lineto(419.20531625,70.37793671)
\lineto(419.20531625,79.74707733)
\closepath
\moveto(419.20531625,83.28906952)
\lineto(420.81371469,83.28906952)
\lineto(420.81371469,81.49610077)
\lineto(419.20531625,81.49610077)
\lineto(419.20531625,83.28906952)
\closepath
}
}
{
\newrgbcolor{curcolor}{0 0 0}
\pscustom[linestyle=none,fillstyle=solid,fillcolor=curcolor]
{
\newpath
\moveto(423.25707406,83.28906952)
\lineto(424.83910531,83.28906952)
\lineto(424.83910531,70.37793671)
\lineto(423.25707406,70.37793671)
\lineto(423.25707406,83.28906952)
\closepath
}
}
{
\newrgbcolor{curcolor}{0 0 0}
\pscustom[linestyle=none,fillstyle=solid,fillcolor=curcolor]
{
\newpath
\moveto(427.22094125,79.74707733)
\lineto(428.82933969,79.74707733)
\lineto(428.82933969,70.37793671)
\lineto(427.22094125,70.37793671)
\lineto(427.22094125,79.74707733)
\closepath
\moveto(427.22094125,83.28906952)
\lineto(428.82933969,83.28906952)
\lineto(428.82933969,81.49610077)
\lineto(427.22094125,81.49610077)
\lineto(427.22094125,83.28906952)
\closepath
}
}
{
\newrgbcolor{curcolor}{0 0 0}
\pscustom[linestyle=none,fillstyle=solid,fillcolor=curcolor]
{
\newpath
\moveto(431.22875375,79.79102265)
\lineto(432.79320688,79.79102265)
\lineto(432.79320688,78.45508515)
\curveto(433.16820688,78.91797577)(433.50805063,79.25488983)(433.81273813,79.46582733)
\curveto(434.3342225,79.82324921)(434.92601938,80.00196015)(435.58812875,80.00196015)
\curveto(436.33812875,80.00196015)(436.94164438,79.81738983)(437.39867563,79.44824921)
\curveto(437.65648813,79.23731171)(437.89086313,78.92676483)(438.10180063,78.51660858)
\curveto(438.45336313,79.02051483)(438.86644906,79.39258515)(439.34105844,79.63281952)
\curveto(439.81566781,79.87891327)(440.34887094,80.00196015)(440.94066781,80.00196015)
\curveto(442.20629281,80.00196015)(443.06762094,79.5449289)(443.52465219,78.6308664)
\curveto(443.77074594,78.1386789)(443.89379281,77.47656952)(443.89379281,76.64453827)
\lineto(443.89379281,70.37793671)
\lineto(442.25023813,70.37793671)
\lineto(442.25023813,76.91699921)
\curveto(442.25023813,77.54395233)(442.092035,77.9746164)(441.77562875,78.2089914)
\curveto(441.46508188,78.4433664)(441.0842225,78.5605539)(440.63305063,78.5605539)
\curveto(440.01195688,78.5605539)(439.47582406,78.35254608)(439.02465219,77.93653046)
\curveto(438.57933969,77.52051483)(438.35668344,76.8261789)(438.35668344,75.85352265)
\lineto(438.35668344,70.37793671)
\lineto(436.748285,70.37793671)
\lineto(436.748285,76.5214914)
\curveto(436.748285,77.16016327)(436.67211313,77.62598358)(436.51976938,77.91895233)
\curveto(436.279535,78.35840546)(435.83129281,78.57813202)(435.17504281,78.57813202)
\curveto(434.57738656,78.57813202)(434.03246469,78.34668671)(433.54027719,77.88379608)
\curveto(433.05394906,77.42090546)(432.810785,76.58301483)(432.810785,75.37012421)
\lineto(432.810785,70.37793671)
\lineto(431.22875375,70.37793671)
\lineto(431.22875375,79.79102265)
\closepath
}
}
{
\newrgbcolor{curcolor}{0 0 0}
\pscustom[linestyle=none,fillstyle=solid,fillcolor=curcolor]
{
\newpath
\moveto(450.12523813,80.00196015)
\curveto(450.79320688,80.00196015)(451.44066781,79.84375702)(452.06762094,79.52735077)
\curveto(452.69457406,79.2168039)(453.17211313,78.81250702)(453.50023813,78.31446015)
\curveto(453.81664438,77.83985077)(454.02758188,77.28613983)(454.13305063,76.65332733)
\curveto(454.22680063,76.21973358)(454.27367563,75.52832733)(454.27367563,74.57910858)
\lineto(447.37426156,74.57910858)
\curveto(447.40355844,73.62403046)(447.62914438,72.85645233)(448.05101938,72.27637421)
\curveto(448.47289438,71.70215546)(449.12621469,71.41504608)(450.01098031,71.41504608)
\curveto(450.83715219,71.41504608)(451.49633188,71.68750702)(451.98851938,72.2324289)
\curveto(452.26976938,72.54883515)(452.46898813,72.91504608)(452.58617563,73.33106171)
\lineto(454.14183969,73.33106171)
\curveto(454.10082406,72.98535858)(453.96312875,72.59863983)(453.72875375,72.17090546)
\curveto(453.50023813,71.74903046)(453.24242563,71.40332733)(452.95531625,71.13379608)
\curveto(452.4748475,70.66504608)(451.88012094,70.34863983)(451.17113656,70.18457733)
\curveto(450.79027719,70.09082733)(450.35961313,70.04395233)(449.87914438,70.04395233)
\curveto(448.70726938,70.04395233)(447.71410531,70.46875702)(446.89965219,71.3183664)
\curveto(446.08519906,72.17383515)(445.6779725,73.36914765)(445.6779725,74.9043039)
\curveto(445.6779725,76.41602265)(446.08812875,77.64356171)(446.90844125,78.58692108)
\curveto(447.72875375,79.53028046)(448.80101938,80.00196015)(450.12523813,80.00196015)
\closepath
\moveto(452.64769906,75.83594452)
\curveto(452.58324594,76.5214914)(452.43383188,77.06934296)(452.19945688,77.47949921)
\curveto(451.76586313,78.24121796)(451.04223031,78.62207733)(450.02855844,78.62207733)
\curveto(449.30199594,78.62207733)(448.69262094,78.35840546)(448.20043344,77.83106171)
\curveto(447.70824594,77.30957733)(447.44750375,76.64453827)(447.41820688,75.83594452)
\lineto(452.64769906,75.83594452)
\closepath
\moveto(449.97582406,80.01953827)
\lineto(449.97582406,80.01953827)
\closepath
}
}
{
\newrgbcolor{curcolor}{0 0 0}
\pscustom[linestyle=none,fillstyle=solid,fillcolor=curcolor]
{
\newpath
\moveto(456.2248475,79.79102265)
\lineto(457.72777719,79.79102265)
\lineto(457.72777719,78.45508515)
\curveto(458.17308969,79.0058664)(458.64476938,79.40137421)(459.14281625,79.64160858)
\curveto(459.64086313,79.88184296)(460.19457406,80.00196015)(460.80394906,80.00196015)
\curveto(462.13988656,80.00196015)(463.04223031,79.53613983)(463.51098031,78.60449921)
\curveto(463.76879281,78.09473358)(463.89769906,77.3652414)(463.89769906,76.41602265)
\lineto(463.89769906,70.37793671)
\lineto(462.28930063,70.37793671)
\lineto(462.28930063,76.3105539)
\curveto(462.28930063,76.88477265)(462.20433969,77.34766327)(462.03441781,77.69922577)
\curveto(461.75316781,78.28516327)(461.24340219,78.57813202)(460.50512094,78.57813202)
\curveto(460.13012094,78.57813202)(459.82250375,78.54004608)(459.58226938,78.46387421)
\curveto(459.14867563,78.33496796)(458.76781625,78.07715546)(458.43969125,77.69043671)
\curveto(458.17601938,77.37988983)(458.00316781,77.05762421)(457.92113656,76.72363983)
\curveto(457.84496469,76.39551483)(457.80687875,75.92383515)(457.80687875,75.30860077)
\lineto(457.80687875,70.37793671)
\lineto(456.2248475,70.37793671)
\lineto(456.2248475,79.79102265)
\closepath
\moveto(459.94262094,80.01953827)
\lineto(459.94262094,80.01953827)
\closepath
}
}
{
\newrgbcolor{curcolor}{0 0 0}
\pscustom[linestyle=none,fillstyle=solid,fillcolor=curcolor]
{
\newpath
\moveto(466.560785,82.41895233)
\lineto(468.16039438,82.41895233)
\lineto(468.16039438,79.79102265)
\lineto(469.66332406,79.79102265)
\lineto(469.66332406,78.49903046)
\lineto(468.16039438,78.49903046)
\lineto(468.16039438,72.35547577)
\curveto(468.16039438,72.02735077)(468.2717225,71.80762421)(468.49437875,71.69629608)
\curveto(468.61742563,71.63184296)(468.82250375,71.5996164)(469.10961313,71.5996164)
\lineto(469.35570688,71.5996164)
\curveto(469.4435975,71.60547577)(469.54613656,71.61426483)(469.66332406,71.62598358)
\lineto(469.66332406,70.37793671)
\curveto(469.48168344,70.32520233)(469.29125375,70.2871164)(469.092035,70.2636789)
\curveto(468.89867563,70.2402414)(468.68773813,70.22852265)(468.4592225,70.22852265)
\curveto(467.72094125,70.22852265)(467.21996469,70.41602265)(466.95629281,70.79102265)
\curveto(466.69262094,71.17188202)(466.560785,71.66406952)(466.560785,72.26758515)
\lineto(466.560785,78.49903046)
\lineto(465.28637094,78.49903046)
\lineto(465.28637094,79.79102265)
\lineto(466.560785,79.79102265)
\lineto(466.560785,82.41895233)
\closepath
}
}
{
\newrgbcolor{curcolor}{0 0 0}
\pscustom[linestyle=none,fillstyle=solid,fillcolor=curcolor]
{
\newpath
\moveto(139.48949594,13.28906952)
\lineto(145.29906625,13.28906952)
\curveto(146.44750375,13.28906952)(147.373285,12.96387421)(148.07641,12.31348358)
\curveto(148.779535,11.66895233)(149.1310975,10.76074921)(149.1310975,9.58887421)
\curveto(149.1310975,8.58106171)(148.81762094,7.70215546)(148.19066781,6.95215546)
\curveto(147.56371469,6.20801483)(146.5998475,5.83594452)(145.29906625,5.83594452)
\lineto(141.23851938,5.83594452)
\lineto(141.23851938,0.37793671)
\lineto(139.48949594,0.37793671)
\lineto(139.48949594,13.28906952)
\closepath
\moveto(147.36449594,9.58008515)
\curveto(147.36449594,10.5293039)(147.01293344,11.17383515)(146.30980844,11.5136789)
\curveto(145.92308969,11.69531952)(145.39281625,11.78613983)(144.71898813,11.78613983)
\lineto(141.23851938,11.78613983)
\lineto(141.23851938,7.31250702)
\lineto(144.71898813,7.31250702)
\curveto(145.50414438,7.31250702)(146.13988656,7.47949921)(146.62621469,7.81348358)
\curveto(147.11840219,8.14746796)(147.36449594,8.73633515)(147.36449594,9.58008515)
\closepath
}
}
{
\newrgbcolor{curcolor}{0 0 0}
\pscustom[linestyle=none,fillstyle=solid,fillcolor=curcolor]
{
\newpath
\moveto(151.17894906,9.79102265)
\lineto(152.68187875,9.79102265)
\lineto(152.68187875,8.16504608)
\curveto(152.80492563,8.48145233)(153.10668344,8.8652414)(153.58715219,9.31641327)
\curveto(154.06762094,9.77344452)(154.62133188,10.00196015)(155.248285,10.00196015)
\curveto(155.27758188,10.00196015)(155.32738656,9.99903046)(155.39769906,9.99317108)
\curveto(155.46801156,9.98731171)(155.58812875,9.97559296)(155.75805063,9.95801483)
\lineto(155.75805063,8.28809296)
\curveto(155.66430063,8.30567108)(155.57641,8.31738983)(155.49437875,8.32324921)
\curveto(155.41820688,8.32910858)(155.33324594,8.33203827)(155.23949594,8.33203827)
\curveto(154.44262094,8.33203827)(153.83031625,8.07422577)(153.40258188,7.55860077)
\curveto(152.9748475,7.04883515)(152.76098031,6.45996796)(152.76098031,5.79199921)
\lineto(152.76098031,0.37793671)
\lineto(151.17894906,0.37793671)
\lineto(151.17894906,9.79102265)
\closepath
}
}
{
\newrgbcolor{curcolor}{0 0 0}
\pscustom[linestyle=none,fillstyle=solid,fillcolor=curcolor]
{
\newpath
\moveto(157.1467225,9.74707733)
\lineto(158.75512094,9.74707733)
\lineto(158.75512094,0.37793671)
\lineto(157.1467225,0.37793671)
\lineto(157.1467225,9.74707733)
\closepath
\moveto(157.1467225,13.28906952)
\lineto(158.75512094,13.28906952)
\lineto(158.75512094,11.49610077)
\lineto(157.1467225,11.49610077)
\lineto(157.1467225,13.28906952)
\closepath
}
}
{
\newrgbcolor{curcolor}{0 0 0}
\pscustom[linestyle=none,fillstyle=solid,fillcolor=curcolor]
{
\newpath
\moveto(161.154535,9.79102265)
\lineto(162.65746469,9.79102265)
\lineto(162.65746469,8.45508515)
\curveto(163.10277719,9.0058664)(163.57445688,9.40137421)(164.07250375,9.64160858)
\curveto(164.57055063,9.88184296)(165.12426156,10.00196015)(165.73363656,10.00196015)
\curveto(167.06957406,10.00196015)(167.97191781,9.53613983)(168.44066781,8.60449921)
\curveto(168.69848031,8.09473358)(168.82738656,7.3652414)(168.82738656,6.41602265)
\lineto(168.82738656,0.37793671)
\lineto(167.21898813,0.37793671)
\lineto(167.21898813,6.3105539)
\curveto(167.21898813,6.88477265)(167.13402719,7.34766327)(166.96410531,7.69922577)
\curveto(166.68285531,8.28516327)(166.17308969,8.57813202)(165.43480844,8.57813202)
\curveto(165.05980844,8.57813202)(164.75219125,8.54004608)(164.51195688,8.46387421)
\curveto(164.07836313,8.33496796)(163.69750375,8.07715546)(163.36937875,7.69043671)
\curveto(163.10570688,7.37988983)(162.93285531,7.05762421)(162.85082406,6.72363983)
\curveto(162.77465219,6.39551483)(162.73656625,5.92383515)(162.73656625,5.30860077)
\lineto(162.73656625,0.37793671)
\lineto(161.154535,0.37793671)
\lineto(161.154535,9.79102265)
\closepath
\moveto(164.87230844,10.01953827)
\lineto(164.87230844,10.01953827)
\closepath
}
}
{
\newrgbcolor{curcolor}{0 0 0}
\pscustom[linestyle=none,fillstyle=solid,fillcolor=curcolor]
{
\newpath
\moveto(171.4904725,12.41895233)
\lineto(173.09008188,12.41895233)
\lineto(173.09008188,9.79102265)
\lineto(174.59301156,9.79102265)
\lineto(174.59301156,8.49903046)
\lineto(173.09008188,8.49903046)
\lineto(173.09008188,2.35547577)
\curveto(173.09008188,2.02735077)(173.20141,1.80762421)(173.42406625,1.69629608)
\curveto(173.54711313,1.63184296)(173.75219125,1.5996164)(174.03930063,1.5996164)
\lineto(174.28539438,1.5996164)
\curveto(174.373285,1.60547577)(174.47582406,1.61426483)(174.59301156,1.62598358)
\lineto(174.59301156,0.37793671)
\curveto(174.41137094,0.32520233)(174.22094125,0.2871164)(174.0217225,0.2636789)
\curveto(173.82836313,0.2402414)(173.61742563,0.22852265)(173.38891,0.22852265)
\curveto(172.65062875,0.22852265)(172.14965219,0.41602265)(171.88598031,0.79102265)
\curveto(171.62230844,1.17188202)(171.4904725,1.66406952)(171.4904725,2.26758515)
\lineto(171.4904725,8.49903046)
\lineto(170.21605844,8.49903046)
\lineto(170.21605844,9.79102265)
\lineto(171.4904725,9.79102265)
\lineto(171.4904725,12.41895233)
\closepath
}
}
{
\newrgbcolor{curcolor}{0 0 0}
\pscustom[linestyle=none,fillstyle=solid,fillcolor=curcolor]
{
\newpath
\moveto(182.54711313,4.54395233)
\curveto(182.58812875,3.81153046)(182.76098031,3.2168039)(183.06566781,2.75977265)
\curveto(183.64574594,1.9043039)(184.66820688,1.47656952)(186.13305063,1.47656952)
\curveto(186.78930063,1.47656952)(187.38695688,1.57031952)(187.92601938,1.75781952)
\curveto(188.96898813,2.12110077)(189.4904725,2.7714914)(189.4904725,3.7089914)
\curveto(189.4904725,4.4121164)(189.27074594,4.91309296)(188.83129281,5.21192108)
\curveto(188.38598031,5.50488983)(187.68871469,5.75977265)(186.73949594,5.97656952)
\lineto(184.9904725,6.37207733)
\curveto(183.84789438,6.62988983)(183.03930063,6.91406952)(182.56469125,7.2246164)
\curveto(181.74437875,7.7636789)(181.3342225,8.56934296)(181.3342225,9.64160858)
\curveto(181.3342225,10.80176483)(181.73558969,11.75391327)(182.53832406,12.4980539)
\curveto(183.34105844,13.24219452)(184.47777719,13.61426483)(185.94848031,13.61426483)
\curveto(187.30199594,13.61426483)(188.45043344,13.28613983)(189.39379281,12.62988983)
\curveto(190.34301156,11.97949921)(190.81762094,10.93653046)(190.81762094,9.50098358)
\lineto(189.17406625,9.50098358)
\curveto(189.08617563,10.19238983)(188.89867563,10.72266327)(188.61156625,11.0918039)
\curveto(188.07836313,11.76563202)(187.17308969,12.10254608)(185.89574594,12.10254608)
\curveto(184.86449594,12.10254608)(184.123285,11.88574921)(183.67211313,11.45215546)
\curveto(183.22094125,11.01856171)(182.99535531,10.51465546)(182.99535531,9.94043671)
\curveto(182.99535531,9.30762421)(183.25902719,8.84473358)(183.78637094,8.55176483)
\curveto(184.13207406,8.36426483)(184.91430063,8.12988983)(186.13305063,7.84863983)
\lineto(187.9435975,7.4355539)
\curveto(188.81664438,7.23633515)(189.4904725,6.96387421)(189.96508188,6.61817108)
\curveto(190.78539438,6.01465546)(191.19555063,5.1386789)(191.19555063,3.9902414)
\curveto(191.19555063,2.5605539)(190.67406625,1.53809296)(189.6310975,0.92285858)
\curveto(188.59398813,0.30762421)(187.38695688,0.00000702)(186.01000375,0.00000702)
\curveto(184.404535,0.00000702)(183.14769906,0.41016327)(182.23949594,1.23047577)
\curveto(181.33129281,2.0449289)(180.88598031,3.14942108)(180.90355844,4.54395233)
\lineto(182.54711313,4.54395233)
\closepath
\moveto(186.08031625,13.64063202)
\lineto(186.08031625,13.64063202)
\closepath
}
}
{
\newrgbcolor{curcolor}{0 0 0}
\pscustom[linestyle=none,fillstyle=solid,fillcolor=curcolor]
{
\newpath
\moveto(194.76391,9.79102265)
\lineto(194.76391,3.54199921)
\curveto(194.76391,3.06153046)(194.84008188,2.66895233)(194.99242563,2.36426483)
\curveto(195.27367563,1.80176483)(195.79808969,1.52051483)(196.56566781,1.52051483)
\curveto(197.66723031,1.52051483)(198.41723031,2.01270233)(198.81566781,2.99707733)
\curveto(199.03246469,3.52442108)(199.14086313,4.2480539)(199.14086313,5.16797577)
\lineto(199.14086313,9.79102265)
\lineto(200.72289438,9.79102265)
\lineto(200.72289438,0.37793671)
\lineto(199.22875375,0.37793671)
\lineto(199.24633188,1.76660858)
\curveto(199.04125375,1.40918671)(198.78637094,1.1074289)(198.48168344,0.86133515)
\curveto(197.87816781,0.36914765)(197.14574594,0.1230539)(196.28441781,0.1230539)
\curveto(194.94262094,0.1230539)(194.02855844,0.57129608)(193.54223031,1.46778046)
\curveto(193.27855844,1.94824921)(193.1467225,2.58985077)(193.1467225,3.39258515)
\lineto(193.1467225,9.79102265)
\lineto(194.76391,9.79102265)
\closepath
\moveto(196.93480844,10.01953827)
\lineto(196.93480844,10.01953827)
\closepath
}
}
{
\newrgbcolor{curcolor}{0 0 0}
\pscustom[linestyle=none,fillstyle=solid,fillcolor=curcolor]
{
\newpath
\moveto(203.24535531,9.79102265)
\lineto(204.748285,9.79102265)
\lineto(204.748285,8.16504608)
\curveto(204.87133188,8.48145233)(205.17308969,8.8652414)(205.65355844,9.31641327)
\curveto(206.13402719,9.77344452)(206.68773813,10.00196015)(207.31469125,10.00196015)
\curveto(207.34398813,10.00196015)(207.39379281,9.99903046)(207.46410531,9.99317108)
\curveto(207.53441781,9.98731171)(207.654535,9.97559296)(207.82445688,9.95801483)
\lineto(207.82445688,8.28809296)
\curveto(207.73070688,8.30567108)(207.64281625,8.31738983)(207.560785,8.32324921)
\curveto(207.48461313,8.32910858)(207.39965219,8.33203827)(207.30590219,8.33203827)
\curveto(206.50902719,8.33203827)(205.8967225,8.07422577)(205.46898813,7.55860077)
\curveto(205.04125375,7.04883515)(204.82738656,6.45996796)(204.82738656,5.79199921)
\lineto(204.82738656,0.37793671)
\lineto(203.24535531,0.37793671)
\lineto(203.24535531,9.79102265)
\closepath
}
}
{
\newrgbcolor{curcolor}{0 0 0}
\pscustom[linestyle=none,fillstyle=solid,fillcolor=curcolor]
{
\newpath
\moveto(209.60863656,11.22363983)
\curveto(209.63207406,11.87988983)(209.74633188,12.36035858)(209.95141,12.66504608)
\curveto(210.32055063,13.20410858)(211.03246469,13.47363983)(212.08715219,13.47363983)
\curveto(212.18676156,13.47363983)(212.28930063,13.47071015)(212.39476938,13.46485077)
\curveto(212.50023813,13.4589914)(212.62035531,13.45020233)(212.75512094,13.43848358)
\lineto(212.75512094,11.99707733)
\curveto(212.59105844,12.00879608)(212.47094125,12.01465546)(212.39476938,12.01465546)
\curveto(212.32445688,12.02051483)(212.25707406,12.02344452)(212.19262094,12.02344452)
\curveto(211.71215219,12.02344452)(211.42504281,11.89746796)(211.33129281,11.64551483)
\curveto(211.23754281,11.39942108)(211.19066781,10.76660858)(211.19066781,9.74707733)
\lineto(212.75512094,9.74707733)
\lineto(212.75512094,8.49903046)
\lineto(211.17308969,8.49903046)
\lineto(211.17308969,0.37793671)
\lineto(209.60863656,0.37793671)
\lineto(209.60863656,8.49903046)
\lineto(208.29906625,8.49903046)
\lineto(208.29906625,9.74707733)
\lineto(209.60863656,9.74707733)
\lineto(209.60863656,11.22363983)
\closepath
}
}
{
\newrgbcolor{curcolor}{0 0 0}
\pscustom[linestyle=none,fillstyle=solid,fillcolor=curcolor]
{
\newpath
\moveto(215.45336313,2.88281952)
\curveto(215.45336313,2.42578827)(215.62035531,2.06543671)(215.95433969,1.80176483)
\curveto(216.28832406,1.53809296)(216.68383188,1.40625702)(217.14086313,1.40625702)
\curveto(217.69750375,1.40625702)(218.23656625,1.53516327)(218.75805063,1.79297577)
\curveto(219.63695688,2.22071015)(220.07641,2.92090546)(220.07641,3.89356171)
\lineto(220.07641,5.16797577)
\curveto(219.88305063,5.0449289)(219.63402719,4.94238983)(219.32933969,4.86035858)
\curveto(219.02465219,4.77832733)(218.72582406,4.71973358)(218.43285531,4.68457733)
\lineto(217.4748475,4.56153046)
\curveto(216.90062875,4.48535858)(216.46996469,4.3652414)(216.18285531,4.2011789)
\curveto(215.69652719,3.92578827)(215.45336313,3.48633515)(215.45336313,2.88281952)
\closepath
\moveto(219.28539438,6.08203827)
\curveto(219.64867563,6.12891327)(219.89183969,6.28125702)(220.01488656,6.53906952)
\curveto(220.08519906,6.67969452)(220.12035531,6.88184296)(220.12035531,7.14551483)
\curveto(220.12035531,7.68457733)(219.92699594,8.07422577)(219.54027719,8.31446015)
\curveto(219.15941781,8.5605539)(218.61156625,8.68360077)(217.8967225,8.68360077)
\curveto(217.07055063,8.68360077)(216.48461313,8.46094452)(216.13891,8.01563202)
\curveto(215.94555063,7.76953827)(215.81957406,7.40332733)(215.76098031,6.91699921)
\lineto(214.28441781,6.91699921)
\curveto(214.31371469,8.07715546)(214.68871469,8.88281952)(215.40941781,9.3339914)
\curveto(216.13598031,9.79102265)(216.97680063,10.01953827)(217.93187875,10.01953827)
\curveto(219.03930063,10.01953827)(219.93871469,9.80860077)(220.63012094,9.38672577)
\curveto(221.31566781,8.96485077)(221.65844125,8.30860077)(221.65844125,7.41797577)
\lineto(221.65844125,1.99512421)
\curveto(221.65844125,1.83106171)(221.69066781,1.69922577)(221.75512094,1.5996164)
\curveto(221.82543344,1.50000702)(221.96898813,1.45020233)(222.185785,1.45020233)
\curveto(222.2560975,1.45020233)(222.33519906,1.45313202)(222.42308969,1.4589914)
\curveto(222.51098031,1.47071015)(222.60473031,1.48535858)(222.70433969,1.50293671)
\lineto(222.70433969,0.3339914)
\curveto(222.45824594,0.2636789)(222.27074594,0.21973358)(222.14183969,0.20215546)
\curveto(222.01293344,0.18457733)(221.83715219,0.17578827)(221.61449594,0.17578827)
\curveto(221.06957406,0.17578827)(220.67406625,0.36914765)(220.4279725,0.7558664)
\curveto(220.29906625,0.96094452)(220.20824594,1.25098358)(220.15551156,1.62598358)
\curveto(219.83324594,1.20410858)(219.37035531,0.83789765)(218.76683969,0.52735077)
\curveto(218.16332406,0.2168039)(217.498285,0.06153046)(216.7717225,0.06153046)
\curveto(215.89867563,0.06153046)(215.18383188,0.32520233)(214.62719125,0.85254608)
\curveto(214.07641,1.38574921)(213.80101938,2.05078827)(213.80101938,2.84766327)
\curveto(213.80101938,3.72071015)(214.07348031,4.39746796)(214.61840219,4.87793671)
\curveto(215.16332406,5.35840546)(215.87816781,5.6543039)(216.76293344,5.76563202)
\lineto(219.28539438,6.08203827)
\closepath
\moveto(217.97582406,10.01953827)
\lineto(217.97582406,10.01953827)
\closepath
}
}
{
\newrgbcolor{curcolor}{0 0 0}
\pscustom[linestyle=none,fillstyle=solid,fillcolor=curcolor]
{
\newpath
\moveto(227.88988656,10.06348358)
\curveto(228.95043344,10.06348358)(229.81176156,9.80567108)(230.47387094,9.29004608)
\curveto(231.14183969,8.77442108)(231.54320688,7.88672577)(231.6779725,6.62696015)
\lineto(230.13988656,6.62696015)
\curveto(230.04613656,7.20703827)(229.83226938,7.68750702)(229.498285,8.0683664)
\curveto(229.16430063,8.45508515)(228.62816781,8.64844452)(227.88988656,8.64844452)
\curveto(226.88207406,8.64844452)(226.16137094,8.15625702)(225.72777719,7.17188202)
\curveto(225.44652719,6.53321015)(225.30590219,5.74512421)(225.30590219,4.80762421)
\curveto(225.30590219,3.86426483)(225.50512094,3.07031952)(225.90355844,2.42578827)
\curveto(226.30199594,1.78125702)(226.92894906,1.4589914)(227.78441781,1.4589914)
\curveto(228.44066781,1.4589914)(228.9592225,1.65821015)(229.34008188,2.05664765)
\curveto(229.72680063,2.46094452)(229.99340219,3.01172577)(230.13988656,3.7089914)
\lineto(231.6779725,3.7089914)
\curveto(231.50219125,2.46094452)(231.06273813,1.54688202)(230.35961313,0.9668039)
\curveto(229.65648813,0.39258515)(228.75707406,0.10547577)(227.66137094,0.10547577)
\curveto(226.43090219,0.10547577)(225.44945688,0.55371796)(224.717035,1.45020233)
\curveto(223.98461313,2.35254608)(223.61840219,3.47754608)(223.61840219,4.82520233)
\curveto(223.61840219,6.47754608)(224.01976938,7.7636789)(224.82250375,8.68360077)
\curveto(225.62523813,9.60352265)(226.64769906,10.06348358)(227.88988656,10.06348358)
\closepath
\moveto(227.64379281,10.01953827)
\lineto(227.64379281,10.01953827)
\closepath
}
}
{
\newrgbcolor{curcolor}{0 0 0}
\pscustom[linestyle=none,fillstyle=solid,fillcolor=curcolor]
{
\newpath
\moveto(237.17992563,10.00196015)
\curveto(237.84789438,10.00196015)(238.49535531,9.84375702)(239.12230844,9.52735077)
\curveto(239.74926156,9.2168039)(240.22680063,8.81250702)(240.55492563,8.31446015)
\curveto(240.87133188,7.83985077)(241.08226938,7.28613983)(241.18773813,6.65332733)
\curveto(241.28148813,6.21973358)(241.32836313,5.52832733)(241.32836313,4.57910858)
\lineto(234.42894906,4.57910858)
\curveto(234.45824594,3.62403046)(234.68383188,2.85645233)(235.10570688,2.27637421)
\curveto(235.52758188,1.70215546)(236.18090219,1.41504608)(237.06566781,1.41504608)
\curveto(237.89183969,1.41504608)(238.55101938,1.68750702)(239.04320688,2.2324289)
\curveto(239.32445688,2.54883515)(239.52367563,2.91504608)(239.64086313,3.33106171)
\lineto(241.19652719,3.33106171)
\curveto(241.15551156,2.98535858)(241.01781625,2.59863983)(240.78344125,2.17090546)
\curveto(240.55492563,1.74903046)(240.29711313,1.40332733)(240.01000375,1.13379608)
\curveto(239.529535,0.66504608)(238.93480844,0.34863983)(238.22582406,0.18457733)
\curveto(237.84496469,0.09082733)(237.41430063,0.04395233)(236.93383188,0.04395233)
\curveto(235.76195688,0.04395233)(234.76879281,0.46875702)(233.95433969,1.3183664)
\curveto(233.13988656,2.17383515)(232.73266,3.36914765)(232.73266,4.9043039)
\curveto(232.73266,6.41602265)(233.14281625,7.64356171)(233.96312875,8.58692108)
\curveto(234.78344125,9.53028046)(235.85570688,10.00196015)(237.17992563,10.00196015)
\closepath
\moveto(239.70238656,5.83594452)
\curveto(239.63793344,6.5214914)(239.48851938,7.06934296)(239.25414438,7.47949921)
\curveto(238.82055063,8.24121796)(238.09691781,8.62207733)(237.08324594,8.62207733)
\curveto(236.35668344,8.62207733)(235.74730844,8.35840546)(235.25512094,7.83106171)
\curveto(234.76293344,7.30957733)(234.50219125,6.64453827)(234.47289438,5.83594452)
\lineto(239.70238656,5.83594452)
\closepath
\moveto(237.03051156,10.01953827)
\lineto(237.03051156,10.01953827)
\closepath
}
}
\end{pspicture}

    \caption{Unsupported Sections--Desired vs Actual}
  \end{figure}
\end{frame}

%
% Need to add some stuff about interior and exterior angles
%
\end{document}
