\documentclass[english,10pt]{beamer}
\usepackage{pstricks}
\usepackage{graphicx}
\DeclareGraphicsExtensions{.pdf}
\DeclareGraphicsRule{.pdf}{pdf}{.pdf}{}
\DeclareGraphicsExtensions{.jpg}
\DeclareGraphicsRule{. jpg}{jpg}{. jpg}{}
\DeclareGraphicsExtensions{.svg}
\DeclareGraphicsRule{. svg}{svg}{. svg}{}

\title{Design for 3D Printing}
\author{Brent Seidel\\modestconsulting@gmail.com\\Modest Consulting}
\institute{Modest Consulting}
\date{\today}


\subtitle{Design for 3D Printing}
\begin{document}
\begin{frame}
  \titlepage
\end{frame}

\begin{frame}
  \frametitle{Outline}
  \tableofcontents
\end{frame}

\section{Introduction}
\begin{frame}
  \frametitle{Introduction}
  A number of items need to be considered when designing an object for 3D printing.  Some of them are generic while others may depend on the 3D printing technology used.
\end{frame}

\section{Measurements}
\begin{frame}
  \frametitle{Measurements}
  A good question is, ``What does one unit in the CAD program translate to when printed out?''.  Depending on your application, the answer may range from mildly interesting to vitally important.  If you are creating a stand alone decorative object, you may not care much as long as the result is a reasonable size.  If, however, you are designing a part to physically interface with some existing items, the answer is vitally important.
\end{frame}

\section{First Layer}
\begin{frame}
  \frametitle{First Layer}
  Depending on the type of printer and its settings, the first layer printed may be thinner or thicker that the subsequent layers.  This is usually accompanied with the material extending beyond or within the subsequent layers.  This effect is sometimes called ``elephant foot''.  The slicer program often has a setting for ``elephant foot compensation''.
\end{frame}
\begin{frame}
  \frametitle{First Layer}
  \begin{figure}
    %LaTeX with PSTricks extensions
%%Creator: inkscape 0.92.2
%%Please note this file requires PSTricks extensions
\psset{xunit=.5pt,yunit=.5pt,runit=.5pt}
\begin{pspicture}(500.00001922,230.3632601)
{
\newrgbcolor{curcolor}{0.40000001 0.40000001 0.40000001}
\pscustom[linestyle=none,fillstyle=solid,fillcolor=curcolor]
{
\newpath
\moveto(0.00000006,10.00000341)
\lineto(500.00001928,10.00000341)
\lineto(500.00001928,0.00000371)
\lineto(0.00000006,0.00000371)
\closepath
}
}
{
\newrgbcolor{curcolor}{0 0 0}
\pscustom[linestyle=none,fillstyle=solid,fillcolor=curcolor]
{
\newpath
\moveto(64.15107667,191.52872306)
\lineto(73.107131,191.52872306)
\lineto(73.107131,189.94669187)
\lineto(65.90010003,189.94669187)
\lineto(65.90010003,186.02677015)
\lineto(72.23701384,186.02677015)
\lineto(72.23701384,184.48868428)
\lineto(65.90010003,184.48868428)
\lineto(65.90010003,178.61759076)
\lineto(64.15107667,178.61759076)
\closepath
}
}
{
\newrgbcolor{curcolor}{0 0 0}
\pscustom[linestyle=none,fillstyle=solid,fillcolor=curcolor]
{
\newpath
\moveto(74.77705306,187.98673101)
\lineto(76.38545144,187.98673101)
\lineto(76.38545144,178.61759076)
\lineto(74.77705306,178.61759076)
\closepath
\moveto(74.77705306,191.52872306)
\lineto(76.38545144,191.52872306)
\lineto(76.38545144,189.73575438)
\lineto(74.77705306,189.73575438)
\closepath
}
}
{
\newrgbcolor{curcolor}{0 0 0}
\pscustom[linestyle=none,fillstyle=solid,fillcolor=curcolor]
{
\newpath
\moveto(78.82881033,188.03067632)
\lineto(80.33173996,188.03067632)
\lineto(80.33173996,186.40469983)
\curveto(80.45478683,186.72110606)(80.75654463,187.10489511)(81.23701336,187.55606697)
\curveto(81.71748209,188.0130982)(82.27119301,188.24161382)(82.89814611,188.24161382)
\curveto(82.92744298,188.24161382)(82.97724767,188.23868413)(83.04756017,188.23282475)
\curveto(83.11787266,188.22696538)(83.23798984,188.21524663)(83.40791171,188.1976685)
\lineto(83.40791171,186.5277467)
\curveto(83.31416172,186.54532482)(83.2262711,186.55704357)(83.14423985,186.56290295)
\curveto(83.06806798,186.56876232)(82.98310704,186.57169201)(82.88935705,186.57169201)
\curveto(82.09248208,186.57169201)(81.48017742,186.31387952)(81.05244306,185.79825454)
\curveto(80.6247087,185.28848893)(80.41084152,184.69962177)(80.41084152,184.03165305)
\lineto(80.41084152,178.61759076)
\lineto(78.82881033,178.61759076)
\closepath
}
}
{
\newrgbcolor{curcolor}{0 0 0}
\pscustom[linestyle=none,fillstyle=solid,fillcolor=curcolor]
{
\newpath
\moveto(85.73701306,181.57071565)
\curveto(85.78388806,181.04337192)(85.91572399,180.63907506)(86.13252086,180.35782507)
\curveto(86.53095834,179.84805946)(87.22236457,179.59317666)(88.20673953,179.59317666)
\curveto(88.792677,179.59317666)(89.30830198,179.71915322)(89.75361446,179.97110633)
\curveto(90.19892695,180.22891882)(90.42158319,180.62442662)(90.42158319,181.15762972)
\curveto(90.42158319,181.56192658)(90.24287226,181.86954376)(89.8854504,182.08048125)
\curveto(89.65693478,182.20938749)(89.20576292,182.35880155)(88.53193483,182.52872342)
\lineto(87.27509894,182.84512966)
\curveto(86.4723646,183.0443484)(85.88056774,183.26700464)(85.49970838,183.51309838)
\curveto(84.82002091,183.94083274)(84.48017718,184.53262959)(84.48017718,185.28848893)
\curveto(84.48017718,186.1791139)(84.7995131,186.89981699)(85.43818495,187.45059822)
\curveto(86.08271617,188.00137945)(86.94697395,188.27677006)(88.03095828,188.27677006)
\curveto(89.44892698,188.27677006)(90.47138787,187.86075446)(91.09834097,187.02872324)
\curveto(91.49091908,186.50137951)(91.68134876,185.93302016)(91.66963001,185.32364518)
\lineto(90.17548945,185.32364518)
\curveto(90.14619257,185.68106704)(90.02021602,186.00626234)(89.79755978,186.29923108)
\curveto(89.43427854,186.71524669)(88.80439575,186.92325449)(87.90791141,186.92325449)
\curveto(87.31025519,186.92325449)(86.85615364,186.80899669)(86.54560678,186.58048107)
\curveto(86.24091929,186.35196545)(86.08857555,186.05020765)(86.08857555,185.67520767)
\curveto(86.08857555,185.26505143)(86.29072398,184.93692645)(86.69502084,184.69083271)
\curveto(86.92939583,184.54434834)(87.27509894,184.41544209)(87.73213017,184.30411397)
\lineto(88.77802857,184.04923117)
\curveto(89.91474727,183.77384056)(90.67646599,183.50723901)(91.06318472,183.24942652)
\curveto(91.67841907,182.84512966)(91.98603625,182.20938749)(91.98603625,181.34220003)
\curveto(91.98603625,180.50430944)(91.66670033,179.78067665)(91.02802848,179.17130168)
\curveto(90.395216,178.5619267)(89.42841916,178.25723922)(88.12763797,178.25723922)
\curveto(86.7272474,178.25723922)(85.73408338,178.57364545)(85.1481459,179.20645793)
\curveto(84.5680678,179.84512978)(84.25752093,180.63321568)(84.21650531,181.57071565)
\closepath
\moveto(88.07490359,188.25919194)
\closepath
}
}
{
\newrgbcolor{curcolor}{0 0 0}
\pscustom[linestyle=none,fillstyle=solid,fillcolor=curcolor]
{
\newpath
\moveto(94.11298929,190.65860591)
\lineto(95.7125986,190.65860591)
\lineto(95.7125986,188.03067632)
\lineto(97.21552823,188.03067632)
\lineto(97.21552823,186.73868419)
\lineto(95.7125986,186.73868419)
\lineto(95.7125986,180.59512975)
\curveto(95.7125986,180.26700476)(95.82392672,180.04727821)(96.04658296,179.93595009)
\curveto(96.16962983,179.87149696)(96.37470795,179.8392704)(96.66181731,179.8392704)
\lineto(96.90791105,179.8392704)
\curveto(96.99580167,179.84512978)(97.09834073,179.85391884)(97.21552823,179.86563759)
\lineto(97.21552823,178.61759076)
\curveto(97.03388761,178.56485639)(96.84345793,178.52677045)(96.64423919,178.50333296)
\curveto(96.45087982,178.47989546)(96.23994233,178.46817671)(96.01142671,178.46817671)
\curveto(95.27314549,178.46817671)(94.77216895,178.6556767)(94.50849709,179.03067668)
\curveto(94.24482522,179.41153604)(94.11298929,179.90372352)(94.11298929,180.50723913)
\lineto(94.11298929,186.73868419)
\lineto(92.83857528,186.73868419)
\lineto(92.83857528,188.03067632)
\lineto(94.11298929,188.03067632)
\closepath
}
}
{
\newrgbcolor{curcolor}{0 0 0}
\pscustom[linestyle=none,fillstyle=solid,fillcolor=curcolor]
{
\newpath
\moveto(103.82490192,191.52872306)
\lineto(105.40693311,191.52872306)
\lineto(105.40693311,178.61759076)
\lineto(103.82490192,178.61759076)
\closepath
}
}
{
\newrgbcolor{curcolor}{0 0 0}
\pscustom[linestyle=none,fillstyle=solid,fillcolor=curcolor]
{
\newpath
\moveto(109.00166185,181.12247348)
\curveto(109.00166185,180.66544224)(109.16865403,180.3050907)(109.50263839,180.04141883)
\curveto(109.83662275,179.77774697)(110.23213055,179.64591103)(110.68916178,179.64591103)
\curveto(111.24580238,179.64591103)(111.78486486,179.77481728)(112.30634922,180.03262977)
\curveto(113.18525543,180.46036413)(113.62470854,181.16055941)(113.62470854,182.13321562)
\lineto(113.62470854,183.40762963)
\curveto(113.43134917,183.28458276)(113.18232574,183.18204371)(112.87763826,183.10001246)
\curveto(112.57295077,183.01798121)(112.27412266,182.95938746)(111.98115392,182.92423122)
\lineto(111.02314614,182.80118435)
\curveto(110.44892742,182.72501247)(110.01826337,182.60489529)(109.73115401,182.4408328)
\curveto(109.2448259,182.16544218)(109.00166185,181.72598908)(109.00166185,181.12247348)
\closepath
\moveto(112.83369295,184.3216921)
\curveto(113.19697418,184.3685671)(113.44013823,184.52091084)(113.5631851,184.77872333)
\curveto(113.6334976,184.91934832)(113.66865385,185.12149675)(113.66865385,185.38516862)
\curveto(113.66865385,185.9242311)(113.47529448,186.31387952)(113.08857575,186.55411388)
\curveto(112.70771639,186.80020762)(112.15986485,186.92325449)(111.44502113,186.92325449)
\curveto(110.61884928,186.92325449)(110.03291181,186.70059825)(109.6872087,186.25528577)
\curveto(109.49384933,186.00919203)(109.36787277,185.64298111)(109.30927902,185.156653)
\lineto(107.83271658,185.156653)
\curveto(107.86201346,186.31680921)(108.23701344,187.12247324)(108.95771654,187.57364509)
\curveto(109.68427901,188.03067632)(110.52509929,188.25919194)(111.48017738,188.25919194)
\curveto(112.58759921,188.25919194)(113.48701323,188.04825445)(114.17841945,187.62637947)
\curveto(114.8639663,187.20450448)(115.20673973,186.54825451)(115.20673973,185.65762954)
\lineto(115.20673973,180.2347782)
\curveto(115.20673973,180.07071571)(115.23896629,179.93887977)(115.30341941,179.8392704)
\curveto(115.37373191,179.73966103)(115.51728659,179.68985635)(115.73408346,179.68985635)
\curveto(115.80439595,179.68985635)(115.88349751,179.69278603)(115.97138813,179.69864541)
\curveto(116.05927875,179.71036416)(116.15302875,179.72501259)(116.25263812,179.74259072)
\lineto(116.25263812,178.57364545)
\curveto(116.00654438,178.50333296)(115.81904439,178.45938764)(115.69013814,178.44180952)
\curveto(115.5612319,178.4242314)(115.38545066,178.41544233)(115.16279442,178.41544233)
\curveto(114.61787256,178.41544233)(114.22236477,178.6088017)(113.97627103,178.99552044)
\curveto(113.84736478,179.20059855)(113.75654447,179.4906376)(113.7038101,179.86563759)
\curveto(113.38154449,179.44376261)(112.91865388,179.07755168)(112.31513828,178.76700482)
\curveto(111.71162268,178.45645796)(111.04658364,178.30118453)(110.32002117,178.30118453)
\curveto(109.44697433,178.30118453)(108.73213061,178.56485639)(108.17549001,179.09220012)
\curveto(107.62470878,179.62540322)(107.34931817,180.29044226)(107.34931817,181.08731723)
\curveto(107.34931817,181.96036407)(107.62177909,182.63712185)(108.16670095,183.11759058)
\curveto(108.7116228,183.59805931)(109.42646652,183.89395774)(110.31123211,184.00528586)
\closepath
\moveto(111.52412269,188.25919194)
\closepath
}
}
{
\newrgbcolor{curcolor}{0 0 0}
\pscustom[linestyle=none,fillstyle=solid,fillcolor=curcolor]
{
\newpath
\moveto(123.6881851,188.03067632)
\lineto(125.43720847,188.03067632)
\curveto(125.21455223,187.42716072)(124.71943506,186.05020765)(123.95185696,183.89981711)
\curveto(123.37763824,182.28262968)(122.89716951,180.96427036)(122.51045077,179.94473915)
\curveto(121.59638831,177.54239549)(120.95185708,176.0775518)(120.5768571,175.55020807)
\curveto(120.20185711,175.02286434)(119.55732589,174.75919248)(118.64326343,174.75919248)
\curveto(118.42060719,174.75919248)(118.24775563,174.76798154)(118.12470876,174.78555967)
\curveto(118.00752127,174.80313779)(117.8610369,174.83536435)(117.68525565,174.88223935)
\lineto(117.68525565,176.32364554)
\curveto(117.96064627,176.24747367)(118.15986501,176.20059867)(118.28291188,176.18302055)
\curveto(118.40595875,176.16544242)(118.51435718,176.15665336)(118.60810718,176.15665336)
\curveto(118.90107592,176.15665336)(119.1149431,176.20645805)(119.24970872,176.30606742)
\curveto(119.39033371,176.39981741)(119.50752121,176.51700491)(119.6012712,176.6576299)
\curveto(119.63056808,176.7045049)(119.73603682,176.94473927)(119.91767744,177.378333)
\curveto(120.09931806,177.81192673)(120.23115399,178.13419235)(120.31318524,178.34512984)
\lineto(116.83271662,188.03067632)
\lineto(118.6256853,188.03067632)
\lineto(121.14814614,180.36661413)
\closepath
\moveto(121.13935708,188.25919194)
\closepath
}
}
{
\newrgbcolor{curcolor}{0 0 0}
\pscustom[linestyle=none,fillstyle=solid,fillcolor=curcolor]
{
\newpath
\moveto(130.72822208,188.24161382)
\curveto(131.3961908,188.24161382)(132.04365171,188.0834107)(132.67060481,187.76700446)
\curveto(133.29755791,187.4564576)(133.77509696,187.05216074)(134.10322194,186.55411388)
\curveto(134.41962818,186.07950453)(134.63056567,185.52579361)(134.73603442,184.89298114)
\curveto(134.82978441,184.4593874)(134.87665941,183.76798118)(134.87665941,182.81876247)
\lineto(127.97724563,182.81876247)
\curveto(128.0065425,181.86368438)(128.23212843,181.09610629)(128.65400341,180.51602819)
\curveto(129.07587839,179.94180946)(129.72919868,179.6547001)(130.61396427,179.6547001)
\curveto(131.44013611,179.6547001)(132.09931577,179.92716102)(132.59150325,180.47208288)
\curveto(132.87275324,180.78848911)(133.07197198,181.15470004)(133.18915948,181.57071565)
\lineto(134.74482348,181.57071565)
\curveto(134.70380786,181.22501253)(134.56611255,180.8382938)(134.33173756,180.41055944)
\curveto(134.10322194,179.98868446)(133.84540945,179.64298135)(133.55830009,179.37345011)
\curveto(133.07783136,178.90470013)(132.48310482,178.58829389)(131.77412047,178.4242314)
\curveto(131.39326111,178.3304814)(130.96259707,178.2836064)(130.48212834,178.2836064)
\curveto(129.31025339,178.2836064)(128.31708936,178.70841107)(127.50263627,179.55802041)
\curveto(126.68818318,180.41348913)(126.28095663,181.60880158)(126.28095663,183.14395777)
\curveto(126.28095663,184.65567646)(126.69111287,185.88321547)(127.51142533,186.82657481)
\curveto(128.3317378,187.76993415)(129.40400338,188.24161382)(130.72822208,188.24161382)
\closepath
\moveto(133.25068292,184.07559836)
\curveto(133.18622979,184.7611452)(133.03681574,185.30899675)(132.80244075,185.71915298)
\curveto(132.36884701,186.4808717)(131.64521423,186.86173106)(130.63154239,186.86173106)
\curveto(129.90497992,186.86173106)(129.29560495,186.59805919)(128.80341747,186.07071546)
\curveto(128.31122999,185.54923111)(128.05048781,184.88419207)(128.02119094,184.07559836)
\closepath
\moveto(130.57880802,188.25919194)
\closepath
}
}
{
\newrgbcolor{curcolor}{0 0 0}
\pscustom[linestyle=none,fillstyle=solid,fillcolor=curcolor]
{
\newpath
\moveto(136.87177692,188.03067632)
\lineto(138.37470655,188.03067632)
\lineto(138.37470655,186.40469983)
\curveto(138.49775342,186.72110606)(138.79951122,187.10489511)(139.27997995,187.55606697)
\curveto(139.76044868,188.0130982)(140.3141596,188.24161382)(140.9411127,188.24161382)
\curveto(140.97040957,188.24161382)(141.02021426,188.23868413)(141.09052675,188.23282475)
\curveto(141.16083925,188.22696538)(141.28095643,188.21524663)(141.4508783,188.1976685)
\lineto(141.4508783,186.5277467)
\curveto(141.3571283,186.54532482)(141.26923768,186.55704357)(141.18720644,186.56290295)
\curveto(141.11103456,186.56876232)(141.02607363,186.57169201)(140.93232363,186.57169201)
\curveto(140.13544867,186.57169201)(139.523144,186.31387952)(139.09540964,185.79825454)
\curveto(138.66767529,185.28848893)(138.45380811,184.69962177)(138.45380811,184.03165305)
\lineto(138.45380811,178.61759076)
\lineto(136.87177692,178.61759076)
\closepath
}
}
{
\newrgbcolor{curcolor}{0 0 0}
\pscustom[linestyle=none,fillstyle=solid,fillcolor=curcolor]
{
\newpath
\moveto(148.14814121,190.65860591)
\lineto(149.74775053,190.65860591)
\lineto(149.74775053,188.03067632)
\lineto(151.25068015,188.03067632)
\lineto(151.25068015,186.73868419)
\lineto(149.74775053,186.73868419)
\lineto(149.74775053,180.59512975)
\curveto(149.74775053,180.26700476)(149.85907865,180.04727821)(150.08173489,179.93595009)
\curveto(150.20478176,179.87149696)(150.40985987,179.8392704)(150.69696924,179.8392704)
\lineto(150.94306298,179.8392704)
\curveto(151.0309536,179.84512978)(151.13349266,179.85391884)(151.25068015,179.86563759)
\lineto(151.25068015,178.61759076)
\curveto(151.06903953,178.56485639)(150.87860986,178.52677045)(150.67939111,178.50333296)
\curveto(150.48603175,178.47989546)(150.27509425,178.46817671)(150.04657864,178.46817671)
\curveto(149.30829742,178.46817671)(148.80732088,178.6556767)(148.54364901,179.03067668)
\curveto(148.27997715,179.41153604)(148.14814121,179.90372352)(148.14814121,180.50723913)
\lineto(148.14814121,186.73868419)
\lineto(146.8737272,186.73868419)
\lineto(146.8737272,188.03067632)
\lineto(148.14814121,188.03067632)
\closepath
}
}
{
\newrgbcolor{curcolor}{0 0 0}
\pscustom[linestyle=none,fillstyle=solid,fillcolor=curcolor]
{
\newpath
\moveto(156.55927677,179.63712197)
\curveto(157.60810486,179.63712197)(158.32587827,180.03262977)(158.712597,180.82364536)
\curveto(159.10517511,181.62052033)(159.30146416,182.50528592)(159.30146416,183.47794213)
\curveto(159.30146416,184.35684835)(159.16083917,185.07169207)(158.87958918,185.6224733)
\curveto(158.4342767,186.48966076)(157.6666986,186.92325449)(156.5768549,186.92325449)
\curveto(155.61005806,186.92325449)(154.90693309,186.55411388)(154.46747998,185.81583266)
\curveto(154.02802687,185.07755144)(153.80830032,184.18692648)(153.80830032,183.14395777)
\curveto(153.80830032,182.14200468)(154.02802687,181.30704378)(154.46747998,180.63907506)
\curveto(154.90693309,179.97110633)(155.60419869,179.63712197)(156.55927677,179.63712197)
\closepath
\moveto(156.62080021,188.30313725)
\curveto(157.83369078,188.30313725)(158.85908137,187.89884039)(159.69697196,187.09024667)
\curveto(160.53486255,186.28165296)(160.95380785,185.09219988)(160.95380785,183.52188744)
\curveto(160.95380785,182.00430938)(160.58466724,180.75040318)(159.84638602,179.76016884)
\curveto(159.1081048,178.76993451)(157.96259703,178.27481734)(156.40986272,178.27481734)
\curveto(155.11494089,178.27481734)(154.08662062,178.71134076)(153.3249019,179.5843876)
\curveto(152.56318318,180.46329381)(152.18232382,181.64102814)(152.18232382,183.11759058)
\curveto(152.18232382,184.69962177)(152.583691,185.95938734)(153.38642534,186.89688731)
\curveto(154.18915968,187.83438727)(155.26728464,188.30313725)(156.62080021,188.30313725)
\closepath
\moveto(156.56806584,188.25919194)
\closepath
}
}
{
\newrgbcolor{curcolor}{0 0 0}
\pscustom[linestyle=none,fillstyle=solid,fillcolor=curcolor]
{
\newpath
\moveto(166.57880802,179.63712197)
\curveto(167.62763611,179.63712197)(168.34540952,180.03262977)(168.73212825,180.82364536)
\curveto(169.12470636,181.62052033)(169.32099541,182.50528592)(169.32099541,183.47794213)
\curveto(169.32099541,184.35684835)(169.18037042,185.07169207)(168.89912043,185.6224733)
\curveto(168.45380795,186.48966076)(167.68622985,186.92325449)(166.59638615,186.92325449)
\curveto(165.62958931,186.92325449)(164.92646434,186.55411388)(164.48701123,185.81583266)
\curveto(164.04755812,185.07755144)(163.82783157,184.18692648)(163.82783157,183.14395777)
\curveto(163.82783157,182.14200468)(164.04755812,181.30704378)(164.48701123,180.63907506)
\curveto(164.92646434,179.97110633)(165.62372994,179.63712197)(166.57880802,179.63712197)
\closepath
\moveto(166.64033146,188.30313725)
\curveto(167.85322203,188.30313725)(168.87861262,187.89884039)(169.71650321,187.09024667)
\curveto(170.5543938,186.28165296)(170.9733391,185.09219988)(170.9733391,183.52188744)
\curveto(170.9733391,182.00430938)(170.60419849,180.75040318)(169.86591727,179.76016884)
\curveto(169.12763605,178.76993451)(167.98212828,178.27481734)(166.42939397,178.27481734)
\curveto(165.13447214,178.27481734)(164.10615187,178.71134076)(163.34443315,179.5843876)
\curveto(162.58271443,180.46329381)(162.20185507,181.64102814)(162.20185507,183.11759058)
\curveto(162.20185507,184.69962177)(162.60322225,185.95938734)(163.40595659,186.89688731)
\curveto(164.20869093,187.83438727)(165.28681589,188.30313725)(166.64033146,188.30313725)
\closepath
\moveto(166.58759709,188.25919194)
\closepath
}
}
{
\newrgbcolor{curcolor}{0 0 0}
\pscustom[linestyle=none,fillstyle=solid,fillcolor=curcolor]
{
\newpath
\moveto(178.17158448,190.65860591)
\lineto(179.77119379,190.65860591)
\lineto(179.77119379,188.03067632)
\lineto(181.27412342,188.03067632)
\lineto(181.27412342,186.73868419)
\lineto(179.77119379,186.73868419)
\lineto(179.77119379,180.59512975)
\curveto(179.77119379,180.26700476)(179.88252191,180.04727821)(180.10517815,179.93595009)
\curveto(180.22822502,179.87149696)(180.43330314,179.8392704)(180.7204125,179.8392704)
\lineto(180.96650624,179.8392704)
\curveto(181.05439687,179.84512978)(181.15693592,179.85391884)(181.27412342,179.86563759)
\lineto(181.27412342,178.61759076)
\curveto(181.0924828,178.56485639)(180.90205312,178.52677045)(180.70283438,178.50333296)
\curveto(180.50947501,178.47989546)(180.29853752,178.46817671)(180.07002191,178.46817671)
\curveto(179.33174069,178.46817671)(178.83076414,178.6556767)(178.56709228,179.03067668)
\curveto(178.30342041,179.41153604)(178.17158448,179.90372352)(178.17158448,180.50723913)
\lineto(178.17158448,186.73868419)
\lineto(176.89717047,186.73868419)
\lineto(176.89717047,188.03067632)
\lineto(178.17158448,188.03067632)
\closepath
}
}
{
\newrgbcolor{curcolor}{0 0 0}
\pscustom[linestyle=none,fillstyle=solid,fillcolor=curcolor]
{
\newpath
\moveto(182.84736142,191.57266837)
\lineto(184.42939261,191.57266837)
\lineto(184.42939261,186.75626231)
\curveto(184.80439259,187.23087167)(185.14130664,187.56485603)(185.44013475,187.7582154)
\curveto(185.94990036,188.09219976)(186.58564252,188.25919194)(187.34736124,188.25919194)
\curveto(188.71259556,188.25919194)(189.63837677,187.7816529)(190.12470488,186.82657481)
\curveto(190.38837674,186.30509046)(190.52021267,185.58145767)(190.52021267,184.65567646)
\lineto(190.52021267,178.61759076)
\lineto(188.89423618,178.61759076)
\lineto(188.89423618,184.55020771)
\curveto(188.89423618,185.24161394)(188.80634556,185.74844985)(188.63056431,186.07071546)
\curveto(188.34345495,186.58634044)(187.80439247,186.84415293)(187.01337688,186.84415293)
\curveto(186.3571269,186.84415293)(185.76240036,186.61856701)(185.22919726,186.16739515)
\curveto(184.69599416,185.71622329)(184.42939261,184.86368426)(184.42939261,183.60977806)
\lineto(184.42939261,178.61759076)
\lineto(182.84736142,178.61759076)
\closepath
}
}
{
\newrgbcolor{curcolor}{0 0 0}
\pscustom[linestyle=none,fillstyle=solid,fillcolor=curcolor]
{
\newpath
\moveto(192.86689988,187.98673101)
\lineto(194.47529825,187.98673101)
\lineto(194.47529825,178.61759076)
\lineto(192.86689988,178.61759076)
\closepath
\moveto(192.86689988,191.52872306)
\lineto(194.47529825,191.52872306)
\lineto(194.47529825,189.73575438)
\lineto(192.86689988,189.73575438)
\closepath
}
}
{
\newrgbcolor{curcolor}{0 0 0}
\pscustom[linestyle=none,fillstyle=solid,fillcolor=curcolor]
{
\newpath
\moveto(196.87470373,188.03067632)
\lineto(198.37763335,188.03067632)
\lineto(198.37763335,186.69473888)
\curveto(198.82294584,187.24552011)(199.29462551,187.6410279)(199.79267236,187.88126227)
\curveto(200.29071922,188.12149663)(200.84443013,188.24161382)(201.45380511,188.24161382)
\curveto(202.78974255,188.24161382)(203.69208627,187.77579352)(204.16083625,186.84415293)
\curveto(204.41864874,186.33438733)(204.54755498,185.60489517)(204.54755498,184.65567646)
\lineto(204.54755498,178.61759076)
\lineto(202.93915661,178.61759076)
\lineto(202.93915661,184.55020771)
\curveto(202.93915661,185.12442644)(202.85419568,185.58731705)(202.68427381,185.93887953)
\curveto(202.40302382,186.52481701)(201.89325821,186.81778575)(201.15497699,186.81778575)
\curveto(200.77997701,186.81778575)(200.47235983,186.77969981)(200.23212547,186.70352794)
\curveto(199.79853173,186.57462169)(199.41767238,186.31680921)(199.08954739,185.93009047)
\curveto(198.82587552,185.61954361)(198.65302397,185.297278)(198.57099272,184.96329363)
\curveto(198.49482085,184.63516865)(198.45673491,184.16348898)(198.45673491,183.54825463)
\lineto(198.45673491,178.61759076)
\lineto(196.87470373,178.61759076)
\closepath
\moveto(200.59247702,188.25919194)
\closepath
}
}
{
\newrgbcolor{curcolor}{0 0 0}
\pscustom[linestyle=none,fillstyle=solid,fillcolor=curcolor]
{
\newpath
\moveto(316.91587072,230.31931384)
\lineto(325.87192505,230.31931384)
\lineto(325.87192505,228.73728265)
\lineto(318.66489409,228.73728265)
\lineto(318.66489409,224.81736094)
\lineto(325.0018079,224.81736094)
\lineto(325.0018079,223.27927506)
\lineto(318.66489409,223.27927506)
\lineto(318.66489409,217.40818155)
\lineto(316.91587072,217.40818155)
\closepath
}
}
{
\newrgbcolor{curcolor}{0 0 0}
\pscustom[linestyle=none,fillstyle=solid,fillcolor=curcolor]
{
\newpath
\moveto(327.54184712,226.7773218)
\lineto(329.15024549,226.7773218)
\lineto(329.15024549,217.40818155)
\lineto(327.54184712,217.40818155)
\closepath
\moveto(327.54184712,230.31931384)
\lineto(329.15024549,230.31931384)
\lineto(329.15024549,228.52634516)
\lineto(327.54184712,228.52634516)
\closepath
}
}
{
\newrgbcolor{curcolor}{0 0 0}
\pscustom[linestyle=none,fillstyle=solid,fillcolor=curcolor]
{
\newpath
\moveto(331.59360439,226.82126711)
\lineto(333.09653402,226.82126711)
\lineto(333.09653402,225.19529061)
\curveto(333.21958089,225.51169685)(333.52133869,225.89548589)(334.00180742,226.34665775)
\curveto(334.48227615,226.80368898)(335.03598707,227.0322046)(335.66294017,227.0322046)
\curveto(335.69223704,227.0322046)(335.74204172,227.02927491)(335.81235422,227.02341554)
\curveto(335.88266672,227.01755616)(336.0027839,227.00583741)(336.17270577,226.98825929)
\lineto(336.17270577,225.31833748)
\curveto(336.07895577,225.3359156)(335.99106515,225.34763435)(335.90903391,225.35349373)
\curveto(335.83286203,225.3593531)(335.7479011,225.36228279)(335.6541511,225.36228279)
\curveto(334.85727614,225.36228279)(334.24497147,225.1044703)(333.81723711,224.58884532)
\curveto(333.38950276,224.07907972)(333.17563558,223.49021255)(333.17563558,222.82224383)
\lineto(333.17563558,217.40818155)
\lineto(331.59360439,217.40818155)
\closepath
}
}
{
\newrgbcolor{curcolor}{0 0 0}
\pscustom[linestyle=none,fillstyle=solid,fillcolor=curcolor]
{
\newpath
\moveto(338.50180712,220.36130643)
\curveto(338.54868212,219.8339627)(338.68051805,219.42966584)(338.89731492,219.14841585)
\curveto(339.2957524,218.63865025)(339.98715862,218.38376744)(340.97153358,218.38376744)
\curveto(341.55747106,218.38376744)(342.07309604,218.509744)(342.51840852,218.76169712)
\curveto(342.963721,219.01950961)(343.18637724,219.4150174)(343.18637724,219.94822051)
\curveto(343.18637724,220.35251736)(343.00766631,220.66013454)(342.65024445,220.87107203)
\curveto(342.42172884,220.99997828)(341.97055698,221.14939233)(341.29672888,221.3193142)
\lineto(340.039893,221.63572044)
\curveto(339.23715865,221.83493918)(338.6453618,222.05759542)(338.26450244,222.30368916)
\curveto(337.58481497,222.73142352)(337.24497123,223.32322037)(337.24497123,224.07907972)
\curveto(337.24497123,224.96970468)(337.56430716,225.69040778)(338.20297901,226.241189)
\curveto(338.84751023,226.79197023)(339.71176801,227.06736085)(340.79575234,227.06736085)
\curveto(342.21372103,227.06736085)(343.23618193,226.65134524)(343.86313503,225.81931402)
\curveto(344.25571314,225.29197029)(344.44614282,224.72361094)(344.43442407,224.11423596)
\lineto(342.9402835,224.11423596)
\curveto(342.91098663,224.47165782)(342.78501007,224.79685312)(342.56235383,225.08982186)
\curveto(342.1990726,225.50583747)(341.56918981,225.71384528)(340.67270547,225.71384528)
\curveto(340.07504924,225.71384528)(339.6209477,225.59958747)(339.31040084,225.37107185)
\curveto(339.00571335,225.14255624)(338.85336961,224.84079844)(338.85336961,224.46579845)
\curveto(338.85336961,224.05564222)(339.05551803,223.72751723)(339.45981489,223.48142349)
\curveto(339.69418988,223.33493912)(340.039893,223.20603288)(340.49692423,223.09470476)
\lineto(341.54282262,222.83982195)
\curveto(342.67954133,222.56443134)(343.44126005,222.29782979)(343.82797878,222.0400173)
\curveto(344.44321313,221.63572044)(344.75083031,220.99997828)(344.75083031,220.13279081)
\curveto(344.75083031,219.29490022)(344.43149438,218.57126744)(343.79282253,217.96189246)
\curveto(343.16001006,217.35251749)(342.19321322,217.04783)(340.89243202,217.04783)
\curveto(339.49204145,217.04783)(338.49887743,217.36423623)(337.91293996,217.99704871)
\curveto(337.33286185,218.63572056)(337.02231499,219.42380646)(336.98129937,220.36130643)
\closepath
\moveto(340.83969765,227.04978272)
\closepath
}
}
{
\newrgbcolor{curcolor}{0 0 0}
\pscustom[linestyle=none,fillstyle=solid,fillcolor=curcolor]
{
\newpath
\moveto(346.87778335,229.44919669)
\lineto(348.47739266,229.44919669)
\lineto(348.47739266,226.82126711)
\lineto(349.98032228,226.82126711)
\lineto(349.98032228,225.52927497)
\lineto(348.47739266,225.52927497)
\lineto(348.47739266,219.38572053)
\curveto(348.47739266,219.05759554)(348.58872078,218.83786899)(348.81137702,218.72654087)
\curveto(348.93442389,218.66208775)(349.13950201,218.62986118)(349.42661137,218.62986118)
\lineto(349.67270511,218.62986118)
\curveto(349.76059573,218.63572056)(349.86313479,218.64450962)(349.98032228,218.65622837)
\lineto(349.98032228,217.40818155)
\curveto(349.79868167,217.35544717)(349.60825199,217.31736124)(349.40903325,217.29392374)
\curveto(349.21567388,217.27048624)(349.00473639,217.25876749)(348.77622077,217.25876749)
\curveto(348.03793955,217.25876749)(347.53696301,217.44626748)(347.27329114,217.82126747)
\curveto(347.00961928,218.20212683)(346.87778335,218.69431431)(346.87778335,219.29782991)
\lineto(346.87778335,225.52927497)
\lineto(345.60336934,225.52927497)
\lineto(345.60336934,226.82126711)
\lineto(346.87778335,226.82126711)
\closepath
}
}
{
\newrgbcolor{curcolor}{0 0 0}
\pscustom[linestyle=none,fillstyle=solid,fillcolor=curcolor]
{
\newpath
\moveto(356.58969598,230.31931384)
\lineto(358.17172716,230.31931384)
\lineto(358.17172716,217.40818155)
\lineto(356.58969598,217.40818155)
\closepath
}
}
{
\newrgbcolor{curcolor}{0 0 0}
\pscustom[linestyle=none,fillstyle=solid,fillcolor=curcolor]
{
\newpath
\moveto(361.76645591,219.91306426)
\curveto(361.76645591,219.45603303)(361.93344809,219.09568148)(362.26743245,218.83200961)
\curveto(362.60141681,218.56833775)(362.99692461,218.43650182)(363.45395584,218.43650182)
\curveto(364.01059644,218.43650182)(364.54965892,218.56540806)(365.07114327,218.82322055)
\curveto(365.95004949,219.25095491)(366.3895026,219.95115019)(366.3895026,220.9238064)
\lineto(366.3895026,222.19822042)
\curveto(366.19614323,222.07517355)(365.9471198,221.97263449)(365.64243231,221.89060324)
\curveto(365.33774483,221.80857199)(365.03891671,221.74997825)(364.74594797,221.714822)
\lineto(363.7879402,221.59177513)
\curveto(363.21372147,221.51560326)(362.78305743,221.39548607)(362.49594806,221.23142358)
\curveto(362.00961996,220.95603297)(361.76645591,220.51657986)(361.76645591,219.91306426)
\closepath
\moveto(365.598487,223.11228288)
\curveto(365.96176824,223.15915788)(366.20493229,223.31150162)(366.32797916,223.56931411)
\curveto(366.39829166,223.70993911)(366.43344791,223.91208753)(366.43344791,224.1757594)
\curveto(366.43344791,224.71482188)(366.24008854,225.1044703)(365.85336981,225.34470466)
\curveto(365.47251045,225.59079841)(364.92465891,225.71384528)(364.20981518,225.71384528)
\curveto(363.38364334,225.71384528)(362.79770587,225.49118903)(362.45200275,225.04587655)
\curveto(362.25864339,224.79978281)(362.13266683,224.43357189)(362.07407308,223.94724378)
\lineto(360.59751064,223.94724378)
\curveto(360.62680751,225.10739999)(361.0018075,225.91306402)(361.7225106,226.36423587)
\curveto(362.44907307,226.82126711)(363.28989335,227.04978272)(364.24497143,227.04978272)
\curveto(365.35239326,227.04978272)(366.25180729,226.83884523)(366.94321351,226.41697025)
\curveto(367.62876036,225.99509526)(367.97153378,225.33884529)(367.97153378,224.44822033)
\lineto(367.97153378,219.02536898)
\curveto(367.97153378,218.86130649)(368.00376034,218.72947055)(368.06821347,218.62986118)
\curveto(368.13852596,218.53025181)(368.28208065,218.48044713)(368.49887751,218.48044713)
\curveto(368.56919001,218.48044713)(368.64829157,218.48337681)(368.73618219,218.48923619)
\curveto(368.82407281,218.50095494)(368.91782281,218.51560338)(369.01743218,218.5331815)
\lineto(369.01743218,217.36423623)
\curveto(368.77133844,217.29392374)(368.58383845,217.24997843)(368.4549322,217.2324003)
\curveto(368.32602596,217.21482218)(368.15024471,217.20603312)(367.92758847,217.20603312)
\curveto(367.38266662,217.20603312)(366.98715882,217.39939248)(366.74106508,217.78611122)
\curveto(366.61215884,217.99118933)(366.52133853,218.28122839)(366.46860416,218.65622837)
\curveto(366.14633854,218.23435339)(365.68344794,217.86814246)(365.07993234,217.5575956)
\curveto(364.47641674,217.24704874)(363.8113777,217.09177531)(363.08481523,217.09177531)
\curveto(362.21176839,217.09177531)(361.49692467,217.35544717)(360.94028406,217.8827909)
\curveto(360.38950284,218.415994)(360.11411222,219.08103304)(360.11411222,219.87790801)
\curveto(360.11411222,220.75095485)(360.38657315,221.42771263)(360.931495,221.90818137)
\curveto(361.47641686,222.3886501)(362.19126058,222.68454852)(363.07602617,222.79587664)
\closepath
\moveto(364.28891674,227.04978272)
\closepath
}
}
{
\newrgbcolor{curcolor}{0 0 0}
\pscustom[linestyle=none,fillstyle=solid,fillcolor=curcolor]
{
\newpath
\moveto(376.45297916,226.82126711)
\lineto(378.20200252,226.82126711)
\curveto(377.97934628,226.2177515)(377.48422912,224.84079844)(376.71665102,222.6904079)
\curveto(376.14243229,221.07322046)(375.66196356,219.75486114)(375.27524483,218.73532993)
\curveto(374.36118237,216.33298628)(373.71665114,214.86814258)(373.34165116,214.34079886)
\curveto(372.96665117,213.81345513)(372.32211995,213.54978326)(371.40805748,213.54978326)
\curveto(371.18540124,213.54978326)(371.01254969,213.55857232)(370.88950282,213.57615045)
\curveto(370.77231532,213.59372857)(370.62583095,213.62595513)(370.45004971,213.67283013)
\lineto(370.45004971,215.11423632)
\curveto(370.72544032,215.03806445)(370.92465907,214.99118945)(371.04770594,214.97361133)
\curveto(371.17075281,214.95603321)(371.27915124,214.94724414)(371.37290124,214.94724414)
\curveto(371.66586997,214.94724414)(371.87973715,214.99704883)(372.01450277,215.0966582)
\curveto(372.15512777,215.1904082)(372.27231526,215.30759569)(372.36606526,215.44822069)
\curveto(372.39536213,215.49509568)(372.50083088,215.73533005)(372.6824715,216.16892378)
\curveto(372.86411211,216.60251752)(372.99594805,216.92478313)(373.07797929,217.13572062)
\lineto(369.59751068,226.82126711)
\lineto(371.39047936,226.82126711)
\lineto(373.9129402,219.15720491)
\closepath
\moveto(373.90415113,227.04978272)
\closepath
}
}
{
\newrgbcolor{curcolor}{0 0 0}
\pscustom[linestyle=none,fillstyle=solid,fillcolor=curcolor]
{
\newpath
\moveto(383.49301614,227.0322046)
\curveto(384.16098486,227.0322046)(384.80844577,226.87400148)(385.43539887,226.55759524)
\curveto(386.06235197,226.24704838)(386.53989101,225.84275152)(386.868016,225.34470466)
\curveto(387.18442224,224.87009531)(387.39535973,224.31638439)(387.50082847,223.68357192)
\curveto(387.59457847,223.24997819)(387.64145347,222.55857196)(387.64145347,221.60935325)
\lineto(380.74203968,221.60935325)
\curveto(380.77133656,220.65427517)(380.99692249,219.88669707)(381.41879747,219.30661897)
\curveto(381.84067245,218.73240024)(382.49399274,218.44529088)(383.37875833,218.44529088)
\curveto(384.20493017,218.44529088)(384.86410983,218.71775181)(385.35629731,219.26267366)
\curveto(385.6375473,219.5790799)(385.83676604,219.94529082)(385.95395354,220.36130643)
\lineto(387.50961754,220.36130643)
\curveto(387.46860191,220.01560332)(387.33090661,219.62888458)(387.09653162,219.20115022)
\curveto(386.868016,218.77927524)(386.61020351,218.43357213)(386.32309415,218.16404089)
\curveto(385.84262542,217.69529091)(385.24789888,217.37888467)(384.53891453,217.21482218)
\curveto(384.15805517,217.12107218)(383.72739113,217.07419718)(383.2469224,217.07419718)
\curveto(382.07504744,217.07419718)(381.08188342,217.49900185)(380.26743033,218.3486112)
\curveto(379.45297723,219.20407991)(379.04575069,220.39939236)(379.04575069,221.93454855)
\curveto(379.04575069,223.44626724)(379.45590692,224.67380625)(380.27621939,225.61716559)
\curveto(381.09653186,226.56052493)(382.16879744,227.0322046)(383.49301614,227.0322046)
\closepath
\moveto(386.01547697,222.86618914)
\curveto(385.95102385,223.55173599)(385.80160979,224.09958753)(385.5672348,224.50974376)
\curveto(385.13364107,225.27146248)(384.41000829,225.65232184)(383.39633645,225.65232184)
\curveto(382.66977398,225.65232184)(382.06039901,225.38864998)(381.56821153,224.86130625)
\curveto(381.07602404,224.33982189)(380.81528187,223.67478286)(380.78598499,222.86618914)
\closepath
\moveto(383.34360208,227.04978272)
\closepath
}
}
{
\newrgbcolor{curcolor}{0 0 0}
\pscustom[linestyle=none,fillstyle=solid,fillcolor=curcolor]
{
\newpath
\moveto(389.63657098,226.82126711)
\lineto(391.13950061,226.82126711)
\lineto(391.13950061,225.19529061)
\curveto(391.26254748,225.51169685)(391.56430528,225.89548589)(392.04477401,226.34665775)
\curveto(392.52524274,226.80368898)(393.07895365,227.0322046)(393.70590675,227.0322046)
\curveto(393.73520363,227.0322046)(393.78500831,227.02927491)(393.85532081,227.02341554)
\curveto(393.92563331,227.01755616)(394.04575049,227.00583741)(394.21567236,226.98825929)
\lineto(394.21567236,225.31833748)
\curveto(394.12192236,225.3359156)(394.03403174,225.34763435)(393.95200049,225.35349373)
\curveto(393.87582862,225.3593531)(393.79086769,225.36228279)(393.69711769,225.36228279)
\curveto(392.90024272,225.36228279)(392.28793806,225.1044703)(391.8602037,224.58884532)
\curveto(391.43246934,224.07907972)(391.21860216,223.49021255)(391.21860216,222.82224383)
\lineto(391.21860216,217.40818155)
\lineto(389.63657098,217.40818155)
\closepath
}
}
{
\newrgbcolor{curcolor}{0 0 0}
\pscustom[linestyle=none,fillstyle=solid,fillcolor=curcolor]
{
\newpath
\moveto(400.91293527,229.44919669)
\lineto(402.51254458,229.44919669)
\lineto(402.51254458,226.82126711)
\lineto(404.01547421,226.82126711)
\lineto(404.01547421,225.52927497)
\lineto(402.51254458,225.52927497)
\lineto(402.51254458,219.38572053)
\curveto(402.51254458,219.05759554)(402.6238727,218.83786899)(402.84652894,218.72654087)
\curveto(402.96957581,218.66208775)(403.17465393,218.62986118)(403.46176329,218.62986118)
\lineto(403.70785703,218.62986118)
\curveto(403.79574766,218.63572056)(403.89828671,218.64450962)(404.01547421,218.65622837)
\lineto(404.01547421,217.40818155)
\curveto(403.83383359,217.35544717)(403.64340391,217.31736124)(403.44418517,217.29392374)
\curveto(403.2508258,217.27048624)(403.03988831,217.25876749)(402.8113727,217.25876749)
\curveto(402.07309147,217.25876749)(401.57211493,217.44626748)(401.30844307,217.82126747)
\curveto(401.0447712,218.20212683)(400.91293527,218.69431431)(400.91293527,219.29782991)
\lineto(400.91293527,225.52927497)
\lineto(399.63852126,225.52927497)
\lineto(399.63852126,226.82126711)
\lineto(400.91293527,226.82126711)
\closepath
}
}
{
\newrgbcolor{curcolor}{0 0 0}
\pscustom[linestyle=none,fillstyle=solid,fillcolor=curcolor]
{
\newpath
\moveto(409.32407083,218.42771275)
\curveto(410.37289891,218.42771275)(411.09067232,218.82322055)(411.47739106,219.61423614)
\curveto(411.86996917,220.41111111)(412.06625822,221.2958767)(412.06625822,222.26853291)
\curveto(412.06625822,223.14743913)(411.92563323,223.86228285)(411.64438324,224.41306408)
\curveto(411.19907076,225.28025154)(410.43149266,225.71384528)(409.34164895,225.71384528)
\curveto(408.37485212,225.71384528)(407.67172715,225.34470466)(407.23227404,224.60642344)
\curveto(406.79282093,223.86814222)(406.57309438,222.97751726)(406.57309438,221.93454855)
\curveto(406.57309438,220.93259547)(406.79282093,220.09763456)(407.23227404,219.42966584)
\curveto(407.67172715,218.76169712)(408.36899274,218.42771275)(409.32407083,218.42771275)
\closepath
\moveto(409.38559427,227.09372803)
\curveto(410.59848484,227.09372803)(411.62387543,226.68943117)(412.46176602,225.88083746)
\curveto(413.29965661,225.07224374)(413.7186019,223.88279066)(413.7186019,222.31247822)
\curveto(413.7186019,220.79490016)(413.34946129,219.54099396)(412.61118007,218.55075962)
\curveto(411.87289885,217.56052529)(410.72739109,217.06540812)(409.17465677,217.06540812)
\curveto(407.87973495,217.06540812)(406.85141468,217.50193154)(406.08969596,218.37497838)
\curveto(405.32797724,219.2538846)(404.94711788,220.43161892)(404.94711788,221.90818137)
\curveto(404.94711788,223.49021255)(405.34848505,224.74997813)(406.15121939,225.68747809)
\curveto(406.95395374,226.62497805)(408.03207869,227.09372803)(409.38559427,227.09372803)
\closepath
\moveto(409.33285989,227.04978272)
\closepath
}
}
{
\newrgbcolor{curcolor}{0 0 0}
\pscustom[linestyle=none,fillstyle=solid,fillcolor=curcolor]
{
\newpath
\moveto(419.34360208,218.42771275)
\curveto(420.39243016,218.42771275)(421.11020357,218.82322055)(421.49692231,219.61423614)
\curveto(421.88950042,220.41111111)(422.08578947,221.2958767)(422.08578947,222.26853291)
\curveto(422.08578947,223.14743913)(421.94516448,223.86228285)(421.66391449,224.41306408)
\curveto(421.21860201,225.28025154)(420.45102391,225.71384528)(419.3611802,225.71384528)
\curveto(418.39438337,225.71384528)(417.6912584,225.34470466)(417.25180529,224.60642344)
\curveto(416.81235218,223.86814222)(416.59262563,222.97751726)(416.59262563,221.93454855)
\curveto(416.59262563,220.93259547)(416.81235218,220.09763456)(417.25180529,219.42966584)
\curveto(417.6912584,218.76169712)(418.38852399,218.42771275)(419.34360208,218.42771275)
\closepath
\moveto(419.40512552,227.09372803)
\curveto(420.61801609,227.09372803)(421.64340668,226.68943117)(422.48129727,225.88083746)
\curveto(423.31918786,225.07224374)(423.73813315,223.88279066)(423.73813315,222.31247822)
\curveto(423.73813315,220.79490016)(423.36899254,219.54099396)(422.63071132,218.55075962)
\curveto(421.8924301,217.56052529)(420.74692234,217.06540812)(419.19418802,217.06540812)
\curveto(417.8992662,217.06540812)(416.87094593,217.50193154)(416.10922721,218.37497838)
\curveto(415.34750849,219.2538846)(414.96664913,220.43161892)(414.96664913,221.90818137)
\curveto(414.96664913,223.49021255)(415.3680163,224.74997813)(416.17075064,225.68747809)
\curveto(416.97348499,226.62497805)(418.05160994,227.09372803)(419.40512552,227.09372803)
\closepath
\moveto(419.35239114,227.04978272)
\closepath
}
}
{
\newrgbcolor{curcolor}{0 0 0}
\pscustom[linestyle=none,fillstyle=solid,fillcolor=curcolor]
{
\newpath
\moveto(430.93637854,229.44919669)
\lineto(432.53598785,229.44919669)
\lineto(432.53598785,226.82126711)
\lineto(434.03891748,226.82126711)
\lineto(434.03891748,225.52927497)
\lineto(432.53598785,225.52927497)
\lineto(432.53598785,219.38572053)
\curveto(432.53598785,219.05759554)(432.64731597,218.83786899)(432.86997221,218.72654087)
\curveto(432.99301908,218.66208775)(433.1980972,218.62986118)(433.48520656,218.62986118)
\lineto(433.7313003,218.62986118)
\curveto(433.81919092,218.63572056)(433.92172998,218.64450962)(434.03891748,218.65622837)
\lineto(434.03891748,217.40818155)
\curveto(433.85727686,217.35544717)(433.66684718,217.31736124)(433.46762844,217.29392374)
\curveto(433.27426907,217.27048624)(433.06333158,217.25876749)(432.83481596,217.25876749)
\curveto(432.09653474,217.25876749)(431.5955582,217.44626748)(431.33188633,217.82126747)
\curveto(431.06821447,218.20212683)(430.93637854,218.69431431)(430.93637854,219.29782991)
\lineto(430.93637854,225.52927497)
\lineto(429.66196453,225.52927497)
\lineto(429.66196453,226.82126711)
\lineto(430.93637854,226.82126711)
\closepath
}
}
{
\newrgbcolor{curcolor}{0 0 0}
\pscustom[linestyle=none,fillstyle=solid,fillcolor=curcolor]
{
\newpath
\moveto(435.61215548,230.36325915)
\lineto(437.19418666,230.36325915)
\lineto(437.19418666,225.54685309)
\curveto(437.56918665,226.02146245)(437.9061007,226.35544681)(438.20492881,226.54880618)
\curveto(438.71469441,226.88279054)(439.35043658,227.04978272)(440.1121553,227.04978272)
\curveto(441.47738962,227.04978272)(442.40317083,226.57224368)(442.88949893,225.61716559)
\curveto(443.1531708,225.09568124)(443.28500673,224.37204845)(443.28500673,223.44626724)
\lineto(443.28500673,217.40818155)
\lineto(441.65903023,217.40818155)
\lineto(441.65903023,223.3407985)
\curveto(441.65903023,224.03220472)(441.57113961,224.53904063)(441.39535837,224.86130625)
\curveto(441.10824901,225.37693123)(440.56918653,225.63474372)(439.77817093,225.63474372)
\curveto(439.12192096,225.63474372)(438.52719442,225.40915779)(437.99399132,224.95798593)
\curveto(437.46078821,224.50681407)(437.19418666,223.65427505)(437.19418666,222.40036885)
\lineto(437.19418666,217.40818155)
\lineto(435.61215548,217.40818155)
\closepath
}
}
{
\newrgbcolor{curcolor}{0 0 0}
\pscustom[linestyle=none,fillstyle=solid,fillcolor=curcolor]
{
\newpath
\moveto(445.63169393,226.7773218)
\lineto(447.24009231,226.7773218)
\lineto(447.24009231,217.40818155)
\lineto(445.63169393,217.40818155)
\closepath
\moveto(445.63169393,230.31931384)
\lineto(447.24009231,230.31931384)
\lineto(447.24009231,228.52634516)
\lineto(445.63169393,228.52634516)
\closepath
}
}
{
\newrgbcolor{curcolor}{0 0 0}
\pscustom[linestyle=none,fillstyle=solid,fillcolor=curcolor]
{
\newpath
\moveto(453.26938045,227.09372803)
\curveto(454.32992728,227.09372803)(455.19125537,226.83591554)(455.85336472,226.32029056)
\curveto(456.52133345,225.80466558)(456.92270062,224.91697031)(457.05746624,223.65720473)
\lineto(455.51938036,223.65720473)
\curveto(455.42563036,224.23728283)(455.21176319,224.71775157)(454.87777882,225.09861092)
\curveto(454.54379446,225.48532966)(454.00766167,225.67868903)(453.26938045,225.67868903)
\curveto(452.26156799,225.67868903)(451.5408649,225.18650155)(451.10727116,224.20212659)
\curveto(450.82602117,223.56345474)(450.68539618,222.77536883)(450.68539618,221.83786887)
\curveto(450.68539618,220.89450953)(450.88461492,220.10056425)(451.28305241,219.45603303)
\curveto(451.68148989,218.8115018)(452.30844299,218.48923619)(453.16391171,218.48923619)
\curveto(453.82016168,218.48923619)(454.33871635,218.68845493)(454.71957571,219.08689242)
\curveto(455.10629444,219.49118927)(455.37289599,220.0419705)(455.51938036,220.7392361)
\lineto(457.05746624,220.7392361)
\curveto(456.88168499,219.49118927)(456.44223189,218.57712681)(455.73910691,217.99704871)
\curveto(455.03598194,217.42282998)(454.13656792,217.13572062)(453.04086484,217.13572062)
\curveto(451.81039613,217.13572062)(450.82895086,217.58396279)(450.09652902,218.48044713)
\curveto(449.36410717,219.38279084)(448.99789625,220.5077908)(448.99789625,221.85544699)
\curveto(448.99789625,223.50779068)(449.39926342,224.79392344)(450.20199776,225.71384528)
\curveto(451.0047321,226.63376711)(452.027193,227.09372803)(453.26938045,227.09372803)
\closepath
\moveto(453.02328671,227.04978272)
\closepath
}
}
{
\newrgbcolor{curcolor}{0 0 0}
\pscustom[linestyle=none,fillstyle=solid,fillcolor=curcolor]
{
\newpath
\moveto(458.60435019,230.31931384)
\lineto(460.12485794,230.31931384)
\lineto(460.12485794,222.82224383)
\lineto(464.18540465,226.82126711)
\lineto(466.20688894,226.82126711)
\lineto(462.60337346,223.29685318)
\lineto(466.40903737,217.40818155)
\lineto(464.38755308,217.40818155)
\lineto(461.45200632,222.15427511)
\lineto(460.12485794,220.94138453)
\lineto(460.12485794,217.40818155)
\lineto(458.60435019,217.40818155)
\closepath
}
}
{
\newrgbcolor{curcolor}{1 0 0}
\pscustom[linestyle=none,fillstyle=solid,fillcolor=curcolor]
{
\newpath
\moveto(418.20882848,49.99231394)
\curveto(418.20882848,27.86986211)(400.68528683,9.93607149)(379.06890274,9.93607149)
\curveto(357.45251866,9.93607149)(339.928977,27.86986211)(339.928977,49.99231394)
\curveto(339.928977,72.11476578)(357.45251866,90.0485564)(379.06890274,90.0485564)
\curveto(400.68528683,90.0485564)(418.20882848,72.11476578)(418.20882848,49.99231394)
\closepath
}
}
{
\newrgbcolor{curcolor}{1 0 0}
\pscustom[linestyle=none,fillstyle=solid,fillcolor=curcolor]
{
\newpath
\moveto(380.00000126,89.99996496)
\lineto(419.99999645,89.99996496)
\lineto(419.99999645,9.99996737)
\lineto(380.00000126,9.99996737)
\closepath
}
}
{
\newrgbcolor{curcolor}{1 0 0}
\pscustom[linestyle=none,fillstyle=solid,fillcolor=curcolor]
{
\newpath
\moveto(460.15270449,49.3443801)
\curveto(460.15270449,27.5797822)(442.04896698,9.93608951)(419.71687503,9.93608951)
\curveto(397.38478308,9.93608951)(379.28104557,27.5797822)(379.28104557,49.3443801)
\curveto(379.28104557,71.10897801)(397.38478308,88.7526707)(419.71687503,88.7526707)
\curveto(442.04896698,88.7526707)(460.15270449,71.10897801)(460.15270449,49.3443801)
\closepath
}
}
{
\newrgbcolor{curcolor}{1 0 0}
\pscustom[linestyle=none,fillstyle=solid,fillcolor=curcolor]
{
\newpath
\moveto(69.86273824,29.25791203)
\curveto(69.86273824,17.87108014)(61.04213611,8.64022904)(50.16137197,8.64022904)
\curveto(39.28060783,8.64022904)(30.46000569,17.87108014)(30.46000569,29.25791203)
\curveto(30.46000569,40.64474392)(39.28060783,49.87559502)(50.16137197,49.87559502)
\curveto(61.04213611,49.87559502)(69.86273824,40.64474392)(69.86273824,29.25791203)
\closepath
}
}
{
\newrgbcolor{curcolor}{1 0 0}
\pscustom[linestyle=none,fillstyle=solid,fillcolor=curcolor]
{
\newpath
\moveto(49.9999508,50.69082526)
\lineto(209.99994599,50.69082526)
\lineto(209.99994599,10.00000341)
\lineto(49.9999508,10.00000341)
\closepath
}
}
{
\newrgbcolor{curcolor}{1 0 0}
\pscustom[linestyle=none,fillstyle=solid,fillcolor=curcolor]
{
\newpath
\moveto(229.86288482,29.25791203)
\curveto(229.86288482,17.87108014)(221.04228269,8.64022904)(210.16151855,8.64022904)
\curveto(199.28075441,8.64022904)(190.46015227,17.87108014)(190.46015227,29.25791203)
\curveto(190.46015227,40.64474392)(199.28075441,49.87559502)(210.16151855,49.87559502)
\curveto(221.04228269,49.87559502)(229.86288482,40.64474392)(229.86288482,29.25791203)
\closepath
}
}
{
\newrgbcolor{curcolor}{0 0 1}
\pscustom[linestyle=none,fillstyle=solid,fillcolor=curcolor]
{
\newpath
\moveto(123.06614439,79.53858724)
\curveto(123.06614439,62.42609045)(109.16882767,48.55367085)(92.02561872,48.55367085)
\curveto(74.88240978,48.55367085)(60.98509306,62.42609045)(60.98509306,79.53858724)
\curveto(60.98509306,96.65108403)(74.88240978,110.52350363)(92.02561872,110.52350363)
\curveto(109.16882767,110.52350363)(123.06614439,96.65108403)(123.06614439,79.53858724)
\closepath
}
}
{
\newrgbcolor{curcolor}{0 0 1}
\pscustom[linestyle=none,fillstyle=solid,fillcolor=curcolor]
{
\newpath
\moveto(90.99996858,110.69082526)
\lineto(170.99996618,110.69082526)
\lineto(170.99996618,50.00000581)
\lineto(90.99996858,50.00000581)
\closepath
}
}
{
\newrgbcolor{curcolor}{0 0 1}
\pscustom[linestyle=none,fillstyle=solid,fillcolor=curcolor]
{
\newpath
\moveto(201.06621936,79.53858724)
\curveto(201.06621936,62.42609045)(187.16890264,48.55367085)(170.0256937,48.55367085)
\curveto(152.88248475,48.55367085)(138.98516803,62.42609045)(138.98516803,79.53858724)
\curveto(138.98516803,96.65108403)(152.88248475,110.52350363)(170.0256937,110.52350363)
\curveto(187.16890264,110.52350363)(201.06621936,96.65108403)(201.06621936,79.53858724)
\closepath
}
}
{
\newrgbcolor{curcolor}{0 0 1}
\pscustom[linestyle=none,fillstyle=solid,fillcolor=curcolor]
{
\newpath
\moveto(123.06614439,139.53860236)
\curveto(123.06614439,122.42610557)(109.16882767,108.55368597)(92.02561872,108.55368597)
\curveto(74.88240978,108.55368597)(60.98509306,122.42610557)(60.98509306,139.53860236)
\curveto(60.98509306,156.65109915)(74.88240978,170.52351875)(92.02561872,170.52351875)
\curveto(109.16882767,170.52351875)(123.06614439,156.65109915)(123.06614439,139.53860236)
\closepath
}
}
{
\newrgbcolor{curcolor}{0 0 1}
\pscustom[linestyle=none,fillstyle=solid,fillcolor=curcolor]
{
\newpath
\moveto(90.99996858,170.69084038)
\lineto(170.99996618,170.69084038)
\lineto(170.99996618,110.00002093)
\lineto(90.99996858,110.00002093)
\closepath
}
}
{
\newrgbcolor{curcolor}{0 0 1}
\pscustom[linestyle=none,fillstyle=solid,fillcolor=curcolor]
{
\newpath
\moveto(201.06621936,139.53860236)
\curveto(201.06621936,122.42610557)(187.16890264,108.55368597)(170.0256937,108.55368597)
\curveto(152.88248475,108.55368597)(138.98516803,122.42610557)(138.98516803,139.53860236)
\curveto(138.98516803,156.65109915)(152.88248475,170.52351875)(170.0256937,170.52351875)
\curveto(187.16890264,170.52351875)(201.06621936,156.65109915)(201.06621936,139.53860236)
\closepath
}
}
{
\newrgbcolor{curcolor}{0 0 1}
\pscustom[linestyle=none,fillstyle=solid,fillcolor=curcolor]
{
\newpath
\moveto(391.06629053,119.53858976)
\curveto(391.06629053,102.42609297)(377.16897381,88.55367337)(360.02576487,88.55367337)
\curveto(342.88255592,88.55367337)(328.9852392,102.42609297)(328.9852392,119.53858976)
\curveto(328.9852392,136.65108655)(342.88255592,150.52350615)(360.02576487,150.52350615)
\curveto(377.16897381,150.52350615)(391.06629053,136.65108655)(391.06629053,119.53858976)
\closepath
}
}
{
\newrgbcolor{curcolor}{0 0 1}
\pscustom[linestyle=none,fillstyle=solid,fillcolor=curcolor]
{
\newpath
\moveto(359.00011472,150.69082778)
\lineto(439.00011232,150.69082778)
\lineto(439.00011232,90.00000833)
\lineto(359.00011472,90.00000833)
\closepath
}
}
{
\newrgbcolor{curcolor}{0 0 1}
\pscustom[linestyle=none,fillstyle=solid,fillcolor=curcolor]
{
\newpath
\moveto(469.0663655,119.53858976)
\curveto(469.0663655,102.42609297)(455.16904878,88.55367337)(438.02583984,88.55367337)
\curveto(420.8826309,88.55367337)(406.98531418,102.42609297)(406.98531418,119.53858976)
\curveto(406.98531418,136.65108655)(420.8826309,150.52350615)(438.02583984,150.52350615)
\curveto(455.16904878,150.52350615)(469.0663655,136.65108655)(469.0663655,119.53858976)
\closepath
}
}
{
\newrgbcolor{curcolor}{0 0 1}
\pscustom[linestyle=none,fillstyle=solid,fillcolor=curcolor]
{
\newpath
\moveto(391.06629053,179.53862756)
\curveto(391.06629053,162.42613077)(377.16897381,148.55371117)(360.02576487,148.55371117)
\curveto(342.88255592,148.55371117)(328.9852392,162.42613077)(328.9852392,179.53862756)
\curveto(328.9852392,196.65112435)(342.88255592,210.52354394)(360.02576487,210.52354394)
\curveto(377.16897381,210.52354394)(391.06629053,196.65112435)(391.06629053,179.53862756)
\closepath
}
}
{
\newrgbcolor{curcolor}{0 0 1}
\pscustom[linestyle=none,fillstyle=solid,fillcolor=curcolor]
{
\newpath
\moveto(359.00011472,210.69086558)
\lineto(439.00011232,210.69086558)
\lineto(439.00011232,150.00004613)
\lineto(359.00011472,150.00004613)
\closepath
}
}
{
\newrgbcolor{curcolor}{0 0 1}
\pscustom[linestyle=none,fillstyle=solid,fillcolor=curcolor]
{
\newpath
\moveto(469.0663655,179.53862756)
\curveto(469.0663655,162.42613077)(455.16904878,148.55371117)(438.02583984,148.55371117)
\curveto(420.8826309,148.55371117)(406.98531418,162.42613077)(406.98531418,179.53862756)
\curveto(406.98531418,196.65112435)(420.8826309,210.52354394)(438.02583984,210.52354394)
\curveto(455.16904878,210.52354394)(469.0663655,196.65112435)(469.0663655,179.53862756)
\closepath
}
}
\end{pspicture}

    \caption{First Layer Problems}
  \end{figure}
\end{frame}

\section{Overhangs}
\begin{frame}
  \frametitle{Overhangs}
\end{frame}

\section{Holes and Voids}
\begin{frame}
  \frametitle{Holes and Voids}
  \begin{figure}
    %LaTeX with PSTricks extensions
%%Creator: inkscape 0.91
%%Please note this file requires PSTricks extensions
\psset{xunit=.5pt,yunit=.5pt,runit=.5pt}
\begin{pspicture}(340.01193309,346.67858514)
{
\newrgbcolor{curcolor}{0.50196081 0.50196081 0.50196081}
\pscustom[linestyle=none,fillstyle=solid,fillcolor=curcolor]
{
\newpath
\moveto(0,290.57926201)
\lineto(340.01193237,290.57926201)
\lineto(340.01193237,0.00000786)
\lineto(0,0.00000786)
\closepath
}
}
{
\newrgbcolor{curcolor}{1 1 1}
\pscustom[linestyle=none,fillstyle=solid,fillcolor=curcolor]
{
\newpath
\moveto(269.72229004,135.8629229)
\curveto(269.72229004,88.91871918)(231.66649376,50.8629229)(184.72229004,50.8629229)
\curveto(137.77808632,50.8629229)(99.72229004,88.91871918)(99.72229004,135.8629229)
\curveto(99.72229004,182.80712661)(137.77808632,220.8629229)(184.72229004,220.8629229)
\curveto(231.66649376,220.8629229)(269.72229004,182.80712661)(269.72229004,135.8629229)
\closepath
}
}
{
\newrgbcolor{curcolor}{0 0 0}
\pscustom[linewidth=2.5,linecolor=curcolor]
{
\newpath
\moveto(269.72229004,135.8629229)
\curveto(269.72229004,88.91871918)(231.66649376,50.8629229)(184.72229004,50.8629229)
\curveto(137.77808632,50.8629229)(99.72229004,88.91871918)(99.72229004,135.8629229)
\curveto(99.72229004,182.80712661)(137.77808632,220.8629229)(184.72229004,220.8629229)
\curveto(231.66649376,220.8629229)(269.72229004,182.80712661)(269.72229004,135.8629229)
\closepath
}
}
{
\newrgbcolor{curcolor}{1 1 1}
\pscustom[linestyle=none,fillstyle=solid,fillcolor=curcolor]
{
\newpath
\moveto(257.34315061,93.74599668)
\lineto(184.33911529,50.27747053)
\lineto(111.95017831,91.88561141)
\lineto(112.56527666,176.96227845)
\lineto(185.56931198,220.43080461)
\lineto(257.95824896,178.82266373)
\closepath
}
}
{
\newrgbcolor{curcolor}{0 0 0}
\pscustom[linewidth=2.74393634,linecolor=curcolor,linestyle=dashed,dash=19.33600873 6.44533624]
{
\newpath
\moveto(257.34315061,93.74599668)
\lineto(184.33911529,50.27747053)
\lineto(111.95017831,91.88561141)
\lineto(112.56527666,176.96227845)
\lineto(185.56931198,220.43080461)
\lineto(257.95824896,178.82266373)
\closepath
}
}
{
\newrgbcolor{curcolor}{0 0 0}
\pscustom[linewidth=2.5,linecolor=curcolor]
{
\newpath
\moveto(0,340.57927514)
\lineto(30,340.57927514)
}
}
{
\newrgbcolor{curcolor}{0 0 0}
\pscustom[linestyle=none,fillstyle=solid,fillcolor=curcolor]
{
\newpath
\moveto(46.328125,335.21764397)
\curveto(46.91992188,335.21764397)(47.40625,335.2791674)(47.78710938,335.40221428)
\curveto(48.46679688,335.6307299)(49.0234375,336.07018303)(49.45703125,336.72057365)
\curveto(49.80273438,337.24205803)(50.05175781,337.91002678)(50.20410156,338.7244799)
\curveto(50.29199219,339.21080803)(50.3359375,339.6619799)(50.3359375,340.07799553)
\curveto(50.3359375,341.6776049)(50.01660156,342.9197924)(49.37792969,343.80455803)
\curveto(48.74511719,344.68932365)(47.72265625,345.13170647)(46.31054688,345.13170647)
\lineto(43.20800781,345.13170647)
\lineto(43.20800781,335.21764397)
\lineto(46.328125,335.21764397)
\closepath
\moveto(41.45019531,346.63463615)
\lineto(46.6796875,346.63463615)
\curveto(48.45507812,346.63463615)(49.83203125,346.00475334)(50.81054688,344.74498772)
\curveto(51.68359375,343.60826897)(52.12011719,342.15221428)(52.12011719,340.37682365)
\curveto(52.12011719,339.0057299)(51.86230469,337.76647209)(51.34667969,336.65905022)
\curveto(50.43847656,334.70201897)(48.87695312,333.72350334)(46.66210938,333.72350334)
\lineto(41.45019531,333.72350334)
\lineto(41.45019531,346.63463615)
\closepath
}
}
{
\newrgbcolor{curcolor}{0 0 0}
\pscustom[linestyle=none,fillstyle=solid,fillcolor=curcolor]
{
\newpath
\moveto(58.08789062,343.34752678)
\curveto(58.75585938,343.34752678)(59.40332031,343.18932365)(60.03027344,342.8729174)
\curveto(60.65722656,342.56237053)(61.13476562,342.15807365)(61.46289062,341.66002678)
\curveto(61.77929688,341.1854174)(61.99023438,340.63170647)(62.09570312,339.99889397)
\curveto(62.18945312,339.56530022)(62.23632812,338.87389397)(62.23632812,337.92467522)
\lineto(55.33691406,337.92467522)
\curveto(55.36621094,336.96959709)(55.59179688,336.20201897)(56.01367188,335.62194084)
\curveto(56.43554688,335.04772209)(57.08886719,334.76061272)(57.97363281,334.76061272)
\curveto(58.79980469,334.76061272)(59.45898438,335.03307365)(59.95117188,335.57799553)
\curveto(60.23242188,335.89440178)(60.43164062,336.26061272)(60.54882812,336.67662834)
\lineto(62.10449219,336.67662834)
\curveto(62.06347656,336.33092522)(61.92578125,335.94420647)(61.69140625,335.51647209)
\curveto(61.46289062,335.09459709)(61.20507812,334.74889397)(60.91796875,334.47936272)
\curveto(60.4375,334.01061272)(59.84277344,333.69420647)(59.13378906,333.53014397)
\curveto(58.75292969,333.43639397)(58.32226562,333.38951897)(57.84179688,333.38951897)
\curveto(56.66992188,333.38951897)(55.67675781,333.81432365)(54.86230469,334.66393303)
\curveto(54.04785156,335.51940178)(53.640625,336.71471428)(53.640625,338.24987053)
\curveto(53.640625,339.76158928)(54.05078125,340.98912834)(54.87109375,341.93248772)
\curveto(55.69140625,342.87584709)(56.76367188,343.34752678)(58.08789062,343.34752678)
\closepath
\moveto(60.61035156,339.18151115)
\curveto(60.54589844,339.86705803)(60.39648438,340.41490959)(60.16210938,340.82506584)
\curveto(59.72851562,341.58678459)(59.00488281,341.96764397)(57.99121094,341.96764397)
\curveto(57.26464844,341.96764397)(56.65527344,341.70397209)(56.16308594,341.17662834)
\curveto(55.67089844,340.65514397)(55.41015625,339.9901049)(55.38085938,339.18151115)
\lineto(60.61035156,339.18151115)
\closepath
\moveto(57.93847656,343.3651049)
\lineto(57.93847656,343.3651049)
\closepath
}
}
{
\newrgbcolor{curcolor}{0 0 0}
\pscustom[linestyle=none,fillstyle=solid,fillcolor=curcolor]
{
\newpath
\moveto(65.12792969,336.67662834)
\curveto(65.17480469,336.14928459)(65.30664062,335.74498772)(65.5234375,335.46373772)
\curveto(65.921875,334.95397209)(66.61328125,334.69908928)(67.59765625,334.69908928)
\curveto(68.18359375,334.69908928)(68.69921875,334.82506584)(69.14453125,335.07701897)
\curveto(69.58984375,335.33483147)(69.8125,335.73033928)(69.8125,336.2635424)
\curveto(69.8125,336.66783928)(69.63378906,336.97545647)(69.27636719,337.18639397)
\curveto(69.04785156,337.31530022)(68.59667969,337.46471428)(67.92285156,337.63463615)
\lineto(66.66601562,337.9510424)
\curveto(65.86328125,338.15026115)(65.27148438,338.3729174)(64.890625,338.61901115)
\curveto(64.2109375,339.04674553)(63.87109375,339.6385424)(63.87109375,340.39440178)
\curveto(63.87109375,341.28502678)(64.19042969,342.0057299)(64.82910156,342.55651115)
\curveto(65.47363281,343.1072924)(66.33789062,343.38268303)(67.421875,343.38268303)
\curveto(68.83984375,343.38268303)(69.86230469,342.9666674)(70.48925781,342.13463615)
\curveto(70.88183594,341.6072924)(71.07226562,341.03893303)(71.06054688,340.42955803)
\lineto(69.56640625,340.42955803)
\curveto(69.53710938,340.7869799)(69.41113281,341.11217522)(69.18847656,341.40514397)
\curveto(68.82519531,341.82115959)(68.1953125,342.0291674)(67.29882812,342.0291674)
\curveto(66.70117188,342.0291674)(66.24707031,341.91490959)(65.93652344,341.68639397)
\curveto(65.63183594,341.45787834)(65.47949219,341.15612053)(65.47949219,340.78112053)
\curveto(65.47949219,340.37096428)(65.68164062,340.04283928)(66.0859375,339.79674553)
\curveto(66.3203125,339.65026115)(66.66601562,339.5213549)(67.12304688,339.41002678)
\lineto(68.16894531,339.15514397)
\curveto(69.30566406,338.87975334)(70.06738281,338.61315178)(70.45410156,338.35533928)
\curveto(71.06933594,337.9510424)(71.37695312,337.31530022)(71.37695312,336.44811272)
\curveto(71.37695312,335.61022209)(71.05761719,334.88658928)(70.41894531,334.27721428)
\curveto(69.78613281,333.66783928)(68.81933594,333.36315178)(67.51855469,333.36315178)
\curveto(66.11816406,333.36315178)(65.125,333.67955803)(64.5390625,334.31237053)
\curveto(63.95898438,334.9510424)(63.6484375,335.73912834)(63.60742188,336.67662834)
\lineto(65.12792969,336.67662834)
\closepath
\moveto(67.46582031,343.3651049)
\lineto(67.46582031,343.3651049)
\closepath
}
}
{
\newrgbcolor{curcolor}{0 0 0}
\pscustom[linestyle=none,fillstyle=solid,fillcolor=curcolor]
{
\newpath
\moveto(73.22265625,343.09264397)
\lineto(74.83105469,343.09264397)
\lineto(74.83105469,333.72350334)
\lineto(73.22265625,333.72350334)
\lineto(73.22265625,343.09264397)
\closepath
\moveto(73.22265625,346.63463615)
\lineto(74.83105469,346.63463615)
\lineto(74.83105469,344.8416674)
\lineto(73.22265625,344.8416674)
\lineto(73.22265625,346.63463615)
\closepath
}
}
{
\newrgbcolor{curcolor}{0 0 0}
\pscustom[linestyle=none,fillstyle=solid,fillcolor=curcolor]
{
\newpath
\moveto(77.27441406,343.13658928)
\lineto(78.77734375,343.13658928)
\lineto(78.77734375,341.51061272)
\curveto(78.90039062,341.82701897)(79.20214844,342.21080803)(79.68261719,342.6619799)
\curveto(80.16308594,343.11901115)(80.71679688,343.34752678)(81.34375,343.34752678)
\curveto(81.37304688,343.34752678)(81.42285156,343.34459709)(81.49316406,343.33873772)
\curveto(81.56347656,343.33287834)(81.68359375,343.32115959)(81.85351562,343.30358147)
\lineto(81.85351562,341.63365959)
\curveto(81.75976562,341.65123772)(81.671875,341.66295647)(81.58984375,341.66881584)
\curveto(81.51367188,341.67467522)(81.42871094,341.6776049)(81.33496094,341.6776049)
\curveto(80.53808594,341.6776049)(79.92578125,341.4197924)(79.49804688,340.9041674)
\curveto(79.0703125,340.39440178)(78.85644531,339.80553459)(78.85644531,339.13756584)
\lineto(78.85644531,333.72350334)
\lineto(77.27441406,333.72350334)
\lineto(77.27441406,343.13658928)
\closepath
}
}
{
\newrgbcolor{curcolor}{0 0 0}
\pscustom[linestyle=none,fillstyle=solid,fillcolor=curcolor]
{
\newpath
\moveto(87.16210938,343.34752678)
\curveto(87.83007812,343.34752678)(88.47753906,343.18932365)(89.10449219,342.8729174)
\curveto(89.73144531,342.56237053)(90.20898438,342.15807365)(90.53710938,341.66002678)
\curveto(90.85351562,341.1854174)(91.06445312,340.63170647)(91.16992188,339.99889397)
\curveto(91.26367188,339.56530022)(91.31054688,338.87389397)(91.31054688,337.92467522)
\lineto(84.41113281,337.92467522)
\curveto(84.44042969,336.96959709)(84.66601562,336.20201897)(85.08789062,335.62194084)
\curveto(85.50976562,335.04772209)(86.16308594,334.76061272)(87.04785156,334.76061272)
\curveto(87.87402344,334.76061272)(88.53320312,335.03307365)(89.02539062,335.57799553)
\curveto(89.30664062,335.89440178)(89.50585938,336.26061272)(89.62304688,336.67662834)
\lineto(91.17871094,336.67662834)
\curveto(91.13769531,336.33092522)(91,335.94420647)(90.765625,335.51647209)
\curveto(90.53710938,335.09459709)(90.27929688,334.74889397)(89.9921875,334.47936272)
\curveto(89.51171875,334.01061272)(88.91699219,333.69420647)(88.20800781,333.53014397)
\curveto(87.82714844,333.43639397)(87.39648438,333.38951897)(86.91601562,333.38951897)
\curveto(85.74414062,333.38951897)(84.75097656,333.81432365)(83.93652344,334.66393303)
\curveto(83.12207031,335.51940178)(82.71484375,336.71471428)(82.71484375,338.24987053)
\curveto(82.71484375,339.76158928)(83.125,340.98912834)(83.9453125,341.93248772)
\curveto(84.765625,342.87584709)(85.83789062,343.34752678)(87.16210938,343.34752678)
\closepath
\moveto(89.68457031,339.18151115)
\curveto(89.62011719,339.86705803)(89.47070312,340.41490959)(89.23632812,340.82506584)
\curveto(88.80273438,341.58678459)(88.07910156,341.96764397)(87.06542969,341.96764397)
\curveto(86.33886719,341.96764397)(85.72949219,341.70397209)(85.23730469,341.17662834)
\curveto(84.74511719,340.65514397)(84.484375,339.9901049)(84.45507812,339.18151115)
\lineto(89.68457031,339.18151115)
\closepath
\moveto(87.01269531,343.3651049)
\lineto(87.01269531,343.3651049)
\closepath
}
}
{
\newrgbcolor{curcolor}{0 0 0}
\pscustom[linestyle=none,fillstyle=solid,fillcolor=curcolor]
{
\newpath
\moveto(94.26367188,338.32018303)
\curveto(94.26367188,337.31237053)(94.47753906,336.46862053)(94.90527344,335.78893303)
\curveto(95.33300781,335.10924553)(96.01855469,334.76940178)(96.96191406,334.76940178)
\curveto(97.69433594,334.76940178)(98.29492188,335.08287834)(98.76367188,335.70983147)
\curveto(99.23828125,336.34264397)(99.47558594,337.2479174)(99.47558594,338.42565178)
\curveto(99.47558594,339.6151049)(99.23242188,340.49401115)(98.74609375,341.06237053)
\curveto(98.25976562,341.63658928)(97.65917969,341.92369865)(96.94433594,341.92369865)
\curveto(96.14746094,341.92369865)(95.5,341.61901115)(95.00195312,341.00963615)
\curveto(94.50976562,340.40026115)(94.26367188,339.50377678)(94.26367188,338.32018303)
\closepath
\moveto(96.64550781,343.30358147)
\curveto(97.36621094,343.30358147)(97.96972656,343.15123772)(98.45605469,342.84655022)
\curveto(98.73730469,342.67076897)(99.05664062,342.36315178)(99.4140625,341.92369865)
\lineto(99.4140625,346.67858147)
\lineto(100.93457031,346.67858147)
\lineto(100.93457031,333.72350334)
\lineto(99.51074219,333.72350334)
\lineto(99.51074219,335.03307365)
\curveto(99.14160156,334.45299553)(98.70507812,334.03405022)(98.20117188,333.77623772)
\curveto(97.69726562,333.51842522)(97.12011719,333.38951897)(96.46972656,333.38951897)
\curveto(95.42089844,333.38951897)(94.51269531,333.82897209)(93.74511719,334.70787834)
\curveto(92.97753906,335.59264397)(92.59375,336.76744865)(92.59375,338.2322924)
\curveto(92.59375,339.60338615)(92.94238281,340.78990959)(93.63964844,341.79186272)
\curveto(94.34277344,342.79967522)(95.34472656,343.30358147)(96.64550781,343.30358147)
\closepath
}
}
{
\newrgbcolor{curcolor}{0 0 0}
\pscustom[linestyle=none,fillstyle=solid,fillcolor=curcolor]
{
\newpath
\moveto(108.2734375,346.67858147)
\lineto(109.85546875,346.67858147)
\lineto(109.85546875,341.86217522)
\curveto(110.23046875,342.33678459)(110.56738281,342.67076897)(110.86621094,342.86412834)
\curveto(111.37597656,343.19811272)(112.01171875,343.3651049)(112.7734375,343.3651049)
\curveto(114.13867188,343.3651049)(115.06445312,342.88756584)(115.55078125,341.93248772)
\curveto(115.81445312,341.41100334)(115.94628906,340.68737053)(115.94628906,339.76158928)
\lineto(115.94628906,333.72350334)
\lineto(114.3203125,333.72350334)
\lineto(114.3203125,339.65612053)
\curveto(114.3203125,340.34752678)(114.23242188,340.85436272)(114.05664062,341.17662834)
\curveto(113.76953125,341.69225334)(113.23046875,341.95006584)(112.43945312,341.95006584)
\curveto(111.78320312,341.95006584)(111.18847656,341.7244799)(110.65527344,341.27330803)
\curveto(110.12207031,340.82213615)(109.85546875,339.96959709)(109.85546875,338.71569084)
\lineto(109.85546875,333.72350334)
\lineto(108.2734375,333.72350334)
\lineto(108.2734375,346.67858147)
\closepath
}
}
{
\newrgbcolor{curcolor}{0 0 0}
\pscustom[linestyle=none,fillstyle=solid,fillcolor=curcolor]
{
\newpath
\moveto(122.02832031,334.74303459)
\curveto(123.07714844,334.74303459)(123.79492188,335.1385424)(124.18164062,335.92955803)
\curveto(124.57421875,336.72643303)(124.77050781,337.61119865)(124.77050781,338.5838549)
\curveto(124.77050781,339.46276115)(124.62988281,340.1776049)(124.34863281,340.72838615)
\curveto(123.90332031,341.59557365)(123.13574219,342.0291674)(122.04589844,342.0291674)
\curveto(121.07910156,342.0291674)(120.37597656,341.66002678)(119.93652344,340.92174553)
\curveto(119.49707031,340.18346428)(119.27734375,339.29283928)(119.27734375,338.24987053)
\curveto(119.27734375,337.2479174)(119.49707031,336.41295647)(119.93652344,335.74498772)
\curveto(120.37597656,335.07701897)(121.07324219,334.74303459)(122.02832031,334.74303459)
\closepath
\moveto(122.08984375,343.40905022)
\curveto(123.30273438,343.40905022)(124.328125,343.00475334)(125.16601562,342.19615959)
\curveto(126.00390625,341.38756584)(126.42285156,340.19811272)(126.42285156,338.62780022)
\curveto(126.42285156,337.11022209)(126.05371094,335.85631584)(125.31542969,334.86608147)
\curveto(124.57714844,333.87584709)(123.43164062,333.3807299)(121.87890625,333.3807299)
\curveto(120.58398438,333.3807299)(119.55566406,333.81725334)(118.79394531,334.69030022)
\curveto(118.03222656,335.56920647)(117.65136719,336.74694084)(117.65136719,338.22350334)
\curveto(117.65136719,339.80553459)(118.05273438,341.06530022)(118.85546875,342.00280022)
\curveto(119.65820312,342.94030022)(120.73632812,343.40905022)(122.08984375,343.40905022)
\closepath
\moveto(122.03710938,343.3651049)
\lineto(122.03710938,343.3651049)
\closepath
}
}
{
\newrgbcolor{curcolor}{0 0 0}
\pscustom[linestyle=none,fillstyle=solid,fillcolor=curcolor]
{
\newpath
\moveto(128.35644531,346.63463615)
\lineto(129.93847656,346.63463615)
\lineto(129.93847656,333.72350334)
\lineto(128.35644531,333.72350334)
\lineto(128.35644531,346.63463615)
\closepath
}
}
{
\newrgbcolor{curcolor}{0 0 0}
\pscustom[linestyle=none,fillstyle=solid,fillcolor=curcolor]
{
\newpath
\moveto(136.24023438,343.34752678)
\curveto(136.90820312,343.34752678)(137.55566406,343.18932365)(138.18261719,342.8729174)
\curveto(138.80957031,342.56237053)(139.28710938,342.15807365)(139.61523438,341.66002678)
\curveto(139.93164062,341.1854174)(140.14257812,340.63170647)(140.24804688,339.99889397)
\curveto(140.34179688,339.56530022)(140.38867188,338.87389397)(140.38867188,337.92467522)
\lineto(133.48925781,337.92467522)
\curveto(133.51855469,336.96959709)(133.74414062,336.20201897)(134.16601562,335.62194084)
\curveto(134.58789062,335.04772209)(135.24121094,334.76061272)(136.12597656,334.76061272)
\curveto(136.95214844,334.76061272)(137.61132812,335.03307365)(138.10351562,335.57799553)
\curveto(138.38476562,335.89440178)(138.58398438,336.26061272)(138.70117188,336.67662834)
\lineto(140.25683594,336.67662834)
\curveto(140.21582031,336.33092522)(140.078125,335.94420647)(139.84375,335.51647209)
\curveto(139.61523438,335.09459709)(139.35742188,334.74889397)(139.0703125,334.47936272)
\curveto(138.58984375,334.01061272)(137.99511719,333.69420647)(137.28613281,333.53014397)
\curveto(136.90527344,333.43639397)(136.47460938,333.38951897)(135.99414062,333.38951897)
\curveto(134.82226562,333.38951897)(133.82910156,333.81432365)(133.01464844,334.66393303)
\curveto(132.20019531,335.51940178)(131.79296875,336.71471428)(131.79296875,338.24987053)
\curveto(131.79296875,339.76158928)(132.203125,340.98912834)(133.0234375,341.93248772)
\curveto(133.84375,342.87584709)(134.91601562,343.34752678)(136.24023438,343.34752678)
\closepath
\moveto(138.76269531,339.18151115)
\curveto(138.69824219,339.86705803)(138.54882812,340.41490959)(138.31445312,340.82506584)
\curveto(137.88085938,341.58678459)(137.15722656,341.96764397)(136.14355469,341.96764397)
\curveto(135.41699219,341.96764397)(134.80761719,341.70397209)(134.31542969,341.17662834)
\curveto(133.82324219,340.65514397)(133.5625,339.9901049)(133.53320312,339.18151115)
\lineto(138.76269531,339.18151115)
\closepath
\moveto(136.09082031,343.3651049)
\lineto(136.09082031,343.3651049)
\closepath
}
}
{
\newrgbcolor{curcolor}{0 0 0}
\pscustom[linewidth=2.5,linecolor=curcolor,linestyle=dashed,dash=7.5 2.5]
{
\newpath
\moveto(0,310.57927514)
\lineto(30,310.57927514)
}
}
{
\newrgbcolor{curcolor}{0 0 0}
\pscustom[linestyle=none,fillstyle=solid,fillcolor=curcolor]
{
\newpath
\moveto(47.99804688,310.72846245)
\lineto(46.03808594,316.43256401)
\lineto(43.95507812,310.72846245)
\lineto(47.99804688,310.72846245)
\closepath
\moveto(45.12402344,318.34857963)
\lineto(47.1015625,318.34857963)
\lineto(51.78613281,305.43744682)
\lineto(49.87011719,305.43744682)
\lineto(48.56054688,309.30463432)
\lineto(43.45410156,309.30463432)
\lineto(42.05664062,305.43744682)
\lineto(40.26367188,305.43744682)
\lineto(45.12402344,318.34857963)
\closepath
\moveto(46.02929688,318.34857963)
\lineto(46.02929688,318.34857963)
\closepath
}
}
{
\newrgbcolor{curcolor}{0 0 0}
\pscustom[linestyle=none,fillstyle=solid,fillcolor=curcolor]
{
\newpath
\moveto(56.81347656,315.1229937)
\curveto(57.87402344,315.1229937)(58.73535156,314.8651812)(59.39746094,314.3495562)
\curveto(60.06542969,313.8339312)(60.46679688,312.94623588)(60.6015625,311.68647026)
\lineto(59.06347656,311.68647026)
\curveto(58.96972656,312.26654838)(58.75585938,312.74701713)(58.421875,313.12787651)
\curveto(58.08789062,313.51459526)(57.55175781,313.70795463)(56.81347656,313.70795463)
\curveto(55.80566406,313.70795463)(55.08496094,313.21576713)(54.65136719,312.23139213)
\curveto(54.37011719,311.59272026)(54.22949219,310.80463432)(54.22949219,309.86713432)
\curveto(54.22949219,308.92377495)(54.42871094,308.12982963)(54.82714844,307.48529838)
\curveto(55.22558594,306.84076713)(55.85253906,306.51850151)(56.70800781,306.51850151)
\curveto(57.36425781,306.51850151)(57.8828125,306.71772026)(58.26367188,307.11615776)
\curveto(58.65039062,307.52045463)(58.91699219,308.07123588)(59.06347656,308.76850151)
\lineto(60.6015625,308.76850151)
\curveto(60.42578125,307.52045463)(59.98632812,306.60639213)(59.28320312,306.02631401)
\curveto(58.58007812,305.45209526)(57.68066406,305.16498588)(56.58496094,305.16498588)
\curveto(55.35449219,305.16498588)(54.37304688,305.61322807)(53.640625,306.50971245)
\curveto(52.90820312,307.4120562)(52.54199219,308.5370562)(52.54199219,309.88471245)
\curveto(52.54199219,311.5370562)(52.94335938,312.82318901)(53.74609375,313.74311088)
\curveto(54.54882812,314.66303276)(55.57128906,315.1229937)(56.81347656,315.1229937)
\closepath
\moveto(56.56738281,315.07904838)
\lineto(56.56738281,315.07904838)
\closepath
}
}
{
\newrgbcolor{curcolor}{0 0 0}
\pscustom[linestyle=none,fillstyle=solid,fillcolor=curcolor]
{
\newpath
\moveto(62.5,317.47846245)
\lineto(64.09960938,317.47846245)
\lineto(64.09960938,314.85053276)
\lineto(65.60253906,314.85053276)
\lineto(65.60253906,313.55854057)
\lineto(64.09960938,313.55854057)
\lineto(64.09960938,307.41498588)
\curveto(64.09960938,307.08686088)(64.2109375,306.86713432)(64.43359375,306.7558062)
\curveto(64.55664062,306.69135307)(64.76171875,306.65912651)(65.04882812,306.65912651)
\lineto(65.29492188,306.65912651)
\curveto(65.3828125,306.66498588)(65.48535156,306.67377495)(65.60253906,306.6854937)
\lineto(65.60253906,305.43744682)
\curveto(65.42089844,305.38471245)(65.23046875,305.34662651)(65.03125,305.32318901)
\curveto(64.83789062,305.29975151)(64.62695312,305.28803276)(64.3984375,305.28803276)
\curveto(63.66015625,305.28803276)(63.15917969,305.47553276)(62.89550781,305.85053276)
\curveto(62.63183594,306.23139213)(62.5,306.72357963)(62.5,307.32709526)
\lineto(62.5,313.55854057)
\lineto(61.22558594,313.55854057)
\lineto(61.22558594,314.85053276)
\lineto(62.5,314.85053276)
\lineto(62.5,317.47846245)
\closepath
}
}
{
\newrgbcolor{curcolor}{0 0 0}
\pscustom[linestyle=none,fillstyle=solid,fillcolor=curcolor]
{
\newpath
\moveto(68.79296875,314.85053276)
\lineto(68.79296875,308.60150932)
\curveto(68.79296875,308.12104057)(68.86914062,307.72846245)(69.02148438,307.42377495)
\curveto(69.30273438,306.86127495)(69.82714844,306.58002495)(70.59472656,306.58002495)
\curveto(71.69628906,306.58002495)(72.44628906,307.07221245)(72.84472656,308.05658745)
\curveto(73.06152344,308.5839312)(73.16992188,309.30756401)(73.16992188,310.22748588)
\lineto(73.16992188,314.85053276)
\lineto(74.75195312,314.85053276)
\lineto(74.75195312,305.43744682)
\lineto(73.2578125,305.43744682)
\lineto(73.27539062,306.8261187)
\curveto(73.0703125,306.46869682)(72.81542969,306.16693901)(72.51074219,305.92084526)
\curveto(71.90722656,305.42865776)(71.17480469,305.18256401)(70.31347656,305.18256401)
\curveto(68.97167969,305.18256401)(68.05761719,305.6308062)(67.57128906,306.52729057)
\curveto(67.30761719,307.00775932)(67.17578125,307.64936088)(67.17578125,308.45209526)
\lineto(67.17578125,314.85053276)
\lineto(68.79296875,314.85053276)
\closepath
\moveto(70.96386719,315.07904838)
\lineto(70.96386719,315.07904838)
\closepath
}
}
{
\newrgbcolor{curcolor}{0 0 0}
\pscustom[linestyle=none,fillstyle=solid,fillcolor=curcolor]
{
\newpath
\moveto(78.44335938,307.94232963)
\curveto(78.44335938,307.48529838)(78.61035156,307.12494682)(78.94433594,306.86127495)
\curveto(79.27832031,306.59760307)(79.67382812,306.46576713)(80.13085938,306.46576713)
\curveto(80.6875,306.46576713)(81.2265625,306.59467338)(81.74804688,306.85248588)
\curveto(82.62695312,307.28022026)(83.06640625,307.98041557)(83.06640625,308.95307182)
\lineto(83.06640625,310.22748588)
\curveto(82.87304688,310.10443901)(82.62402344,310.00189995)(82.31933594,309.9198687)
\curveto(82.01464844,309.83783745)(81.71582031,309.7792437)(81.42285156,309.74408745)
\lineto(80.46484375,309.62104057)
\curveto(79.890625,309.5448687)(79.45996094,309.42475151)(79.17285156,309.26068901)
\curveto(78.68652344,308.98529838)(78.44335938,308.54584526)(78.44335938,307.94232963)
\closepath
\moveto(82.27539062,311.14154838)
\curveto(82.63867188,311.18842338)(82.88183594,311.34076713)(83.00488281,311.59857963)
\curveto(83.07519531,311.73920463)(83.11035156,311.94135307)(83.11035156,312.20502495)
\curveto(83.11035156,312.74408745)(82.91699219,313.13373588)(82.53027344,313.37397026)
\curveto(82.14941406,313.62006401)(81.6015625,313.74311088)(80.88671875,313.74311088)
\curveto(80.06054688,313.74311088)(79.47460938,313.52045463)(79.12890625,313.07514213)
\curveto(78.93554688,312.82904838)(78.80957031,312.46283745)(78.75097656,311.97650932)
\lineto(77.27441406,311.97650932)
\curveto(77.30371094,313.13666557)(77.67871094,313.94232963)(78.39941406,314.39350151)
\curveto(79.12597656,314.85053276)(79.96679688,315.07904838)(80.921875,315.07904838)
\curveto(82.02929688,315.07904838)(82.92871094,314.86811088)(83.62011719,314.44623588)
\curveto(84.30566406,314.02436088)(84.6484375,313.36811088)(84.6484375,312.47748588)
\lineto(84.6484375,307.05463432)
\curveto(84.6484375,306.89057182)(84.68066406,306.75873588)(84.74511719,306.65912651)
\curveto(84.81542969,306.55951713)(84.95898438,306.50971245)(85.17578125,306.50971245)
\curveto(85.24609375,306.50971245)(85.32519531,306.51264213)(85.41308594,306.51850151)
\curveto(85.50097656,306.53022026)(85.59472656,306.5448687)(85.69433594,306.56244682)
\lineto(85.69433594,305.39350151)
\curveto(85.44824219,305.32318901)(85.26074219,305.2792437)(85.13183594,305.26166557)
\curveto(85.00292969,305.24408745)(84.82714844,305.23529838)(84.60449219,305.23529838)
\curveto(84.05957031,305.23529838)(83.6640625,305.42865776)(83.41796875,305.81537651)
\curveto(83.2890625,306.02045463)(83.19824219,306.3104937)(83.14550781,306.6854937)
\curveto(82.82324219,306.2636187)(82.36035156,305.89740776)(81.75683594,305.58686088)
\curveto(81.15332031,305.27631401)(80.48828125,305.12104057)(79.76171875,305.12104057)
\curveto(78.88867188,305.12104057)(78.17382812,305.38471245)(77.6171875,305.9120562)
\curveto(77.06640625,306.44525932)(76.79101562,307.11029838)(76.79101562,307.90717338)
\curveto(76.79101562,308.78022026)(77.06347656,309.45697807)(77.60839844,309.93744682)
\curveto(78.15332031,310.41791557)(78.86816406,310.71381401)(79.75292969,310.82514213)
\lineto(82.27539062,311.14154838)
\closepath
\moveto(80.96582031,315.07904838)
\lineto(80.96582031,315.07904838)
\closepath
}
}
{
\newrgbcolor{curcolor}{0 0 0}
\pscustom[linestyle=none,fillstyle=solid,fillcolor=curcolor]
{
\newpath
\moveto(87.29394531,318.34857963)
\lineto(88.87597656,318.34857963)
\lineto(88.87597656,305.43744682)
\lineto(87.29394531,305.43744682)
\lineto(87.29394531,318.34857963)
\closepath
}
}
{
\newrgbcolor{curcolor}{0 0 0}
\pscustom[linestyle=none,fillstyle=solid,fillcolor=curcolor]
{
\newpath
\moveto(96.25,318.39252495)
\lineto(97.83203125,318.39252495)
\lineto(97.83203125,313.5761187)
\curveto(98.20703125,314.05072807)(98.54394531,314.38471245)(98.84277344,314.57807182)
\curveto(99.35253906,314.9120562)(99.98828125,315.07904838)(100.75,315.07904838)
\curveto(102.11523438,315.07904838)(103.04101562,314.60150932)(103.52734375,313.6464312)
\curveto(103.79101562,313.12494682)(103.92285156,312.40131401)(103.92285156,311.47553276)
\lineto(103.92285156,305.43744682)
\lineto(102.296875,305.43744682)
\lineto(102.296875,311.37006401)
\curveto(102.296875,312.06147026)(102.20898438,312.5683062)(102.03320312,312.89057182)
\curveto(101.74609375,313.40619682)(101.20703125,313.66400932)(100.41601562,313.66400932)
\curveto(99.75976562,313.66400932)(99.16503906,313.43842338)(98.63183594,312.98725151)
\curveto(98.09863281,312.53607963)(97.83203125,311.68354057)(97.83203125,310.42963432)
\lineto(97.83203125,305.43744682)
\lineto(96.25,305.43744682)
\lineto(96.25,318.39252495)
\closepath
}
}
{
\newrgbcolor{curcolor}{0 0 0}
\pscustom[linestyle=none,fillstyle=solid,fillcolor=curcolor]
{
\newpath
\moveto(110.00488281,306.45697807)
\curveto(111.05371094,306.45697807)(111.77148438,306.85248588)(112.15820312,307.64350151)
\curveto(112.55078125,308.44037651)(112.74707031,309.32514213)(112.74707031,310.29779838)
\curveto(112.74707031,311.17670463)(112.60644531,311.89154838)(112.32519531,312.44232963)
\curveto(111.87988281,313.30951713)(111.11230469,313.74311088)(110.02246094,313.74311088)
\curveto(109.05566406,313.74311088)(108.35253906,313.37397026)(107.91308594,312.63568901)
\curveto(107.47363281,311.89740776)(107.25390625,311.00678276)(107.25390625,309.96381401)
\curveto(107.25390625,308.96186088)(107.47363281,308.12689995)(107.91308594,307.4589312)
\curveto(108.35253906,306.79096245)(109.04980469,306.45697807)(110.00488281,306.45697807)
\closepath
\moveto(110.06640625,315.1229937)
\curveto(111.27929688,315.1229937)(112.3046875,314.71869682)(113.14257812,313.91010307)
\curveto(113.98046875,313.10150932)(114.39941406,311.9120562)(114.39941406,310.3417437)
\curveto(114.39941406,308.82416557)(114.03027344,307.57025932)(113.29199219,306.58002495)
\curveto(112.55371094,305.58979057)(111.40820312,305.09467338)(109.85546875,305.09467338)
\curveto(108.56054688,305.09467338)(107.53222656,305.53119682)(106.77050781,306.4042437)
\curveto(106.00878906,307.28314995)(105.62792969,308.46088432)(105.62792969,309.93744682)
\curveto(105.62792969,311.51947807)(106.02929688,312.7792437)(106.83203125,313.7167437)
\curveto(107.63476562,314.6542437)(108.71289062,315.1229937)(110.06640625,315.1229937)
\closepath
\moveto(110.01367188,315.07904838)
\lineto(110.01367188,315.07904838)
\closepath
}
}
{
\newrgbcolor{curcolor}{0 0 0}
\pscustom[linestyle=none,fillstyle=solid,fillcolor=curcolor]
{
\newpath
\moveto(116.33300781,318.34857963)
\lineto(117.91503906,318.34857963)
\lineto(117.91503906,305.43744682)
\lineto(116.33300781,305.43744682)
\lineto(116.33300781,318.34857963)
\closepath
}
}
{
\newrgbcolor{curcolor}{0 0 0}
\pscustom[linestyle=none,fillstyle=solid,fillcolor=curcolor]
{
\newpath
\moveto(124.21679688,315.06147026)
\curveto(124.88476562,315.06147026)(125.53222656,314.90326713)(126.15917969,314.58686088)
\curveto(126.78613281,314.27631401)(127.26367188,313.87201713)(127.59179688,313.37397026)
\curveto(127.90820312,312.89936088)(128.11914062,312.34564995)(128.22460938,311.71283745)
\curveto(128.31835938,311.2792437)(128.36523438,310.58783745)(128.36523438,309.6386187)
\lineto(121.46582031,309.6386187)
\curveto(121.49511719,308.68354057)(121.72070312,307.91596245)(122.14257812,307.33588432)
\curveto(122.56445312,306.76166557)(123.21777344,306.4745562)(124.10253906,306.4745562)
\curveto(124.92871094,306.4745562)(125.58789062,306.74701713)(126.08007812,307.29193901)
\curveto(126.36132812,307.60834526)(126.56054688,307.9745562)(126.67773438,308.39057182)
\lineto(128.23339844,308.39057182)
\curveto(128.19238281,308.0448687)(128.0546875,307.65814995)(127.8203125,307.23041557)
\curveto(127.59179688,306.80854057)(127.33398438,306.46283745)(127.046875,306.1933062)
\curveto(126.56640625,305.7245562)(125.97167969,305.40814995)(125.26269531,305.24408745)
\curveto(124.88183594,305.15033745)(124.45117188,305.10346245)(123.97070312,305.10346245)
\curveto(122.79882812,305.10346245)(121.80566406,305.52826713)(120.99121094,306.37787651)
\curveto(120.17675781,307.23334526)(119.76953125,308.42865776)(119.76953125,309.96381401)
\curveto(119.76953125,311.47553276)(120.1796875,312.70307182)(121,313.6464312)
\curveto(121.8203125,314.58979057)(122.89257812,315.06147026)(124.21679688,315.06147026)
\closepath
\moveto(126.73925781,310.89545463)
\curveto(126.67480469,311.58100151)(126.52539062,312.12885307)(126.29101562,312.53900932)
\curveto(125.85742188,313.30072807)(125.13378906,313.68158745)(124.12011719,313.68158745)
\curveto(123.39355469,313.68158745)(122.78417969,313.41791557)(122.29199219,312.89057182)
\curveto(121.79980469,312.36908745)(121.5390625,311.70404838)(121.50976562,310.89545463)
\lineto(126.73925781,310.89545463)
\closepath
\moveto(124.06738281,315.07904838)
\lineto(124.06738281,315.07904838)
\closepath
}
}
\end{pspicture}

    \caption{Desired Hole vs Actual Hole}
  \end{figure}
\end{frame}

\end{document}
