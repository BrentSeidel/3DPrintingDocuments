\documentclass[english,10pt]{beamer}
\usepackage{pstricks}
\usepackage{graphicx}
\DeclareGraphicsExtensions{.pdf}
\DeclareGraphicsRule{.pdf}{pdf}{.pdf}{}
\DeclareGraphicsExtensions{.jpg}
\DeclareGraphicsRule{. jpg}{jpg}{. jpg}{}
\DeclareGraphicsExtensions{.svg}
\DeclareGraphicsRule{. svg}{svg}{. svg}{}

\title{Design for 3D Printing}
\usepackage[yyyymmdd]{datetime}
\renewcommand{\dateseparator}{-}

\author{Brent Seidel\\modestconsulting@gmail.com\\Modest Consulting}
\institute{Modest Consulting}
\date{Copyright \today\ CC-BY-SA}


\subtitle{Design for 3D Printing}
\begin{document}
\begin{frame}
  \titlepage
\end{frame}

\begin{frame}
  \frametitle{Outline}
  \tableofcontents
\end{frame}

\section{Introduction}
\begin{frame}
  \frametitle{Introduction}
  A number of items need to be considered when designing an object for 3D printing.  Some of them are generic while others may depend on the 3D printing technology used.
\end{frame}

\section{Measurements}
\begin{frame}
  \frametitle{Measurements (units)}
  A good question is, ``What does one unit in the CAD program translate to when printed out?''.  Depending on your application, the answer may range from mildly interesting to vitally important.  If you are creating a stand alone decorative object, you may not care much as long as the result is a reasonable size.  If, however, you are designing a part to physically interface with some existing items, the answer is vitally important.
\end{frame}

\begin{frame}
  \frametitle{Measurements (accuracy)}
  There are three different items that impact the detail and accuracy of the print.
  \begin{itemize}
    \item What is the resolution for printing bits of material? (note that x and y resolution may be different)
    \item What is the smallest bit of material that can be printed?
    \item What is the layer height?
  \end{itemize}
  Some or all of these may be adjustable on your printer.  In general, reducing the resolution makes the print go faster.
\end{frame}

\begin{frame}
  \frametitle{Measurements (Resolution)}
  \begin{figure}
    %LaTeX with PSTricks extensions
%%Creator: inkscape 0.91
%%Please note this file requires PSTricks extensions
\psset{xunit=.5pt,yunit=.5pt,runit=.5pt}
\begin{pspicture}(435.04902397,317.45960206)
{
\newrgbcolor{curcolor}{1 1 1}
\pscustom[linestyle=none,fillstyle=solid,fillcolor=curcolor,opacity=0]
{
\newpath
\moveto(211.11230469,173.754939)
\curveto(211.11230469,130.17981467)(175.59600265,94.85519687)(131.78442383,94.85519687)
\curveto(87.97284501,94.85519687)(52.45654297,130.17981467)(52.45654297,173.754939)
\curveto(52.45654297,217.33006332)(87.97284501,252.65468113)(131.78442383,252.65468113)
\curveto(175.59600265,252.65468113)(211.11230469,217.33006332)(211.11230469,173.754939)
\closepath
}
}
{
\newrgbcolor{curcolor}{0 0 0}
\pscustom[linewidth=1.97783792,linecolor=curcolor,linestyle=dashed,dash=3.95567594 1.97783797]
{
\newpath
\moveto(211.11230469,173.754939)
\curveto(211.11230469,130.17981467)(175.59600265,94.85519687)(131.78442383,94.85519687)
\curveto(87.97284501,94.85519687)(52.45654297,130.17981467)(52.45654297,173.754939)
\curveto(52.45654297,217.33006332)(87.97284501,252.65468113)(131.78442383,252.65468113)
\curveto(175.59600265,252.65468113)(211.11230469,217.33006332)(211.11230469,173.754939)
\closepath
}
}
{
\newrgbcolor{curcolor}{0 0 0}
\pscustom[linestyle=none,fillstyle=solid,fillcolor=curcolor]
{
\newpath
\moveto(20.5,253.99787418)
\curveto(20.5,248.47502668)(16.0228475,243.99787418)(10.5,243.99787418)
\curveto(4.9771525,243.99787418)(0.5,248.47502668)(0.5,253.99787418)
\curveto(0.5,259.52072168)(4.9771525,263.99787418)(10.5,263.99787418)
\curveto(16.0228475,263.99787418)(20.5,259.52072168)(20.5,253.99787418)
\closepath
}
}
{
\newrgbcolor{curcolor}{0 0 0}
\pscustom[linewidth=1,linecolor=curcolor]
{
\newpath
\moveto(20.5,253.99787418)
\curveto(20.5,248.47502668)(16.0228475,243.99787418)(10.5,243.99787418)
\curveto(4.9771525,243.99787418)(0.5,248.47502668)(0.5,253.99787418)
\curveto(0.5,259.52072168)(4.9771525,263.99787418)(10.5,263.99787418)
\curveto(16.0228475,263.99787418)(20.5,259.52072168)(20.5,253.99787418)
\closepath
}
}
{
\newrgbcolor{curcolor}{0 0 0}
\pscustom[linestyle=none,fillstyle=solid,fillcolor=curcolor]
{
\newpath
\moveto(61.60121155,253.99787418)
\curveto(61.60121155,248.47502668)(57.12405904,243.99787418)(51.60121155,243.99787418)
\curveto(46.07836405,243.99787418)(41.60121155,248.47502668)(41.60121155,253.99787418)
\curveto(41.60121155,259.52072168)(46.07836405,263.99787418)(51.60121155,263.99787418)
\curveto(57.12405904,263.99787418)(61.60121155,259.52072168)(61.60121155,253.99787418)
\closepath
}
}
{
\newrgbcolor{curcolor}{0 0 0}
\pscustom[linewidth=1,linecolor=curcolor]
{
\newpath
\moveto(61.60121155,253.99787418)
\curveto(61.60121155,248.47502668)(57.12405904,243.99787418)(51.60121155,243.99787418)
\curveto(46.07836405,243.99787418)(41.60121155,248.47502668)(41.60121155,253.99787418)
\curveto(41.60121155,259.52072168)(46.07836405,263.99787418)(51.60121155,263.99787418)
\curveto(57.12405904,263.99787418)(61.60121155,259.52072168)(61.60121155,253.99787418)
\closepath
}
}
{
\newrgbcolor{curcolor}{0 0 0}
\pscustom[linestyle=none,fillstyle=solid,fillcolor=curcolor]
{
\newpath
\moveto(100.98986816,253.14159671)
\curveto(100.98986816,247.61874922)(96.51271566,243.14159671)(90.98986816,243.14159671)
\curveto(85.46702067,243.14159671)(80.98986816,247.61874922)(80.98986816,253.14159671)
\curveto(80.98986816,258.66444421)(85.46702067,263.14159671)(90.98986816,263.14159671)
\curveto(96.51271566,263.14159671)(100.98986816,258.66444421)(100.98986816,253.14159671)
\closepath
}
}
{
\newrgbcolor{curcolor}{0 0 0}
\pscustom[linewidth=1,linecolor=curcolor]
{
\newpath
\moveto(100.98986816,253.14159671)
\curveto(100.98986816,247.61874922)(96.51271566,243.14159671)(90.98986816,243.14159671)
\curveto(85.46702067,243.14159671)(80.98986816,247.61874922)(80.98986816,253.14159671)
\curveto(80.98986816,258.66444421)(85.46702067,263.14159671)(90.98986816,263.14159671)
\curveto(96.51271566,263.14159671)(100.98986816,258.66444421)(100.98986816,253.14159671)
\closepath
}
}
{
\newrgbcolor{curcolor}{0 0 0}
\pscustom[linestyle=none,fillstyle=solid,fillcolor=curcolor]
{
\newpath
\moveto(142.09107971,253.99787418)
\curveto(142.09107971,248.47502668)(137.61392721,243.99787418)(132.09107971,243.99787418)
\curveto(126.56823222,243.99787418)(122.09107971,248.47502668)(122.09107971,253.99787418)
\curveto(122.09107971,259.52072168)(126.56823222,263.99787418)(132.09107971,263.99787418)
\curveto(137.61392721,263.99787418)(142.09107971,259.52072168)(142.09107971,253.99787418)
\closepath
}
}
{
\newrgbcolor{curcolor}{0 0 0}
\pscustom[linewidth=1,linecolor=curcolor]
{
\newpath
\moveto(142.09107971,253.99787418)
\curveto(142.09107971,248.47502668)(137.61392721,243.99787418)(132.09107971,243.99787418)
\curveto(126.56823222,243.99787418)(122.09107971,248.47502668)(122.09107971,253.99787418)
\curveto(122.09107971,259.52072168)(126.56823222,263.99787418)(132.09107971,263.99787418)
\curveto(137.61392721,263.99787418)(142.09107971,259.52072168)(142.09107971,253.99787418)
\closepath
}
}
{
\newrgbcolor{curcolor}{0 0 0}
\pscustom[linestyle=none,fillstyle=solid,fillcolor=curcolor]
{
\newpath
\moveto(181.47973633,253.99787418)
\curveto(181.47973633,248.47502668)(177.00258382,243.99787418)(171.47973633,243.99787418)
\curveto(165.95688883,243.99787418)(161.47973633,248.47502668)(161.47973633,253.99787418)
\curveto(161.47973633,259.52072168)(165.95688883,263.99787418)(171.47973633,263.99787418)
\curveto(177.00258382,263.99787418)(181.47973633,259.52072168)(181.47973633,253.99787418)
\closepath
}
}
{
\newrgbcolor{curcolor}{0 0 0}
\pscustom[linewidth=1,linecolor=curcolor]
{
\newpath
\moveto(181.47973633,253.99787418)
\curveto(181.47973633,248.47502668)(177.00258382,243.99787418)(171.47973633,243.99787418)
\curveto(165.95688883,243.99787418)(161.47973633,248.47502668)(161.47973633,253.99787418)
\curveto(161.47973633,259.52072168)(165.95688883,263.99787418)(171.47973633,263.99787418)
\curveto(177.00258382,263.99787418)(181.47973633,259.52072168)(181.47973633,253.99787418)
\closepath
}
}
{
\newrgbcolor{curcolor}{0 0 0}
\pscustom[linestyle=none,fillstyle=solid,fillcolor=curcolor]
{
\newpath
\moveto(221.47973633,253.99787418)
\curveto(221.47973633,248.47502668)(217.00258382,243.99787418)(211.47973633,243.99787418)
\curveto(205.95688883,243.99787418)(201.47973633,248.47502668)(201.47973633,253.99787418)
\curveto(201.47973633,259.52072168)(205.95688883,263.99787418)(211.47973633,263.99787418)
\curveto(217.00258382,263.99787418)(221.47973633,259.52072168)(221.47973633,253.99787418)
\closepath
}
}
{
\newrgbcolor{curcolor}{0 0 0}
\pscustom[linewidth=1,linecolor=curcolor]
{
\newpath
\moveto(221.47973633,253.99787418)
\curveto(221.47973633,248.47502668)(217.00258382,243.99787418)(211.47973633,243.99787418)
\curveto(205.95688883,243.99787418)(201.47973633,248.47502668)(201.47973633,253.99787418)
\curveto(201.47973633,259.52072168)(205.95688883,263.99787418)(211.47973633,263.99787418)
\curveto(217.00258382,263.99787418)(221.47973633,259.52072168)(221.47973633,253.99787418)
\closepath
}
}
{
\newrgbcolor{curcolor}{0 0 0}
\pscustom[linestyle=none,fillstyle=solid,fillcolor=curcolor]
{
\newpath
\moveto(20.5,213.99787418)
\curveto(20.5,208.47502668)(16.0228475,203.99787418)(10.5,203.99787418)
\curveto(4.9771525,203.99787418)(0.5,208.47502668)(0.5,213.99787418)
\curveto(0.5,219.52072168)(4.9771525,223.99787418)(10.5,223.99787418)
\curveto(16.0228475,223.99787418)(20.5,219.52072168)(20.5,213.99787418)
\closepath
}
}
{
\newrgbcolor{curcolor}{0 0 0}
\pscustom[linewidth=1,linecolor=curcolor]
{
\newpath
\moveto(20.5,213.99787418)
\curveto(20.5,208.47502668)(16.0228475,203.99787418)(10.5,203.99787418)
\curveto(4.9771525,203.99787418)(0.5,208.47502668)(0.5,213.99787418)
\curveto(0.5,219.52072168)(4.9771525,223.99787418)(10.5,223.99787418)
\curveto(16.0228475,223.99787418)(20.5,219.52072168)(20.5,213.99787418)
\closepath
}
}
{
\newrgbcolor{curcolor}{0 0 0}
\pscustom[linestyle=none,fillstyle=solid,fillcolor=curcolor]
{
\newpath
\moveto(61.60121155,213.99787418)
\curveto(61.60121155,208.47502668)(57.12405904,203.99787418)(51.60121155,203.99787418)
\curveto(46.07836405,203.99787418)(41.60121155,208.47502668)(41.60121155,213.99787418)
\curveto(41.60121155,219.52072168)(46.07836405,223.99787418)(51.60121155,223.99787418)
\curveto(57.12405904,223.99787418)(61.60121155,219.52072168)(61.60121155,213.99787418)
\closepath
}
}
{
\newrgbcolor{curcolor}{0 0 0}
\pscustom[linewidth=1,linecolor=curcolor]
{
\newpath
\moveto(61.60121155,213.99787418)
\curveto(61.60121155,208.47502668)(57.12405904,203.99787418)(51.60121155,203.99787418)
\curveto(46.07836405,203.99787418)(41.60121155,208.47502668)(41.60121155,213.99787418)
\curveto(41.60121155,219.52072168)(46.07836405,223.99787418)(51.60121155,223.99787418)
\curveto(57.12405904,223.99787418)(61.60121155,219.52072168)(61.60121155,213.99787418)
\closepath
}
}
{
\newrgbcolor{curcolor}{0 0 0}
\pscustom[linestyle=none,fillstyle=solid,fillcolor=curcolor]
{
\newpath
\moveto(100.98986816,213.14159671)
\curveto(100.98986816,207.61874922)(96.51271566,203.14159671)(90.98986816,203.14159671)
\curveto(85.46702067,203.14159671)(80.98986816,207.61874922)(80.98986816,213.14159671)
\curveto(80.98986816,218.66444421)(85.46702067,223.14159671)(90.98986816,223.14159671)
\curveto(96.51271566,223.14159671)(100.98986816,218.66444421)(100.98986816,213.14159671)
\closepath
}
}
{
\newrgbcolor{curcolor}{0 0 0}
\pscustom[linewidth=1,linecolor=curcolor]
{
\newpath
\moveto(100.98986816,213.14159671)
\curveto(100.98986816,207.61874922)(96.51271566,203.14159671)(90.98986816,203.14159671)
\curveto(85.46702067,203.14159671)(80.98986816,207.61874922)(80.98986816,213.14159671)
\curveto(80.98986816,218.66444421)(85.46702067,223.14159671)(90.98986816,223.14159671)
\curveto(96.51271566,223.14159671)(100.98986816,218.66444421)(100.98986816,213.14159671)
\closepath
}
}
{
\newrgbcolor{curcolor}{0 0 0}
\pscustom[linestyle=none,fillstyle=solid,fillcolor=curcolor]
{
\newpath
\moveto(142.09107971,213.99787418)
\curveto(142.09107971,208.47502668)(137.61392721,203.99787418)(132.09107971,203.99787418)
\curveto(126.56823222,203.99787418)(122.09107971,208.47502668)(122.09107971,213.99787418)
\curveto(122.09107971,219.52072168)(126.56823222,223.99787418)(132.09107971,223.99787418)
\curveto(137.61392721,223.99787418)(142.09107971,219.52072168)(142.09107971,213.99787418)
\closepath
}
}
{
\newrgbcolor{curcolor}{0 0 0}
\pscustom[linewidth=1,linecolor=curcolor]
{
\newpath
\moveto(142.09107971,213.99787418)
\curveto(142.09107971,208.47502668)(137.61392721,203.99787418)(132.09107971,203.99787418)
\curveto(126.56823222,203.99787418)(122.09107971,208.47502668)(122.09107971,213.99787418)
\curveto(122.09107971,219.52072168)(126.56823222,223.99787418)(132.09107971,223.99787418)
\curveto(137.61392721,223.99787418)(142.09107971,219.52072168)(142.09107971,213.99787418)
\closepath
}
}
{
\newrgbcolor{curcolor}{0 0 0}
\pscustom[linestyle=none,fillstyle=solid,fillcolor=curcolor]
{
\newpath
\moveto(181.47973633,213.99787418)
\curveto(181.47973633,208.47502668)(177.00258382,203.99787418)(171.47973633,203.99787418)
\curveto(165.95688883,203.99787418)(161.47973633,208.47502668)(161.47973633,213.99787418)
\curveto(161.47973633,219.52072168)(165.95688883,223.99787418)(171.47973633,223.99787418)
\curveto(177.00258382,223.99787418)(181.47973633,219.52072168)(181.47973633,213.99787418)
\closepath
}
}
{
\newrgbcolor{curcolor}{0 0 0}
\pscustom[linewidth=1,linecolor=curcolor]
{
\newpath
\moveto(181.47973633,213.99787418)
\curveto(181.47973633,208.47502668)(177.00258382,203.99787418)(171.47973633,203.99787418)
\curveto(165.95688883,203.99787418)(161.47973633,208.47502668)(161.47973633,213.99787418)
\curveto(161.47973633,219.52072168)(165.95688883,223.99787418)(171.47973633,223.99787418)
\curveto(177.00258382,223.99787418)(181.47973633,219.52072168)(181.47973633,213.99787418)
\closepath
}
}
{
\newrgbcolor{curcolor}{0 0 0}
\pscustom[linestyle=none,fillstyle=solid,fillcolor=curcolor]
{
\newpath
\moveto(221.47973633,213.99787418)
\curveto(221.47973633,208.47502668)(217.00258382,203.99787418)(211.47973633,203.99787418)
\curveto(205.95688883,203.99787418)(201.47973633,208.47502668)(201.47973633,213.99787418)
\curveto(201.47973633,219.52072168)(205.95688883,223.99787418)(211.47973633,223.99787418)
\curveto(217.00258382,223.99787418)(221.47973633,219.52072168)(221.47973633,213.99787418)
\closepath
}
}
{
\newrgbcolor{curcolor}{0 0 0}
\pscustom[linewidth=1,linecolor=curcolor]
{
\newpath
\moveto(221.47973633,213.99787418)
\curveto(221.47973633,208.47502668)(217.00258382,203.99787418)(211.47973633,203.99787418)
\curveto(205.95688883,203.99787418)(201.47973633,208.47502668)(201.47973633,213.99787418)
\curveto(201.47973633,219.52072168)(205.95688883,223.99787418)(211.47973633,223.99787418)
\curveto(217.00258382,223.99787418)(221.47973633,219.52072168)(221.47973633,213.99787418)
\closepath
}
}
{
\newrgbcolor{curcolor}{0 0 0}
\pscustom[linestyle=none,fillstyle=solid,fillcolor=curcolor]
{
\newpath
\moveto(20.5,173.99788944)
\curveto(20.5,168.47504194)(16.0228475,163.99788944)(10.5,163.99788944)
\curveto(4.9771525,163.99788944)(0.5,168.47504194)(0.5,173.99788944)
\curveto(0.5,179.52073693)(4.9771525,183.99788944)(10.5,183.99788944)
\curveto(16.0228475,183.99788944)(20.5,179.52073693)(20.5,173.99788944)
\closepath
}
}
{
\newrgbcolor{curcolor}{0 0 0}
\pscustom[linewidth=1,linecolor=curcolor]
{
\newpath
\moveto(20.5,173.99788944)
\curveto(20.5,168.47504194)(16.0228475,163.99788944)(10.5,163.99788944)
\curveto(4.9771525,163.99788944)(0.5,168.47504194)(0.5,173.99788944)
\curveto(0.5,179.52073693)(4.9771525,183.99788944)(10.5,183.99788944)
\curveto(16.0228475,183.99788944)(20.5,179.52073693)(20.5,173.99788944)
\closepath
}
}
{
\newrgbcolor{curcolor}{0 0 0}
\pscustom[linestyle=none,fillstyle=solid,fillcolor=curcolor]
{
\newpath
\moveto(61.60121155,173.99788944)
\curveto(61.60121155,168.47504194)(57.12405904,163.99788944)(51.60121155,163.99788944)
\curveto(46.07836405,163.99788944)(41.60121155,168.47504194)(41.60121155,173.99788944)
\curveto(41.60121155,179.52073693)(46.07836405,183.99788944)(51.60121155,183.99788944)
\curveto(57.12405904,183.99788944)(61.60121155,179.52073693)(61.60121155,173.99788944)
\closepath
}
}
{
\newrgbcolor{curcolor}{0 0 0}
\pscustom[linewidth=1,linecolor=curcolor]
{
\newpath
\moveto(61.60121155,173.99788944)
\curveto(61.60121155,168.47504194)(57.12405904,163.99788944)(51.60121155,163.99788944)
\curveto(46.07836405,163.99788944)(41.60121155,168.47504194)(41.60121155,173.99788944)
\curveto(41.60121155,179.52073693)(46.07836405,183.99788944)(51.60121155,183.99788944)
\curveto(57.12405904,183.99788944)(61.60121155,179.52073693)(61.60121155,173.99788944)
\closepath
}
}
{
\newrgbcolor{curcolor}{0 0 0}
\pscustom[linestyle=none,fillstyle=solid,fillcolor=curcolor]
{
\newpath
\moveto(100.98986816,173.14159671)
\curveto(100.98986816,167.61874922)(96.51271566,163.14159671)(90.98986816,163.14159671)
\curveto(85.46702067,163.14159671)(80.98986816,167.61874922)(80.98986816,173.14159671)
\curveto(80.98986816,178.66444421)(85.46702067,183.14159671)(90.98986816,183.14159671)
\curveto(96.51271566,183.14159671)(100.98986816,178.66444421)(100.98986816,173.14159671)
\closepath
}
}
{
\newrgbcolor{curcolor}{0 0 0}
\pscustom[linewidth=1,linecolor=curcolor]
{
\newpath
\moveto(100.98986816,173.14159671)
\curveto(100.98986816,167.61874922)(96.51271566,163.14159671)(90.98986816,163.14159671)
\curveto(85.46702067,163.14159671)(80.98986816,167.61874922)(80.98986816,173.14159671)
\curveto(80.98986816,178.66444421)(85.46702067,183.14159671)(90.98986816,183.14159671)
\curveto(96.51271566,183.14159671)(100.98986816,178.66444421)(100.98986816,173.14159671)
\closepath
}
}
{
\newrgbcolor{curcolor}{0 0 0}
\pscustom[linestyle=none,fillstyle=solid,fillcolor=curcolor]
{
\newpath
\moveto(142.09107971,173.99788944)
\curveto(142.09107971,168.47504194)(137.61392721,163.99788944)(132.09107971,163.99788944)
\curveto(126.56823222,163.99788944)(122.09107971,168.47504194)(122.09107971,173.99788944)
\curveto(122.09107971,179.52073693)(126.56823222,183.99788944)(132.09107971,183.99788944)
\curveto(137.61392721,183.99788944)(142.09107971,179.52073693)(142.09107971,173.99788944)
\closepath
}
}
{
\newrgbcolor{curcolor}{0 0 0}
\pscustom[linewidth=1,linecolor=curcolor]
{
\newpath
\moveto(142.09107971,173.99788944)
\curveto(142.09107971,168.47504194)(137.61392721,163.99788944)(132.09107971,163.99788944)
\curveto(126.56823222,163.99788944)(122.09107971,168.47504194)(122.09107971,173.99788944)
\curveto(122.09107971,179.52073693)(126.56823222,183.99788944)(132.09107971,183.99788944)
\curveto(137.61392721,183.99788944)(142.09107971,179.52073693)(142.09107971,173.99788944)
\closepath
}
}
{
\newrgbcolor{curcolor}{0 0 0}
\pscustom[linestyle=none,fillstyle=solid,fillcolor=curcolor]
{
\newpath
\moveto(181.47973633,173.99788944)
\curveto(181.47973633,168.47504194)(177.00258382,163.99788944)(171.47973633,163.99788944)
\curveto(165.95688883,163.99788944)(161.47973633,168.47504194)(161.47973633,173.99788944)
\curveto(161.47973633,179.52073693)(165.95688883,183.99788944)(171.47973633,183.99788944)
\curveto(177.00258382,183.99788944)(181.47973633,179.52073693)(181.47973633,173.99788944)
\closepath
}
}
{
\newrgbcolor{curcolor}{0 0 0}
\pscustom[linewidth=1,linecolor=curcolor]
{
\newpath
\moveto(181.47973633,173.99788944)
\curveto(181.47973633,168.47504194)(177.00258382,163.99788944)(171.47973633,163.99788944)
\curveto(165.95688883,163.99788944)(161.47973633,168.47504194)(161.47973633,173.99788944)
\curveto(161.47973633,179.52073693)(165.95688883,183.99788944)(171.47973633,183.99788944)
\curveto(177.00258382,183.99788944)(181.47973633,179.52073693)(181.47973633,173.99788944)
\closepath
}
}
{
\newrgbcolor{curcolor}{0 0 0}
\pscustom[linestyle=none,fillstyle=solid,fillcolor=curcolor]
{
\newpath
\moveto(221.47973633,173.99788944)
\curveto(221.47973633,168.47504194)(217.00258382,163.99788944)(211.47973633,163.99788944)
\curveto(205.95688883,163.99788944)(201.47973633,168.47504194)(201.47973633,173.99788944)
\curveto(201.47973633,179.52073693)(205.95688883,183.99788944)(211.47973633,183.99788944)
\curveto(217.00258382,183.99788944)(221.47973633,179.52073693)(221.47973633,173.99788944)
\closepath
}
}
{
\newrgbcolor{curcolor}{0 0 0}
\pscustom[linewidth=1,linecolor=curcolor]
{
\newpath
\moveto(221.47973633,173.99788944)
\curveto(221.47973633,168.47504194)(217.00258382,163.99788944)(211.47973633,163.99788944)
\curveto(205.95688883,163.99788944)(201.47973633,168.47504194)(201.47973633,173.99788944)
\curveto(201.47973633,179.52073693)(205.95688883,183.99788944)(211.47973633,183.99788944)
\curveto(217.00258382,183.99788944)(221.47973633,179.52073693)(221.47973633,173.99788944)
\closepath
}
}
{
\newrgbcolor{curcolor}{0 0 0}
\pscustom[linestyle=none,fillstyle=solid,fillcolor=curcolor]
{
\newpath
\moveto(20.5,133.99788944)
\curveto(20.5,128.47504194)(16.0228475,123.99788944)(10.5,123.99788944)
\curveto(4.9771525,123.99788944)(0.5,128.47504194)(0.5,133.99788944)
\curveto(0.5,139.52073693)(4.9771525,143.99788944)(10.5,143.99788944)
\curveto(16.0228475,143.99788944)(20.5,139.52073693)(20.5,133.99788944)
\closepath
}
}
{
\newrgbcolor{curcolor}{0 0 0}
\pscustom[linewidth=1,linecolor=curcolor]
{
\newpath
\moveto(20.5,133.99788944)
\curveto(20.5,128.47504194)(16.0228475,123.99788944)(10.5,123.99788944)
\curveto(4.9771525,123.99788944)(0.5,128.47504194)(0.5,133.99788944)
\curveto(0.5,139.52073693)(4.9771525,143.99788944)(10.5,143.99788944)
\curveto(16.0228475,143.99788944)(20.5,139.52073693)(20.5,133.99788944)
\closepath
}
}
{
\newrgbcolor{curcolor}{0 0 0}
\pscustom[linestyle=none,fillstyle=solid,fillcolor=curcolor]
{
\newpath
\moveto(61.60121155,133.99788944)
\curveto(61.60121155,128.47504194)(57.12405904,123.99788944)(51.60121155,123.99788944)
\curveto(46.07836405,123.99788944)(41.60121155,128.47504194)(41.60121155,133.99788944)
\curveto(41.60121155,139.52073693)(46.07836405,143.99788944)(51.60121155,143.99788944)
\curveto(57.12405904,143.99788944)(61.60121155,139.52073693)(61.60121155,133.99788944)
\closepath
}
}
{
\newrgbcolor{curcolor}{0 0 0}
\pscustom[linewidth=1,linecolor=curcolor]
{
\newpath
\moveto(61.60121155,133.99788944)
\curveto(61.60121155,128.47504194)(57.12405904,123.99788944)(51.60121155,123.99788944)
\curveto(46.07836405,123.99788944)(41.60121155,128.47504194)(41.60121155,133.99788944)
\curveto(41.60121155,139.52073693)(46.07836405,143.99788944)(51.60121155,143.99788944)
\curveto(57.12405904,143.99788944)(61.60121155,139.52073693)(61.60121155,133.99788944)
\closepath
}
}
{
\newrgbcolor{curcolor}{0 0 0}
\pscustom[linestyle=none,fillstyle=solid,fillcolor=curcolor]
{
\newpath
\moveto(100.98986816,133.14159671)
\curveto(100.98986816,127.61874922)(96.51271566,123.14159671)(90.98986816,123.14159671)
\curveto(85.46702067,123.14159671)(80.98986816,127.61874922)(80.98986816,133.14159671)
\curveto(80.98986816,138.66444421)(85.46702067,143.14159671)(90.98986816,143.14159671)
\curveto(96.51271566,143.14159671)(100.98986816,138.66444421)(100.98986816,133.14159671)
\closepath
}
}
{
\newrgbcolor{curcolor}{0 0 0}
\pscustom[linewidth=1,linecolor=curcolor]
{
\newpath
\moveto(100.98986816,133.14159671)
\curveto(100.98986816,127.61874922)(96.51271566,123.14159671)(90.98986816,123.14159671)
\curveto(85.46702067,123.14159671)(80.98986816,127.61874922)(80.98986816,133.14159671)
\curveto(80.98986816,138.66444421)(85.46702067,143.14159671)(90.98986816,143.14159671)
\curveto(96.51271566,143.14159671)(100.98986816,138.66444421)(100.98986816,133.14159671)
\closepath
}
}
{
\newrgbcolor{curcolor}{0 0 0}
\pscustom[linestyle=none,fillstyle=solid,fillcolor=curcolor]
{
\newpath
\moveto(142.09107971,133.99788944)
\curveto(142.09107971,128.47504194)(137.61392721,123.99788944)(132.09107971,123.99788944)
\curveto(126.56823222,123.99788944)(122.09107971,128.47504194)(122.09107971,133.99788944)
\curveto(122.09107971,139.52073693)(126.56823222,143.99788944)(132.09107971,143.99788944)
\curveto(137.61392721,143.99788944)(142.09107971,139.52073693)(142.09107971,133.99788944)
\closepath
}
}
{
\newrgbcolor{curcolor}{0 0 0}
\pscustom[linewidth=1,linecolor=curcolor]
{
\newpath
\moveto(142.09107971,133.99788944)
\curveto(142.09107971,128.47504194)(137.61392721,123.99788944)(132.09107971,123.99788944)
\curveto(126.56823222,123.99788944)(122.09107971,128.47504194)(122.09107971,133.99788944)
\curveto(122.09107971,139.52073693)(126.56823222,143.99788944)(132.09107971,143.99788944)
\curveto(137.61392721,143.99788944)(142.09107971,139.52073693)(142.09107971,133.99788944)
\closepath
}
}
{
\newrgbcolor{curcolor}{0 0 0}
\pscustom[linestyle=none,fillstyle=solid,fillcolor=curcolor]
{
\newpath
\moveto(181.47973633,133.99788944)
\curveto(181.47973633,128.47504194)(177.00258382,123.99788944)(171.47973633,123.99788944)
\curveto(165.95688883,123.99788944)(161.47973633,128.47504194)(161.47973633,133.99788944)
\curveto(161.47973633,139.52073693)(165.95688883,143.99788944)(171.47973633,143.99788944)
\curveto(177.00258382,143.99788944)(181.47973633,139.52073693)(181.47973633,133.99788944)
\closepath
}
}
{
\newrgbcolor{curcolor}{0 0 0}
\pscustom[linewidth=1,linecolor=curcolor]
{
\newpath
\moveto(181.47973633,133.99788944)
\curveto(181.47973633,128.47504194)(177.00258382,123.99788944)(171.47973633,123.99788944)
\curveto(165.95688883,123.99788944)(161.47973633,128.47504194)(161.47973633,133.99788944)
\curveto(161.47973633,139.52073693)(165.95688883,143.99788944)(171.47973633,143.99788944)
\curveto(177.00258382,143.99788944)(181.47973633,139.52073693)(181.47973633,133.99788944)
\closepath
}
}
{
\newrgbcolor{curcolor}{0 0 0}
\pscustom[linestyle=none,fillstyle=solid,fillcolor=curcolor]
{
\newpath
\moveto(221.47973633,133.99788944)
\curveto(221.47973633,128.47504194)(217.00258382,123.99788944)(211.47973633,123.99788944)
\curveto(205.95688883,123.99788944)(201.47973633,128.47504194)(201.47973633,133.99788944)
\curveto(201.47973633,139.52073693)(205.95688883,143.99788944)(211.47973633,143.99788944)
\curveto(217.00258382,143.99788944)(221.47973633,139.52073693)(221.47973633,133.99788944)
\closepath
}
}
{
\newrgbcolor{curcolor}{0 0 0}
\pscustom[linewidth=1,linecolor=curcolor]
{
\newpath
\moveto(221.47973633,133.99788944)
\curveto(221.47973633,128.47504194)(217.00258382,123.99788944)(211.47973633,123.99788944)
\curveto(205.95688883,123.99788944)(201.47973633,128.47504194)(201.47973633,133.99788944)
\curveto(201.47973633,139.52073693)(205.95688883,143.99788944)(211.47973633,143.99788944)
\curveto(217.00258382,143.99788944)(221.47973633,139.52073693)(221.47973633,133.99788944)
\closepath
}
}
{
\newrgbcolor{curcolor}{0 0 0}
\pscustom[linestyle=none,fillstyle=solid,fillcolor=curcolor]
{
\newpath
\moveto(20.5,93.99788944)
\curveto(20.5,88.47504194)(16.0228475,83.99788944)(10.5,83.99788944)
\curveto(4.9771525,83.99788944)(0.5,88.47504194)(0.5,93.99788944)
\curveto(0.5,99.52073693)(4.9771525,103.99788944)(10.5,103.99788944)
\curveto(16.0228475,103.99788944)(20.5,99.52073693)(20.5,93.99788944)
\closepath
}
}
{
\newrgbcolor{curcolor}{0 0 0}
\pscustom[linewidth=1,linecolor=curcolor]
{
\newpath
\moveto(20.5,93.99788944)
\curveto(20.5,88.47504194)(16.0228475,83.99788944)(10.5,83.99788944)
\curveto(4.9771525,83.99788944)(0.5,88.47504194)(0.5,93.99788944)
\curveto(0.5,99.52073693)(4.9771525,103.99788944)(10.5,103.99788944)
\curveto(16.0228475,103.99788944)(20.5,99.52073693)(20.5,93.99788944)
\closepath
}
}
{
\newrgbcolor{curcolor}{0 0 0}
\pscustom[linestyle=none,fillstyle=solid,fillcolor=curcolor]
{
\newpath
\moveto(61.60121155,93.99788944)
\curveto(61.60121155,88.47504194)(57.12405904,83.99788944)(51.60121155,83.99788944)
\curveto(46.07836405,83.99788944)(41.60121155,88.47504194)(41.60121155,93.99788944)
\curveto(41.60121155,99.52073693)(46.07836405,103.99788944)(51.60121155,103.99788944)
\curveto(57.12405904,103.99788944)(61.60121155,99.52073693)(61.60121155,93.99788944)
\closepath
}
}
{
\newrgbcolor{curcolor}{0 0 0}
\pscustom[linewidth=1,linecolor=curcolor]
{
\newpath
\moveto(61.60121155,93.99788944)
\curveto(61.60121155,88.47504194)(57.12405904,83.99788944)(51.60121155,83.99788944)
\curveto(46.07836405,83.99788944)(41.60121155,88.47504194)(41.60121155,93.99788944)
\curveto(41.60121155,99.52073693)(46.07836405,103.99788944)(51.60121155,103.99788944)
\curveto(57.12405904,103.99788944)(61.60121155,99.52073693)(61.60121155,93.99788944)
\closepath
}
}
{
\newrgbcolor{curcolor}{0 0 0}
\pscustom[linestyle=none,fillstyle=solid,fillcolor=curcolor]
{
\newpath
\moveto(100.98986816,93.14159671)
\curveto(100.98986816,87.61874922)(96.51271566,83.14159671)(90.98986816,83.14159671)
\curveto(85.46702067,83.14159671)(80.98986816,87.61874922)(80.98986816,93.14159671)
\curveto(80.98986816,98.66444421)(85.46702067,103.14159671)(90.98986816,103.14159671)
\curveto(96.51271566,103.14159671)(100.98986816,98.66444421)(100.98986816,93.14159671)
\closepath
}
}
{
\newrgbcolor{curcolor}{0 0 0}
\pscustom[linewidth=1,linecolor=curcolor]
{
\newpath
\moveto(100.98986816,93.14159671)
\curveto(100.98986816,87.61874922)(96.51271566,83.14159671)(90.98986816,83.14159671)
\curveto(85.46702067,83.14159671)(80.98986816,87.61874922)(80.98986816,93.14159671)
\curveto(80.98986816,98.66444421)(85.46702067,103.14159671)(90.98986816,103.14159671)
\curveto(96.51271566,103.14159671)(100.98986816,98.66444421)(100.98986816,93.14159671)
\closepath
}
}
{
\newrgbcolor{curcolor}{0 0 0}
\pscustom[linestyle=none,fillstyle=solid,fillcolor=curcolor]
{
\newpath
\moveto(142.09107971,93.99788944)
\curveto(142.09107971,88.47504194)(137.61392721,83.99788944)(132.09107971,83.99788944)
\curveto(126.56823222,83.99788944)(122.09107971,88.47504194)(122.09107971,93.99788944)
\curveto(122.09107971,99.52073693)(126.56823222,103.99788944)(132.09107971,103.99788944)
\curveto(137.61392721,103.99788944)(142.09107971,99.52073693)(142.09107971,93.99788944)
\closepath
}
}
{
\newrgbcolor{curcolor}{0 0 0}
\pscustom[linewidth=1,linecolor=curcolor]
{
\newpath
\moveto(142.09107971,93.99788944)
\curveto(142.09107971,88.47504194)(137.61392721,83.99788944)(132.09107971,83.99788944)
\curveto(126.56823222,83.99788944)(122.09107971,88.47504194)(122.09107971,93.99788944)
\curveto(122.09107971,99.52073693)(126.56823222,103.99788944)(132.09107971,103.99788944)
\curveto(137.61392721,103.99788944)(142.09107971,99.52073693)(142.09107971,93.99788944)
\closepath
}
}
{
\newrgbcolor{curcolor}{0 0 0}
\pscustom[linestyle=none,fillstyle=solid,fillcolor=curcolor]
{
\newpath
\moveto(181.47973633,93.99788944)
\curveto(181.47973633,88.47504194)(177.00258382,83.99788944)(171.47973633,83.99788944)
\curveto(165.95688883,83.99788944)(161.47973633,88.47504194)(161.47973633,93.99788944)
\curveto(161.47973633,99.52073693)(165.95688883,103.99788944)(171.47973633,103.99788944)
\curveto(177.00258382,103.99788944)(181.47973633,99.52073693)(181.47973633,93.99788944)
\closepath
}
}
{
\newrgbcolor{curcolor}{0 0 0}
\pscustom[linewidth=1,linecolor=curcolor]
{
\newpath
\moveto(181.47973633,93.99788944)
\curveto(181.47973633,88.47504194)(177.00258382,83.99788944)(171.47973633,83.99788944)
\curveto(165.95688883,83.99788944)(161.47973633,88.47504194)(161.47973633,93.99788944)
\curveto(161.47973633,99.52073693)(165.95688883,103.99788944)(171.47973633,103.99788944)
\curveto(177.00258382,103.99788944)(181.47973633,99.52073693)(181.47973633,93.99788944)
\closepath
}
}
{
\newrgbcolor{curcolor}{0 0 0}
\pscustom[linestyle=none,fillstyle=solid,fillcolor=curcolor]
{
\newpath
\moveto(221.47973633,93.99788944)
\curveto(221.47973633,88.47504194)(217.00258382,83.99788944)(211.47973633,83.99788944)
\curveto(205.95688883,83.99788944)(201.47973633,88.47504194)(201.47973633,93.99788944)
\curveto(201.47973633,99.52073693)(205.95688883,103.99788944)(211.47973633,103.99788944)
\curveto(217.00258382,103.99788944)(221.47973633,99.52073693)(221.47973633,93.99788944)
\closepath
}
}
{
\newrgbcolor{curcolor}{0 0 0}
\pscustom[linewidth=1,linecolor=curcolor]
{
\newpath
\moveto(221.47973633,93.99788944)
\curveto(221.47973633,88.47504194)(217.00258382,83.99788944)(211.47973633,83.99788944)
\curveto(205.95688883,83.99788944)(201.47973633,88.47504194)(201.47973633,93.99788944)
\curveto(201.47973633,99.52073693)(205.95688883,103.99788944)(211.47973633,103.99788944)
\curveto(217.00258382,103.99788944)(221.47973633,99.52073693)(221.47973633,93.99788944)
\closepath
}
}
{
\newrgbcolor{curcolor}{0 0 0}
\pscustom[linestyle=none,fillstyle=solid,fillcolor=curcolor]
{
\newpath
\moveto(11.57163239,304.54847232)
\lineto(9.46225739,304.54847232)
\lineto(14.08530426,311.16663639)
\lineto(9.75229645,317.45960514)
\lineto(11.94077301,317.45960514)
\lineto(15.23667145,312.52015201)
\lineto(18.5062027,317.45960514)
\lineto(20.58921051,317.45960514)
\lineto(16.2562027,311.16663639)
\lineto(20.80014801,304.54847232)
\lineto(18.62924957,304.54847232)
\lineto(15.14878082,309.85706607)
\lineto(11.57163239,304.54847232)
\closepath
}
}
{
\newrgbcolor{curcolor}{0 0 0}
\pscustom[linestyle=none,fillstyle=solid,fillcolor=curcolor]
{
\newpath
\moveto(34.10678864,309.83948795)
\lineto(32.1468277,315.54358951)
\lineto(30.06381989,309.83948795)
\lineto(34.10678864,309.83948795)
\closepath
\moveto(31.2327652,317.45960514)
\lineto(33.21030426,317.45960514)
\lineto(37.89487457,304.54847232)
\lineto(35.97885895,304.54847232)
\lineto(34.66928864,308.41565982)
\lineto(29.56284332,308.41565982)
\lineto(28.16538239,304.54847232)
\lineto(26.37241364,304.54847232)
\lineto(31.2327652,317.45960514)
\closepath
\moveto(32.13803864,317.45960514)
\lineto(32.13803864,317.45960514)
\closepath
}
}
{
\newrgbcolor{curcolor}{0 0 0}
\pscustom[linestyle=none,fillstyle=solid,fillcolor=curcolor]
{
\newpath
\moveto(38.39585114,313.96155826)
\lineto(40.4437027,313.96155826)
\lineto(42.60581207,310.6480817)
\lineto(44.79428864,313.96155826)
\lineto(46.71909332,313.91761295)
\lineto(43.54624176,309.37366764)
\lineto(46.85971832,304.54847232)
\lineto(44.83823395,304.54847232)
\lineto(42.50034332,308.08167545)
\lineto(40.2327652,304.54847232)
\lineto(38.22885895,304.54847232)
\lineto(41.54233551,309.37366764)
\lineto(38.39585114,313.96155826)
\closepath
}
}
{
\newrgbcolor{curcolor}{0 0 0}
\pscustom[linestyle=none,fillstyle=solid,fillcolor=curcolor]
{
\newpath
\moveto(48.29233551,313.91761295)
\lineto(49.90073395,313.91761295)
\lineto(49.90073395,304.54847232)
\lineto(48.29233551,304.54847232)
\lineto(48.29233551,313.91761295)
\closepath
\moveto(48.29233551,317.45960514)
\lineto(49.90073395,317.45960514)
\lineto(49.90073395,315.66663639)
\lineto(48.29233551,315.66663639)
\lineto(48.29233551,317.45960514)
\closepath
}
}
{
\newrgbcolor{curcolor}{0 0 0}
\pscustom[linestyle=none,fillstyle=solid,fillcolor=curcolor]
{
\newpath
\moveto(53.2405777,307.50159732)
\curveto(53.2874527,306.97425357)(53.41928864,306.5699567)(53.63608551,306.2887067)
\curveto(54.03452301,305.77894107)(54.72592926,305.52405826)(55.71030426,305.52405826)
\curveto(56.29624176,305.52405826)(56.81186676,305.65003482)(57.25717926,305.90198795)
\curveto(57.70249176,306.15980045)(57.92514801,306.55530826)(57.92514801,307.08851139)
\curveto(57.92514801,307.49280826)(57.74643707,307.80042545)(57.3890152,308.01136295)
\curveto(57.16049957,308.1402692)(56.7093277,308.28968326)(56.03549957,308.45960514)
\lineto(54.77866364,308.77601139)
\curveto(53.97592926,308.97523014)(53.38413239,309.19788639)(53.00327301,309.44398014)
\curveto(52.32358551,309.87171451)(51.98374176,310.46351139)(51.98374176,311.21937076)
\curveto(51.98374176,312.10999576)(52.3030777,312.83069889)(52.94174957,313.38148014)
\curveto(53.58628082,313.93226139)(54.45053864,314.20765201)(55.53452301,314.20765201)
\curveto(56.95249176,314.20765201)(57.9749527,313.79163639)(58.60190582,312.95960514)
\curveto(58.99448395,312.43226139)(59.18491364,311.86390201)(59.17319489,311.25452701)
\lineto(57.67905426,311.25452701)
\curveto(57.64975739,311.61194889)(57.52378082,311.9371442)(57.30112457,312.23011295)
\curveto(56.93784332,312.64612857)(56.30796051,312.85413639)(55.41147614,312.85413639)
\curveto(54.81381989,312.85413639)(54.35971832,312.73987857)(54.04917145,312.51136295)
\curveto(53.74448395,312.28284732)(53.5921402,311.98108951)(53.5921402,311.60608951)
\curveto(53.5921402,311.19593326)(53.79428864,310.86780826)(54.19858551,310.62171451)
\curveto(54.43296051,310.47523014)(54.77866364,310.34632389)(55.23569489,310.23499576)
\lineto(56.28159332,309.98011295)
\curveto(57.41831207,309.70472232)(58.18003082,309.43812076)(58.56674957,309.18030826)
\curveto(59.18198395,308.77601139)(59.48960114,308.1402692)(59.48960114,307.2730817)
\curveto(59.48960114,306.43519107)(59.1702652,305.71155826)(58.53159332,305.10218326)
\curveto(57.89878082,304.49280826)(56.93198395,304.18812076)(55.6312027,304.18812076)
\curveto(54.23081207,304.18812076)(53.23764801,304.50452701)(52.65171051,305.13733951)
\curveto(52.07163239,305.77601139)(51.76108551,306.56409732)(51.72006989,307.50159732)
\lineto(53.2405777,307.50159732)
\closepath
\moveto(55.57846832,314.19007389)
\lineto(55.57846832,314.19007389)
\closepath
}
}
{
\newrgbcolor{curcolor}{0 0 0}
\pscustom[linestyle=none,fillstyle=solid,fillcolor=curcolor]
{
\newpath
\moveto(72.54135895,311.54456607)
\curveto(73.36167145,311.54456607)(74.00913239,311.70862857)(74.48374176,312.03675357)
\curveto(74.96421051,312.36487857)(75.20444489,312.95667545)(75.20444489,313.8121442)
\curveto(75.20444489,314.73206607)(74.87046051,315.3590192)(74.20249176,315.69300357)
\curveto(73.84506989,315.86878482)(73.36753082,315.95667545)(72.76987457,315.95667545)
\lineto(68.4983902,315.95667545)
\lineto(68.4983902,311.54456607)
\lineto(72.54135895,311.54456607)
\closepath
\moveto(66.74936676,317.45960514)
\lineto(72.72592926,317.45960514)
\curveto(73.71030426,317.45960514)(74.5218277,317.31605045)(75.16049957,317.02894107)
\curveto(76.3733902,316.47815982)(76.97983551,315.46155826)(76.97983551,313.97913639)
\curveto(76.97983551,313.20569889)(76.8187027,312.57288639)(76.49643707,312.08069889)
\curveto(76.18003082,311.58851139)(75.73471832,311.19300357)(75.16049957,310.89417545)
\curveto(75.66440582,310.68909732)(76.04233551,310.41956607)(76.29428864,310.0855817)
\curveto(76.55210114,309.75159732)(76.69565582,309.20960514)(76.7249527,308.45960514)
\lineto(76.78647614,306.72815982)
\curveto(76.80405426,306.23597232)(76.84506989,305.86976139)(76.90952301,305.62952701)
\curveto(77.01499176,305.21937076)(77.20249176,304.95569889)(77.47202301,304.83851139)
\lineto(77.47202301,304.54847232)
\lineto(75.32749176,304.54847232)
\curveto(75.26889801,304.65980045)(75.22202301,304.80335514)(75.18686676,304.97913639)
\curveto(75.15171051,305.15491764)(75.12241364,305.49476139)(75.09897614,305.99866764)
\lineto(74.99350739,308.15198795)
\curveto(74.95249176,308.99573795)(74.6390152,309.56116764)(74.0530777,309.84827701)
\curveto(73.71909332,310.00648014)(73.19467926,310.0855817)(72.47983551,310.0855817)
\lineto(68.4983902,310.0855817)
\lineto(68.4983902,304.54847232)
\lineto(66.74936676,304.54847232)
\lineto(66.74936676,317.45960514)
\closepath
}
}
{
\newrgbcolor{curcolor}{0 0 0}
\pscustom[linestyle=none,fillstyle=solid,fillcolor=curcolor]
{
\newpath
\moveto(83.25522614,314.17249576)
\curveto(83.92319489,314.17249576)(84.57065582,314.01429264)(85.19760895,313.69788639)
\curveto(85.82456207,313.38733951)(86.30210114,312.98304264)(86.63022614,312.48499576)
\curveto(86.94663239,312.01038639)(87.15756989,311.45667545)(87.26303864,310.82386295)
\curveto(87.35678864,310.3902692)(87.40366364,309.69886295)(87.40366364,308.7496442)
\lineto(80.50424957,308.7496442)
\curveto(80.53354645,307.79456607)(80.75913239,307.02698795)(81.18100739,306.44690982)
\curveto(81.60288239,305.87269107)(82.2562027,305.5855817)(83.14096832,305.5855817)
\curveto(83.9671402,305.5855817)(84.62631989,305.85804264)(85.11850739,306.40296451)
\curveto(85.39975739,306.71937076)(85.59897614,307.0855817)(85.71616364,307.50159732)
\lineto(87.2718277,307.50159732)
\curveto(87.23081207,307.1558942)(87.09311676,306.76917545)(86.85874176,306.34144107)
\curveto(86.63022614,305.91956607)(86.37241364,305.57386295)(86.08530426,305.3043317)
\curveto(85.60483551,304.8355817)(85.01010895,304.51917545)(84.30112457,304.35511295)
\curveto(83.9202652,304.26136295)(83.48960114,304.21448795)(83.00913239,304.21448795)
\curveto(81.83725739,304.21448795)(80.84409332,304.63929264)(80.0296402,305.48890201)
\curveto(79.21518707,306.34437076)(78.80796051,307.53968326)(78.80796051,309.07483951)
\curveto(78.80796051,310.58655826)(79.21811676,311.81409732)(80.03842926,312.7574567)
\curveto(80.85874176,313.70081607)(81.93100739,314.17249576)(83.25522614,314.17249576)
\closepath
\moveto(85.77768707,310.00648014)
\curveto(85.71323395,310.69202701)(85.56381989,311.23987857)(85.32944489,311.65003482)
\curveto(84.89585114,312.41175357)(84.17221832,312.79261295)(83.15854645,312.79261295)
\curveto(82.43198395,312.79261295)(81.82260895,312.52894107)(81.33042145,312.00159732)
\curveto(80.83823395,311.48011295)(80.57749176,310.81507389)(80.54819489,310.00648014)
\lineto(85.77768707,310.00648014)
\closepath
\moveto(83.10581207,314.19007389)
\lineto(83.10581207,314.19007389)
\closepath
}
}
{
\newrgbcolor{curcolor}{0 0 0}
\pscustom[linestyle=none,fillstyle=solid,fillcolor=curcolor]
{
\newpath
\moveto(90.2952652,307.50159732)
\curveto(90.3421402,306.97425357)(90.47397614,306.5699567)(90.69077301,306.2887067)
\curveto(91.08921051,305.77894107)(91.78061676,305.52405826)(92.76499176,305.52405826)
\curveto(93.35092926,305.52405826)(93.86655426,305.65003482)(94.31186676,305.90198795)
\curveto(94.75717926,306.15980045)(94.97983551,306.55530826)(94.97983551,307.08851139)
\curveto(94.97983551,307.49280826)(94.80112457,307.80042545)(94.4437027,308.01136295)
\curveto(94.21518707,308.1402692)(93.7640152,308.28968326)(93.09018707,308.45960514)
\lineto(91.83335114,308.77601139)
\curveto(91.03061676,308.97523014)(90.43881989,309.19788639)(90.05796051,309.44398014)
\curveto(89.37827301,309.87171451)(89.03842926,310.46351139)(89.03842926,311.21937076)
\curveto(89.03842926,312.10999576)(89.3577652,312.83069889)(89.99643707,313.38148014)
\curveto(90.64096832,313.93226139)(91.50522614,314.20765201)(92.58921051,314.20765201)
\curveto(94.00717926,314.20765201)(95.0296402,313.79163639)(95.65659332,312.95960514)
\curveto(96.04917145,312.43226139)(96.23960114,311.86390201)(96.22788239,311.25452701)
\lineto(94.73374176,311.25452701)
\curveto(94.70444489,311.61194889)(94.57846832,311.9371442)(94.35581207,312.23011295)
\curveto(93.99253082,312.64612857)(93.36264801,312.85413639)(92.46616364,312.85413639)
\curveto(91.86850739,312.85413639)(91.41440582,312.73987857)(91.10385895,312.51136295)
\curveto(90.79917145,312.28284732)(90.6468277,311.98108951)(90.6468277,311.60608951)
\curveto(90.6468277,311.19593326)(90.84897614,310.86780826)(91.25327301,310.62171451)
\curveto(91.48764801,310.47523014)(91.83335114,310.34632389)(92.29038239,310.23499576)
\lineto(93.33628082,309.98011295)
\curveto(94.47299957,309.70472232)(95.23471832,309.43812076)(95.62143707,309.18030826)
\curveto(96.23667145,308.77601139)(96.54428864,308.1402692)(96.54428864,307.2730817)
\curveto(96.54428864,306.43519107)(96.2249527,305.71155826)(95.58628082,305.10218326)
\curveto(94.95346832,304.49280826)(93.98667145,304.18812076)(92.6858902,304.18812076)
\curveto(91.28549957,304.18812076)(90.29233551,304.50452701)(89.70639801,305.13733951)
\curveto(89.12631989,305.77601139)(88.81577301,306.56409732)(88.77475739,307.50159732)
\lineto(90.2952652,307.50159732)
\closepath
\moveto(92.63315582,314.19007389)
\lineto(92.63315582,314.19007389)
\closepath
}
}
{
\newrgbcolor{curcolor}{0 0 0}
\pscustom[linestyle=none,fillstyle=solid,fillcolor=curcolor]
{
\newpath
\moveto(102.12534332,305.56800357)
\curveto(103.17417145,305.56800357)(103.89194489,305.96351139)(104.27866364,306.75452701)
\curveto(104.67124176,307.55140201)(104.86753082,308.43616764)(104.86753082,309.40882389)
\curveto(104.86753082,310.28773014)(104.72690582,311.00257389)(104.44565582,311.55335514)
\curveto(104.00034332,312.42054264)(103.2327652,312.85413639)(102.14292145,312.85413639)
\curveto(101.17612457,312.85413639)(100.47299957,312.48499576)(100.03354645,311.74671451)
\curveto(99.59409332,311.00843326)(99.37436676,310.11780826)(99.37436676,309.07483951)
\curveto(99.37436676,308.07288639)(99.59409332,307.23792545)(100.03354645,306.5699567)
\curveto(100.47299957,305.90198795)(101.1702652,305.56800357)(102.12534332,305.56800357)
\closepath
\moveto(102.18686676,314.2340192)
\curveto(103.39975739,314.2340192)(104.42514801,313.82972232)(105.26303864,313.02112857)
\curveto(106.10092926,312.21253482)(106.51987457,311.0230817)(106.51987457,309.4527692)
\curveto(106.51987457,307.93519107)(106.15073395,306.68128482)(105.4124527,305.69105045)
\curveto(104.67417145,304.70081607)(103.52866364,304.20569889)(101.97592926,304.20569889)
\curveto(100.68100739,304.20569889)(99.65268707,304.64222232)(98.89096832,305.5152692)
\curveto(98.12924957,306.39417545)(97.7483902,307.57190982)(97.7483902,309.04847232)
\curveto(97.7483902,310.63050357)(98.14975739,311.8902692)(98.95249176,312.8277692)
\curveto(99.75522614,313.7652692)(100.83335114,314.2340192)(102.18686676,314.2340192)
\closepath
\moveto(102.13413239,314.19007389)
\lineto(102.13413239,314.19007389)
\closepath
}
}
{
\newrgbcolor{curcolor}{0 0 0}
\pscustom[linestyle=none,fillstyle=solid,fillcolor=curcolor]
{
\newpath
\moveto(108.45346832,317.45960514)
\lineto(110.03549957,317.45960514)
\lineto(110.03549957,304.54847232)
\lineto(108.45346832,304.54847232)
\lineto(108.45346832,317.45960514)
\closepath
}
}
{
\newrgbcolor{curcolor}{0 0 0}
\pscustom[linestyle=none,fillstyle=solid,fillcolor=curcolor]
{
\newpath
\moveto(113.99936676,313.96155826)
\lineto(113.99936676,307.71253482)
\curveto(113.99936676,307.23206607)(114.07553864,306.83948795)(114.22788239,306.53480045)
\curveto(114.50913239,305.97230045)(115.03354645,305.69105045)(115.80112457,305.69105045)
\curveto(116.90268707,305.69105045)(117.65268707,306.18323795)(118.05112457,307.16761295)
\curveto(118.26792145,307.6949567)(118.37631989,308.41858951)(118.37631989,309.33851139)
\lineto(118.37631989,313.96155826)
\lineto(119.95835114,313.96155826)
\lineto(119.95835114,304.54847232)
\lineto(118.46421051,304.54847232)
\lineto(118.48178864,305.9371442)
\curveto(118.27671051,305.57972232)(118.0218277,305.27796451)(117.7171402,305.03187076)
\curveto(117.11362457,304.53968326)(116.3812027,304.29358951)(115.51987457,304.29358951)
\curveto(114.1780777,304.29358951)(113.2640152,304.7418317)(112.77768707,305.63831607)
\curveto(112.5140152,306.11878482)(112.38217926,306.76038639)(112.38217926,307.56312076)
\lineto(112.38217926,313.96155826)
\lineto(113.99936676,313.96155826)
\closepath
\moveto(116.1702652,314.19007389)
\lineto(116.1702652,314.19007389)
\closepath
}
}
{
\newrgbcolor{curcolor}{0 0 0}
\pscustom[linestyle=none,fillstyle=solid,fillcolor=curcolor]
{
\newpath
\moveto(122.75327301,316.58948795)
\lineto(124.35288239,316.58948795)
\lineto(124.35288239,313.96155826)
\lineto(125.85581207,313.96155826)
\lineto(125.85581207,312.66956607)
\lineto(124.35288239,312.66956607)
\lineto(124.35288239,306.52601139)
\curveto(124.35288239,306.19788639)(124.46421051,305.97815982)(124.68686676,305.8668317)
\curveto(124.80991364,305.80237857)(125.01499176,305.77015201)(125.30210114,305.77015201)
\lineto(125.54819489,305.77015201)
\curveto(125.63608551,305.77601139)(125.73862457,305.78480045)(125.85581207,305.7965192)
\lineto(125.85581207,304.54847232)
\curveto(125.67417145,304.49573795)(125.48374176,304.45765201)(125.28452301,304.43421451)
\curveto(125.09116364,304.41077701)(124.88022614,304.39905826)(124.65171051,304.39905826)
\curveto(123.91342926,304.39905826)(123.4124527,304.58655826)(123.14878082,304.96155826)
\curveto(122.88510895,305.34241764)(122.75327301,305.83460514)(122.75327301,306.43812076)
\lineto(122.75327301,312.66956607)
\lineto(121.47885895,312.66956607)
\lineto(121.47885895,313.96155826)
\lineto(122.75327301,313.96155826)
\lineto(122.75327301,316.58948795)
\closepath
}
}
{
\newrgbcolor{curcolor}{0 0 0}
\pscustom[linestyle=none,fillstyle=solid,fillcolor=curcolor]
{
\newpath
\moveto(127.46421051,313.91761295)
\lineto(129.07260895,313.91761295)
\lineto(129.07260895,304.54847232)
\lineto(127.46421051,304.54847232)
\lineto(127.46421051,313.91761295)
\closepath
\moveto(127.46421051,317.45960514)
\lineto(129.07260895,317.45960514)
\lineto(129.07260895,315.66663639)
\lineto(127.46421051,315.66663639)
\lineto(127.46421051,317.45960514)
\closepath
}
}
{
\newrgbcolor{curcolor}{0 0 0}
\pscustom[linestyle=none,fillstyle=solid,fillcolor=curcolor]
{
\newpath
\moveto(135.20737457,305.56800357)
\curveto(136.2562027,305.56800357)(136.97397614,305.96351139)(137.36069489,306.75452701)
\curveto(137.75327301,307.55140201)(137.94956207,308.43616764)(137.94956207,309.40882389)
\curveto(137.94956207,310.28773014)(137.80893707,311.00257389)(137.52768707,311.55335514)
\curveto(137.08237457,312.42054264)(136.31479645,312.85413639)(135.2249527,312.85413639)
\curveto(134.25815582,312.85413639)(133.55503082,312.48499576)(133.1155777,311.74671451)
\curveto(132.67612457,311.00843326)(132.45639801,310.11780826)(132.45639801,309.07483951)
\curveto(132.45639801,308.07288639)(132.67612457,307.23792545)(133.1155777,306.5699567)
\curveto(133.55503082,305.90198795)(134.25229645,305.56800357)(135.20737457,305.56800357)
\closepath
\moveto(135.26889801,314.2340192)
\curveto(136.48178864,314.2340192)(137.50717926,313.82972232)(138.34506989,313.02112857)
\curveto(139.18296051,312.21253482)(139.60190582,311.0230817)(139.60190582,309.4527692)
\curveto(139.60190582,307.93519107)(139.2327652,306.68128482)(138.49448395,305.69105045)
\curveto(137.7562027,304.70081607)(136.61069489,304.20569889)(135.05796051,304.20569889)
\curveto(133.76303864,304.20569889)(132.73471832,304.64222232)(131.97299957,305.5152692)
\curveto(131.21128082,306.39417545)(130.83042145,307.57190982)(130.83042145,309.04847232)
\curveto(130.83042145,310.63050357)(131.23178864,311.8902692)(132.03452301,312.8277692)
\curveto(132.83725739,313.7652692)(133.91538239,314.2340192)(135.26889801,314.2340192)
\closepath
\moveto(135.21616364,314.19007389)
\lineto(135.21616364,314.19007389)
\closepath
}
}
{
\newrgbcolor{curcolor}{0 0 0}
\pscustom[linestyle=none,fillstyle=solid,fillcolor=curcolor]
{
\newpath
\moveto(141.49155426,313.96155826)
\lineto(142.99448395,313.96155826)
\lineto(142.99448395,312.62562076)
\curveto(143.43979645,313.17640201)(143.91147614,313.57190982)(144.40952301,313.8121442)
\curveto(144.90756989,314.05237857)(145.46128082,314.17249576)(146.07065582,314.17249576)
\curveto(147.40659332,314.17249576)(148.30893707,313.70667545)(148.77768707,312.77503482)
\curveto(149.03549957,312.2652692)(149.16440582,311.53577701)(149.16440582,310.58655826)
\lineto(149.16440582,304.54847232)
\lineto(147.55600739,304.54847232)
\lineto(147.55600739,310.48108951)
\curveto(147.55600739,311.05530826)(147.47104645,311.51819889)(147.30112457,311.86976139)
\curveto(147.01987457,312.45569889)(146.51010895,312.74866764)(145.7718277,312.74866764)
\curveto(145.3968277,312.74866764)(145.08921051,312.7105817)(144.84897614,312.63440982)
\curveto(144.41538239,312.50550357)(144.03452301,312.24769107)(143.70639801,311.86097232)
\curveto(143.44272614,311.55042545)(143.26987457,311.22815982)(143.18784332,310.89417545)
\curveto(143.11167145,310.56605045)(143.07358551,310.09437076)(143.07358551,309.47913639)
\lineto(143.07358551,304.54847232)
\lineto(141.49155426,304.54847232)
\lineto(141.49155426,313.96155826)
\closepath
\moveto(145.2093277,314.19007389)
\lineto(145.2093277,314.19007389)
\closepath
}
}
{
\newrgbcolor{curcolor}{0 0 0}
\pscustom[linestyle=none,fillstyle=solid,fillcolor=curcolor]
{
\newpath
\moveto(295.34686279,201.43529789)
\lineto(297.38592529,201.43529789)
\lineto(301.09490967,195.23021976)
\lineto(304.80389404,201.43529789)
\lineto(306.85174561,201.43529789)
\lineto(301.97381592,193.72729007)
\lineto(301.97381592,188.52416507)
\lineto(300.22479248,188.52416507)
\lineto(300.22479248,193.72729007)
\lineto(295.34686279,201.43529789)
\closepath
\moveto(301.12127686,201.43529789)
\lineto(301.12127686,201.43529789)
\closepath
}
}
{
\newrgbcolor{curcolor}{0 0 0}
\pscustom[linestyle=none,fillstyle=solid,fillcolor=curcolor]
{
\newpath
\moveto(319.99139404,193.8151807)
\lineto(318.03143311,199.51928226)
\lineto(315.94842529,193.8151807)
\lineto(319.99139404,193.8151807)
\closepath
\moveto(317.11737061,201.43529789)
\lineto(319.09490967,201.43529789)
\lineto(323.77947998,188.52416507)
\lineto(321.86346436,188.52416507)
\lineto(320.55389404,192.39135257)
\lineto(315.44744873,192.39135257)
\lineto(314.04998779,188.52416507)
\lineto(312.25701904,188.52416507)
\lineto(317.11737061,201.43529789)
\closepath
\moveto(318.02264404,201.43529789)
\lineto(318.02264404,201.43529789)
\closepath
}
}
{
\newrgbcolor{curcolor}{0 0 0}
\pscustom[linestyle=none,fillstyle=solid,fillcolor=curcolor]
{
\newpath
\moveto(324.28045654,197.93725101)
\lineto(326.32830811,197.93725101)
\lineto(328.49041748,194.62377445)
\lineto(330.67889404,197.93725101)
\lineto(332.60369873,197.8933057)
\lineto(329.43084717,193.34936039)
\lineto(332.74432373,188.52416507)
\lineto(330.72283936,188.52416507)
\lineto(328.38494873,192.0573682)
\lineto(326.11737061,188.52416507)
\lineto(324.11346436,188.52416507)
\lineto(327.42694092,193.34936039)
\lineto(324.28045654,197.93725101)
\closepath
}
}
{
\newrgbcolor{curcolor}{0 0 0}
\pscustom[linestyle=none,fillstyle=solid,fillcolor=curcolor]
{
\newpath
\moveto(334.17694092,197.8933057)
\lineto(335.78533936,197.8933057)
\lineto(335.78533936,188.52416507)
\lineto(334.17694092,188.52416507)
\lineto(334.17694092,197.8933057)
\closepath
\moveto(334.17694092,201.43529789)
\lineto(335.78533936,201.43529789)
\lineto(335.78533936,199.64232914)
\lineto(334.17694092,199.64232914)
\lineto(334.17694092,201.43529789)
\closepath
}
}
{
\newrgbcolor{curcolor}{0 0 0}
\pscustom[linestyle=none,fillstyle=solid,fillcolor=curcolor]
{
\newpath
\moveto(339.12518311,191.47729007)
\curveto(339.17205811,190.94994632)(339.30389404,190.54564945)(339.52069092,190.26439945)
\curveto(339.91912842,189.75463382)(340.61053467,189.49975101)(341.59490967,189.49975101)
\curveto(342.18084717,189.49975101)(342.69647217,189.62572757)(343.14178467,189.8776807)
\curveto(343.58709717,190.1354932)(343.80975342,190.53100101)(343.80975342,191.06420414)
\curveto(343.80975342,191.46850101)(343.63104248,191.7761182)(343.27362061,191.9870557)
\curveto(343.04510498,192.11596195)(342.59393311,192.26537601)(341.92010498,192.43529789)
\lineto(340.66326904,192.75170414)
\curveto(339.86053467,192.95092289)(339.26873779,193.17357914)(338.88787842,193.41967289)
\curveto(338.20819092,193.84740726)(337.86834717,194.43920414)(337.86834717,195.19506351)
\curveto(337.86834717,196.08568851)(338.18768311,196.80639164)(338.82635498,197.35717289)
\curveto(339.47088623,197.90795414)(340.33514404,198.18334476)(341.41912842,198.18334476)
\curveto(342.83709717,198.18334476)(343.85955811,197.76732914)(344.48651123,196.93529789)
\curveto(344.87908936,196.40795414)(345.06951904,195.83959476)(345.05780029,195.23021976)
\lineto(343.56365967,195.23021976)
\curveto(343.53436279,195.58764164)(343.40838623,195.91283695)(343.18572998,196.2058057)
\curveto(342.82244873,196.62182132)(342.19256592,196.82982914)(341.29608154,196.82982914)
\curveto(340.69842529,196.82982914)(340.24432373,196.71557132)(339.93377686,196.4870557)
\curveto(339.62908936,196.25854007)(339.47674561,195.95678226)(339.47674561,195.58178226)
\curveto(339.47674561,195.17162601)(339.67889404,194.84350101)(340.08319092,194.59740726)
\curveto(340.31756592,194.45092289)(340.66326904,194.32201664)(341.12030029,194.21068851)
\lineto(342.16619873,193.9558057)
\curveto(343.30291748,193.68041507)(344.06463623,193.41381351)(344.45135498,193.15600101)
\curveto(345.06658936,192.75170414)(345.37420654,192.11596195)(345.37420654,191.24877445)
\curveto(345.37420654,190.41088382)(345.05487061,189.68725101)(344.41619873,189.07787601)
\curveto(343.78338623,188.46850101)(342.81658936,188.16381351)(341.51580811,188.16381351)
\curveto(340.11541748,188.16381351)(339.12225342,188.48021976)(338.53631592,189.11303226)
\curveto(337.95623779,189.75170414)(337.64569092,190.53979007)(337.60467529,191.47729007)
\lineto(339.12518311,191.47729007)
\closepath
\moveto(341.46307373,198.16576664)
\lineto(341.46307373,198.16576664)
\closepath
}
}
{
\newrgbcolor{curcolor}{0 0 0}
\pscustom[linestyle=none,fillstyle=solid,fillcolor=curcolor]
{
\newpath
\moveto(358.42596436,195.52025882)
\curveto(359.24627686,195.52025882)(359.89373779,195.68432132)(360.36834717,196.01244632)
\curveto(360.84881592,196.34057132)(361.08905029,196.9323682)(361.08905029,197.78783695)
\curveto(361.08905029,198.70775882)(360.75506592,199.33471195)(360.08709717,199.66869632)
\curveto(359.72967529,199.84447757)(359.25213623,199.9323682)(358.65447998,199.9323682)
\lineto(354.38299561,199.9323682)
\lineto(354.38299561,195.52025882)
\lineto(358.42596436,195.52025882)
\closepath
\moveto(352.63397217,201.43529789)
\lineto(358.61053467,201.43529789)
\curveto(359.59490967,201.43529789)(360.40643311,201.2917432)(361.04510498,201.00463382)
\curveto(362.25799561,200.45385257)(362.86444092,199.43725101)(362.86444092,197.95482914)
\curveto(362.86444092,197.18139164)(362.70330811,196.54857914)(362.38104248,196.05639164)
\curveto(362.06463623,195.56420414)(361.61932373,195.16869632)(361.04510498,194.8698682)
\curveto(361.54901123,194.66479007)(361.92694092,194.39525882)(362.17889404,194.06127445)
\curveto(362.43670654,193.72729007)(362.58026123,193.18529789)(362.60955811,192.43529789)
\lineto(362.67108154,190.70385257)
\curveto(362.68865967,190.21166507)(362.72967529,189.84545414)(362.79412842,189.60521976)
\curveto(362.89959717,189.19506351)(363.08709717,188.93139164)(363.35662842,188.81420414)
\lineto(363.35662842,188.52416507)
\lineto(361.21209717,188.52416507)
\curveto(361.15350342,188.6354932)(361.10662842,188.77904789)(361.07147217,188.95482914)
\curveto(361.03631592,189.13061039)(361.00701904,189.47045414)(360.98358154,189.97436039)
\lineto(360.87811279,192.1276807)
\curveto(360.83709717,192.9714307)(360.52362061,193.53686039)(359.93768311,193.82396976)
\curveto(359.60369873,193.98217289)(359.07928467,194.06127445)(358.36444092,194.06127445)
\lineto(354.38299561,194.06127445)
\lineto(354.38299561,188.52416507)
\lineto(352.63397217,188.52416507)
\lineto(352.63397217,201.43529789)
\closepath
}
}
{
\newrgbcolor{curcolor}{0 0 0}
\pscustom[linestyle=none,fillstyle=solid,fillcolor=curcolor]
{
\newpath
\moveto(369.13983154,198.14818851)
\curveto(369.80780029,198.14818851)(370.45526123,197.98998539)(371.08221436,197.67357914)
\curveto(371.70916748,197.36303226)(372.18670654,196.95873539)(372.51483154,196.46068851)
\curveto(372.83123779,195.98607914)(373.04217529,195.4323682)(373.14764404,194.7995557)
\curveto(373.24139404,194.36596195)(373.28826904,193.6745557)(373.28826904,192.72533695)
\lineto(366.38885498,192.72533695)
\curveto(366.41815186,191.77025882)(366.64373779,191.0026807)(367.06561279,190.42260257)
\curveto(367.48748779,189.84838382)(368.14080811,189.56127445)(369.02557373,189.56127445)
\curveto(369.85174561,189.56127445)(370.51092529,189.83373539)(371.00311279,190.37865726)
\curveto(371.28436279,190.69506351)(371.48358154,191.06127445)(371.60076904,191.47729007)
\lineto(373.15643311,191.47729007)
\curveto(373.11541748,191.13158695)(372.97772217,190.7448682)(372.74334717,190.31713382)
\curveto(372.51483154,189.89525882)(372.25701904,189.5495557)(371.96990967,189.28002445)
\curveto(371.48944092,188.81127445)(370.89471436,188.4948682)(370.18572998,188.3308057)
\curveto(369.80487061,188.2370557)(369.37420654,188.1901807)(368.89373779,188.1901807)
\curveto(367.72186279,188.1901807)(366.72869873,188.61498539)(365.91424561,189.46459476)
\curveto(365.09979248,190.32006351)(364.69256592,191.51537601)(364.69256592,193.05053226)
\curveto(364.69256592,194.56225101)(365.10272217,195.78979007)(365.92303467,196.73314945)
\curveto(366.74334717,197.67650882)(367.81561279,198.14818851)(369.13983154,198.14818851)
\closepath
\moveto(371.66229248,193.98217289)
\curveto(371.59783936,194.66771976)(371.44842529,195.21557132)(371.21405029,195.62572757)
\curveto(370.78045654,196.38744632)(370.05682373,196.7683057)(369.04315186,196.7683057)
\curveto(368.31658936,196.7683057)(367.70721436,196.50463382)(367.21502686,195.97729007)
\curveto(366.72283936,195.4558057)(366.46209717,194.79076664)(366.43280029,193.98217289)
\lineto(371.66229248,193.98217289)
\closepath
\moveto(368.99041748,198.16576664)
\lineto(368.99041748,198.16576664)
\closepath
}
}
{
\newrgbcolor{curcolor}{0 0 0}
\pscustom[linestyle=none,fillstyle=solid,fillcolor=curcolor]
{
\newpath
\moveto(376.17987061,191.47729007)
\curveto(376.22674561,190.94994632)(376.35858154,190.54564945)(376.57537842,190.26439945)
\curveto(376.97381592,189.75463382)(377.66522217,189.49975101)(378.64959717,189.49975101)
\curveto(379.23553467,189.49975101)(379.75115967,189.62572757)(380.19647217,189.8776807)
\curveto(380.64178467,190.1354932)(380.86444092,190.53100101)(380.86444092,191.06420414)
\curveto(380.86444092,191.46850101)(380.68572998,191.7761182)(380.32830811,191.9870557)
\curveto(380.09979248,192.11596195)(379.64862061,192.26537601)(378.97479248,192.43529789)
\lineto(377.71795654,192.75170414)
\curveto(376.91522217,192.95092289)(376.32342529,193.17357914)(375.94256592,193.41967289)
\curveto(375.26287842,193.84740726)(374.92303467,194.43920414)(374.92303467,195.19506351)
\curveto(374.92303467,196.08568851)(375.24237061,196.80639164)(375.88104248,197.35717289)
\curveto(376.52557373,197.90795414)(377.38983154,198.18334476)(378.47381592,198.18334476)
\curveto(379.89178467,198.18334476)(380.91424561,197.76732914)(381.54119873,196.93529789)
\curveto(381.93377686,196.40795414)(382.12420654,195.83959476)(382.11248779,195.23021976)
\lineto(380.61834717,195.23021976)
\curveto(380.58905029,195.58764164)(380.46307373,195.91283695)(380.24041748,196.2058057)
\curveto(379.87713623,196.62182132)(379.24725342,196.82982914)(378.35076904,196.82982914)
\curveto(377.75311279,196.82982914)(377.29901123,196.71557132)(376.98846436,196.4870557)
\curveto(376.68377686,196.25854007)(376.53143311,195.95678226)(376.53143311,195.58178226)
\curveto(376.53143311,195.17162601)(376.73358154,194.84350101)(377.13787842,194.59740726)
\curveto(377.37225342,194.45092289)(377.71795654,194.32201664)(378.17498779,194.21068851)
\lineto(379.22088623,193.9558057)
\curveto(380.35760498,193.68041507)(381.11932373,193.41381351)(381.50604248,193.15600101)
\curveto(382.12127686,192.75170414)(382.42889404,192.11596195)(382.42889404,191.24877445)
\curveto(382.42889404,190.41088382)(382.10955811,189.68725101)(381.47088623,189.07787601)
\curveto(380.83807373,188.46850101)(379.87127686,188.16381351)(378.57049561,188.16381351)
\curveto(377.17010498,188.16381351)(376.17694092,188.48021976)(375.59100342,189.11303226)
\curveto(375.01092529,189.75170414)(374.70037842,190.53979007)(374.65936279,191.47729007)
\lineto(376.17987061,191.47729007)
\closepath
\moveto(378.51776123,198.16576664)
\lineto(378.51776123,198.16576664)
\closepath
}
}
{
\newrgbcolor{curcolor}{0 0 0}
\pscustom[linestyle=none,fillstyle=solid,fillcolor=curcolor]
{
\newpath
\moveto(388.00994873,189.54369632)
\curveto(389.05877686,189.54369632)(389.77655029,189.93920414)(390.16326904,190.73021976)
\curveto(390.55584717,191.52709476)(390.75213623,192.41186039)(390.75213623,193.38451664)
\curveto(390.75213623,194.26342289)(390.61151123,194.97826664)(390.33026123,195.52904789)
\curveto(389.88494873,196.39623539)(389.11737061,196.82982914)(388.02752686,196.82982914)
\curveto(387.06072998,196.82982914)(386.35760498,196.46068851)(385.91815186,195.72240726)
\curveto(385.47869873,194.98412601)(385.25897217,194.09350101)(385.25897217,193.05053226)
\curveto(385.25897217,192.04857914)(385.47869873,191.2136182)(385.91815186,190.54564945)
\curveto(386.35760498,189.8776807)(387.05487061,189.54369632)(388.00994873,189.54369632)
\closepath
\moveto(388.07147217,198.20971195)
\curveto(389.28436279,198.20971195)(390.30975342,197.80541507)(391.14764404,196.99682132)
\curveto(391.98553467,196.18822757)(392.40447998,194.99877445)(392.40447998,193.42846195)
\curveto(392.40447998,191.91088382)(392.03533936,190.65697757)(391.29705811,189.6667432)
\curveto(390.55877686,188.67650882)(389.41326904,188.18139164)(387.86053467,188.18139164)
\curveto(386.56561279,188.18139164)(385.53729248,188.61791507)(384.77557373,189.49096195)
\curveto(384.01385498,190.3698682)(383.63299561,191.54760257)(383.63299561,193.02416507)
\curveto(383.63299561,194.60619632)(384.03436279,195.86596195)(384.83709717,196.80346195)
\curveto(385.63983154,197.74096195)(386.71795654,198.20971195)(388.07147217,198.20971195)
\closepath
\moveto(388.01873779,198.16576664)
\lineto(388.01873779,198.16576664)
\closepath
}
}
{
\newrgbcolor{curcolor}{0 0 0}
\pscustom[linestyle=none,fillstyle=solid,fillcolor=curcolor]
{
\newpath
\moveto(394.33807373,201.43529789)
\lineto(395.92010498,201.43529789)
\lineto(395.92010498,188.52416507)
\lineto(394.33807373,188.52416507)
\lineto(394.33807373,201.43529789)
\closepath
}
}
{
\newrgbcolor{curcolor}{0 0 0}
\pscustom[linestyle=none,fillstyle=solid,fillcolor=curcolor]
{
\newpath
\moveto(399.88397217,197.93725101)
\lineto(399.88397217,191.68822757)
\curveto(399.88397217,191.20775882)(399.96014404,190.8151807)(400.11248779,190.5104932)
\curveto(400.39373779,189.9479932)(400.91815186,189.6667432)(401.68572998,189.6667432)
\curveto(402.78729248,189.6667432)(403.53729248,190.1589307)(403.93572998,191.1433057)
\curveto(404.15252686,191.67064945)(404.26092529,192.39428226)(404.26092529,193.31420414)
\lineto(404.26092529,197.93725101)
\lineto(405.84295654,197.93725101)
\lineto(405.84295654,188.52416507)
\lineto(404.34881592,188.52416507)
\lineto(404.36639404,189.91283695)
\curveto(404.16131592,189.55541507)(403.90643311,189.25365726)(403.60174561,189.00756351)
\curveto(402.99822998,188.51537601)(402.26580811,188.26928226)(401.40447998,188.26928226)
\curveto(400.06268311,188.26928226)(399.14862061,188.71752445)(398.66229248,189.61400882)
\curveto(398.39862061,190.09447757)(398.26678467,190.73607914)(398.26678467,191.53881351)
\lineto(398.26678467,197.93725101)
\lineto(399.88397217,197.93725101)
\closepath
\moveto(402.05487061,198.16576664)
\lineto(402.05487061,198.16576664)
\closepath
}
}
{
\newrgbcolor{curcolor}{0 0 0}
\pscustom[linestyle=none,fillstyle=solid,fillcolor=curcolor]
{
\newpath
\moveto(408.63787842,200.5651807)
\lineto(410.23748779,200.5651807)
\lineto(410.23748779,197.93725101)
\lineto(411.74041748,197.93725101)
\lineto(411.74041748,196.64525882)
\lineto(410.23748779,196.64525882)
\lineto(410.23748779,190.50170414)
\curveto(410.23748779,190.17357914)(410.34881592,189.95385257)(410.57147217,189.84252445)
\curveto(410.69451904,189.77807132)(410.89959717,189.74584476)(411.18670654,189.74584476)
\lineto(411.43280029,189.74584476)
\curveto(411.52069092,189.75170414)(411.62322998,189.7604932)(411.74041748,189.77221195)
\lineto(411.74041748,188.52416507)
\curveto(411.55877686,188.4714307)(411.36834717,188.43334476)(411.16912842,188.40990726)
\curveto(410.97576904,188.38646976)(410.76483154,188.37475101)(410.53631592,188.37475101)
\curveto(409.79803467,188.37475101)(409.29705811,188.56225101)(409.03338623,188.93725101)
\curveto(408.76971436,189.31811039)(408.63787842,189.81029789)(408.63787842,190.41381351)
\lineto(408.63787842,196.64525882)
\lineto(407.36346436,196.64525882)
\lineto(407.36346436,197.93725101)
\lineto(408.63787842,197.93725101)
\lineto(408.63787842,200.5651807)
\closepath
}
}
{
\newrgbcolor{curcolor}{0 0 0}
\pscustom[linestyle=none,fillstyle=solid,fillcolor=curcolor]
{
\newpath
\moveto(413.34881592,197.8933057)
\lineto(414.95721436,197.8933057)
\lineto(414.95721436,188.52416507)
\lineto(413.34881592,188.52416507)
\lineto(413.34881592,197.8933057)
\closepath
\moveto(413.34881592,201.43529789)
\lineto(414.95721436,201.43529789)
\lineto(414.95721436,199.64232914)
\lineto(413.34881592,199.64232914)
\lineto(413.34881592,201.43529789)
\closepath
}
}
{
\newrgbcolor{curcolor}{0 0 0}
\pscustom[linestyle=none,fillstyle=solid,fillcolor=curcolor]
{
\newpath
\moveto(421.09197998,189.54369632)
\curveto(422.14080811,189.54369632)(422.85858154,189.93920414)(423.24530029,190.73021976)
\curveto(423.63787842,191.52709476)(423.83416748,192.41186039)(423.83416748,193.38451664)
\curveto(423.83416748,194.26342289)(423.69354248,194.97826664)(423.41229248,195.52904789)
\curveto(422.96697998,196.39623539)(422.19940186,196.82982914)(421.10955811,196.82982914)
\curveto(420.14276123,196.82982914)(419.43963623,196.46068851)(419.00018311,195.72240726)
\curveto(418.56072998,194.98412601)(418.34100342,194.09350101)(418.34100342,193.05053226)
\curveto(418.34100342,192.04857914)(418.56072998,191.2136182)(419.00018311,190.54564945)
\curveto(419.43963623,189.8776807)(420.13690186,189.54369632)(421.09197998,189.54369632)
\closepath
\moveto(421.15350342,198.20971195)
\curveto(422.36639404,198.20971195)(423.39178467,197.80541507)(424.22967529,196.99682132)
\curveto(425.06756592,196.18822757)(425.48651123,194.99877445)(425.48651123,193.42846195)
\curveto(425.48651123,191.91088382)(425.11737061,190.65697757)(424.37908936,189.6667432)
\curveto(423.64080811,188.67650882)(422.49530029,188.18139164)(420.94256592,188.18139164)
\curveto(419.64764404,188.18139164)(418.61932373,188.61791507)(417.85760498,189.49096195)
\curveto(417.09588623,190.3698682)(416.71502686,191.54760257)(416.71502686,193.02416507)
\curveto(416.71502686,194.60619632)(417.11639404,195.86596195)(417.91912842,196.80346195)
\curveto(418.72186279,197.74096195)(419.79998779,198.20971195)(421.15350342,198.20971195)
\closepath
\moveto(421.10076904,198.16576664)
\lineto(421.10076904,198.16576664)
\closepath
}
}
{
\newrgbcolor{curcolor}{0 0 0}
\pscustom[linestyle=none,fillstyle=solid,fillcolor=curcolor]
{
\newpath
\moveto(427.37615967,197.93725101)
\lineto(428.87908936,197.93725101)
\lineto(428.87908936,196.60131351)
\curveto(429.32440186,197.15209476)(429.79608154,197.54760257)(430.29412842,197.78783695)
\curveto(430.79217529,198.02807132)(431.34588623,198.14818851)(431.95526123,198.14818851)
\curveto(433.29119873,198.14818851)(434.19354248,197.6823682)(434.66229248,196.75072757)
\curveto(434.92010498,196.24096195)(435.04901123,195.51146976)(435.04901123,194.56225101)
\lineto(435.04901123,188.52416507)
\lineto(433.44061279,188.52416507)
\lineto(433.44061279,194.45678226)
\curveto(433.44061279,195.03100101)(433.35565186,195.49389164)(433.18572998,195.84545414)
\curveto(432.90447998,196.43139164)(432.39471436,196.72436039)(431.65643311,196.72436039)
\curveto(431.28143311,196.72436039)(430.97381592,196.68627445)(430.73358154,196.61010257)
\curveto(430.29998779,196.48119632)(429.91912842,196.22338382)(429.59100342,195.83666507)
\curveto(429.32733154,195.5261182)(429.15447998,195.20385257)(429.07244873,194.8698682)
\curveto(428.99627686,194.5417432)(428.95819092,194.07006351)(428.95819092,193.45482914)
\lineto(428.95819092,188.52416507)
\lineto(427.37615967,188.52416507)
\lineto(427.37615967,197.93725101)
\closepath
\moveto(431.09393311,198.16576664)
\lineto(431.09393311,198.16576664)
\closepath
}
}
{
\newrgbcolor{curcolor}{0 0 0}
\pscustom[linestyle=none,fillstyle=solid,fillcolor=curcolor]
{
\newpath
\moveto(43.01367188,7.91894046)
\curveto(43.0546875,7.18651859)(43.22753906,6.59179203)(43.53222656,6.13476078)
\curveto(44.11230469,5.27929203)(45.13476562,4.85155765)(46.59960938,4.85155765)
\curveto(47.25585938,4.85155765)(47.85351562,4.94530765)(48.39257812,5.13280765)
\curveto(49.43554688,5.4960889)(49.95703125,6.14647953)(49.95703125,7.08397953)
\curveto(49.95703125,7.78710453)(49.73730469,8.28808109)(49.29785156,8.58690921)
\curveto(48.85253906,8.87987796)(48.15527344,9.13476078)(47.20605469,9.35155765)
\lineto(45.45703125,9.74706546)
\curveto(44.31445312,10.00487796)(43.50585938,10.28905765)(43.03125,10.59960453)
\curveto(42.2109375,11.13866703)(41.80078125,11.94433109)(41.80078125,13.01659671)
\curveto(41.80078125,14.17675296)(42.20214844,15.1289014)(43.00488281,15.87304203)
\curveto(43.80761719,16.61718265)(44.94433594,16.98925296)(46.41503906,16.98925296)
\curveto(47.76855469,16.98925296)(48.91699219,16.66112796)(49.86035156,16.00487796)
\curveto(50.80957031,15.35448734)(51.28417969,14.31151859)(51.28417969,12.87597171)
\lineto(49.640625,12.87597171)
\curveto(49.55273438,13.56737796)(49.36523438,14.0976514)(49.078125,14.46679203)
\curveto(48.54492188,15.14062015)(47.63964844,15.47753421)(46.36230469,15.47753421)
\curveto(45.33105469,15.47753421)(44.58984375,15.26073734)(44.13867188,14.82714359)
\curveto(43.6875,14.39354984)(43.46191406,13.88964359)(43.46191406,13.31542484)
\curveto(43.46191406,12.68261234)(43.72558594,12.21972171)(44.25292969,11.92675296)
\curveto(44.59863281,11.73925296)(45.38085938,11.50487796)(46.59960938,11.22362796)
\lineto(48.41015625,10.81054203)
\curveto(49.28320312,10.61132328)(49.95703125,10.33886234)(50.43164062,9.99315921)
\curveto(51.25195312,9.38964359)(51.66210938,8.51366703)(51.66210938,7.36522953)
\curveto(51.66210938,5.93554203)(51.140625,4.91308109)(50.09765625,4.29784671)
\curveto(49.06054688,3.68261234)(47.85351562,3.37499515)(46.4765625,3.37499515)
\curveto(44.87109375,3.37499515)(43.61425781,3.7851514)(42.70605469,4.6054639)
\curveto(41.79785156,5.41991703)(41.35253906,6.52440921)(41.37011719,7.91894046)
\lineto(43.01367188,7.91894046)
\closepath
\moveto(46.546875,17.01562015)
\lineto(46.546875,17.01562015)
\closepath
}
}
{
\newrgbcolor{curcolor}{0 0 0}
\pscustom[linestyle=none,fillstyle=solid,fillcolor=curcolor]
{
\newpath
\moveto(53.6484375,13.16601078)
\lineto(55.21289062,13.16601078)
\lineto(55.21289062,11.83007328)
\curveto(55.58789062,12.2929639)(55.92773438,12.62987796)(56.23242188,12.84081546)
\curveto(56.75390625,13.19823734)(57.34570312,13.37694828)(58.0078125,13.37694828)
\curveto(58.7578125,13.37694828)(59.36132812,13.19237796)(59.81835938,12.82323734)
\curveto(60.07617188,12.61229984)(60.31054688,12.30175296)(60.52148438,11.89159671)
\curveto(60.87304688,12.39550296)(61.28613281,12.76757328)(61.76074219,13.00780765)
\curveto(62.23535156,13.2539014)(62.76855469,13.37694828)(63.36035156,13.37694828)
\curveto(64.62597656,13.37694828)(65.48730469,12.91991703)(65.94433594,12.00585453)
\curveto(66.19042969,11.51366703)(66.31347656,10.85155765)(66.31347656,10.0195264)
\lineto(66.31347656,3.75292484)
\lineto(64.66992188,3.75292484)
\lineto(64.66992188,10.29198734)
\curveto(64.66992188,10.91894046)(64.51171875,11.34960453)(64.1953125,11.58397953)
\curveto(63.88476562,11.81835453)(63.50390625,11.93554203)(63.05273438,11.93554203)
\curveto(62.43164062,11.93554203)(61.89550781,11.72753421)(61.44433594,11.31151859)
\curveto(60.99902344,10.89550296)(60.77636719,10.20116703)(60.77636719,9.22851078)
\lineto(60.77636719,3.75292484)
\lineto(59.16796875,3.75292484)
\lineto(59.16796875,9.89647953)
\curveto(59.16796875,10.5351514)(59.09179688,11.00097171)(58.93945312,11.29394046)
\curveto(58.69921875,11.73339359)(58.25097656,11.95312015)(57.59472656,11.95312015)
\curveto(56.99707031,11.95312015)(56.45214844,11.72167484)(55.95996094,11.25878421)
\curveto(55.47363281,10.79589359)(55.23046875,9.95800296)(55.23046875,8.74511234)
\lineto(55.23046875,3.75292484)
\lineto(53.6484375,3.75292484)
\lineto(53.6484375,13.16601078)
\closepath
}
}
{
\newrgbcolor{curcolor}{0 0 0}
\pscustom[linestyle=none,fillstyle=solid,fillcolor=curcolor]
{
\newpath
\moveto(69.83789062,6.25780765)
\curveto(69.83789062,5.8007764)(70.00488281,5.44042484)(70.33886719,5.17675296)
\curveto(70.67285156,4.91308109)(71.06835938,4.78124515)(71.52539062,4.78124515)
\curveto(72.08203125,4.78124515)(72.62109375,4.9101514)(73.14257812,5.1679639)
\curveto(74.02148438,5.59569828)(74.4609375,6.29589359)(74.4609375,7.26854984)
\lineto(74.4609375,8.5429639)
\curveto(74.26757812,8.41991703)(74.01855469,8.31737796)(73.71386719,8.23534671)
\curveto(73.40917969,8.15331546)(73.11035156,8.09472171)(72.81738281,8.05956546)
\lineto(71.859375,7.93651859)
\curveto(71.28515625,7.86034671)(70.85449219,7.74022953)(70.56738281,7.57616703)
\curveto(70.08105469,7.3007764)(69.83789062,6.86132328)(69.83789062,6.25780765)
\closepath
\moveto(73.66992188,9.4570264)
\curveto(74.03320312,9.5039014)(74.27636719,9.65624515)(74.39941406,9.91405765)
\curveto(74.46972656,10.05468265)(74.50488281,10.25683109)(74.50488281,10.52050296)
\curveto(74.50488281,11.05956546)(74.31152344,11.4492139)(73.92480469,11.68944828)
\curveto(73.54394531,11.93554203)(72.99609375,12.0585889)(72.28125,12.0585889)
\curveto(71.45507812,12.0585889)(70.86914062,11.83593265)(70.5234375,11.39062015)
\curveto(70.33007812,11.1445264)(70.20410156,10.77831546)(70.14550781,10.29198734)
\lineto(68.66894531,10.29198734)
\curveto(68.69824219,11.45214359)(69.07324219,12.25780765)(69.79394531,12.70897953)
\curveto(70.52050781,13.16601078)(71.36132812,13.3945264)(72.31640625,13.3945264)
\curveto(73.42382812,13.3945264)(74.32324219,13.1835889)(75.01464844,12.7617139)
\curveto(75.70019531,12.3398389)(76.04296875,11.6835889)(76.04296875,10.7929639)
\lineto(76.04296875,5.37011234)
\curveto(76.04296875,5.20604984)(76.07519531,5.0742139)(76.13964844,4.97460453)
\curveto(76.20996094,4.87499515)(76.35351562,4.82519046)(76.5703125,4.82519046)
\curveto(76.640625,4.82519046)(76.71972656,4.82812015)(76.80761719,4.83397953)
\curveto(76.89550781,4.84569828)(76.98925781,4.86034671)(77.08886719,4.87792484)
\lineto(77.08886719,3.70897953)
\curveto(76.84277344,3.63866703)(76.65527344,3.59472171)(76.52636719,3.57714359)
\curveto(76.39746094,3.55956546)(76.22167969,3.5507764)(75.99902344,3.5507764)
\curveto(75.45410156,3.5507764)(75.05859375,3.74413578)(74.8125,4.13085453)
\curveto(74.68359375,4.33593265)(74.59277344,4.62597171)(74.54003906,5.00097171)
\curveto(74.21777344,4.57909671)(73.75488281,4.21288578)(73.15136719,3.9023389)
\curveto(72.54785156,3.59179203)(71.8828125,3.43651859)(71.15625,3.43651859)
\curveto(70.28320312,3.43651859)(69.56835938,3.70019046)(69.01171875,4.22753421)
\curveto(68.4609375,4.76073734)(68.18554688,5.4257764)(68.18554688,6.2226514)
\curveto(68.18554688,7.09569828)(68.45800781,7.77245609)(69.00292969,8.25292484)
\curveto(69.54785156,8.73339359)(70.26269531,9.02929203)(71.14746094,9.14062015)
\lineto(73.66992188,9.4570264)
\closepath
\moveto(72.36035156,13.3945264)
\lineto(72.36035156,13.3945264)
\closepath
}
}
{
\newrgbcolor{curcolor}{0 0 0}
\pscustom[linestyle=none,fillstyle=solid,fillcolor=curcolor]
{
\newpath
\moveto(78.68847656,16.66405765)
\lineto(80.27050781,16.66405765)
\lineto(80.27050781,3.75292484)
\lineto(78.68847656,3.75292484)
\lineto(78.68847656,16.66405765)
\closepath
}
}
{
\newrgbcolor{curcolor}{0 0 0}
\pscustom[linestyle=none,fillstyle=solid,fillcolor=curcolor]
{
\newpath
\moveto(82.69628906,16.66405765)
\lineto(84.27832031,16.66405765)
\lineto(84.27832031,3.75292484)
\lineto(82.69628906,3.75292484)
\lineto(82.69628906,16.66405765)
\closepath
}
}
{
\newrgbcolor{curcolor}{0 0 0}
\pscustom[linestyle=none,fillstyle=solid,fillcolor=curcolor]
{
\newpath
\moveto(90.58007812,13.37694828)
\curveto(91.24804688,13.37694828)(91.89550781,13.21874515)(92.52246094,12.9023389)
\curveto(93.14941406,12.59179203)(93.62695312,12.18749515)(93.95507812,11.68944828)
\curveto(94.27148438,11.2148389)(94.48242188,10.66112796)(94.58789062,10.02831546)
\curveto(94.68164062,9.59472171)(94.72851562,8.90331546)(94.72851562,7.95409671)
\lineto(87.82910156,7.95409671)
\curveto(87.85839844,6.99901859)(88.08398438,6.23144046)(88.50585938,5.65136234)
\curveto(88.92773438,5.07714359)(89.58105469,4.79003421)(90.46582031,4.79003421)
\curveto(91.29199219,4.79003421)(91.95117188,5.06249515)(92.44335938,5.60741703)
\curveto(92.72460938,5.92382328)(92.92382812,6.29003421)(93.04101562,6.70604984)
\lineto(94.59667969,6.70604984)
\curveto(94.55566406,6.36034671)(94.41796875,5.97362796)(94.18359375,5.54589359)
\curveto(93.95507812,5.12401859)(93.69726562,4.77831546)(93.41015625,4.50878421)
\curveto(92.9296875,4.04003421)(92.33496094,3.72362796)(91.62597656,3.55956546)
\curveto(91.24511719,3.46581546)(90.81445312,3.41894046)(90.33398438,3.41894046)
\curveto(89.16210938,3.41894046)(88.16894531,3.84374515)(87.35449219,4.69335453)
\curveto(86.54003906,5.54882328)(86.1328125,6.74413578)(86.1328125,8.27929203)
\curveto(86.1328125,9.79101078)(86.54296875,11.01854984)(87.36328125,11.96190921)
\curveto(88.18359375,12.90526859)(89.25585938,13.37694828)(90.58007812,13.37694828)
\closepath
\moveto(93.10253906,9.21093265)
\curveto(93.03808594,9.89647953)(92.88867188,10.44433109)(92.65429688,10.85448734)
\curveto(92.22070312,11.61620609)(91.49707031,11.99706546)(90.48339844,11.99706546)
\curveto(89.75683594,11.99706546)(89.14746094,11.73339359)(88.65527344,11.20604984)
\curveto(88.16308594,10.68456546)(87.90234375,10.0195264)(87.87304688,9.21093265)
\lineto(93.10253906,9.21093265)
\closepath
\moveto(90.43066406,13.3945264)
\lineto(90.43066406,13.3945264)
\closepath
}
}
{
\newrgbcolor{curcolor}{0 0 0}
\pscustom[linestyle=none,fillstyle=solid,fillcolor=curcolor]
{
\newpath
\moveto(97.62011719,6.70604984)
\curveto(97.66699219,6.17870609)(97.79882812,5.77440921)(98.015625,5.49315921)
\curveto(98.4140625,4.98339359)(99.10546875,4.72851078)(100.08984375,4.72851078)
\curveto(100.67578125,4.72851078)(101.19140625,4.85448734)(101.63671875,5.10644046)
\curveto(102.08203125,5.36425296)(102.3046875,5.75976078)(102.3046875,6.2929639)
\curveto(102.3046875,6.69726078)(102.12597656,7.00487796)(101.76855469,7.21581546)
\curveto(101.54003906,7.34472171)(101.08886719,7.49413578)(100.41503906,7.66405765)
\lineto(99.15820312,7.9804639)
\curveto(98.35546875,8.17968265)(97.76367188,8.4023389)(97.3828125,8.64843265)
\curveto(96.703125,9.07616703)(96.36328125,9.6679639)(96.36328125,10.42382328)
\curveto(96.36328125,11.31444828)(96.68261719,12.0351514)(97.32128906,12.58593265)
\curveto(97.96582031,13.1367139)(98.83007812,13.41210453)(99.9140625,13.41210453)
\curveto(101.33203125,13.41210453)(102.35449219,12.9960889)(102.98144531,12.16405765)
\curveto(103.37402344,11.6367139)(103.56445312,11.06835453)(103.55273438,10.45897953)
\lineto(102.05859375,10.45897953)
\curveto(102.02929688,10.8164014)(101.90332031,11.14159671)(101.68066406,11.43456546)
\curveto(101.31738281,11.85058109)(100.6875,12.0585889)(99.79101562,12.0585889)
\curveto(99.19335938,12.0585889)(98.73925781,11.94433109)(98.42871094,11.71581546)
\curveto(98.12402344,11.48729984)(97.97167969,11.18554203)(97.97167969,10.81054203)
\curveto(97.97167969,10.40038578)(98.17382812,10.07226078)(98.578125,9.82616703)
\curveto(98.8125,9.67968265)(99.15820312,9.5507764)(99.61523438,9.43944828)
\lineto(100.66113281,9.18456546)
\curveto(101.79785156,8.90917484)(102.55957031,8.64257328)(102.94628906,8.38476078)
\curveto(103.56152344,7.9804639)(103.86914062,7.34472171)(103.86914062,6.47753421)
\curveto(103.86914062,5.63964359)(103.54980469,4.91601078)(102.91113281,4.30663578)
\curveto(102.27832031,3.69726078)(101.31152344,3.39257328)(100.01074219,3.39257328)
\curveto(98.61035156,3.39257328)(97.6171875,3.70897953)(97.03125,4.34179203)
\curveto(96.45117188,4.9804639)(96.140625,5.76854984)(96.09960938,6.70604984)
\lineto(97.62011719,6.70604984)
\closepath
\moveto(99.95800781,13.3945264)
\lineto(99.95800781,13.3945264)
\closepath
}
}
{
\newrgbcolor{curcolor}{0 0 0}
\pscustom[linestyle=none,fillstyle=solid,fillcolor=curcolor]
{
\newpath
\moveto(106.03125,15.79394046)
\lineto(107.63085938,15.79394046)
\lineto(107.63085938,13.16601078)
\lineto(109.13378906,13.16601078)
\lineto(109.13378906,11.87401859)
\lineto(107.63085938,11.87401859)
\lineto(107.63085938,5.7304639)
\curveto(107.63085938,5.4023389)(107.7421875,5.18261234)(107.96484375,5.07128421)
\curveto(108.08789062,5.00683109)(108.29296875,4.97460453)(108.58007812,4.97460453)
\lineto(108.82617188,4.97460453)
\curveto(108.9140625,4.9804639)(109.01660156,4.98925296)(109.13378906,5.00097171)
\lineto(109.13378906,3.75292484)
\curveto(108.95214844,3.70019046)(108.76171875,3.66210453)(108.5625,3.63866703)
\curveto(108.36914062,3.61522953)(108.15820312,3.60351078)(107.9296875,3.60351078)
\curveto(107.19140625,3.60351078)(106.69042969,3.79101078)(106.42675781,4.16601078)
\curveto(106.16308594,4.54687015)(106.03125,5.03905765)(106.03125,5.64257328)
\lineto(106.03125,11.87401859)
\lineto(104.75683594,11.87401859)
\lineto(104.75683594,13.16601078)
\lineto(106.03125,13.16601078)
\lineto(106.03125,15.79394046)
\closepath
}
}
{
\newrgbcolor{curcolor}{0 0 0}
\pscustom[linestyle=none,fillstyle=solid,fillcolor=curcolor]
{
\newpath
\moveto(119.70703125,4.8164014)
\curveto(120.4453125,4.8164014)(121.05761719,5.12401859)(121.54394531,5.73925296)
\curveto(122.03613281,6.36034671)(122.28222656,7.28612796)(122.28222656,8.51659671)
\curveto(122.28222656,9.26659671)(122.17382812,9.91112796)(121.95703125,10.45019046)
\curveto(121.546875,11.48729984)(120.796875,12.00585453)(119.70703125,12.00585453)
\curveto(118.61132812,12.00585453)(117.86132812,11.45800296)(117.45703125,10.36229984)
\curveto(117.24023438,9.77636234)(117.13183594,9.03222171)(117.13183594,8.12987796)
\curveto(117.13183594,7.40331546)(117.24023438,6.7851514)(117.45703125,6.27538578)
\curveto(117.8671875,5.30272953)(118.6171875,4.8164014)(119.70703125,4.8164014)
\closepath
\moveto(115.61132812,13.12206546)
\lineto(117.14941406,13.12206546)
\lineto(117.14941406,11.87401859)
\curveto(117.46582031,12.30175296)(117.81152344,12.63280765)(118.18652344,12.86718265)
\curveto(118.71972656,13.21874515)(119.34667969,13.3945264)(120.06738281,13.3945264)
\curveto(121.13378906,13.3945264)(122.0390625,12.98437015)(122.78320312,12.16405765)
\curveto(123.52734375,11.34960453)(123.89941406,10.1835889)(123.89941406,8.66601078)
\curveto(123.89941406,6.61522953)(123.36328125,5.15038578)(122.29101562,4.27147953)
\curveto(121.61132812,3.7148389)(120.8203125,3.43651859)(119.91796875,3.43651859)
\curveto(119.20898438,3.43651859)(118.61425781,3.59179203)(118.13378906,3.9023389)
\curveto(117.85253906,4.07812015)(117.5390625,4.37987796)(117.19335938,4.80761234)
\lineto(117.19335938,-0.00000485)
\lineto(115.61132812,-0.00000485)
\lineto(115.61132812,13.12206546)
\closepath
}
}
{
\newrgbcolor{curcolor}{0 0 0}
\pscustom[linestyle=none,fillstyle=solid,fillcolor=curcolor]
{
\newpath
\moveto(125.79785156,13.16601078)
\lineto(127.30078125,13.16601078)
\lineto(127.30078125,11.54003421)
\curveto(127.42382812,11.85644046)(127.72558594,12.24022953)(128.20605469,12.6914014)
\curveto(128.68652344,13.14843265)(129.24023438,13.37694828)(129.8671875,13.37694828)
\curveto(129.89648438,13.37694828)(129.94628906,13.37401859)(130.01660156,13.36815921)
\curveto(130.08691406,13.36229984)(130.20703125,13.35058109)(130.37695312,13.33300296)
\lineto(130.37695312,11.66308109)
\curveto(130.28320312,11.68065921)(130.1953125,11.69237796)(130.11328125,11.69823734)
\curveto(130.03710938,11.70409671)(129.95214844,11.7070264)(129.85839844,11.7070264)
\curveto(129.06152344,11.7070264)(128.44921875,11.4492139)(128.02148438,10.9335889)
\curveto(127.59375,10.42382328)(127.37988281,9.83495609)(127.37988281,9.16698734)
\lineto(127.37988281,3.75292484)
\lineto(125.79785156,3.75292484)
\lineto(125.79785156,13.16601078)
\closepath
}
}
{
\newrgbcolor{curcolor}{0 0 0}
\pscustom[linestyle=none,fillstyle=solid,fillcolor=curcolor]
{
\newpath
\moveto(131.765625,13.12206546)
\lineto(133.37402344,13.12206546)
\lineto(133.37402344,3.75292484)
\lineto(131.765625,3.75292484)
\lineto(131.765625,13.12206546)
\closepath
\moveto(131.765625,16.66405765)
\lineto(133.37402344,16.66405765)
\lineto(133.37402344,14.8710889)
\lineto(131.765625,14.8710889)
\lineto(131.765625,16.66405765)
\closepath
}
}
{
\newrgbcolor{curcolor}{0 0 0}
\pscustom[linestyle=none,fillstyle=solid,fillcolor=curcolor]
{
\newpath
\moveto(135.7734375,13.16601078)
\lineto(137.27636719,13.16601078)
\lineto(137.27636719,11.83007328)
\curveto(137.72167969,12.38085453)(138.19335938,12.77636234)(138.69140625,13.01659671)
\curveto(139.18945312,13.25683109)(139.74316406,13.37694828)(140.35253906,13.37694828)
\curveto(141.68847656,13.37694828)(142.59082031,12.91112796)(143.05957031,11.97948734)
\curveto(143.31738281,11.46972171)(143.44628906,10.74022953)(143.44628906,9.79101078)
\lineto(143.44628906,3.75292484)
\lineto(141.83789062,3.75292484)
\lineto(141.83789062,9.68554203)
\curveto(141.83789062,10.25976078)(141.75292969,10.7226514)(141.58300781,11.0742139)
\curveto(141.30175781,11.6601514)(140.79199219,11.95312015)(140.05371094,11.95312015)
\curveto(139.67871094,11.95312015)(139.37109375,11.91503421)(139.13085938,11.83886234)
\curveto(138.69726562,11.70995609)(138.31640625,11.45214359)(137.98828125,11.06542484)
\curveto(137.72460938,10.75487796)(137.55175781,10.43261234)(137.46972656,10.09862796)
\curveto(137.39355469,9.77050296)(137.35546875,9.29882328)(137.35546875,8.6835889)
\lineto(137.35546875,3.75292484)
\lineto(135.7734375,3.75292484)
\lineto(135.7734375,13.16601078)
\closepath
\moveto(139.49121094,13.3945264)
\lineto(139.49121094,13.3945264)
\closepath
}
}
{
\newrgbcolor{curcolor}{0 0 0}
\pscustom[linestyle=none,fillstyle=solid,fillcolor=curcolor]
{
\newpath
\moveto(146.109375,15.79394046)
\lineto(147.70898438,15.79394046)
\lineto(147.70898438,13.16601078)
\lineto(149.21191406,13.16601078)
\lineto(149.21191406,11.87401859)
\lineto(147.70898438,11.87401859)
\lineto(147.70898438,5.7304639)
\curveto(147.70898438,5.4023389)(147.8203125,5.18261234)(148.04296875,5.07128421)
\curveto(148.16601562,5.00683109)(148.37109375,4.97460453)(148.65820312,4.97460453)
\lineto(148.90429688,4.97460453)
\curveto(148.9921875,4.9804639)(149.09472656,4.98925296)(149.21191406,5.00097171)
\lineto(149.21191406,3.75292484)
\curveto(149.03027344,3.70019046)(148.83984375,3.66210453)(148.640625,3.63866703)
\curveto(148.44726562,3.61522953)(148.23632812,3.60351078)(148.0078125,3.60351078)
\curveto(147.26953125,3.60351078)(146.76855469,3.79101078)(146.50488281,4.16601078)
\curveto(146.24121094,4.54687015)(146.109375,5.03905765)(146.109375,5.64257328)
\lineto(146.109375,11.87401859)
\lineto(144.83496094,11.87401859)
\lineto(144.83496094,13.16601078)
\lineto(146.109375,13.16601078)
\lineto(146.109375,15.79394046)
\closepath
}
}
{
\newrgbcolor{curcolor}{0 0 0}
\pscustom[linestyle=none,fillstyle=solid,fillcolor=curcolor]
{
\newpath
\moveto(152.03320312,6.25780765)
\curveto(152.03320312,5.8007764)(152.20019531,5.44042484)(152.53417969,5.17675296)
\curveto(152.86816406,4.91308109)(153.26367188,4.78124515)(153.72070312,4.78124515)
\curveto(154.27734375,4.78124515)(154.81640625,4.9101514)(155.33789062,5.1679639)
\curveto(156.21679688,5.59569828)(156.65625,6.29589359)(156.65625,7.26854984)
\lineto(156.65625,8.5429639)
\curveto(156.46289062,8.41991703)(156.21386719,8.31737796)(155.90917969,8.23534671)
\curveto(155.60449219,8.15331546)(155.30566406,8.09472171)(155.01269531,8.05956546)
\lineto(154.0546875,7.93651859)
\curveto(153.48046875,7.86034671)(153.04980469,7.74022953)(152.76269531,7.57616703)
\curveto(152.27636719,7.3007764)(152.03320312,6.86132328)(152.03320312,6.25780765)
\closepath
\moveto(155.86523438,9.4570264)
\curveto(156.22851562,9.5039014)(156.47167969,9.65624515)(156.59472656,9.91405765)
\curveto(156.66503906,10.05468265)(156.70019531,10.25683109)(156.70019531,10.52050296)
\curveto(156.70019531,11.05956546)(156.50683594,11.4492139)(156.12011719,11.68944828)
\curveto(155.73925781,11.93554203)(155.19140625,12.0585889)(154.4765625,12.0585889)
\curveto(153.65039062,12.0585889)(153.06445312,11.83593265)(152.71875,11.39062015)
\curveto(152.52539062,11.1445264)(152.39941406,10.77831546)(152.34082031,10.29198734)
\lineto(150.86425781,10.29198734)
\curveto(150.89355469,11.45214359)(151.26855469,12.25780765)(151.98925781,12.70897953)
\curveto(152.71582031,13.16601078)(153.55664062,13.3945264)(154.51171875,13.3945264)
\curveto(155.61914062,13.3945264)(156.51855469,13.1835889)(157.20996094,12.7617139)
\curveto(157.89550781,12.3398389)(158.23828125,11.6835889)(158.23828125,10.7929639)
\lineto(158.23828125,5.37011234)
\curveto(158.23828125,5.20604984)(158.27050781,5.0742139)(158.33496094,4.97460453)
\curveto(158.40527344,4.87499515)(158.54882812,4.82519046)(158.765625,4.82519046)
\curveto(158.8359375,4.82519046)(158.91503906,4.82812015)(159.00292969,4.83397953)
\curveto(159.09082031,4.84569828)(159.18457031,4.86034671)(159.28417969,4.87792484)
\lineto(159.28417969,3.70897953)
\curveto(159.03808594,3.63866703)(158.85058594,3.59472171)(158.72167969,3.57714359)
\curveto(158.59277344,3.55956546)(158.41699219,3.5507764)(158.19433594,3.5507764)
\curveto(157.64941406,3.5507764)(157.25390625,3.74413578)(157.0078125,4.13085453)
\curveto(156.87890625,4.33593265)(156.78808594,4.62597171)(156.73535156,5.00097171)
\curveto(156.41308594,4.57909671)(155.95019531,4.21288578)(155.34667969,3.9023389)
\curveto(154.74316406,3.59179203)(154.078125,3.43651859)(153.3515625,3.43651859)
\curveto(152.47851562,3.43651859)(151.76367188,3.70019046)(151.20703125,4.22753421)
\curveto(150.65625,4.76073734)(150.38085938,5.4257764)(150.38085938,6.2226514)
\curveto(150.38085938,7.09569828)(150.65332031,7.77245609)(151.19824219,8.25292484)
\curveto(151.74316406,8.73339359)(152.45800781,9.02929203)(153.34277344,9.14062015)
\lineto(155.86523438,9.4570264)
\closepath
\moveto(154.55566406,13.3945264)
\lineto(154.55566406,13.3945264)
\closepath
}
}
{
\newrgbcolor{curcolor}{0 0 0}
\pscustom[linestyle=none,fillstyle=solid,fillcolor=curcolor]
{
\newpath
\moveto(160.71679688,16.70800296)
\lineto(162.25488281,16.70800296)
\lineto(162.25488281,12.02343265)
\curveto(162.60058594,12.47460453)(163.01367188,12.81737796)(163.49414062,13.05175296)
\curveto(163.97460938,13.29198734)(164.49609375,13.41210453)(165.05859375,13.41210453)
\curveto(166.23046875,13.41210453)(167.1796875,13.00780765)(167.90625,12.1992139)
\curveto(168.63867188,11.39647953)(169.00488281,10.20995609)(169.00488281,8.63964359)
\curveto(169.00488281,7.15136234)(168.64453125,5.91503421)(167.92382812,4.93065921)
\curveto(167.203125,3.94628421)(166.20410156,3.45409671)(164.92675781,3.45409671)
\curveto(164.21191406,3.45409671)(163.60839844,3.62694828)(163.11621094,3.9726514)
\curveto(162.82324219,4.17772953)(162.50976562,4.50585453)(162.17578125,4.9570264)
\lineto(162.17578125,3.75292484)
\lineto(160.71679688,3.75292484)
\lineto(160.71679688,16.70800296)
\closepath
\moveto(164.83007812,4.85155765)
\curveto(165.68554688,4.85155765)(166.32421875,5.1914014)(166.74609375,5.8710889)
\curveto(167.17382812,6.5507764)(167.38769531,7.44726078)(167.38769531,8.56054203)
\curveto(167.38769531,9.5507764)(167.17382812,10.3710889)(166.74609375,11.02147953)
\curveto(166.32421875,11.67187015)(165.70019531,11.99706546)(164.87402344,11.99706546)
\curveto(164.15332031,11.99706546)(163.52050781,11.7304639)(162.97558594,11.19726078)
\curveto(162.43652344,10.66405765)(162.16699219,9.7851514)(162.16699219,8.56054203)
\curveto(162.16699219,7.6757764)(162.27832031,6.95800296)(162.50097656,6.40722171)
\curveto(162.91699219,5.37011234)(163.69335938,4.85155765)(164.83007812,4.85155765)
\closepath
}
}
{
\newrgbcolor{curcolor}{0 0 0}
\pscustom[linestyle=none,fillstyle=solid,fillcolor=curcolor]
{
\newpath
\moveto(170.90332031,16.66405765)
\lineto(172.48535156,16.66405765)
\lineto(172.48535156,3.75292484)
\lineto(170.90332031,3.75292484)
\lineto(170.90332031,16.66405765)
\closepath
}
}
{
\newrgbcolor{curcolor}{0 0 0}
\pscustom[linestyle=none,fillstyle=solid,fillcolor=curcolor]
{
\newpath
\moveto(178.78710938,13.37694828)
\curveto(179.45507812,13.37694828)(180.10253906,13.21874515)(180.72949219,12.9023389)
\curveto(181.35644531,12.59179203)(181.83398438,12.18749515)(182.16210938,11.68944828)
\curveto(182.47851562,11.2148389)(182.68945312,10.66112796)(182.79492188,10.02831546)
\curveto(182.88867188,9.59472171)(182.93554688,8.90331546)(182.93554688,7.95409671)
\lineto(176.03613281,7.95409671)
\curveto(176.06542969,6.99901859)(176.29101562,6.23144046)(176.71289062,5.65136234)
\curveto(177.13476562,5.07714359)(177.78808594,4.79003421)(178.67285156,4.79003421)
\curveto(179.49902344,4.79003421)(180.15820312,5.06249515)(180.65039062,5.60741703)
\curveto(180.93164062,5.92382328)(181.13085938,6.29003421)(181.24804688,6.70604984)
\lineto(182.80371094,6.70604984)
\curveto(182.76269531,6.36034671)(182.625,5.97362796)(182.390625,5.54589359)
\curveto(182.16210938,5.12401859)(181.90429688,4.77831546)(181.6171875,4.50878421)
\curveto(181.13671875,4.04003421)(180.54199219,3.72362796)(179.83300781,3.55956546)
\curveto(179.45214844,3.46581546)(179.02148438,3.41894046)(178.54101562,3.41894046)
\curveto(177.36914062,3.41894046)(176.37597656,3.84374515)(175.56152344,4.69335453)
\curveto(174.74707031,5.54882328)(174.33984375,6.74413578)(174.33984375,8.27929203)
\curveto(174.33984375,9.79101078)(174.75,11.01854984)(175.5703125,11.96190921)
\curveto(176.390625,12.90526859)(177.46289062,13.37694828)(178.78710938,13.37694828)
\closepath
\moveto(181.30957031,9.21093265)
\curveto(181.24511719,9.89647953)(181.09570312,10.44433109)(180.86132812,10.85448734)
\curveto(180.42773438,11.61620609)(179.70410156,11.99706546)(178.69042969,11.99706546)
\curveto(177.96386719,11.99706546)(177.35449219,11.73339359)(176.86230469,11.20604984)
\curveto(176.37011719,10.68456546)(176.109375,10.0195264)(176.08007812,9.21093265)
\lineto(181.30957031,9.21093265)
\closepath
\moveto(178.63769531,13.3945264)
\lineto(178.63769531,13.3945264)
\closepath
}
}
{
\newrgbcolor{curcolor}{0 0 0}
\pscustom[linestyle=none,fillstyle=solid,fillcolor=curcolor]
{
\newpath
\moveto(193.79882812,13.37694828)
\curveto(194.46679688,13.37694828)(195.11425781,13.21874515)(195.74121094,12.9023389)
\curveto(196.36816406,12.59179203)(196.84570312,12.18749515)(197.17382812,11.68944828)
\curveto(197.49023438,11.2148389)(197.70117188,10.66112796)(197.80664062,10.02831546)
\curveto(197.90039062,9.59472171)(197.94726562,8.90331546)(197.94726562,7.95409671)
\lineto(191.04785156,7.95409671)
\curveto(191.07714844,6.99901859)(191.30273438,6.23144046)(191.72460938,5.65136234)
\curveto(192.14648438,5.07714359)(192.79980469,4.79003421)(193.68457031,4.79003421)
\curveto(194.51074219,4.79003421)(195.16992188,5.06249515)(195.66210938,5.60741703)
\curveto(195.94335938,5.92382328)(196.14257812,6.29003421)(196.25976562,6.70604984)
\lineto(197.81542969,6.70604984)
\curveto(197.77441406,6.36034671)(197.63671875,5.97362796)(197.40234375,5.54589359)
\curveto(197.17382812,5.12401859)(196.91601562,4.77831546)(196.62890625,4.50878421)
\curveto(196.1484375,4.04003421)(195.55371094,3.72362796)(194.84472656,3.55956546)
\curveto(194.46386719,3.46581546)(194.03320312,3.41894046)(193.55273438,3.41894046)
\curveto(192.38085938,3.41894046)(191.38769531,3.84374515)(190.57324219,4.69335453)
\curveto(189.75878906,5.54882328)(189.3515625,6.74413578)(189.3515625,8.27929203)
\curveto(189.3515625,9.79101078)(189.76171875,11.01854984)(190.58203125,11.96190921)
\curveto(191.40234375,12.90526859)(192.47460938,13.37694828)(193.79882812,13.37694828)
\closepath
\moveto(196.32128906,9.21093265)
\curveto(196.25683594,9.89647953)(196.10742188,10.44433109)(195.87304688,10.85448734)
\curveto(195.43945312,11.61620609)(194.71582031,11.99706546)(193.70214844,11.99706546)
\curveto(192.97558594,11.99706546)(192.36621094,11.73339359)(191.87402344,11.20604984)
\curveto(191.38183594,10.68456546)(191.12109375,10.0195264)(191.09179688,9.21093265)
\lineto(196.32128906,9.21093265)
\closepath
\moveto(193.64941406,13.3945264)
\lineto(193.64941406,13.3945264)
\closepath
}
}
{
\newrgbcolor{curcolor}{0 0 0}
\pscustom[linestyle=none,fillstyle=solid,fillcolor=curcolor]
{
\newpath
\moveto(199.94238281,16.66405765)
\lineto(201.52441406,16.66405765)
\lineto(201.52441406,3.75292484)
\lineto(199.94238281,3.75292484)
\lineto(199.94238281,16.66405765)
\closepath
}
}
{
\newrgbcolor{curcolor}{0 0 0}
\pscustom[linestyle=none,fillstyle=solid,fillcolor=curcolor]
{
\newpath
\moveto(207.82617188,13.37694828)
\curveto(208.49414062,13.37694828)(209.14160156,13.21874515)(209.76855469,12.9023389)
\curveto(210.39550781,12.59179203)(210.87304688,12.18749515)(211.20117188,11.68944828)
\curveto(211.51757812,11.2148389)(211.72851562,10.66112796)(211.83398438,10.02831546)
\curveto(211.92773438,9.59472171)(211.97460938,8.90331546)(211.97460938,7.95409671)
\lineto(205.07519531,7.95409671)
\curveto(205.10449219,6.99901859)(205.33007812,6.23144046)(205.75195312,5.65136234)
\curveto(206.17382812,5.07714359)(206.82714844,4.79003421)(207.71191406,4.79003421)
\curveto(208.53808594,4.79003421)(209.19726562,5.06249515)(209.68945312,5.60741703)
\curveto(209.97070312,5.92382328)(210.16992188,6.29003421)(210.28710938,6.70604984)
\lineto(211.84277344,6.70604984)
\curveto(211.80175781,6.36034671)(211.6640625,5.97362796)(211.4296875,5.54589359)
\curveto(211.20117188,5.12401859)(210.94335938,4.77831546)(210.65625,4.50878421)
\curveto(210.17578125,4.04003421)(209.58105469,3.72362796)(208.87207031,3.55956546)
\curveto(208.49121094,3.46581546)(208.06054688,3.41894046)(207.58007812,3.41894046)
\curveto(206.40820312,3.41894046)(205.41503906,3.84374515)(204.60058594,4.69335453)
\curveto(203.78613281,5.54882328)(203.37890625,6.74413578)(203.37890625,8.27929203)
\curveto(203.37890625,9.79101078)(203.7890625,11.01854984)(204.609375,11.96190921)
\curveto(205.4296875,12.90526859)(206.50195312,13.37694828)(207.82617188,13.37694828)
\closepath
\moveto(210.34863281,9.21093265)
\curveto(210.28417969,9.89647953)(210.13476562,10.44433109)(209.90039062,10.85448734)
\curveto(209.46679688,11.61620609)(208.74316406,11.99706546)(207.72949219,11.99706546)
\curveto(207.00292969,11.99706546)(206.39355469,11.73339359)(205.90136719,11.20604984)
\curveto(205.40917969,10.68456546)(205.1484375,10.0195264)(205.11914062,9.21093265)
\lineto(210.34863281,9.21093265)
\closepath
\moveto(207.67675781,13.3945264)
\lineto(207.67675781,13.3945264)
\closepath
}
}
{
\newrgbcolor{curcolor}{0 0 0}
\pscustom[linestyle=none,fillstyle=solid,fillcolor=curcolor]
{
\newpath
\moveto(213.92578125,13.16601078)
\lineto(215.49023438,13.16601078)
\lineto(215.49023438,11.83007328)
\curveto(215.86523438,12.2929639)(216.20507812,12.62987796)(216.50976562,12.84081546)
\curveto(217.03125,13.19823734)(217.62304688,13.37694828)(218.28515625,13.37694828)
\curveto(219.03515625,13.37694828)(219.63867188,13.19237796)(220.09570312,12.82323734)
\curveto(220.35351562,12.61229984)(220.58789062,12.30175296)(220.79882812,11.89159671)
\curveto(221.15039062,12.39550296)(221.56347656,12.76757328)(222.03808594,13.00780765)
\curveto(222.51269531,13.2539014)(223.04589844,13.37694828)(223.63769531,13.37694828)
\curveto(224.90332031,13.37694828)(225.76464844,12.91991703)(226.22167969,12.00585453)
\curveto(226.46777344,11.51366703)(226.59082031,10.85155765)(226.59082031,10.0195264)
\lineto(226.59082031,3.75292484)
\lineto(224.94726562,3.75292484)
\lineto(224.94726562,10.29198734)
\curveto(224.94726562,10.91894046)(224.7890625,11.34960453)(224.47265625,11.58397953)
\curveto(224.16210938,11.81835453)(223.78125,11.93554203)(223.33007812,11.93554203)
\curveto(222.70898438,11.93554203)(222.17285156,11.72753421)(221.72167969,11.31151859)
\curveto(221.27636719,10.89550296)(221.05371094,10.20116703)(221.05371094,9.22851078)
\lineto(221.05371094,3.75292484)
\lineto(219.4453125,3.75292484)
\lineto(219.4453125,9.89647953)
\curveto(219.4453125,10.5351514)(219.36914062,11.00097171)(219.21679688,11.29394046)
\curveto(218.9765625,11.73339359)(218.52832031,11.95312015)(217.87207031,11.95312015)
\curveto(217.27441406,11.95312015)(216.72949219,11.72167484)(216.23730469,11.25878421)
\curveto(215.75097656,10.79589359)(215.5078125,9.95800296)(215.5078125,8.74511234)
\lineto(215.5078125,3.75292484)
\lineto(213.92578125,3.75292484)
\lineto(213.92578125,13.16601078)
\closepath
}
}
{
\newrgbcolor{curcolor}{0 0 0}
\pscustom[linestyle=none,fillstyle=solid,fillcolor=curcolor]
{
\newpath
\moveto(232.82226562,13.37694828)
\curveto(233.49023438,13.37694828)(234.13769531,13.21874515)(234.76464844,12.9023389)
\curveto(235.39160156,12.59179203)(235.86914062,12.18749515)(236.19726562,11.68944828)
\curveto(236.51367188,11.2148389)(236.72460938,10.66112796)(236.83007812,10.02831546)
\curveto(236.92382812,9.59472171)(236.97070312,8.90331546)(236.97070312,7.95409671)
\lineto(230.07128906,7.95409671)
\curveto(230.10058594,6.99901859)(230.32617188,6.23144046)(230.74804688,5.65136234)
\curveto(231.16992188,5.07714359)(231.82324219,4.79003421)(232.70800781,4.79003421)
\curveto(233.53417969,4.79003421)(234.19335938,5.06249515)(234.68554688,5.60741703)
\curveto(234.96679688,5.92382328)(235.16601562,6.29003421)(235.28320312,6.70604984)
\lineto(236.83886719,6.70604984)
\curveto(236.79785156,6.36034671)(236.66015625,5.97362796)(236.42578125,5.54589359)
\curveto(236.19726562,5.12401859)(235.93945312,4.77831546)(235.65234375,4.50878421)
\curveto(235.171875,4.04003421)(234.57714844,3.72362796)(233.86816406,3.55956546)
\curveto(233.48730469,3.46581546)(233.05664062,3.41894046)(232.57617188,3.41894046)
\curveto(231.40429688,3.41894046)(230.41113281,3.84374515)(229.59667969,4.69335453)
\curveto(228.78222656,5.54882328)(228.375,6.74413578)(228.375,8.27929203)
\curveto(228.375,9.79101078)(228.78515625,11.01854984)(229.60546875,11.96190921)
\curveto(230.42578125,12.90526859)(231.49804688,13.37694828)(232.82226562,13.37694828)
\closepath
\moveto(235.34472656,9.21093265)
\curveto(235.28027344,9.89647953)(235.13085938,10.44433109)(234.89648438,10.85448734)
\curveto(234.46289062,11.61620609)(233.73925781,11.99706546)(232.72558594,11.99706546)
\curveto(231.99902344,11.99706546)(231.38964844,11.73339359)(230.89746094,11.20604984)
\curveto(230.40527344,10.68456546)(230.14453125,10.0195264)(230.11523438,9.21093265)
\lineto(235.34472656,9.21093265)
\closepath
\moveto(232.67285156,13.3945264)
\lineto(232.67285156,13.3945264)
\closepath
}
}
{
\newrgbcolor{curcolor}{0 0 0}
\pscustom[linestyle=none,fillstyle=solid,fillcolor=curcolor]
{
\newpath
\moveto(238.921875,13.16601078)
\lineto(240.42480469,13.16601078)
\lineto(240.42480469,11.83007328)
\curveto(240.87011719,12.38085453)(241.34179688,12.77636234)(241.83984375,13.01659671)
\curveto(242.33789062,13.25683109)(242.89160156,13.37694828)(243.50097656,13.37694828)
\curveto(244.83691406,13.37694828)(245.73925781,12.91112796)(246.20800781,11.97948734)
\curveto(246.46582031,11.46972171)(246.59472656,10.74022953)(246.59472656,9.79101078)
\lineto(246.59472656,3.75292484)
\lineto(244.98632812,3.75292484)
\lineto(244.98632812,9.68554203)
\curveto(244.98632812,10.25976078)(244.90136719,10.7226514)(244.73144531,11.0742139)
\curveto(244.45019531,11.6601514)(243.94042969,11.95312015)(243.20214844,11.95312015)
\curveto(242.82714844,11.95312015)(242.51953125,11.91503421)(242.27929688,11.83886234)
\curveto(241.84570312,11.70995609)(241.46484375,11.45214359)(241.13671875,11.06542484)
\curveto(240.87304688,10.75487796)(240.70019531,10.43261234)(240.61816406,10.09862796)
\curveto(240.54199219,9.77050296)(240.50390625,9.29882328)(240.50390625,8.6835889)
\lineto(240.50390625,3.75292484)
\lineto(238.921875,3.75292484)
\lineto(238.921875,13.16601078)
\closepath
\moveto(242.63964844,13.3945264)
\lineto(242.63964844,13.3945264)
\closepath
}
}
{
\newrgbcolor{curcolor}{0 0 0}
\pscustom[linestyle=none,fillstyle=solid,fillcolor=curcolor]
{
\newpath
\moveto(249.2578125,15.79394046)
\lineto(250.85742188,15.79394046)
\lineto(250.85742188,13.16601078)
\lineto(252.36035156,13.16601078)
\lineto(252.36035156,11.87401859)
\lineto(250.85742188,11.87401859)
\lineto(250.85742188,5.7304639)
\curveto(250.85742188,5.4023389)(250.96875,5.18261234)(251.19140625,5.07128421)
\curveto(251.31445312,5.00683109)(251.51953125,4.97460453)(251.80664062,4.97460453)
\lineto(252.05273438,4.97460453)
\curveto(252.140625,4.9804639)(252.24316406,4.98925296)(252.36035156,5.00097171)
\lineto(252.36035156,3.75292484)
\curveto(252.17871094,3.70019046)(251.98828125,3.66210453)(251.7890625,3.63866703)
\curveto(251.59570312,3.61522953)(251.38476562,3.60351078)(251.15625,3.60351078)
\curveto(250.41796875,3.60351078)(249.91699219,3.79101078)(249.65332031,4.16601078)
\curveto(249.38964844,4.54687015)(249.2578125,5.03905765)(249.2578125,5.64257328)
\lineto(249.2578125,11.87401859)
\lineto(247.98339844,11.87401859)
\lineto(247.98339844,13.16601078)
\lineto(249.2578125,13.16601078)
\lineto(249.2578125,15.79394046)
\closepath
}
}
{
\newrgbcolor{curcolor}{0 0 0}
\pscustom[linewidth=0.79306203,linecolor=curcolor]
{
\newpath
\moveto(10.928138,273.99787206)
\lineto(10.928138,292.86629206)
}
}
{
\newrgbcolor{curcolor}{0 0 0}
\pscustom[linewidth=0.79306203,linecolor=curcolor]
{
\newpath
\moveto(50.5,273.99787206)
\lineto(50.5,292.86629206)
}
}
{
\newrgbcolor{curcolor}{0 0 0}
\pscustom[linewidth=1,linecolor=curcolor]
{
\newpath
\moveto(10.5,283.99787206)
\lineto(50.5,283.99787206)
}
}
{
\newrgbcolor{curcolor}{0 0 0}
\pscustom[linestyle=none,fillstyle=solid,fillcolor=curcolor]
{
\newpath
\moveto(20.5,283.99787206)
\lineto(24.5,287.99787206)
\lineto(10.5,283.99787206)
\lineto(24.5,279.99787206)
\lineto(20.5,283.99787206)
\closepath
}
}
{
\newrgbcolor{curcolor}{0 0 0}
\pscustom[linewidth=1,linecolor=curcolor]
{
\newpath
\moveto(20.5,283.99787206)
\lineto(24.5,287.99787206)
\lineto(10.5,283.99787206)
\lineto(24.5,279.99787206)
\lineto(20.5,283.99787206)
\closepath
}
}
{
\newrgbcolor{curcolor}{0 0 0}
\pscustom[linestyle=none,fillstyle=solid,fillcolor=curcolor]
{
\newpath
\moveto(40.5,283.99787206)
\lineto(36.5,279.99787206)
\lineto(50.5,283.99787206)
\lineto(36.5,287.99787206)
\lineto(40.5,283.99787206)
\closepath
}
}
{
\newrgbcolor{curcolor}{0 0 0}
\pscustom[linewidth=1,linecolor=curcolor]
{
\newpath
\moveto(40.5,283.99787206)
\lineto(36.5,279.99787206)
\lineto(50.5,283.99787206)
\lineto(36.5,287.99787206)
\lineto(40.5,283.99787206)
\closepath
}
}
{
\newrgbcolor{curcolor}{0 0 0}
\pscustom[linewidth=1,linecolor=curcolor]
{
\newpath
\moveto(271.55262,213.99787206)
\lineto(291.55262,213.99787206)
}
}
{
\newrgbcolor{curcolor}{0 0 0}
\pscustom[linewidth=1,linecolor=curcolor]
{
\newpath
\moveto(271.55262,173.99787206)
\lineto(291.55262,173.99787206)
}
}
{
\newrgbcolor{curcolor}{0 0 0}
\pscustom[linewidth=1,linecolor=curcolor]
{
\newpath
\moveto(281.55262,213.99787206)
\lineto(281.55262,173.99787206)
}
}
{
\newrgbcolor{curcolor}{0 0 0}
\pscustom[linestyle=none,fillstyle=solid,fillcolor=curcolor]
{
\newpath
\moveto(281.55262,203.99787206)
\lineto(285.55262,199.99787206)
\lineto(281.55262,213.99787206)
\lineto(277.55262,199.99787206)
\lineto(281.55262,203.99787206)
\closepath
}
}
{
\newrgbcolor{curcolor}{0 0 0}
\pscustom[linewidth=1,linecolor=curcolor]
{
\newpath
\moveto(281.55262,203.99787206)
\lineto(285.55262,199.99787206)
\lineto(281.55262,213.99787206)
\lineto(277.55262,199.99787206)
\lineto(281.55262,203.99787206)
\closepath
}
}
{
\newrgbcolor{curcolor}{0 0 0}
\pscustom[linestyle=none,fillstyle=solid,fillcolor=curcolor]
{
\newpath
\moveto(281.55262,183.99787206)
\lineto(277.55262,187.99787206)
\lineto(281.55262,173.99787206)
\lineto(285.55262,187.99787206)
\lineto(281.55262,183.99787206)
\closepath
}
}
{
\newrgbcolor{curcolor}{0 0 0}
\pscustom[linewidth=1,linecolor=curcolor]
{
\newpath
\moveto(281.55262,183.99787206)
\lineto(277.55262,187.99787206)
\lineto(281.55262,173.99787206)
\lineto(285.55262,187.99787206)
\lineto(281.55262,183.99787206)
\closepath
}
}
{
\newrgbcolor{curcolor}{0 0 0}
\pscustom[linestyle=none,fillstyle=solid,fillcolor=curcolor]
{
\newpath
\moveto(261.47973633,253.99787418)
\curveto(261.47973633,248.47502668)(257.00258382,243.99787418)(251.47973633,243.99787418)
\curveto(245.95688883,243.99787418)(241.47973633,248.47502668)(241.47973633,253.99787418)
\curveto(241.47973633,259.52072168)(245.95688883,263.99787418)(251.47973633,263.99787418)
\curveto(257.00258382,263.99787418)(261.47973633,259.52072168)(261.47973633,253.99787418)
\closepath
}
}
{
\newrgbcolor{curcolor}{0 0 0}
\pscustom[linewidth=1,linecolor=curcolor]
{
\newpath
\moveto(261.47973633,253.99787418)
\curveto(261.47973633,248.47502668)(257.00258382,243.99787418)(251.47973633,243.99787418)
\curveto(245.95688883,243.99787418)(241.47973633,248.47502668)(241.47973633,253.99787418)
\curveto(241.47973633,259.52072168)(245.95688883,263.99787418)(251.47973633,263.99787418)
\curveto(257.00258382,263.99787418)(261.47973633,259.52072168)(261.47973633,253.99787418)
\closepath
}
}
{
\newrgbcolor{curcolor}{0 0 0}
\pscustom[linestyle=none,fillstyle=solid,fillcolor=curcolor]
{
\newpath
\moveto(261.47973633,213.99787418)
\curveto(261.47973633,208.47502668)(257.00258382,203.99787418)(251.47973633,203.99787418)
\curveto(245.95688883,203.99787418)(241.47973633,208.47502668)(241.47973633,213.99787418)
\curveto(241.47973633,219.52072168)(245.95688883,223.99787418)(251.47973633,223.99787418)
\curveto(257.00258382,223.99787418)(261.47973633,219.52072168)(261.47973633,213.99787418)
\closepath
}
}
{
\newrgbcolor{curcolor}{0 0 0}
\pscustom[linewidth=1,linecolor=curcolor]
{
\newpath
\moveto(261.47973633,213.99787418)
\curveto(261.47973633,208.47502668)(257.00258382,203.99787418)(251.47973633,203.99787418)
\curveto(245.95688883,203.99787418)(241.47973633,208.47502668)(241.47973633,213.99787418)
\curveto(241.47973633,219.52072168)(245.95688883,223.99787418)(251.47973633,223.99787418)
\curveto(257.00258382,223.99787418)(261.47973633,219.52072168)(261.47973633,213.99787418)
\closepath
}
}
{
\newrgbcolor{curcolor}{0 0 0}
\pscustom[linestyle=none,fillstyle=solid,fillcolor=curcolor]
{
\newpath
\moveto(261.47973633,173.99788944)
\curveto(261.47973633,168.47504194)(257.00258382,163.99788944)(251.47973633,163.99788944)
\curveto(245.95688883,163.99788944)(241.47973633,168.47504194)(241.47973633,173.99788944)
\curveto(241.47973633,179.52073693)(245.95688883,183.99788944)(251.47973633,183.99788944)
\curveto(257.00258382,183.99788944)(261.47973633,179.52073693)(261.47973633,173.99788944)
\closepath
}
}
{
\newrgbcolor{curcolor}{0 0 0}
\pscustom[linewidth=1,linecolor=curcolor]
{
\newpath
\moveto(261.47973633,173.99788944)
\curveto(261.47973633,168.47504194)(257.00258382,163.99788944)(251.47973633,163.99788944)
\curveto(245.95688883,163.99788944)(241.47973633,168.47504194)(241.47973633,173.99788944)
\curveto(241.47973633,179.52073693)(245.95688883,183.99788944)(251.47973633,183.99788944)
\curveto(257.00258382,183.99788944)(261.47973633,179.52073693)(261.47973633,173.99788944)
\closepath
}
}
{
\newrgbcolor{curcolor}{0 0 0}
\pscustom[linestyle=none,fillstyle=solid,fillcolor=curcolor]
{
\newpath
\moveto(261.47973633,133.99788944)
\curveto(261.47973633,128.47504194)(257.00258382,123.99788944)(251.47973633,123.99788944)
\curveto(245.95688883,123.99788944)(241.47973633,128.47504194)(241.47973633,133.99788944)
\curveto(241.47973633,139.52073693)(245.95688883,143.99788944)(251.47973633,143.99788944)
\curveto(257.00258382,143.99788944)(261.47973633,139.52073693)(261.47973633,133.99788944)
\closepath
}
}
{
\newrgbcolor{curcolor}{0 0 0}
\pscustom[linewidth=1,linecolor=curcolor]
{
\newpath
\moveto(261.47973633,133.99788944)
\curveto(261.47973633,128.47504194)(257.00258382,123.99788944)(251.47973633,123.99788944)
\curveto(245.95688883,123.99788944)(241.47973633,128.47504194)(241.47973633,133.99788944)
\curveto(241.47973633,139.52073693)(245.95688883,143.99788944)(251.47973633,143.99788944)
\curveto(257.00258382,143.99788944)(261.47973633,139.52073693)(261.47973633,133.99788944)
\closepath
}
}
{
\newrgbcolor{curcolor}{0 0 0}
\pscustom[linestyle=none,fillstyle=solid,fillcolor=curcolor]
{
\newpath
\moveto(261.47973633,93.99788944)
\curveto(261.47973633,88.47504194)(257.00258382,83.99788944)(251.47973633,83.99788944)
\curveto(245.95688883,83.99788944)(241.47973633,88.47504194)(241.47973633,93.99788944)
\curveto(241.47973633,99.52073693)(245.95688883,103.99788944)(251.47973633,103.99788944)
\curveto(257.00258382,103.99788944)(261.47973633,99.52073693)(261.47973633,93.99788944)
\closepath
}
}
{
\newrgbcolor{curcolor}{0 0 0}
\pscustom[linewidth=1,linecolor=curcolor]
{
\newpath
\moveto(261.47973633,93.99788944)
\curveto(261.47973633,88.47504194)(257.00258382,83.99788944)(251.47973633,83.99788944)
\curveto(245.95688883,83.99788944)(241.47973633,88.47504194)(241.47973633,93.99788944)
\curveto(241.47973633,99.52073693)(245.95688883,103.99788944)(251.47973633,103.99788944)
\curveto(257.00258382,103.99788944)(261.47973633,99.52073693)(261.47973633,93.99788944)
\closepath
}
}
{
\newrgbcolor{curcolor}{0 0 0}
\pscustom[linestyle=none,fillstyle=solid,fillcolor=curcolor]
{
\newpath
\moveto(20.5,53.99788944)
\curveto(20.5,48.47504194)(16.0228475,43.99788944)(10.5,43.99788944)
\curveto(4.9771525,43.99788944)(0.5,48.47504194)(0.5,53.99788944)
\curveto(0.5,59.52073693)(4.9771525,63.99788944)(10.5,63.99788944)
\curveto(16.0228475,63.99788944)(20.5,59.52073693)(20.5,53.99788944)
\closepath
}
}
{
\newrgbcolor{curcolor}{0 0 0}
\pscustom[linewidth=1,linecolor=curcolor]
{
\newpath
\moveto(20.5,53.99788944)
\curveto(20.5,48.47504194)(16.0228475,43.99788944)(10.5,43.99788944)
\curveto(4.9771525,43.99788944)(0.5,48.47504194)(0.5,53.99788944)
\curveto(0.5,59.52073693)(4.9771525,63.99788944)(10.5,63.99788944)
\curveto(16.0228475,63.99788944)(20.5,59.52073693)(20.5,53.99788944)
\closepath
}
}
{
\newrgbcolor{curcolor}{0 0 0}
\pscustom[linestyle=none,fillstyle=solid,fillcolor=curcolor]
{
\newpath
\moveto(61.60121155,53.99788944)
\curveto(61.60121155,48.47504194)(57.12405904,43.99788944)(51.60121155,43.99788944)
\curveto(46.07836405,43.99788944)(41.60121155,48.47504194)(41.60121155,53.99788944)
\curveto(41.60121155,59.52073693)(46.07836405,63.99788944)(51.60121155,63.99788944)
\curveto(57.12405904,63.99788944)(61.60121155,59.52073693)(61.60121155,53.99788944)
\closepath
}
}
{
\newrgbcolor{curcolor}{0 0 0}
\pscustom[linewidth=1,linecolor=curcolor]
{
\newpath
\moveto(61.60121155,53.99788944)
\curveto(61.60121155,48.47504194)(57.12405904,43.99788944)(51.60121155,43.99788944)
\curveto(46.07836405,43.99788944)(41.60121155,48.47504194)(41.60121155,53.99788944)
\curveto(41.60121155,59.52073693)(46.07836405,63.99788944)(51.60121155,63.99788944)
\curveto(57.12405904,63.99788944)(61.60121155,59.52073693)(61.60121155,53.99788944)
\closepath
}
}
{
\newrgbcolor{curcolor}{0 0 0}
\pscustom[linestyle=none,fillstyle=solid,fillcolor=curcolor]
{
\newpath
\moveto(100.98986816,53.14159671)
\curveto(100.98986816,47.61874922)(96.51271566,43.14159671)(90.98986816,43.14159671)
\curveto(85.46702067,43.14159671)(80.98986816,47.61874922)(80.98986816,53.14159671)
\curveto(80.98986816,58.66444421)(85.46702067,63.14159671)(90.98986816,63.14159671)
\curveto(96.51271566,63.14159671)(100.98986816,58.66444421)(100.98986816,53.14159671)
\closepath
}
}
{
\newrgbcolor{curcolor}{0 0 0}
\pscustom[linewidth=1,linecolor=curcolor]
{
\newpath
\moveto(100.98986816,53.14159671)
\curveto(100.98986816,47.61874922)(96.51271566,43.14159671)(90.98986816,43.14159671)
\curveto(85.46702067,43.14159671)(80.98986816,47.61874922)(80.98986816,53.14159671)
\curveto(80.98986816,58.66444421)(85.46702067,63.14159671)(90.98986816,63.14159671)
\curveto(96.51271566,63.14159671)(100.98986816,58.66444421)(100.98986816,53.14159671)
\closepath
}
}
{
\newrgbcolor{curcolor}{0 0 0}
\pscustom[linestyle=none,fillstyle=solid,fillcolor=curcolor]
{
\newpath
\moveto(142.09107971,53.99788944)
\curveto(142.09107971,48.47504194)(137.61392721,43.99788944)(132.09107971,43.99788944)
\curveto(126.56823222,43.99788944)(122.09107971,48.47504194)(122.09107971,53.99788944)
\curveto(122.09107971,59.52073693)(126.56823222,63.99788944)(132.09107971,63.99788944)
\curveto(137.61392721,63.99788944)(142.09107971,59.52073693)(142.09107971,53.99788944)
\closepath
}
}
{
\newrgbcolor{curcolor}{0 0 0}
\pscustom[linewidth=1,linecolor=curcolor]
{
\newpath
\moveto(142.09107971,53.99788944)
\curveto(142.09107971,48.47504194)(137.61392721,43.99788944)(132.09107971,43.99788944)
\curveto(126.56823222,43.99788944)(122.09107971,48.47504194)(122.09107971,53.99788944)
\curveto(122.09107971,59.52073693)(126.56823222,63.99788944)(132.09107971,63.99788944)
\curveto(137.61392721,63.99788944)(142.09107971,59.52073693)(142.09107971,53.99788944)
\closepath
}
}
{
\newrgbcolor{curcolor}{0 0 0}
\pscustom[linestyle=none,fillstyle=solid,fillcolor=curcolor]
{
\newpath
\moveto(181.47973633,53.99788944)
\curveto(181.47973633,48.47504194)(177.00258382,43.99788944)(171.47973633,43.99788944)
\curveto(165.95688883,43.99788944)(161.47973633,48.47504194)(161.47973633,53.99788944)
\curveto(161.47973633,59.52073693)(165.95688883,63.99788944)(171.47973633,63.99788944)
\curveto(177.00258382,63.99788944)(181.47973633,59.52073693)(181.47973633,53.99788944)
\closepath
}
}
{
\newrgbcolor{curcolor}{0 0 0}
\pscustom[linewidth=1,linecolor=curcolor]
{
\newpath
\moveto(181.47973633,53.99788944)
\curveto(181.47973633,48.47504194)(177.00258382,43.99788944)(171.47973633,43.99788944)
\curveto(165.95688883,43.99788944)(161.47973633,48.47504194)(161.47973633,53.99788944)
\curveto(161.47973633,59.52073693)(165.95688883,63.99788944)(171.47973633,63.99788944)
\curveto(177.00258382,63.99788944)(181.47973633,59.52073693)(181.47973633,53.99788944)
\closepath
}
}
{
\newrgbcolor{curcolor}{0 0 0}
\pscustom[linestyle=none,fillstyle=solid,fillcolor=curcolor]
{
\newpath
\moveto(221.47973633,53.99788944)
\curveto(221.47973633,48.47504194)(217.00258382,43.99788944)(211.47973633,43.99788944)
\curveto(205.95688883,43.99788944)(201.47973633,48.47504194)(201.47973633,53.99788944)
\curveto(201.47973633,59.52073693)(205.95688883,63.99788944)(211.47973633,63.99788944)
\curveto(217.00258382,63.99788944)(221.47973633,59.52073693)(221.47973633,53.99788944)
\closepath
}
}
{
\newrgbcolor{curcolor}{0 0 0}
\pscustom[linewidth=1,linecolor=curcolor]
{
\newpath
\moveto(221.47973633,53.99788944)
\curveto(221.47973633,48.47504194)(217.00258382,43.99788944)(211.47973633,43.99788944)
\curveto(205.95688883,43.99788944)(201.47973633,48.47504194)(201.47973633,53.99788944)
\curveto(201.47973633,59.52073693)(205.95688883,63.99788944)(211.47973633,63.99788944)
\curveto(217.00258382,63.99788944)(221.47973633,59.52073693)(221.47973633,53.99788944)
\closepath
}
}
{
\newrgbcolor{curcolor}{0 0 0}
\pscustom[linestyle=none,fillstyle=solid,fillcolor=curcolor]
{
\newpath
\moveto(261.47973633,53.99788944)
\curveto(261.47973633,48.47504194)(257.00258382,43.99788944)(251.47973633,43.99788944)
\curveto(245.95688883,43.99788944)(241.47973633,48.47504194)(241.47973633,53.99788944)
\curveto(241.47973633,59.52073693)(245.95688883,63.99788944)(251.47973633,63.99788944)
\curveto(257.00258382,63.99788944)(261.47973633,59.52073693)(261.47973633,53.99788944)
\closepath
}
}
{
\newrgbcolor{curcolor}{0 0 0}
\pscustom[linewidth=1,linecolor=curcolor]
{
\newpath
\moveto(261.47973633,53.99788944)
\curveto(261.47973633,48.47504194)(257.00258382,43.99788944)(251.47973633,43.99788944)
\curveto(245.95688883,43.99788944)(241.47973633,48.47504194)(241.47973633,53.99788944)
\curveto(241.47973633,59.52073693)(245.95688883,63.99788944)(251.47973633,63.99788944)
\curveto(257.00258382,63.99788944)(261.47973633,59.52073693)(261.47973633,53.99788944)
\closepath
}
}
{
\newrgbcolor{curcolor}{0 0 0}
\pscustom[linewidth=1,linecolor=curcolor]
{
\newpath
\moveto(150.5,93.99787206)
\lineto(150.5,23.99787206)
}
}
{
\newrgbcolor{curcolor}{0 0 0}
\pscustom[linestyle=none,fillstyle=solid,fillcolor=curcolor]
{
\newpath
\moveto(150.5,83.99787206)
\lineto(154.5,79.99787206)
\lineto(150.5,93.99787206)
\lineto(146.5,79.99787206)
\lineto(150.5,83.99787206)
\closepath
}
}
{
\newrgbcolor{curcolor}{0 0 0}
\pscustom[linewidth=1,linecolor=curcolor]
{
\newpath
\moveto(150.5,83.99787206)
\lineto(154.5,79.99787206)
\lineto(150.5,93.99787206)
\lineto(146.5,79.99787206)
\lineto(150.5,83.99787206)
\closepath
}
}
\end{pspicture}

    \caption{Measurements and Resolution}
  \end{figure}
\end{frame}

\section{First Layer}
\begin{frame}
  \frametitle{First Layer}
  Depending on the type of printer and its settings, the first layer printed may be thinner or thicker that the subsequent layers.  This is usually accompanied with the material extending beyond or within the subsequent layers.  This effect is sometimes called ``elephant foot''.  The slicer program often has a setting for ``elephant foot compensation''.
\end{frame}
\begin{frame}
  \frametitle{First Layer}
  \begin{figure}
    %LaTeX with PSTricks extensions
%%Creator: inkscape 0.92.2
%%Please note this file requires PSTricks extensions
\psset{xunit=.5pt,yunit=.5pt,runit=.5pt}
\begin{pspicture}(500.00001922,230.3632601)
{
\newrgbcolor{curcolor}{0.40000001 0.40000001 0.40000001}
\pscustom[linestyle=none,fillstyle=solid,fillcolor=curcolor]
{
\newpath
\moveto(0.00000006,10.00000341)
\lineto(500.00001928,10.00000341)
\lineto(500.00001928,0.00000371)
\lineto(0.00000006,0.00000371)
\closepath
}
}
{
\newrgbcolor{curcolor}{0 0 0}
\pscustom[linestyle=none,fillstyle=solid,fillcolor=curcolor]
{
\newpath
\moveto(64.15107667,191.52872306)
\lineto(73.107131,191.52872306)
\lineto(73.107131,189.94669187)
\lineto(65.90010003,189.94669187)
\lineto(65.90010003,186.02677015)
\lineto(72.23701384,186.02677015)
\lineto(72.23701384,184.48868428)
\lineto(65.90010003,184.48868428)
\lineto(65.90010003,178.61759076)
\lineto(64.15107667,178.61759076)
\closepath
}
}
{
\newrgbcolor{curcolor}{0 0 0}
\pscustom[linestyle=none,fillstyle=solid,fillcolor=curcolor]
{
\newpath
\moveto(74.77705306,187.98673101)
\lineto(76.38545144,187.98673101)
\lineto(76.38545144,178.61759076)
\lineto(74.77705306,178.61759076)
\closepath
\moveto(74.77705306,191.52872306)
\lineto(76.38545144,191.52872306)
\lineto(76.38545144,189.73575438)
\lineto(74.77705306,189.73575438)
\closepath
}
}
{
\newrgbcolor{curcolor}{0 0 0}
\pscustom[linestyle=none,fillstyle=solid,fillcolor=curcolor]
{
\newpath
\moveto(78.82881033,188.03067632)
\lineto(80.33173996,188.03067632)
\lineto(80.33173996,186.40469983)
\curveto(80.45478683,186.72110606)(80.75654463,187.10489511)(81.23701336,187.55606697)
\curveto(81.71748209,188.0130982)(82.27119301,188.24161382)(82.89814611,188.24161382)
\curveto(82.92744298,188.24161382)(82.97724767,188.23868413)(83.04756017,188.23282475)
\curveto(83.11787266,188.22696538)(83.23798984,188.21524663)(83.40791171,188.1976685)
\lineto(83.40791171,186.5277467)
\curveto(83.31416172,186.54532482)(83.2262711,186.55704357)(83.14423985,186.56290295)
\curveto(83.06806798,186.56876232)(82.98310704,186.57169201)(82.88935705,186.57169201)
\curveto(82.09248208,186.57169201)(81.48017742,186.31387952)(81.05244306,185.79825454)
\curveto(80.6247087,185.28848893)(80.41084152,184.69962177)(80.41084152,184.03165305)
\lineto(80.41084152,178.61759076)
\lineto(78.82881033,178.61759076)
\closepath
}
}
{
\newrgbcolor{curcolor}{0 0 0}
\pscustom[linestyle=none,fillstyle=solid,fillcolor=curcolor]
{
\newpath
\moveto(85.73701306,181.57071565)
\curveto(85.78388806,181.04337192)(85.91572399,180.63907506)(86.13252086,180.35782507)
\curveto(86.53095834,179.84805946)(87.22236457,179.59317666)(88.20673953,179.59317666)
\curveto(88.792677,179.59317666)(89.30830198,179.71915322)(89.75361446,179.97110633)
\curveto(90.19892695,180.22891882)(90.42158319,180.62442662)(90.42158319,181.15762972)
\curveto(90.42158319,181.56192658)(90.24287226,181.86954376)(89.8854504,182.08048125)
\curveto(89.65693478,182.20938749)(89.20576292,182.35880155)(88.53193483,182.52872342)
\lineto(87.27509894,182.84512966)
\curveto(86.4723646,183.0443484)(85.88056774,183.26700464)(85.49970838,183.51309838)
\curveto(84.82002091,183.94083274)(84.48017718,184.53262959)(84.48017718,185.28848893)
\curveto(84.48017718,186.1791139)(84.7995131,186.89981699)(85.43818495,187.45059822)
\curveto(86.08271617,188.00137945)(86.94697395,188.27677006)(88.03095828,188.27677006)
\curveto(89.44892698,188.27677006)(90.47138787,187.86075446)(91.09834097,187.02872324)
\curveto(91.49091908,186.50137951)(91.68134876,185.93302016)(91.66963001,185.32364518)
\lineto(90.17548945,185.32364518)
\curveto(90.14619257,185.68106704)(90.02021602,186.00626234)(89.79755978,186.29923108)
\curveto(89.43427854,186.71524669)(88.80439575,186.92325449)(87.90791141,186.92325449)
\curveto(87.31025519,186.92325449)(86.85615364,186.80899669)(86.54560678,186.58048107)
\curveto(86.24091929,186.35196545)(86.08857555,186.05020765)(86.08857555,185.67520767)
\curveto(86.08857555,185.26505143)(86.29072398,184.93692645)(86.69502084,184.69083271)
\curveto(86.92939583,184.54434834)(87.27509894,184.41544209)(87.73213017,184.30411397)
\lineto(88.77802857,184.04923117)
\curveto(89.91474727,183.77384056)(90.67646599,183.50723901)(91.06318472,183.24942652)
\curveto(91.67841907,182.84512966)(91.98603625,182.20938749)(91.98603625,181.34220003)
\curveto(91.98603625,180.50430944)(91.66670033,179.78067665)(91.02802848,179.17130168)
\curveto(90.395216,178.5619267)(89.42841916,178.25723922)(88.12763797,178.25723922)
\curveto(86.7272474,178.25723922)(85.73408338,178.57364545)(85.1481459,179.20645793)
\curveto(84.5680678,179.84512978)(84.25752093,180.63321568)(84.21650531,181.57071565)
\closepath
\moveto(88.07490359,188.25919194)
\closepath
}
}
{
\newrgbcolor{curcolor}{0 0 0}
\pscustom[linestyle=none,fillstyle=solid,fillcolor=curcolor]
{
\newpath
\moveto(94.11298929,190.65860591)
\lineto(95.7125986,190.65860591)
\lineto(95.7125986,188.03067632)
\lineto(97.21552823,188.03067632)
\lineto(97.21552823,186.73868419)
\lineto(95.7125986,186.73868419)
\lineto(95.7125986,180.59512975)
\curveto(95.7125986,180.26700476)(95.82392672,180.04727821)(96.04658296,179.93595009)
\curveto(96.16962983,179.87149696)(96.37470795,179.8392704)(96.66181731,179.8392704)
\lineto(96.90791105,179.8392704)
\curveto(96.99580167,179.84512978)(97.09834073,179.85391884)(97.21552823,179.86563759)
\lineto(97.21552823,178.61759076)
\curveto(97.03388761,178.56485639)(96.84345793,178.52677045)(96.64423919,178.50333296)
\curveto(96.45087982,178.47989546)(96.23994233,178.46817671)(96.01142671,178.46817671)
\curveto(95.27314549,178.46817671)(94.77216895,178.6556767)(94.50849709,179.03067668)
\curveto(94.24482522,179.41153604)(94.11298929,179.90372352)(94.11298929,180.50723913)
\lineto(94.11298929,186.73868419)
\lineto(92.83857528,186.73868419)
\lineto(92.83857528,188.03067632)
\lineto(94.11298929,188.03067632)
\closepath
}
}
{
\newrgbcolor{curcolor}{0 0 0}
\pscustom[linestyle=none,fillstyle=solid,fillcolor=curcolor]
{
\newpath
\moveto(103.82490192,191.52872306)
\lineto(105.40693311,191.52872306)
\lineto(105.40693311,178.61759076)
\lineto(103.82490192,178.61759076)
\closepath
}
}
{
\newrgbcolor{curcolor}{0 0 0}
\pscustom[linestyle=none,fillstyle=solid,fillcolor=curcolor]
{
\newpath
\moveto(109.00166185,181.12247348)
\curveto(109.00166185,180.66544224)(109.16865403,180.3050907)(109.50263839,180.04141883)
\curveto(109.83662275,179.77774697)(110.23213055,179.64591103)(110.68916178,179.64591103)
\curveto(111.24580238,179.64591103)(111.78486486,179.77481728)(112.30634922,180.03262977)
\curveto(113.18525543,180.46036413)(113.62470854,181.16055941)(113.62470854,182.13321562)
\lineto(113.62470854,183.40762963)
\curveto(113.43134917,183.28458276)(113.18232574,183.18204371)(112.87763826,183.10001246)
\curveto(112.57295077,183.01798121)(112.27412266,182.95938746)(111.98115392,182.92423122)
\lineto(111.02314614,182.80118435)
\curveto(110.44892742,182.72501247)(110.01826337,182.60489529)(109.73115401,182.4408328)
\curveto(109.2448259,182.16544218)(109.00166185,181.72598908)(109.00166185,181.12247348)
\closepath
\moveto(112.83369295,184.3216921)
\curveto(113.19697418,184.3685671)(113.44013823,184.52091084)(113.5631851,184.77872333)
\curveto(113.6334976,184.91934832)(113.66865385,185.12149675)(113.66865385,185.38516862)
\curveto(113.66865385,185.9242311)(113.47529448,186.31387952)(113.08857575,186.55411388)
\curveto(112.70771639,186.80020762)(112.15986485,186.92325449)(111.44502113,186.92325449)
\curveto(110.61884928,186.92325449)(110.03291181,186.70059825)(109.6872087,186.25528577)
\curveto(109.49384933,186.00919203)(109.36787277,185.64298111)(109.30927902,185.156653)
\lineto(107.83271658,185.156653)
\curveto(107.86201346,186.31680921)(108.23701344,187.12247324)(108.95771654,187.57364509)
\curveto(109.68427901,188.03067632)(110.52509929,188.25919194)(111.48017738,188.25919194)
\curveto(112.58759921,188.25919194)(113.48701323,188.04825445)(114.17841945,187.62637947)
\curveto(114.8639663,187.20450448)(115.20673973,186.54825451)(115.20673973,185.65762954)
\lineto(115.20673973,180.2347782)
\curveto(115.20673973,180.07071571)(115.23896629,179.93887977)(115.30341941,179.8392704)
\curveto(115.37373191,179.73966103)(115.51728659,179.68985635)(115.73408346,179.68985635)
\curveto(115.80439595,179.68985635)(115.88349751,179.69278603)(115.97138813,179.69864541)
\curveto(116.05927875,179.71036416)(116.15302875,179.72501259)(116.25263812,179.74259072)
\lineto(116.25263812,178.57364545)
\curveto(116.00654438,178.50333296)(115.81904439,178.45938764)(115.69013814,178.44180952)
\curveto(115.5612319,178.4242314)(115.38545066,178.41544233)(115.16279442,178.41544233)
\curveto(114.61787256,178.41544233)(114.22236477,178.6088017)(113.97627103,178.99552044)
\curveto(113.84736478,179.20059855)(113.75654447,179.4906376)(113.7038101,179.86563759)
\curveto(113.38154449,179.44376261)(112.91865388,179.07755168)(112.31513828,178.76700482)
\curveto(111.71162268,178.45645796)(111.04658364,178.30118453)(110.32002117,178.30118453)
\curveto(109.44697433,178.30118453)(108.73213061,178.56485639)(108.17549001,179.09220012)
\curveto(107.62470878,179.62540322)(107.34931817,180.29044226)(107.34931817,181.08731723)
\curveto(107.34931817,181.96036407)(107.62177909,182.63712185)(108.16670095,183.11759058)
\curveto(108.7116228,183.59805931)(109.42646652,183.89395774)(110.31123211,184.00528586)
\closepath
\moveto(111.52412269,188.25919194)
\closepath
}
}
{
\newrgbcolor{curcolor}{0 0 0}
\pscustom[linestyle=none,fillstyle=solid,fillcolor=curcolor]
{
\newpath
\moveto(123.6881851,188.03067632)
\lineto(125.43720847,188.03067632)
\curveto(125.21455223,187.42716072)(124.71943506,186.05020765)(123.95185696,183.89981711)
\curveto(123.37763824,182.28262968)(122.89716951,180.96427036)(122.51045077,179.94473915)
\curveto(121.59638831,177.54239549)(120.95185708,176.0775518)(120.5768571,175.55020807)
\curveto(120.20185711,175.02286434)(119.55732589,174.75919248)(118.64326343,174.75919248)
\curveto(118.42060719,174.75919248)(118.24775563,174.76798154)(118.12470876,174.78555967)
\curveto(118.00752127,174.80313779)(117.8610369,174.83536435)(117.68525565,174.88223935)
\lineto(117.68525565,176.32364554)
\curveto(117.96064627,176.24747367)(118.15986501,176.20059867)(118.28291188,176.18302055)
\curveto(118.40595875,176.16544242)(118.51435718,176.15665336)(118.60810718,176.15665336)
\curveto(118.90107592,176.15665336)(119.1149431,176.20645805)(119.24970872,176.30606742)
\curveto(119.39033371,176.39981741)(119.50752121,176.51700491)(119.6012712,176.6576299)
\curveto(119.63056808,176.7045049)(119.73603682,176.94473927)(119.91767744,177.378333)
\curveto(120.09931806,177.81192673)(120.23115399,178.13419235)(120.31318524,178.34512984)
\lineto(116.83271662,188.03067632)
\lineto(118.6256853,188.03067632)
\lineto(121.14814614,180.36661413)
\closepath
\moveto(121.13935708,188.25919194)
\closepath
}
}
{
\newrgbcolor{curcolor}{0 0 0}
\pscustom[linestyle=none,fillstyle=solid,fillcolor=curcolor]
{
\newpath
\moveto(130.72822208,188.24161382)
\curveto(131.3961908,188.24161382)(132.04365171,188.0834107)(132.67060481,187.76700446)
\curveto(133.29755791,187.4564576)(133.77509696,187.05216074)(134.10322194,186.55411388)
\curveto(134.41962818,186.07950453)(134.63056567,185.52579361)(134.73603442,184.89298114)
\curveto(134.82978441,184.4593874)(134.87665941,183.76798118)(134.87665941,182.81876247)
\lineto(127.97724563,182.81876247)
\curveto(128.0065425,181.86368438)(128.23212843,181.09610629)(128.65400341,180.51602819)
\curveto(129.07587839,179.94180946)(129.72919868,179.6547001)(130.61396427,179.6547001)
\curveto(131.44013611,179.6547001)(132.09931577,179.92716102)(132.59150325,180.47208288)
\curveto(132.87275324,180.78848911)(133.07197198,181.15470004)(133.18915948,181.57071565)
\lineto(134.74482348,181.57071565)
\curveto(134.70380786,181.22501253)(134.56611255,180.8382938)(134.33173756,180.41055944)
\curveto(134.10322194,179.98868446)(133.84540945,179.64298135)(133.55830009,179.37345011)
\curveto(133.07783136,178.90470013)(132.48310482,178.58829389)(131.77412047,178.4242314)
\curveto(131.39326111,178.3304814)(130.96259707,178.2836064)(130.48212834,178.2836064)
\curveto(129.31025339,178.2836064)(128.31708936,178.70841107)(127.50263627,179.55802041)
\curveto(126.68818318,180.41348913)(126.28095663,181.60880158)(126.28095663,183.14395777)
\curveto(126.28095663,184.65567646)(126.69111287,185.88321547)(127.51142533,186.82657481)
\curveto(128.3317378,187.76993415)(129.40400338,188.24161382)(130.72822208,188.24161382)
\closepath
\moveto(133.25068292,184.07559836)
\curveto(133.18622979,184.7611452)(133.03681574,185.30899675)(132.80244075,185.71915298)
\curveto(132.36884701,186.4808717)(131.64521423,186.86173106)(130.63154239,186.86173106)
\curveto(129.90497992,186.86173106)(129.29560495,186.59805919)(128.80341747,186.07071546)
\curveto(128.31122999,185.54923111)(128.05048781,184.88419207)(128.02119094,184.07559836)
\closepath
\moveto(130.57880802,188.25919194)
\closepath
}
}
{
\newrgbcolor{curcolor}{0 0 0}
\pscustom[linestyle=none,fillstyle=solid,fillcolor=curcolor]
{
\newpath
\moveto(136.87177692,188.03067632)
\lineto(138.37470655,188.03067632)
\lineto(138.37470655,186.40469983)
\curveto(138.49775342,186.72110606)(138.79951122,187.10489511)(139.27997995,187.55606697)
\curveto(139.76044868,188.0130982)(140.3141596,188.24161382)(140.9411127,188.24161382)
\curveto(140.97040957,188.24161382)(141.02021426,188.23868413)(141.09052675,188.23282475)
\curveto(141.16083925,188.22696538)(141.28095643,188.21524663)(141.4508783,188.1976685)
\lineto(141.4508783,186.5277467)
\curveto(141.3571283,186.54532482)(141.26923768,186.55704357)(141.18720644,186.56290295)
\curveto(141.11103456,186.56876232)(141.02607363,186.57169201)(140.93232363,186.57169201)
\curveto(140.13544867,186.57169201)(139.523144,186.31387952)(139.09540964,185.79825454)
\curveto(138.66767529,185.28848893)(138.45380811,184.69962177)(138.45380811,184.03165305)
\lineto(138.45380811,178.61759076)
\lineto(136.87177692,178.61759076)
\closepath
}
}
{
\newrgbcolor{curcolor}{0 0 0}
\pscustom[linestyle=none,fillstyle=solid,fillcolor=curcolor]
{
\newpath
\moveto(148.14814121,190.65860591)
\lineto(149.74775053,190.65860591)
\lineto(149.74775053,188.03067632)
\lineto(151.25068015,188.03067632)
\lineto(151.25068015,186.73868419)
\lineto(149.74775053,186.73868419)
\lineto(149.74775053,180.59512975)
\curveto(149.74775053,180.26700476)(149.85907865,180.04727821)(150.08173489,179.93595009)
\curveto(150.20478176,179.87149696)(150.40985987,179.8392704)(150.69696924,179.8392704)
\lineto(150.94306298,179.8392704)
\curveto(151.0309536,179.84512978)(151.13349266,179.85391884)(151.25068015,179.86563759)
\lineto(151.25068015,178.61759076)
\curveto(151.06903953,178.56485639)(150.87860986,178.52677045)(150.67939111,178.50333296)
\curveto(150.48603175,178.47989546)(150.27509425,178.46817671)(150.04657864,178.46817671)
\curveto(149.30829742,178.46817671)(148.80732088,178.6556767)(148.54364901,179.03067668)
\curveto(148.27997715,179.41153604)(148.14814121,179.90372352)(148.14814121,180.50723913)
\lineto(148.14814121,186.73868419)
\lineto(146.8737272,186.73868419)
\lineto(146.8737272,188.03067632)
\lineto(148.14814121,188.03067632)
\closepath
}
}
{
\newrgbcolor{curcolor}{0 0 0}
\pscustom[linestyle=none,fillstyle=solid,fillcolor=curcolor]
{
\newpath
\moveto(156.55927677,179.63712197)
\curveto(157.60810486,179.63712197)(158.32587827,180.03262977)(158.712597,180.82364536)
\curveto(159.10517511,181.62052033)(159.30146416,182.50528592)(159.30146416,183.47794213)
\curveto(159.30146416,184.35684835)(159.16083917,185.07169207)(158.87958918,185.6224733)
\curveto(158.4342767,186.48966076)(157.6666986,186.92325449)(156.5768549,186.92325449)
\curveto(155.61005806,186.92325449)(154.90693309,186.55411388)(154.46747998,185.81583266)
\curveto(154.02802687,185.07755144)(153.80830032,184.18692648)(153.80830032,183.14395777)
\curveto(153.80830032,182.14200468)(154.02802687,181.30704378)(154.46747998,180.63907506)
\curveto(154.90693309,179.97110633)(155.60419869,179.63712197)(156.55927677,179.63712197)
\closepath
\moveto(156.62080021,188.30313725)
\curveto(157.83369078,188.30313725)(158.85908137,187.89884039)(159.69697196,187.09024667)
\curveto(160.53486255,186.28165296)(160.95380785,185.09219988)(160.95380785,183.52188744)
\curveto(160.95380785,182.00430938)(160.58466724,180.75040318)(159.84638602,179.76016884)
\curveto(159.1081048,178.76993451)(157.96259703,178.27481734)(156.40986272,178.27481734)
\curveto(155.11494089,178.27481734)(154.08662062,178.71134076)(153.3249019,179.5843876)
\curveto(152.56318318,180.46329381)(152.18232382,181.64102814)(152.18232382,183.11759058)
\curveto(152.18232382,184.69962177)(152.583691,185.95938734)(153.38642534,186.89688731)
\curveto(154.18915968,187.83438727)(155.26728464,188.30313725)(156.62080021,188.30313725)
\closepath
\moveto(156.56806584,188.25919194)
\closepath
}
}
{
\newrgbcolor{curcolor}{0 0 0}
\pscustom[linestyle=none,fillstyle=solid,fillcolor=curcolor]
{
\newpath
\moveto(166.57880802,179.63712197)
\curveto(167.62763611,179.63712197)(168.34540952,180.03262977)(168.73212825,180.82364536)
\curveto(169.12470636,181.62052033)(169.32099541,182.50528592)(169.32099541,183.47794213)
\curveto(169.32099541,184.35684835)(169.18037042,185.07169207)(168.89912043,185.6224733)
\curveto(168.45380795,186.48966076)(167.68622985,186.92325449)(166.59638615,186.92325449)
\curveto(165.62958931,186.92325449)(164.92646434,186.55411388)(164.48701123,185.81583266)
\curveto(164.04755812,185.07755144)(163.82783157,184.18692648)(163.82783157,183.14395777)
\curveto(163.82783157,182.14200468)(164.04755812,181.30704378)(164.48701123,180.63907506)
\curveto(164.92646434,179.97110633)(165.62372994,179.63712197)(166.57880802,179.63712197)
\closepath
\moveto(166.64033146,188.30313725)
\curveto(167.85322203,188.30313725)(168.87861262,187.89884039)(169.71650321,187.09024667)
\curveto(170.5543938,186.28165296)(170.9733391,185.09219988)(170.9733391,183.52188744)
\curveto(170.9733391,182.00430938)(170.60419849,180.75040318)(169.86591727,179.76016884)
\curveto(169.12763605,178.76993451)(167.98212828,178.27481734)(166.42939397,178.27481734)
\curveto(165.13447214,178.27481734)(164.10615187,178.71134076)(163.34443315,179.5843876)
\curveto(162.58271443,180.46329381)(162.20185507,181.64102814)(162.20185507,183.11759058)
\curveto(162.20185507,184.69962177)(162.60322225,185.95938734)(163.40595659,186.89688731)
\curveto(164.20869093,187.83438727)(165.28681589,188.30313725)(166.64033146,188.30313725)
\closepath
\moveto(166.58759709,188.25919194)
\closepath
}
}
{
\newrgbcolor{curcolor}{0 0 0}
\pscustom[linestyle=none,fillstyle=solid,fillcolor=curcolor]
{
\newpath
\moveto(178.17158448,190.65860591)
\lineto(179.77119379,190.65860591)
\lineto(179.77119379,188.03067632)
\lineto(181.27412342,188.03067632)
\lineto(181.27412342,186.73868419)
\lineto(179.77119379,186.73868419)
\lineto(179.77119379,180.59512975)
\curveto(179.77119379,180.26700476)(179.88252191,180.04727821)(180.10517815,179.93595009)
\curveto(180.22822502,179.87149696)(180.43330314,179.8392704)(180.7204125,179.8392704)
\lineto(180.96650624,179.8392704)
\curveto(181.05439687,179.84512978)(181.15693592,179.85391884)(181.27412342,179.86563759)
\lineto(181.27412342,178.61759076)
\curveto(181.0924828,178.56485639)(180.90205312,178.52677045)(180.70283438,178.50333296)
\curveto(180.50947501,178.47989546)(180.29853752,178.46817671)(180.07002191,178.46817671)
\curveto(179.33174069,178.46817671)(178.83076414,178.6556767)(178.56709228,179.03067668)
\curveto(178.30342041,179.41153604)(178.17158448,179.90372352)(178.17158448,180.50723913)
\lineto(178.17158448,186.73868419)
\lineto(176.89717047,186.73868419)
\lineto(176.89717047,188.03067632)
\lineto(178.17158448,188.03067632)
\closepath
}
}
{
\newrgbcolor{curcolor}{0 0 0}
\pscustom[linestyle=none,fillstyle=solid,fillcolor=curcolor]
{
\newpath
\moveto(182.84736142,191.57266837)
\lineto(184.42939261,191.57266837)
\lineto(184.42939261,186.75626231)
\curveto(184.80439259,187.23087167)(185.14130664,187.56485603)(185.44013475,187.7582154)
\curveto(185.94990036,188.09219976)(186.58564252,188.25919194)(187.34736124,188.25919194)
\curveto(188.71259556,188.25919194)(189.63837677,187.7816529)(190.12470488,186.82657481)
\curveto(190.38837674,186.30509046)(190.52021267,185.58145767)(190.52021267,184.65567646)
\lineto(190.52021267,178.61759076)
\lineto(188.89423618,178.61759076)
\lineto(188.89423618,184.55020771)
\curveto(188.89423618,185.24161394)(188.80634556,185.74844985)(188.63056431,186.07071546)
\curveto(188.34345495,186.58634044)(187.80439247,186.84415293)(187.01337688,186.84415293)
\curveto(186.3571269,186.84415293)(185.76240036,186.61856701)(185.22919726,186.16739515)
\curveto(184.69599416,185.71622329)(184.42939261,184.86368426)(184.42939261,183.60977806)
\lineto(184.42939261,178.61759076)
\lineto(182.84736142,178.61759076)
\closepath
}
}
{
\newrgbcolor{curcolor}{0 0 0}
\pscustom[linestyle=none,fillstyle=solid,fillcolor=curcolor]
{
\newpath
\moveto(192.86689988,187.98673101)
\lineto(194.47529825,187.98673101)
\lineto(194.47529825,178.61759076)
\lineto(192.86689988,178.61759076)
\closepath
\moveto(192.86689988,191.52872306)
\lineto(194.47529825,191.52872306)
\lineto(194.47529825,189.73575438)
\lineto(192.86689988,189.73575438)
\closepath
}
}
{
\newrgbcolor{curcolor}{0 0 0}
\pscustom[linestyle=none,fillstyle=solid,fillcolor=curcolor]
{
\newpath
\moveto(196.87470373,188.03067632)
\lineto(198.37763335,188.03067632)
\lineto(198.37763335,186.69473888)
\curveto(198.82294584,187.24552011)(199.29462551,187.6410279)(199.79267236,187.88126227)
\curveto(200.29071922,188.12149663)(200.84443013,188.24161382)(201.45380511,188.24161382)
\curveto(202.78974255,188.24161382)(203.69208627,187.77579352)(204.16083625,186.84415293)
\curveto(204.41864874,186.33438733)(204.54755498,185.60489517)(204.54755498,184.65567646)
\lineto(204.54755498,178.61759076)
\lineto(202.93915661,178.61759076)
\lineto(202.93915661,184.55020771)
\curveto(202.93915661,185.12442644)(202.85419568,185.58731705)(202.68427381,185.93887953)
\curveto(202.40302382,186.52481701)(201.89325821,186.81778575)(201.15497699,186.81778575)
\curveto(200.77997701,186.81778575)(200.47235983,186.77969981)(200.23212547,186.70352794)
\curveto(199.79853173,186.57462169)(199.41767238,186.31680921)(199.08954739,185.93009047)
\curveto(198.82587552,185.61954361)(198.65302397,185.297278)(198.57099272,184.96329363)
\curveto(198.49482085,184.63516865)(198.45673491,184.16348898)(198.45673491,183.54825463)
\lineto(198.45673491,178.61759076)
\lineto(196.87470373,178.61759076)
\closepath
\moveto(200.59247702,188.25919194)
\closepath
}
}
{
\newrgbcolor{curcolor}{0 0 0}
\pscustom[linestyle=none,fillstyle=solid,fillcolor=curcolor]
{
\newpath
\moveto(316.91587072,230.31931384)
\lineto(325.87192505,230.31931384)
\lineto(325.87192505,228.73728265)
\lineto(318.66489409,228.73728265)
\lineto(318.66489409,224.81736094)
\lineto(325.0018079,224.81736094)
\lineto(325.0018079,223.27927506)
\lineto(318.66489409,223.27927506)
\lineto(318.66489409,217.40818155)
\lineto(316.91587072,217.40818155)
\closepath
}
}
{
\newrgbcolor{curcolor}{0 0 0}
\pscustom[linestyle=none,fillstyle=solid,fillcolor=curcolor]
{
\newpath
\moveto(327.54184712,226.7773218)
\lineto(329.15024549,226.7773218)
\lineto(329.15024549,217.40818155)
\lineto(327.54184712,217.40818155)
\closepath
\moveto(327.54184712,230.31931384)
\lineto(329.15024549,230.31931384)
\lineto(329.15024549,228.52634516)
\lineto(327.54184712,228.52634516)
\closepath
}
}
{
\newrgbcolor{curcolor}{0 0 0}
\pscustom[linestyle=none,fillstyle=solid,fillcolor=curcolor]
{
\newpath
\moveto(331.59360439,226.82126711)
\lineto(333.09653402,226.82126711)
\lineto(333.09653402,225.19529061)
\curveto(333.21958089,225.51169685)(333.52133869,225.89548589)(334.00180742,226.34665775)
\curveto(334.48227615,226.80368898)(335.03598707,227.0322046)(335.66294017,227.0322046)
\curveto(335.69223704,227.0322046)(335.74204172,227.02927491)(335.81235422,227.02341554)
\curveto(335.88266672,227.01755616)(336.0027839,227.00583741)(336.17270577,226.98825929)
\lineto(336.17270577,225.31833748)
\curveto(336.07895577,225.3359156)(335.99106515,225.34763435)(335.90903391,225.35349373)
\curveto(335.83286203,225.3593531)(335.7479011,225.36228279)(335.6541511,225.36228279)
\curveto(334.85727614,225.36228279)(334.24497147,225.1044703)(333.81723711,224.58884532)
\curveto(333.38950276,224.07907972)(333.17563558,223.49021255)(333.17563558,222.82224383)
\lineto(333.17563558,217.40818155)
\lineto(331.59360439,217.40818155)
\closepath
}
}
{
\newrgbcolor{curcolor}{0 0 0}
\pscustom[linestyle=none,fillstyle=solid,fillcolor=curcolor]
{
\newpath
\moveto(338.50180712,220.36130643)
\curveto(338.54868212,219.8339627)(338.68051805,219.42966584)(338.89731492,219.14841585)
\curveto(339.2957524,218.63865025)(339.98715862,218.38376744)(340.97153358,218.38376744)
\curveto(341.55747106,218.38376744)(342.07309604,218.509744)(342.51840852,218.76169712)
\curveto(342.963721,219.01950961)(343.18637724,219.4150174)(343.18637724,219.94822051)
\curveto(343.18637724,220.35251736)(343.00766631,220.66013454)(342.65024445,220.87107203)
\curveto(342.42172884,220.99997828)(341.97055698,221.14939233)(341.29672888,221.3193142)
\lineto(340.039893,221.63572044)
\curveto(339.23715865,221.83493918)(338.6453618,222.05759542)(338.26450244,222.30368916)
\curveto(337.58481497,222.73142352)(337.24497123,223.32322037)(337.24497123,224.07907972)
\curveto(337.24497123,224.96970468)(337.56430716,225.69040778)(338.20297901,226.241189)
\curveto(338.84751023,226.79197023)(339.71176801,227.06736085)(340.79575234,227.06736085)
\curveto(342.21372103,227.06736085)(343.23618193,226.65134524)(343.86313503,225.81931402)
\curveto(344.25571314,225.29197029)(344.44614282,224.72361094)(344.43442407,224.11423596)
\lineto(342.9402835,224.11423596)
\curveto(342.91098663,224.47165782)(342.78501007,224.79685312)(342.56235383,225.08982186)
\curveto(342.1990726,225.50583747)(341.56918981,225.71384528)(340.67270547,225.71384528)
\curveto(340.07504924,225.71384528)(339.6209477,225.59958747)(339.31040084,225.37107185)
\curveto(339.00571335,225.14255624)(338.85336961,224.84079844)(338.85336961,224.46579845)
\curveto(338.85336961,224.05564222)(339.05551803,223.72751723)(339.45981489,223.48142349)
\curveto(339.69418988,223.33493912)(340.039893,223.20603288)(340.49692423,223.09470476)
\lineto(341.54282262,222.83982195)
\curveto(342.67954133,222.56443134)(343.44126005,222.29782979)(343.82797878,222.0400173)
\curveto(344.44321313,221.63572044)(344.75083031,220.99997828)(344.75083031,220.13279081)
\curveto(344.75083031,219.29490022)(344.43149438,218.57126744)(343.79282253,217.96189246)
\curveto(343.16001006,217.35251749)(342.19321322,217.04783)(340.89243202,217.04783)
\curveto(339.49204145,217.04783)(338.49887743,217.36423623)(337.91293996,217.99704871)
\curveto(337.33286185,218.63572056)(337.02231499,219.42380646)(336.98129937,220.36130643)
\closepath
\moveto(340.83969765,227.04978272)
\closepath
}
}
{
\newrgbcolor{curcolor}{0 0 0}
\pscustom[linestyle=none,fillstyle=solid,fillcolor=curcolor]
{
\newpath
\moveto(346.87778335,229.44919669)
\lineto(348.47739266,229.44919669)
\lineto(348.47739266,226.82126711)
\lineto(349.98032228,226.82126711)
\lineto(349.98032228,225.52927497)
\lineto(348.47739266,225.52927497)
\lineto(348.47739266,219.38572053)
\curveto(348.47739266,219.05759554)(348.58872078,218.83786899)(348.81137702,218.72654087)
\curveto(348.93442389,218.66208775)(349.13950201,218.62986118)(349.42661137,218.62986118)
\lineto(349.67270511,218.62986118)
\curveto(349.76059573,218.63572056)(349.86313479,218.64450962)(349.98032228,218.65622837)
\lineto(349.98032228,217.40818155)
\curveto(349.79868167,217.35544717)(349.60825199,217.31736124)(349.40903325,217.29392374)
\curveto(349.21567388,217.27048624)(349.00473639,217.25876749)(348.77622077,217.25876749)
\curveto(348.03793955,217.25876749)(347.53696301,217.44626748)(347.27329114,217.82126747)
\curveto(347.00961928,218.20212683)(346.87778335,218.69431431)(346.87778335,219.29782991)
\lineto(346.87778335,225.52927497)
\lineto(345.60336934,225.52927497)
\lineto(345.60336934,226.82126711)
\lineto(346.87778335,226.82126711)
\closepath
}
}
{
\newrgbcolor{curcolor}{0 0 0}
\pscustom[linestyle=none,fillstyle=solid,fillcolor=curcolor]
{
\newpath
\moveto(356.58969598,230.31931384)
\lineto(358.17172716,230.31931384)
\lineto(358.17172716,217.40818155)
\lineto(356.58969598,217.40818155)
\closepath
}
}
{
\newrgbcolor{curcolor}{0 0 0}
\pscustom[linestyle=none,fillstyle=solid,fillcolor=curcolor]
{
\newpath
\moveto(361.76645591,219.91306426)
\curveto(361.76645591,219.45603303)(361.93344809,219.09568148)(362.26743245,218.83200961)
\curveto(362.60141681,218.56833775)(362.99692461,218.43650182)(363.45395584,218.43650182)
\curveto(364.01059644,218.43650182)(364.54965892,218.56540806)(365.07114327,218.82322055)
\curveto(365.95004949,219.25095491)(366.3895026,219.95115019)(366.3895026,220.9238064)
\lineto(366.3895026,222.19822042)
\curveto(366.19614323,222.07517355)(365.9471198,221.97263449)(365.64243231,221.89060324)
\curveto(365.33774483,221.80857199)(365.03891671,221.74997825)(364.74594797,221.714822)
\lineto(363.7879402,221.59177513)
\curveto(363.21372147,221.51560326)(362.78305743,221.39548607)(362.49594806,221.23142358)
\curveto(362.00961996,220.95603297)(361.76645591,220.51657986)(361.76645591,219.91306426)
\closepath
\moveto(365.598487,223.11228288)
\curveto(365.96176824,223.15915788)(366.20493229,223.31150162)(366.32797916,223.56931411)
\curveto(366.39829166,223.70993911)(366.43344791,223.91208753)(366.43344791,224.1757594)
\curveto(366.43344791,224.71482188)(366.24008854,225.1044703)(365.85336981,225.34470466)
\curveto(365.47251045,225.59079841)(364.92465891,225.71384528)(364.20981518,225.71384528)
\curveto(363.38364334,225.71384528)(362.79770587,225.49118903)(362.45200275,225.04587655)
\curveto(362.25864339,224.79978281)(362.13266683,224.43357189)(362.07407308,223.94724378)
\lineto(360.59751064,223.94724378)
\curveto(360.62680751,225.10739999)(361.0018075,225.91306402)(361.7225106,226.36423587)
\curveto(362.44907307,226.82126711)(363.28989335,227.04978272)(364.24497143,227.04978272)
\curveto(365.35239326,227.04978272)(366.25180729,226.83884523)(366.94321351,226.41697025)
\curveto(367.62876036,225.99509526)(367.97153378,225.33884529)(367.97153378,224.44822033)
\lineto(367.97153378,219.02536898)
\curveto(367.97153378,218.86130649)(368.00376034,218.72947055)(368.06821347,218.62986118)
\curveto(368.13852596,218.53025181)(368.28208065,218.48044713)(368.49887751,218.48044713)
\curveto(368.56919001,218.48044713)(368.64829157,218.48337681)(368.73618219,218.48923619)
\curveto(368.82407281,218.50095494)(368.91782281,218.51560338)(369.01743218,218.5331815)
\lineto(369.01743218,217.36423623)
\curveto(368.77133844,217.29392374)(368.58383845,217.24997843)(368.4549322,217.2324003)
\curveto(368.32602596,217.21482218)(368.15024471,217.20603312)(367.92758847,217.20603312)
\curveto(367.38266662,217.20603312)(366.98715882,217.39939248)(366.74106508,217.78611122)
\curveto(366.61215884,217.99118933)(366.52133853,218.28122839)(366.46860416,218.65622837)
\curveto(366.14633854,218.23435339)(365.68344794,217.86814246)(365.07993234,217.5575956)
\curveto(364.47641674,217.24704874)(363.8113777,217.09177531)(363.08481523,217.09177531)
\curveto(362.21176839,217.09177531)(361.49692467,217.35544717)(360.94028406,217.8827909)
\curveto(360.38950284,218.415994)(360.11411222,219.08103304)(360.11411222,219.87790801)
\curveto(360.11411222,220.75095485)(360.38657315,221.42771263)(360.931495,221.90818137)
\curveto(361.47641686,222.3886501)(362.19126058,222.68454852)(363.07602617,222.79587664)
\closepath
\moveto(364.28891674,227.04978272)
\closepath
}
}
{
\newrgbcolor{curcolor}{0 0 0}
\pscustom[linestyle=none,fillstyle=solid,fillcolor=curcolor]
{
\newpath
\moveto(376.45297916,226.82126711)
\lineto(378.20200252,226.82126711)
\curveto(377.97934628,226.2177515)(377.48422912,224.84079844)(376.71665102,222.6904079)
\curveto(376.14243229,221.07322046)(375.66196356,219.75486114)(375.27524483,218.73532993)
\curveto(374.36118237,216.33298628)(373.71665114,214.86814258)(373.34165116,214.34079886)
\curveto(372.96665117,213.81345513)(372.32211995,213.54978326)(371.40805748,213.54978326)
\curveto(371.18540124,213.54978326)(371.01254969,213.55857232)(370.88950282,213.57615045)
\curveto(370.77231532,213.59372857)(370.62583095,213.62595513)(370.45004971,213.67283013)
\lineto(370.45004971,215.11423632)
\curveto(370.72544032,215.03806445)(370.92465907,214.99118945)(371.04770594,214.97361133)
\curveto(371.17075281,214.95603321)(371.27915124,214.94724414)(371.37290124,214.94724414)
\curveto(371.66586997,214.94724414)(371.87973715,214.99704883)(372.01450277,215.0966582)
\curveto(372.15512777,215.1904082)(372.27231526,215.30759569)(372.36606526,215.44822069)
\curveto(372.39536213,215.49509568)(372.50083088,215.73533005)(372.6824715,216.16892378)
\curveto(372.86411211,216.60251752)(372.99594805,216.92478313)(373.07797929,217.13572062)
\lineto(369.59751068,226.82126711)
\lineto(371.39047936,226.82126711)
\lineto(373.9129402,219.15720491)
\closepath
\moveto(373.90415113,227.04978272)
\closepath
}
}
{
\newrgbcolor{curcolor}{0 0 0}
\pscustom[linestyle=none,fillstyle=solid,fillcolor=curcolor]
{
\newpath
\moveto(383.49301614,227.0322046)
\curveto(384.16098486,227.0322046)(384.80844577,226.87400148)(385.43539887,226.55759524)
\curveto(386.06235197,226.24704838)(386.53989101,225.84275152)(386.868016,225.34470466)
\curveto(387.18442224,224.87009531)(387.39535973,224.31638439)(387.50082847,223.68357192)
\curveto(387.59457847,223.24997819)(387.64145347,222.55857196)(387.64145347,221.60935325)
\lineto(380.74203968,221.60935325)
\curveto(380.77133656,220.65427517)(380.99692249,219.88669707)(381.41879747,219.30661897)
\curveto(381.84067245,218.73240024)(382.49399274,218.44529088)(383.37875833,218.44529088)
\curveto(384.20493017,218.44529088)(384.86410983,218.71775181)(385.35629731,219.26267366)
\curveto(385.6375473,219.5790799)(385.83676604,219.94529082)(385.95395354,220.36130643)
\lineto(387.50961754,220.36130643)
\curveto(387.46860191,220.01560332)(387.33090661,219.62888458)(387.09653162,219.20115022)
\curveto(386.868016,218.77927524)(386.61020351,218.43357213)(386.32309415,218.16404089)
\curveto(385.84262542,217.69529091)(385.24789888,217.37888467)(384.53891453,217.21482218)
\curveto(384.15805517,217.12107218)(383.72739113,217.07419718)(383.2469224,217.07419718)
\curveto(382.07504744,217.07419718)(381.08188342,217.49900185)(380.26743033,218.3486112)
\curveto(379.45297723,219.20407991)(379.04575069,220.39939236)(379.04575069,221.93454855)
\curveto(379.04575069,223.44626724)(379.45590692,224.67380625)(380.27621939,225.61716559)
\curveto(381.09653186,226.56052493)(382.16879744,227.0322046)(383.49301614,227.0322046)
\closepath
\moveto(386.01547697,222.86618914)
\curveto(385.95102385,223.55173599)(385.80160979,224.09958753)(385.5672348,224.50974376)
\curveto(385.13364107,225.27146248)(384.41000829,225.65232184)(383.39633645,225.65232184)
\curveto(382.66977398,225.65232184)(382.06039901,225.38864998)(381.56821153,224.86130625)
\curveto(381.07602404,224.33982189)(380.81528187,223.67478286)(380.78598499,222.86618914)
\closepath
\moveto(383.34360208,227.04978272)
\closepath
}
}
{
\newrgbcolor{curcolor}{0 0 0}
\pscustom[linestyle=none,fillstyle=solid,fillcolor=curcolor]
{
\newpath
\moveto(389.63657098,226.82126711)
\lineto(391.13950061,226.82126711)
\lineto(391.13950061,225.19529061)
\curveto(391.26254748,225.51169685)(391.56430528,225.89548589)(392.04477401,226.34665775)
\curveto(392.52524274,226.80368898)(393.07895365,227.0322046)(393.70590675,227.0322046)
\curveto(393.73520363,227.0322046)(393.78500831,227.02927491)(393.85532081,227.02341554)
\curveto(393.92563331,227.01755616)(394.04575049,227.00583741)(394.21567236,226.98825929)
\lineto(394.21567236,225.31833748)
\curveto(394.12192236,225.3359156)(394.03403174,225.34763435)(393.95200049,225.35349373)
\curveto(393.87582862,225.3593531)(393.79086769,225.36228279)(393.69711769,225.36228279)
\curveto(392.90024272,225.36228279)(392.28793806,225.1044703)(391.8602037,224.58884532)
\curveto(391.43246934,224.07907972)(391.21860216,223.49021255)(391.21860216,222.82224383)
\lineto(391.21860216,217.40818155)
\lineto(389.63657098,217.40818155)
\closepath
}
}
{
\newrgbcolor{curcolor}{0 0 0}
\pscustom[linestyle=none,fillstyle=solid,fillcolor=curcolor]
{
\newpath
\moveto(400.91293527,229.44919669)
\lineto(402.51254458,229.44919669)
\lineto(402.51254458,226.82126711)
\lineto(404.01547421,226.82126711)
\lineto(404.01547421,225.52927497)
\lineto(402.51254458,225.52927497)
\lineto(402.51254458,219.38572053)
\curveto(402.51254458,219.05759554)(402.6238727,218.83786899)(402.84652894,218.72654087)
\curveto(402.96957581,218.66208775)(403.17465393,218.62986118)(403.46176329,218.62986118)
\lineto(403.70785703,218.62986118)
\curveto(403.79574766,218.63572056)(403.89828671,218.64450962)(404.01547421,218.65622837)
\lineto(404.01547421,217.40818155)
\curveto(403.83383359,217.35544717)(403.64340391,217.31736124)(403.44418517,217.29392374)
\curveto(403.2508258,217.27048624)(403.03988831,217.25876749)(402.8113727,217.25876749)
\curveto(402.07309147,217.25876749)(401.57211493,217.44626748)(401.30844307,217.82126747)
\curveto(401.0447712,218.20212683)(400.91293527,218.69431431)(400.91293527,219.29782991)
\lineto(400.91293527,225.52927497)
\lineto(399.63852126,225.52927497)
\lineto(399.63852126,226.82126711)
\lineto(400.91293527,226.82126711)
\closepath
}
}
{
\newrgbcolor{curcolor}{0 0 0}
\pscustom[linestyle=none,fillstyle=solid,fillcolor=curcolor]
{
\newpath
\moveto(409.32407083,218.42771275)
\curveto(410.37289891,218.42771275)(411.09067232,218.82322055)(411.47739106,219.61423614)
\curveto(411.86996917,220.41111111)(412.06625822,221.2958767)(412.06625822,222.26853291)
\curveto(412.06625822,223.14743913)(411.92563323,223.86228285)(411.64438324,224.41306408)
\curveto(411.19907076,225.28025154)(410.43149266,225.71384528)(409.34164895,225.71384528)
\curveto(408.37485212,225.71384528)(407.67172715,225.34470466)(407.23227404,224.60642344)
\curveto(406.79282093,223.86814222)(406.57309438,222.97751726)(406.57309438,221.93454855)
\curveto(406.57309438,220.93259547)(406.79282093,220.09763456)(407.23227404,219.42966584)
\curveto(407.67172715,218.76169712)(408.36899274,218.42771275)(409.32407083,218.42771275)
\closepath
\moveto(409.38559427,227.09372803)
\curveto(410.59848484,227.09372803)(411.62387543,226.68943117)(412.46176602,225.88083746)
\curveto(413.29965661,225.07224374)(413.7186019,223.88279066)(413.7186019,222.31247822)
\curveto(413.7186019,220.79490016)(413.34946129,219.54099396)(412.61118007,218.55075962)
\curveto(411.87289885,217.56052529)(410.72739109,217.06540812)(409.17465677,217.06540812)
\curveto(407.87973495,217.06540812)(406.85141468,217.50193154)(406.08969596,218.37497838)
\curveto(405.32797724,219.2538846)(404.94711788,220.43161892)(404.94711788,221.90818137)
\curveto(404.94711788,223.49021255)(405.34848505,224.74997813)(406.15121939,225.68747809)
\curveto(406.95395374,226.62497805)(408.03207869,227.09372803)(409.38559427,227.09372803)
\closepath
\moveto(409.33285989,227.04978272)
\closepath
}
}
{
\newrgbcolor{curcolor}{0 0 0}
\pscustom[linestyle=none,fillstyle=solid,fillcolor=curcolor]
{
\newpath
\moveto(419.34360208,218.42771275)
\curveto(420.39243016,218.42771275)(421.11020357,218.82322055)(421.49692231,219.61423614)
\curveto(421.88950042,220.41111111)(422.08578947,221.2958767)(422.08578947,222.26853291)
\curveto(422.08578947,223.14743913)(421.94516448,223.86228285)(421.66391449,224.41306408)
\curveto(421.21860201,225.28025154)(420.45102391,225.71384528)(419.3611802,225.71384528)
\curveto(418.39438337,225.71384528)(417.6912584,225.34470466)(417.25180529,224.60642344)
\curveto(416.81235218,223.86814222)(416.59262563,222.97751726)(416.59262563,221.93454855)
\curveto(416.59262563,220.93259547)(416.81235218,220.09763456)(417.25180529,219.42966584)
\curveto(417.6912584,218.76169712)(418.38852399,218.42771275)(419.34360208,218.42771275)
\closepath
\moveto(419.40512552,227.09372803)
\curveto(420.61801609,227.09372803)(421.64340668,226.68943117)(422.48129727,225.88083746)
\curveto(423.31918786,225.07224374)(423.73813315,223.88279066)(423.73813315,222.31247822)
\curveto(423.73813315,220.79490016)(423.36899254,219.54099396)(422.63071132,218.55075962)
\curveto(421.8924301,217.56052529)(420.74692234,217.06540812)(419.19418802,217.06540812)
\curveto(417.8992662,217.06540812)(416.87094593,217.50193154)(416.10922721,218.37497838)
\curveto(415.34750849,219.2538846)(414.96664913,220.43161892)(414.96664913,221.90818137)
\curveto(414.96664913,223.49021255)(415.3680163,224.74997813)(416.17075064,225.68747809)
\curveto(416.97348499,226.62497805)(418.05160994,227.09372803)(419.40512552,227.09372803)
\closepath
\moveto(419.35239114,227.04978272)
\closepath
}
}
{
\newrgbcolor{curcolor}{0 0 0}
\pscustom[linestyle=none,fillstyle=solid,fillcolor=curcolor]
{
\newpath
\moveto(430.93637854,229.44919669)
\lineto(432.53598785,229.44919669)
\lineto(432.53598785,226.82126711)
\lineto(434.03891748,226.82126711)
\lineto(434.03891748,225.52927497)
\lineto(432.53598785,225.52927497)
\lineto(432.53598785,219.38572053)
\curveto(432.53598785,219.05759554)(432.64731597,218.83786899)(432.86997221,218.72654087)
\curveto(432.99301908,218.66208775)(433.1980972,218.62986118)(433.48520656,218.62986118)
\lineto(433.7313003,218.62986118)
\curveto(433.81919092,218.63572056)(433.92172998,218.64450962)(434.03891748,218.65622837)
\lineto(434.03891748,217.40818155)
\curveto(433.85727686,217.35544717)(433.66684718,217.31736124)(433.46762844,217.29392374)
\curveto(433.27426907,217.27048624)(433.06333158,217.25876749)(432.83481596,217.25876749)
\curveto(432.09653474,217.25876749)(431.5955582,217.44626748)(431.33188633,217.82126747)
\curveto(431.06821447,218.20212683)(430.93637854,218.69431431)(430.93637854,219.29782991)
\lineto(430.93637854,225.52927497)
\lineto(429.66196453,225.52927497)
\lineto(429.66196453,226.82126711)
\lineto(430.93637854,226.82126711)
\closepath
}
}
{
\newrgbcolor{curcolor}{0 0 0}
\pscustom[linestyle=none,fillstyle=solid,fillcolor=curcolor]
{
\newpath
\moveto(435.61215548,230.36325915)
\lineto(437.19418666,230.36325915)
\lineto(437.19418666,225.54685309)
\curveto(437.56918665,226.02146245)(437.9061007,226.35544681)(438.20492881,226.54880618)
\curveto(438.71469441,226.88279054)(439.35043658,227.04978272)(440.1121553,227.04978272)
\curveto(441.47738962,227.04978272)(442.40317083,226.57224368)(442.88949893,225.61716559)
\curveto(443.1531708,225.09568124)(443.28500673,224.37204845)(443.28500673,223.44626724)
\lineto(443.28500673,217.40818155)
\lineto(441.65903023,217.40818155)
\lineto(441.65903023,223.3407985)
\curveto(441.65903023,224.03220472)(441.57113961,224.53904063)(441.39535837,224.86130625)
\curveto(441.10824901,225.37693123)(440.56918653,225.63474372)(439.77817093,225.63474372)
\curveto(439.12192096,225.63474372)(438.52719442,225.40915779)(437.99399132,224.95798593)
\curveto(437.46078821,224.50681407)(437.19418666,223.65427505)(437.19418666,222.40036885)
\lineto(437.19418666,217.40818155)
\lineto(435.61215548,217.40818155)
\closepath
}
}
{
\newrgbcolor{curcolor}{0 0 0}
\pscustom[linestyle=none,fillstyle=solid,fillcolor=curcolor]
{
\newpath
\moveto(445.63169393,226.7773218)
\lineto(447.24009231,226.7773218)
\lineto(447.24009231,217.40818155)
\lineto(445.63169393,217.40818155)
\closepath
\moveto(445.63169393,230.31931384)
\lineto(447.24009231,230.31931384)
\lineto(447.24009231,228.52634516)
\lineto(445.63169393,228.52634516)
\closepath
}
}
{
\newrgbcolor{curcolor}{0 0 0}
\pscustom[linestyle=none,fillstyle=solid,fillcolor=curcolor]
{
\newpath
\moveto(453.26938045,227.09372803)
\curveto(454.32992728,227.09372803)(455.19125537,226.83591554)(455.85336472,226.32029056)
\curveto(456.52133345,225.80466558)(456.92270062,224.91697031)(457.05746624,223.65720473)
\lineto(455.51938036,223.65720473)
\curveto(455.42563036,224.23728283)(455.21176319,224.71775157)(454.87777882,225.09861092)
\curveto(454.54379446,225.48532966)(454.00766167,225.67868903)(453.26938045,225.67868903)
\curveto(452.26156799,225.67868903)(451.5408649,225.18650155)(451.10727116,224.20212659)
\curveto(450.82602117,223.56345474)(450.68539618,222.77536883)(450.68539618,221.83786887)
\curveto(450.68539618,220.89450953)(450.88461492,220.10056425)(451.28305241,219.45603303)
\curveto(451.68148989,218.8115018)(452.30844299,218.48923619)(453.16391171,218.48923619)
\curveto(453.82016168,218.48923619)(454.33871635,218.68845493)(454.71957571,219.08689242)
\curveto(455.10629444,219.49118927)(455.37289599,220.0419705)(455.51938036,220.7392361)
\lineto(457.05746624,220.7392361)
\curveto(456.88168499,219.49118927)(456.44223189,218.57712681)(455.73910691,217.99704871)
\curveto(455.03598194,217.42282998)(454.13656792,217.13572062)(453.04086484,217.13572062)
\curveto(451.81039613,217.13572062)(450.82895086,217.58396279)(450.09652902,218.48044713)
\curveto(449.36410717,219.38279084)(448.99789625,220.5077908)(448.99789625,221.85544699)
\curveto(448.99789625,223.50779068)(449.39926342,224.79392344)(450.20199776,225.71384528)
\curveto(451.0047321,226.63376711)(452.027193,227.09372803)(453.26938045,227.09372803)
\closepath
\moveto(453.02328671,227.04978272)
\closepath
}
}
{
\newrgbcolor{curcolor}{0 0 0}
\pscustom[linestyle=none,fillstyle=solid,fillcolor=curcolor]
{
\newpath
\moveto(458.60435019,230.31931384)
\lineto(460.12485794,230.31931384)
\lineto(460.12485794,222.82224383)
\lineto(464.18540465,226.82126711)
\lineto(466.20688894,226.82126711)
\lineto(462.60337346,223.29685318)
\lineto(466.40903737,217.40818155)
\lineto(464.38755308,217.40818155)
\lineto(461.45200632,222.15427511)
\lineto(460.12485794,220.94138453)
\lineto(460.12485794,217.40818155)
\lineto(458.60435019,217.40818155)
\closepath
}
}
{
\newrgbcolor{curcolor}{1 0 0}
\pscustom[linestyle=none,fillstyle=solid,fillcolor=curcolor]
{
\newpath
\moveto(418.20882848,49.99231394)
\curveto(418.20882848,27.86986211)(400.68528683,9.93607149)(379.06890274,9.93607149)
\curveto(357.45251866,9.93607149)(339.928977,27.86986211)(339.928977,49.99231394)
\curveto(339.928977,72.11476578)(357.45251866,90.0485564)(379.06890274,90.0485564)
\curveto(400.68528683,90.0485564)(418.20882848,72.11476578)(418.20882848,49.99231394)
\closepath
}
}
{
\newrgbcolor{curcolor}{1 0 0}
\pscustom[linestyle=none,fillstyle=solid,fillcolor=curcolor]
{
\newpath
\moveto(380.00000126,89.99996496)
\lineto(419.99999645,89.99996496)
\lineto(419.99999645,9.99996737)
\lineto(380.00000126,9.99996737)
\closepath
}
}
{
\newrgbcolor{curcolor}{1 0 0}
\pscustom[linestyle=none,fillstyle=solid,fillcolor=curcolor]
{
\newpath
\moveto(460.15270449,49.3443801)
\curveto(460.15270449,27.5797822)(442.04896698,9.93608951)(419.71687503,9.93608951)
\curveto(397.38478308,9.93608951)(379.28104557,27.5797822)(379.28104557,49.3443801)
\curveto(379.28104557,71.10897801)(397.38478308,88.7526707)(419.71687503,88.7526707)
\curveto(442.04896698,88.7526707)(460.15270449,71.10897801)(460.15270449,49.3443801)
\closepath
}
}
{
\newrgbcolor{curcolor}{1 0 0}
\pscustom[linestyle=none,fillstyle=solid,fillcolor=curcolor]
{
\newpath
\moveto(69.86273824,29.25791203)
\curveto(69.86273824,17.87108014)(61.04213611,8.64022904)(50.16137197,8.64022904)
\curveto(39.28060783,8.64022904)(30.46000569,17.87108014)(30.46000569,29.25791203)
\curveto(30.46000569,40.64474392)(39.28060783,49.87559502)(50.16137197,49.87559502)
\curveto(61.04213611,49.87559502)(69.86273824,40.64474392)(69.86273824,29.25791203)
\closepath
}
}
{
\newrgbcolor{curcolor}{1 0 0}
\pscustom[linestyle=none,fillstyle=solid,fillcolor=curcolor]
{
\newpath
\moveto(49.9999508,50.69082526)
\lineto(209.99994599,50.69082526)
\lineto(209.99994599,10.00000341)
\lineto(49.9999508,10.00000341)
\closepath
}
}
{
\newrgbcolor{curcolor}{1 0 0}
\pscustom[linestyle=none,fillstyle=solid,fillcolor=curcolor]
{
\newpath
\moveto(229.86288482,29.25791203)
\curveto(229.86288482,17.87108014)(221.04228269,8.64022904)(210.16151855,8.64022904)
\curveto(199.28075441,8.64022904)(190.46015227,17.87108014)(190.46015227,29.25791203)
\curveto(190.46015227,40.64474392)(199.28075441,49.87559502)(210.16151855,49.87559502)
\curveto(221.04228269,49.87559502)(229.86288482,40.64474392)(229.86288482,29.25791203)
\closepath
}
}
{
\newrgbcolor{curcolor}{0 0 1}
\pscustom[linestyle=none,fillstyle=solid,fillcolor=curcolor]
{
\newpath
\moveto(123.06614439,79.53858724)
\curveto(123.06614439,62.42609045)(109.16882767,48.55367085)(92.02561872,48.55367085)
\curveto(74.88240978,48.55367085)(60.98509306,62.42609045)(60.98509306,79.53858724)
\curveto(60.98509306,96.65108403)(74.88240978,110.52350363)(92.02561872,110.52350363)
\curveto(109.16882767,110.52350363)(123.06614439,96.65108403)(123.06614439,79.53858724)
\closepath
}
}
{
\newrgbcolor{curcolor}{0 0 1}
\pscustom[linestyle=none,fillstyle=solid,fillcolor=curcolor]
{
\newpath
\moveto(90.99996858,110.69082526)
\lineto(170.99996618,110.69082526)
\lineto(170.99996618,50.00000581)
\lineto(90.99996858,50.00000581)
\closepath
}
}
{
\newrgbcolor{curcolor}{0 0 1}
\pscustom[linestyle=none,fillstyle=solid,fillcolor=curcolor]
{
\newpath
\moveto(201.06621936,79.53858724)
\curveto(201.06621936,62.42609045)(187.16890264,48.55367085)(170.0256937,48.55367085)
\curveto(152.88248475,48.55367085)(138.98516803,62.42609045)(138.98516803,79.53858724)
\curveto(138.98516803,96.65108403)(152.88248475,110.52350363)(170.0256937,110.52350363)
\curveto(187.16890264,110.52350363)(201.06621936,96.65108403)(201.06621936,79.53858724)
\closepath
}
}
{
\newrgbcolor{curcolor}{0 0 1}
\pscustom[linestyle=none,fillstyle=solid,fillcolor=curcolor]
{
\newpath
\moveto(123.06614439,139.53860236)
\curveto(123.06614439,122.42610557)(109.16882767,108.55368597)(92.02561872,108.55368597)
\curveto(74.88240978,108.55368597)(60.98509306,122.42610557)(60.98509306,139.53860236)
\curveto(60.98509306,156.65109915)(74.88240978,170.52351875)(92.02561872,170.52351875)
\curveto(109.16882767,170.52351875)(123.06614439,156.65109915)(123.06614439,139.53860236)
\closepath
}
}
{
\newrgbcolor{curcolor}{0 0 1}
\pscustom[linestyle=none,fillstyle=solid,fillcolor=curcolor]
{
\newpath
\moveto(90.99996858,170.69084038)
\lineto(170.99996618,170.69084038)
\lineto(170.99996618,110.00002093)
\lineto(90.99996858,110.00002093)
\closepath
}
}
{
\newrgbcolor{curcolor}{0 0 1}
\pscustom[linestyle=none,fillstyle=solid,fillcolor=curcolor]
{
\newpath
\moveto(201.06621936,139.53860236)
\curveto(201.06621936,122.42610557)(187.16890264,108.55368597)(170.0256937,108.55368597)
\curveto(152.88248475,108.55368597)(138.98516803,122.42610557)(138.98516803,139.53860236)
\curveto(138.98516803,156.65109915)(152.88248475,170.52351875)(170.0256937,170.52351875)
\curveto(187.16890264,170.52351875)(201.06621936,156.65109915)(201.06621936,139.53860236)
\closepath
}
}
{
\newrgbcolor{curcolor}{0 0 1}
\pscustom[linestyle=none,fillstyle=solid,fillcolor=curcolor]
{
\newpath
\moveto(391.06629053,119.53858976)
\curveto(391.06629053,102.42609297)(377.16897381,88.55367337)(360.02576487,88.55367337)
\curveto(342.88255592,88.55367337)(328.9852392,102.42609297)(328.9852392,119.53858976)
\curveto(328.9852392,136.65108655)(342.88255592,150.52350615)(360.02576487,150.52350615)
\curveto(377.16897381,150.52350615)(391.06629053,136.65108655)(391.06629053,119.53858976)
\closepath
}
}
{
\newrgbcolor{curcolor}{0 0 1}
\pscustom[linestyle=none,fillstyle=solid,fillcolor=curcolor]
{
\newpath
\moveto(359.00011472,150.69082778)
\lineto(439.00011232,150.69082778)
\lineto(439.00011232,90.00000833)
\lineto(359.00011472,90.00000833)
\closepath
}
}
{
\newrgbcolor{curcolor}{0 0 1}
\pscustom[linestyle=none,fillstyle=solid,fillcolor=curcolor]
{
\newpath
\moveto(469.0663655,119.53858976)
\curveto(469.0663655,102.42609297)(455.16904878,88.55367337)(438.02583984,88.55367337)
\curveto(420.8826309,88.55367337)(406.98531418,102.42609297)(406.98531418,119.53858976)
\curveto(406.98531418,136.65108655)(420.8826309,150.52350615)(438.02583984,150.52350615)
\curveto(455.16904878,150.52350615)(469.0663655,136.65108655)(469.0663655,119.53858976)
\closepath
}
}
{
\newrgbcolor{curcolor}{0 0 1}
\pscustom[linestyle=none,fillstyle=solid,fillcolor=curcolor]
{
\newpath
\moveto(391.06629053,179.53862756)
\curveto(391.06629053,162.42613077)(377.16897381,148.55371117)(360.02576487,148.55371117)
\curveto(342.88255592,148.55371117)(328.9852392,162.42613077)(328.9852392,179.53862756)
\curveto(328.9852392,196.65112435)(342.88255592,210.52354394)(360.02576487,210.52354394)
\curveto(377.16897381,210.52354394)(391.06629053,196.65112435)(391.06629053,179.53862756)
\closepath
}
}
{
\newrgbcolor{curcolor}{0 0 1}
\pscustom[linestyle=none,fillstyle=solid,fillcolor=curcolor]
{
\newpath
\moveto(359.00011472,210.69086558)
\lineto(439.00011232,210.69086558)
\lineto(439.00011232,150.00004613)
\lineto(359.00011472,150.00004613)
\closepath
}
}
{
\newrgbcolor{curcolor}{0 0 1}
\pscustom[linestyle=none,fillstyle=solid,fillcolor=curcolor]
{
\newpath
\moveto(469.0663655,179.53862756)
\curveto(469.0663655,162.42613077)(455.16904878,148.55371117)(438.02583984,148.55371117)
\curveto(420.8826309,148.55371117)(406.98531418,162.42613077)(406.98531418,179.53862756)
\curveto(406.98531418,196.65112435)(420.8826309,210.52354394)(438.02583984,210.52354394)
\curveto(455.16904878,210.52354394)(469.0663655,196.65112435)(469.0663655,179.53862756)
\closepath
}
}
\end{pspicture}

    \caption{First Layer Problems}
  \end{figure}
\end{frame}

\section{Holes and Voids}
\begin{frame}
  \frametitle{Holes and Voids}
  The size of holes in a 3D printed object are generally less than specified because holes are printed as polygons.  The effect is more pronounced when the polygon has a small number of sides.
\end{frame}

\begin{frame}
  \frametitle{Holes and Voids}
  \begin{figure}
    %LaTeX with PSTricks extensions
%%Creator: inkscape 0.91
%%Please note this file requires PSTricks extensions
\psset{xunit=.5pt,yunit=.5pt,runit=.5pt}
\begin{pspicture}(340.01193309,346.67858514)
{
\newrgbcolor{curcolor}{0.50196081 0.50196081 0.50196081}
\pscustom[linestyle=none,fillstyle=solid,fillcolor=curcolor]
{
\newpath
\moveto(0,290.57926201)
\lineto(340.01193237,290.57926201)
\lineto(340.01193237,0.00000786)
\lineto(0,0.00000786)
\closepath
}
}
{
\newrgbcolor{curcolor}{1 1 1}
\pscustom[linestyle=none,fillstyle=solid,fillcolor=curcolor]
{
\newpath
\moveto(269.72229004,135.8629229)
\curveto(269.72229004,88.91871918)(231.66649376,50.8629229)(184.72229004,50.8629229)
\curveto(137.77808632,50.8629229)(99.72229004,88.91871918)(99.72229004,135.8629229)
\curveto(99.72229004,182.80712661)(137.77808632,220.8629229)(184.72229004,220.8629229)
\curveto(231.66649376,220.8629229)(269.72229004,182.80712661)(269.72229004,135.8629229)
\closepath
}
}
{
\newrgbcolor{curcolor}{0 0 0}
\pscustom[linewidth=2.5,linecolor=curcolor]
{
\newpath
\moveto(269.72229004,135.8629229)
\curveto(269.72229004,88.91871918)(231.66649376,50.8629229)(184.72229004,50.8629229)
\curveto(137.77808632,50.8629229)(99.72229004,88.91871918)(99.72229004,135.8629229)
\curveto(99.72229004,182.80712661)(137.77808632,220.8629229)(184.72229004,220.8629229)
\curveto(231.66649376,220.8629229)(269.72229004,182.80712661)(269.72229004,135.8629229)
\closepath
}
}
{
\newrgbcolor{curcolor}{1 1 1}
\pscustom[linestyle=none,fillstyle=solid,fillcolor=curcolor]
{
\newpath
\moveto(257.34315061,93.74599668)
\lineto(184.33911529,50.27747053)
\lineto(111.95017831,91.88561141)
\lineto(112.56527666,176.96227845)
\lineto(185.56931198,220.43080461)
\lineto(257.95824896,178.82266373)
\closepath
}
}
{
\newrgbcolor{curcolor}{0 0 0}
\pscustom[linewidth=2.74393634,linecolor=curcolor,linestyle=dashed,dash=19.33600873 6.44533624]
{
\newpath
\moveto(257.34315061,93.74599668)
\lineto(184.33911529,50.27747053)
\lineto(111.95017831,91.88561141)
\lineto(112.56527666,176.96227845)
\lineto(185.56931198,220.43080461)
\lineto(257.95824896,178.82266373)
\closepath
}
}
{
\newrgbcolor{curcolor}{0 0 0}
\pscustom[linewidth=2.5,linecolor=curcolor]
{
\newpath
\moveto(0,340.57927514)
\lineto(30,340.57927514)
}
}
{
\newrgbcolor{curcolor}{0 0 0}
\pscustom[linestyle=none,fillstyle=solid,fillcolor=curcolor]
{
\newpath
\moveto(46.328125,335.21764397)
\curveto(46.91992188,335.21764397)(47.40625,335.2791674)(47.78710938,335.40221428)
\curveto(48.46679688,335.6307299)(49.0234375,336.07018303)(49.45703125,336.72057365)
\curveto(49.80273438,337.24205803)(50.05175781,337.91002678)(50.20410156,338.7244799)
\curveto(50.29199219,339.21080803)(50.3359375,339.6619799)(50.3359375,340.07799553)
\curveto(50.3359375,341.6776049)(50.01660156,342.9197924)(49.37792969,343.80455803)
\curveto(48.74511719,344.68932365)(47.72265625,345.13170647)(46.31054688,345.13170647)
\lineto(43.20800781,345.13170647)
\lineto(43.20800781,335.21764397)
\lineto(46.328125,335.21764397)
\closepath
\moveto(41.45019531,346.63463615)
\lineto(46.6796875,346.63463615)
\curveto(48.45507812,346.63463615)(49.83203125,346.00475334)(50.81054688,344.74498772)
\curveto(51.68359375,343.60826897)(52.12011719,342.15221428)(52.12011719,340.37682365)
\curveto(52.12011719,339.0057299)(51.86230469,337.76647209)(51.34667969,336.65905022)
\curveto(50.43847656,334.70201897)(48.87695312,333.72350334)(46.66210938,333.72350334)
\lineto(41.45019531,333.72350334)
\lineto(41.45019531,346.63463615)
\closepath
}
}
{
\newrgbcolor{curcolor}{0 0 0}
\pscustom[linestyle=none,fillstyle=solid,fillcolor=curcolor]
{
\newpath
\moveto(58.08789062,343.34752678)
\curveto(58.75585938,343.34752678)(59.40332031,343.18932365)(60.03027344,342.8729174)
\curveto(60.65722656,342.56237053)(61.13476562,342.15807365)(61.46289062,341.66002678)
\curveto(61.77929688,341.1854174)(61.99023438,340.63170647)(62.09570312,339.99889397)
\curveto(62.18945312,339.56530022)(62.23632812,338.87389397)(62.23632812,337.92467522)
\lineto(55.33691406,337.92467522)
\curveto(55.36621094,336.96959709)(55.59179688,336.20201897)(56.01367188,335.62194084)
\curveto(56.43554688,335.04772209)(57.08886719,334.76061272)(57.97363281,334.76061272)
\curveto(58.79980469,334.76061272)(59.45898438,335.03307365)(59.95117188,335.57799553)
\curveto(60.23242188,335.89440178)(60.43164062,336.26061272)(60.54882812,336.67662834)
\lineto(62.10449219,336.67662834)
\curveto(62.06347656,336.33092522)(61.92578125,335.94420647)(61.69140625,335.51647209)
\curveto(61.46289062,335.09459709)(61.20507812,334.74889397)(60.91796875,334.47936272)
\curveto(60.4375,334.01061272)(59.84277344,333.69420647)(59.13378906,333.53014397)
\curveto(58.75292969,333.43639397)(58.32226562,333.38951897)(57.84179688,333.38951897)
\curveto(56.66992188,333.38951897)(55.67675781,333.81432365)(54.86230469,334.66393303)
\curveto(54.04785156,335.51940178)(53.640625,336.71471428)(53.640625,338.24987053)
\curveto(53.640625,339.76158928)(54.05078125,340.98912834)(54.87109375,341.93248772)
\curveto(55.69140625,342.87584709)(56.76367188,343.34752678)(58.08789062,343.34752678)
\closepath
\moveto(60.61035156,339.18151115)
\curveto(60.54589844,339.86705803)(60.39648438,340.41490959)(60.16210938,340.82506584)
\curveto(59.72851562,341.58678459)(59.00488281,341.96764397)(57.99121094,341.96764397)
\curveto(57.26464844,341.96764397)(56.65527344,341.70397209)(56.16308594,341.17662834)
\curveto(55.67089844,340.65514397)(55.41015625,339.9901049)(55.38085938,339.18151115)
\lineto(60.61035156,339.18151115)
\closepath
\moveto(57.93847656,343.3651049)
\lineto(57.93847656,343.3651049)
\closepath
}
}
{
\newrgbcolor{curcolor}{0 0 0}
\pscustom[linestyle=none,fillstyle=solid,fillcolor=curcolor]
{
\newpath
\moveto(65.12792969,336.67662834)
\curveto(65.17480469,336.14928459)(65.30664062,335.74498772)(65.5234375,335.46373772)
\curveto(65.921875,334.95397209)(66.61328125,334.69908928)(67.59765625,334.69908928)
\curveto(68.18359375,334.69908928)(68.69921875,334.82506584)(69.14453125,335.07701897)
\curveto(69.58984375,335.33483147)(69.8125,335.73033928)(69.8125,336.2635424)
\curveto(69.8125,336.66783928)(69.63378906,336.97545647)(69.27636719,337.18639397)
\curveto(69.04785156,337.31530022)(68.59667969,337.46471428)(67.92285156,337.63463615)
\lineto(66.66601562,337.9510424)
\curveto(65.86328125,338.15026115)(65.27148438,338.3729174)(64.890625,338.61901115)
\curveto(64.2109375,339.04674553)(63.87109375,339.6385424)(63.87109375,340.39440178)
\curveto(63.87109375,341.28502678)(64.19042969,342.0057299)(64.82910156,342.55651115)
\curveto(65.47363281,343.1072924)(66.33789062,343.38268303)(67.421875,343.38268303)
\curveto(68.83984375,343.38268303)(69.86230469,342.9666674)(70.48925781,342.13463615)
\curveto(70.88183594,341.6072924)(71.07226562,341.03893303)(71.06054688,340.42955803)
\lineto(69.56640625,340.42955803)
\curveto(69.53710938,340.7869799)(69.41113281,341.11217522)(69.18847656,341.40514397)
\curveto(68.82519531,341.82115959)(68.1953125,342.0291674)(67.29882812,342.0291674)
\curveto(66.70117188,342.0291674)(66.24707031,341.91490959)(65.93652344,341.68639397)
\curveto(65.63183594,341.45787834)(65.47949219,341.15612053)(65.47949219,340.78112053)
\curveto(65.47949219,340.37096428)(65.68164062,340.04283928)(66.0859375,339.79674553)
\curveto(66.3203125,339.65026115)(66.66601562,339.5213549)(67.12304688,339.41002678)
\lineto(68.16894531,339.15514397)
\curveto(69.30566406,338.87975334)(70.06738281,338.61315178)(70.45410156,338.35533928)
\curveto(71.06933594,337.9510424)(71.37695312,337.31530022)(71.37695312,336.44811272)
\curveto(71.37695312,335.61022209)(71.05761719,334.88658928)(70.41894531,334.27721428)
\curveto(69.78613281,333.66783928)(68.81933594,333.36315178)(67.51855469,333.36315178)
\curveto(66.11816406,333.36315178)(65.125,333.67955803)(64.5390625,334.31237053)
\curveto(63.95898438,334.9510424)(63.6484375,335.73912834)(63.60742188,336.67662834)
\lineto(65.12792969,336.67662834)
\closepath
\moveto(67.46582031,343.3651049)
\lineto(67.46582031,343.3651049)
\closepath
}
}
{
\newrgbcolor{curcolor}{0 0 0}
\pscustom[linestyle=none,fillstyle=solid,fillcolor=curcolor]
{
\newpath
\moveto(73.22265625,343.09264397)
\lineto(74.83105469,343.09264397)
\lineto(74.83105469,333.72350334)
\lineto(73.22265625,333.72350334)
\lineto(73.22265625,343.09264397)
\closepath
\moveto(73.22265625,346.63463615)
\lineto(74.83105469,346.63463615)
\lineto(74.83105469,344.8416674)
\lineto(73.22265625,344.8416674)
\lineto(73.22265625,346.63463615)
\closepath
}
}
{
\newrgbcolor{curcolor}{0 0 0}
\pscustom[linestyle=none,fillstyle=solid,fillcolor=curcolor]
{
\newpath
\moveto(77.27441406,343.13658928)
\lineto(78.77734375,343.13658928)
\lineto(78.77734375,341.51061272)
\curveto(78.90039062,341.82701897)(79.20214844,342.21080803)(79.68261719,342.6619799)
\curveto(80.16308594,343.11901115)(80.71679688,343.34752678)(81.34375,343.34752678)
\curveto(81.37304688,343.34752678)(81.42285156,343.34459709)(81.49316406,343.33873772)
\curveto(81.56347656,343.33287834)(81.68359375,343.32115959)(81.85351562,343.30358147)
\lineto(81.85351562,341.63365959)
\curveto(81.75976562,341.65123772)(81.671875,341.66295647)(81.58984375,341.66881584)
\curveto(81.51367188,341.67467522)(81.42871094,341.6776049)(81.33496094,341.6776049)
\curveto(80.53808594,341.6776049)(79.92578125,341.4197924)(79.49804688,340.9041674)
\curveto(79.0703125,340.39440178)(78.85644531,339.80553459)(78.85644531,339.13756584)
\lineto(78.85644531,333.72350334)
\lineto(77.27441406,333.72350334)
\lineto(77.27441406,343.13658928)
\closepath
}
}
{
\newrgbcolor{curcolor}{0 0 0}
\pscustom[linestyle=none,fillstyle=solid,fillcolor=curcolor]
{
\newpath
\moveto(87.16210938,343.34752678)
\curveto(87.83007812,343.34752678)(88.47753906,343.18932365)(89.10449219,342.8729174)
\curveto(89.73144531,342.56237053)(90.20898438,342.15807365)(90.53710938,341.66002678)
\curveto(90.85351562,341.1854174)(91.06445312,340.63170647)(91.16992188,339.99889397)
\curveto(91.26367188,339.56530022)(91.31054688,338.87389397)(91.31054688,337.92467522)
\lineto(84.41113281,337.92467522)
\curveto(84.44042969,336.96959709)(84.66601562,336.20201897)(85.08789062,335.62194084)
\curveto(85.50976562,335.04772209)(86.16308594,334.76061272)(87.04785156,334.76061272)
\curveto(87.87402344,334.76061272)(88.53320312,335.03307365)(89.02539062,335.57799553)
\curveto(89.30664062,335.89440178)(89.50585938,336.26061272)(89.62304688,336.67662834)
\lineto(91.17871094,336.67662834)
\curveto(91.13769531,336.33092522)(91,335.94420647)(90.765625,335.51647209)
\curveto(90.53710938,335.09459709)(90.27929688,334.74889397)(89.9921875,334.47936272)
\curveto(89.51171875,334.01061272)(88.91699219,333.69420647)(88.20800781,333.53014397)
\curveto(87.82714844,333.43639397)(87.39648438,333.38951897)(86.91601562,333.38951897)
\curveto(85.74414062,333.38951897)(84.75097656,333.81432365)(83.93652344,334.66393303)
\curveto(83.12207031,335.51940178)(82.71484375,336.71471428)(82.71484375,338.24987053)
\curveto(82.71484375,339.76158928)(83.125,340.98912834)(83.9453125,341.93248772)
\curveto(84.765625,342.87584709)(85.83789062,343.34752678)(87.16210938,343.34752678)
\closepath
\moveto(89.68457031,339.18151115)
\curveto(89.62011719,339.86705803)(89.47070312,340.41490959)(89.23632812,340.82506584)
\curveto(88.80273438,341.58678459)(88.07910156,341.96764397)(87.06542969,341.96764397)
\curveto(86.33886719,341.96764397)(85.72949219,341.70397209)(85.23730469,341.17662834)
\curveto(84.74511719,340.65514397)(84.484375,339.9901049)(84.45507812,339.18151115)
\lineto(89.68457031,339.18151115)
\closepath
\moveto(87.01269531,343.3651049)
\lineto(87.01269531,343.3651049)
\closepath
}
}
{
\newrgbcolor{curcolor}{0 0 0}
\pscustom[linestyle=none,fillstyle=solid,fillcolor=curcolor]
{
\newpath
\moveto(94.26367188,338.32018303)
\curveto(94.26367188,337.31237053)(94.47753906,336.46862053)(94.90527344,335.78893303)
\curveto(95.33300781,335.10924553)(96.01855469,334.76940178)(96.96191406,334.76940178)
\curveto(97.69433594,334.76940178)(98.29492188,335.08287834)(98.76367188,335.70983147)
\curveto(99.23828125,336.34264397)(99.47558594,337.2479174)(99.47558594,338.42565178)
\curveto(99.47558594,339.6151049)(99.23242188,340.49401115)(98.74609375,341.06237053)
\curveto(98.25976562,341.63658928)(97.65917969,341.92369865)(96.94433594,341.92369865)
\curveto(96.14746094,341.92369865)(95.5,341.61901115)(95.00195312,341.00963615)
\curveto(94.50976562,340.40026115)(94.26367188,339.50377678)(94.26367188,338.32018303)
\closepath
\moveto(96.64550781,343.30358147)
\curveto(97.36621094,343.30358147)(97.96972656,343.15123772)(98.45605469,342.84655022)
\curveto(98.73730469,342.67076897)(99.05664062,342.36315178)(99.4140625,341.92369865)
\lineto(99.4140625,346.67858147)
\lineto(100.93457031,346.67858147)
\lineto(100.93457031,333.72350334)
\lineto(99.51074219,333.72350334)
\lineto(99.51074219,335.03307365)
\curveto(99.14160156,334.45299553)(98.70507812,334.03405022)(98.20117188,333.77623772)
\curveto(97.69726562,333.51842522)(97.12011719,333.38951897)(96.46972656,333.38951897)
\curveto(95.42089844,333.38951897)(94.51269531,333.82897209)(93.74511719,334.70787834)
\curveto(92.97753906,335.59264397)(92.59375,336.76744865)(92.59375,338.2322924)
\curveto(92.59375,339.60338615)(92.94238281,340.78990959)(93.63964844,341.79186272)
\curveto(94.34277344,342.79967522)(95.34472656,343.30358147)(96.64550781,343.30358147)
\closepath
}
}
{
\newrgbcolor{curcolor}{0 0 0}
\pscustom[linestyle=none,fillstyle=solid,fillcolor=curcolor]
{
\newpath
\moveto(108.2734375,346.67858147)
\lineto(109.85546875,346.67858147)
\lineto(109.85546875,341.86217522)
\curveto(110.23046875,342.33678459)(110.56738281,342.67076897)(110.86621094,342.86412834)
\curveto(111.37597656,343.19811272)(112.01171875,343.3651049)(112.7734375,343.3651049)
\curveto(114.13867188,343.3651049)(115.06445312,342.88756584)(115.55078125,341.93248772)
\curveto(115.81445312,341.41100334)(115.94628906,340.68737053)(115.94628906,339.76158928)
\lineto(115.94628906,333.72350334)
\lineto(114.3203125,333.72350334)
\lineto(114.3203125,339.65612053)
\curveto(114.3203125,340.34752678)(114.23242188,340.85436272)(114.05664062,341.17662834)
\curveto(113.76953125,341.69225334)(113.23046875,341.95006584)(112.43945312,341.95006584)
\curveto(111.78320312,341.95006584)(111.18847656,341.7244799)(110.65527344,341.27330803)
\curveto(110.12207031,340.82213615)(109.85546875,339.96959709)(109.85546875,338.71569084)
\lineto(109.85546875,333.72350334)
\lineto(108.2734375,333.72350334)
\lineto(108.2734375,346.67858147)
\closepath
}
}
{
\newrgbcolor{curcolor}{0 0 0}
\pscustom[linestyle=none,fillstyle=solid,fillcolor=curcolor]
{
\newpath
\moveto(122.02832031,334.74303459)
\curveto(123.07714844,334.74303459)(123.79492188,335.1385424)(124.18164062,335.92955803)
\curveto(124.57421875,336.72643303)(124.77050781,337.61119865)(124.77050781,338.5838549)
\curveto(124.77050781,339.46276115)(124.62988281,340.1776049)(124.34863281,340.72838615)
\curveto(123.90332031,341.59557365)(123.13574219,342.0291674)(122.04589844,342.0291674)
\curveto(121.07910156,342.0291674)(120.37597656,341.66002678)(119.93652344,340.92174553)
\curveto(119.49707031,340.18346428)(119.27734375,339.29283928)(119.27734375,338.24987053)
\curveto(119.27734375,337.2479174)(119.49707031,336.41295647)(119.93652344,335.74498772)
\curveto(120.37597656,335.07701897)(121.07324219,334.74303459)(122.02832031,334.74303459)
\closepath
\moveto(122.08984375,343.40905022)
\curveto(123.30273438,343.40905022)(124.328125,343.00475334)(125.16601562,342.19615959)
\curveto(126.00390625,341.38756584)(126.42285156,340.19811272)(126.42285156,338.62780022)
\curveto(126.42285156,337.11022209)(126.05371094,335.85631584)(125.31542969,334.86608147)
\curveto(124.57714844,333.87584709)(123.43164062,333.3807299)(121.87890625,333.3807299)
\curveto(120.58398438,333.3807299)(119.55566406,333.81725334)(118.79394531,334.69030022)
\curveto(118.03222656,335.56920647)(117.65136719,336.74694084)(117.65136719,338.22350334)
\curveto(117.65136719,339.80553459)(118.05273438,341.06530022)(118.85546875,342.00280022)
\curveto(119.65820312,342.94030022)(120.73632812,343.40905022)(122.08984375,343.40905022)
\closepath
\moveto(122.03710938,343.3651049)
\lineto(122.03710938,343.3651049)
\closepath
}
}
{
\newrgbcolor{curcolor}{0 0 0}
\pscustom[linestyle=none,fillstyle=solid,fillcolor=curcolor]
{
\newpath
\moveto(128.35644531,346.63463615)
\lineto(129.93847656,346.63463615)
\lineto(129.93847656,333.72350334)
\lineto(128.35644531,333.72350334)
\lineto(128.35644531,346.63463615)
\closepath
}
}
{
\newrgbcolor{curcolor}{0 0 0}
\pscustom[linestyle=none,fillstyle=solid,fillcolor=curcolor]
{
\newpath
\moveto(136.24023438,343.34752678)
\curveto(136.90820312,343.34752678)(137.55566406,343.18932365)(138.18261719,342.8729174)
\curveto(138.80957031,342.56237053)(139.28710938,342.15807365)(139.61523438,341.66002678)
\curveto(139.93164062,341.1854174)(140.14257812,340.63170647)(140.24804688,339.99889397)
\curveto(140.34179688,339.56530022)(140.38867188,338.87389397)(140.38867188,337.92467522)
\lineto(133.48925781,337.92467522)
\curveto(133.51855469,336.96959709)(133.74414062,336.20201897)(134.16601562,335.62194084)
\curveto(134.58789062,335.04772209)(135.24121094,334.76061272)(136.12597656,334.76061272)
\curveto(136.95214844,334.76061272)(137.61132812,335.03307365)(138.10351562,335.57799553)
\curveto(138.38476562,335.89440178)(138.58398438,336.26061272)(138.70117188,336.67662834)
\lineto(140.25683594,336.67662834)
\curveto(140.21582031,336.33092522)(140.078125,335.94420647)(139.84375,335.51647209)
\curveto(139.61523438,335.09459709)(139.35742188,334.74889397)(139.0703125,334.47936272)
\curveto(138.58984375,334.01061272)(137.99511719,333.69420647)(137.28613281,333.53014397)
\curveto(136.90527344,333.43639397)(136.47460938,333.38951897)(135.99414062,333.38951897)
\curveto(134.82226562,333.38951897)(133.82910156,333.81432365)(133.01464844,334.66393303)
\curveto(132.20019531,335.51940178)(131.79296875,336.71471428)(131.79296875,338.24987053)
\curveto(131.79296875,339.76158928)(132.203125,340.98912834)(133.0234375,341.93248772)
\curveto(133.84375,342.87584709)(134.91601562,343.34752678)(136.24023438,343.34752678)
\closepath
\moveto(138.76269531,339.18151115)
\curveto(138.69824219,339.86705803)(138.54882812,340.41490959)(138.31445312,340.82506584)
\curveto(137.88085938,341.58678459)(137.15722656,341.96764397)(136.14355469,341.96764397)
\curveto(135.41699219,341.96764397)(134.80761719,341.70397209)(134.31542969,341.17662834)
\curveto(133.82324219,340.65514397)(133.5625,339.9901049)(133.53320312,339.18151115)
\lineto(138.76269531,339.18151115)
\closepath
\moveto(136.09082031,343.3651049)
\lineto(136.09082031,343.3651049)
\closepath
}
}
{
\newrgbcolor{curcolor}{0 0 0}
\pscustom[linewidth=2.5,linecolor=curcolor,linestyle=dashed,dash=7.5 2.5]
{
\newpath
\moveto(0,310.57927514)
\lineto(30,310.57927514)
}
}
{
\newrgbcolor{curcolor}{0 0 0}
\pscustom[linestyle=none,fillstyle=solid,fillcolor=curcolor]
{
\newpath
\moveto(47.99804688,310.72846245)
\lineto(46.03808594,316.43256401)
\lineto(43.95507812,310.72846245)
\lineto(47.99804688,310.72846245)
\closepath
\moveto(45.12402344,318.34857963)
\lineto(47.1015625,318.34857963)
\lineto(51.78613281,305.43744682)
\lineto(49.87011719,305.43744682)
\lineto(48.56054688,309.30463432)
\lineto(43.45410156,309.30463432)
\lineto(42.05664062,305.43744682)
\lineto(40.26367188,305.43744682)
\lineto(45.12402344,318.34857963)
\closepath
\moveto(46.02929688,318.34857963)
\lineto(46.02929688,318.34857963)
\closepath
}
}
{
\newrgbcolor{curcolor}{0 0 0}
\pscustom[linestyle=none,fillstyle=solid,fillcolor=curcolor]
{
\newpath
\moveto(56.81347656,315.1229937)
\curveto(57.87402344,315.1229937)(58.73535156,314.8651812)(59.39746094,314.3495562)
\curveto(60.06542969,313.8339312)(60.46679688,312.94623588)(60.6015625,311.68647026)
\lineto(59.06347656,311.68647026)
\curveto(58.96972656,312.26654838)(58.75585938,312.74701713)(58.421875,313.12787651)
\curveto(58.08789062,313.51459526)(57.55175781,313.70795463)(56.81347656,313.70795463)
\curveto(55.80566406,313.70795463)(55.08496094,313.21576713)(54.65136719,312.23139213)
\curveto(54.37011719,311.59272026)(54.22949219,310.80463432)(54.22949219,309.86713432)
\curveto(54.22949219,308.92377495)(54.42871094,308.12982963)(54.82714844,307.48529838)
\curveto(55.22558594,306.84076713)(55.85253906,306.51850151)(56.70800781,306.51850151)
\curveto(57.36425781,306.51850151)(57.8828125,306.71772026)(58.26367188,307.11615776)
\curveto(58.65039062,307.52045463)(58.91699219,308.07123588)(59.06347656,308.76850151)
\lineto(60.6015625,308.76850151)
\curveto(60.42578125,307.52045463)(59.98632812,306.60639213)(59.28320312,306.02631401)
\curveto(58.58007812,305.45209526)(57.68066406,305.16498588)(56.58496094,305.16498588)
\curveto(55.35449219,305.16498588)(54.37304688,305.61322807)(53.640625,306.50971245)
\curveto(52.90820312,307.4120562)(52.54199219,308.5370562)(52.54199219,309.88471245)
\curveto(52.54199219,311.5370562)(52.94335938,312.82318901)(53.74609375,313.74311088)
\curveto(54.54882812,314.66303276)(55.57128906,315.1229937)(56.81347656,315.1229937)
\closepath
\moveto(56.56738281,315.07904838)
\lineto(56.56738281,315.07904838)
\closepath
}
}
{
\newrgbcolor{curcolor}{0 0 0}
\pscustom[linestyle=none,fillstyle=solid,fillcolor=curcolor]
{
\newpath
\moveto(62.5,317.47846245)
\lineto(64.09960938,317.47846245)
\lineto(64.09960938,314.85053276)
\lineto(65.60253906,314.85053276)
\lineto(65.60253906,313.55854057)
\lineto(64.09960938,313.55854057)
\lineto(64.09960938,307.41498588)
\curveto(64.09960938,307.08686088)(64.2109375,306.86713432)(64.43359375,306.7558062)
\curveto(64.55664062,306.69135307)(64.76171875,306.65912651)(65.04882812,306.65912651)
\lineto(65.29492188,306.65912651)
\curveto(65.3828125,306.66498588)(65.48535156,306.67377495)(65.60253906,306.6854937)
\lineto(65.60253906,305.43744682)
\curveto(65.42089844,305.38471245)(65.23046875,305.34662651)(65.03125,305.32318901)
\curveto(64.83789062,305.29975151)(64.62695312,305.28803276)(64.3984375,305.28803276)
\curveto(63.66015625,305.28803276)(63.15917969,305.47553276)(62.89550781,305.85053276)
\curveto(62.63183594,306.23139213)(62.5,306.72357963)(62.5,307.32709526)
\lineto(62.5,313.55854057)
\lineto(61.22558594,313.55854057)
\lineto(61.22558594,314.85053276)
\lineto(62.5,314.85053276)
\lineto(62.5,317.47846245)
\closepath
}
}
{
\newrgbcolor{curcolor}{0 0 0}
\pscustom[linestyle=none,fillstyle=solid,fillcolor=curcolor]
{
\newpath
\moveto(68.79296875,314.85053276)
\lineto(68.79296875,308.60150932)
\curveto(68.79296875,308.12104057)(68.86914062,307.72846245)(69.02148438,307.42377495)
\curveto(69.30273438,306.86127495)(69.82714844,306.58002495)(70.59472656,306.58002495)
\curveto(71.69628906,306.58002495)(72.44628906,307.07221245)(72.84472656,308.05658745)
\curveto(73.06152344,308.5839312)(73.16992188,309.30756401)(73.16992188,310.22748588)
\lineto(73.16992188,314.85053276)
\lineto(74.75195312,314.85053276)
\lineto(74.75195312,305.43744682)
\lineto(73.2578125,305.43744682)
\lineto(73.27539062,306.8261187)
\curveto(73.0703125,306.46869682)(72.81542969,306.16693901)(72.51074219,305.92084526)
\curveto(71.90722656,305.42865776)(71.17480469,305.18256401)(70.31347656,305.18256401)
\curveto(68.97167969,305.18256401)(68.05761719,305.6308062)(67.57128906,306.52729057)
\curveto(67.30761719,307.00775932)(67.17578125,307.64936088)(67.17578125,308.45209526)
\lineto(67.17578125,314.85053276)
\lineto(68.79296875,314.85053276)
\closepath
\moveto(70.96386719,315.07904838)
\lineto(70.96386719,315.07904838)
\closepath
}
}
{
\newrgbcolor{curcolor}{0 0 0}
\pscustom[linestyle=none,fillstyle=solid,fillcolor=curcolor]
{
\newpath
\moveto(78.44335938,307.94232963)
\curveto(78.44335938,307.48529838)(78.61035156,307.12494682)(78.94433594,306.86127495)
\curveto(79.27832031,306.59760307)(79.67382812,306.46576713)(80.13085938,306.46576713)
\curveto(80.6875,306.46576713)(81.2265625,306.59467338)(81.74804688,306.85248588)
\curveto(82.62695312,307.28022026)(83.06640625,307.98041557)(83.06640625,308.95307182)
\lineto(83.06640625,310.22748588)
\curveto(82.87304688,310.10443901)(82.62402344,310.00189995)(82.31933594,309.9198687)
\curveto(82.01464844,309.83783745)(81.71582031,309.7792437)(81.42285156,309.74408745)
\lineto(80.46484375,309.62104057)
\curveto(79.890625,309.5448687)(79.45996094,309.42475151)(79.17285156,309.26068901)
\curveto(78.68652344,308.98529838)(78.44335938,308.54584526)(78.44335938,307.94232963)
\closepath
\moveto(82.27539062,311.14154838)
\curveto(82.63867188,311.18842338)(82.88183594,311.34076713)(83.00488281,311.59857963)
\curveto(83.07519531,311.73920463)(83.11035156,311.94135307)(83.11035156,312.20502495)
\curveto(83.11035156,312.74408745)(82.91699219,313.13373588)(82.53027344,313.37397026)
\curveto(82.14941406,313.62006401)(81.6015625,313.74311088)(80.88671875,313.74311088)
\curveto(80.06054688,313.74311088)(79.47460938,313.52045463)(79.12890625,313.07514213)
\curveto(78.93554688,312.82904838)(78.80957031,312.46283745)(78.75097656,311.97650932)
\lineto(77.27441406,311.97650932)
\curveto(77.30371094,313.13666557)(77.67871094,313.94232963)(78.39941406,314.39350151)
\curveto(79.12597656,314.85053276)(79.96679688,315.07904838)(80.921875,315.07904838)
\curveto(82.02929688,315.07904838)(82.92871094,314.86811088)(83.62011719,314.44623588)
\curveto(84.30566406,314.02436088)(84.6484375,313.36811088)(84.6484375,312.47748588)
\lineto(84.6484375,307.05463432)
\curveto(84.6484375,306.89057182)(84.68066406,306.75873588)(84.74511719,306.65912651)
\curveto(84.81542969,306.55951713)(84.95898438,306.50971245)(85.17578125,306.50971245)
\curveto(85.24609375,306.50971245)(85.32519531,306.51264213)(85.41308594,306.51850151)
\curveto(85.50097656,306.53022026)(85.59472656,306.5448687)(85.69433594,306.56244682)
\lineto(85.69433594,305.39350151)
\curveto(85.44824219,305.32318901)(85.26074219,305.2792437)(85.13183594,305.26166557)
\curveto(85.00292969,305.24408745)(84.82714844,305.23529838)(84.60449219,305.23529838)
\curveto(84.05957031,305.23529838)(83.6640625,305.42865776)(83.41796875,305.81537651)
\curveto(83.2890625,306.02045463)(83.19824219,306.3104937)(83.14550781,306.6854937)
\curveto(82.82324219,306.2636187)(82.36035156,305.89740776)(81.75683594,305.58686088)
\curveto(81.15332031,305.27631401)(80.48828125,305.12104057)(79.76171875,305.12104057)
\curveto(78.88867188,305.12104057)(78.17382812,305.38471245)(77.6171875,305.9120562)
\curveto(77.06640625,306.44525932)(76.79101562,307.11029838)(76.79101562,307.90717338)
\curveto(76.79101562,308.78022026)(77.06347656,309.45697807)(77.60839844,309.93744682)
\curveto(78.15332031,310.41791557)(78.86816406,310.71381401)(79.75292969,310.82514213)
\lineto(82.27539062,311.14154838)
\closepath
\moveto(80.96582031,315.07904838)
\lineto(80.96582031,315.07904838)
\closepath
}
}
{
\newrgbcolor{curcolor}{0 0 0}
\pscustom[linestyle=none,fillstyle=solid,fillcolor=curcolor]
{
\newpath
\moveto(87.29394531,318.34857963)
\lineto(88.87597656,318.34857963)
\lineto(88.87597656,305.43744682)
\lineto(87.29394531,305.43744682)
\lineto(87.29394531,318.34857963)
\closepath
}
}
{
\newrgbcolor{curcolor}{0 0 0}
\pscustom[linestyle=none,fillstyle=solid,fillcolor=curcolor]
{
\newpath
\moveto(96.25,318.39252495)
\lineto(97.83203125,318.39252495)
\lineto(97.83203125,313.5761187)
\curveto(98.20703125,314.05072807)(98.54394531,314.38471245)(98.84277344,314.57807182)
\curveto(99.35253906,314.9120562)(99.98828125,315.07904838)(100.75,315.07904838)
\curveto(102.11523438,315.07904838)(103.04101562,314.60150932)(103.52734375,313.6464312)
\curveto(103.79101562,313.12494682)(103.92285156,312.40131401)(103.92285156,311.47553276)
\lineto(103.92285156,305.43744682)
\lineto(102.296875,305.43744682)
\lineto(102.296875,311.37006401)
\curveto(102.296875,312.06147026)(102.20898438,312.5683062)(102.03320312,312.89057182)
\curveto(101.74609375,313.40619682)(101.20703125,313.66400932)(100.41601562,313.66400932)
\curveto(99.75976562,313.66400932)(99.16503906,313.43842338)(98.63183594,312.98725151)
\curveto(98.09863281,312.53607963)(97.83203125,311.68354057)(97.83203125,310.42963432)
\lineto(97.83203125,305.43744682)
\lineto(96.25,305.43744682)
\lineto(96.25,318.39252495)
\closepath
}
}
{
\newrgbcolor{curcolor}{0 0 0}
\pscustom[linestyle=none,fillstyle=solid,fillcolor=curcolor]
{
\newpath
\moveto(110.00488281,306.45697807)
\curveto(111.05371094,306.45697807)(111.77148438,306.85248588)(112.15820312,307.64350151)
\curveto(112.55078125,308.44037651)(112.74707031,309.32514213)(112.74707031,310.29779838)
\curveto(112.74707031,311.17670463)(112.60644531,311.89154838)(112.32519531,312.44232963)
\curveto(111.87988281,313.30951713)(111.11230469,313.74311088)(110.02246094,313.74311088)
\curveto(109.05566406,313.74311088)(108.35253906,313.37397026)(107.91308594,312.63568901)
\curveto(107.47363281,311.89740776)(107.25390625,311.00678276)(107.25390625,309.96381401)
\curveto(107.25390625,308.96186088)(107.47363281,308.12689995)(107.91308594,307.4589312)
\curveto(108.35253906,306.79096245)(109.04980469,306.45697807)(110.00488281,306.45697807)
\closepath
\moveto(110.06640625,315.1229937)
\curveto(111.27929688,315.1229937)(112.3046875,314.71869682)(113.14257812,313.91010307)
\curveto(113.98046875,313.10150932)(114.39941406,311.9120562)(114.39941406,310.3417437)
\curveto(114.39941406,308.82416557)(114.03027344,307.57025932)(113.29199219,306.58002495)
\curveto(112.55371094,305.58979057)(111.40820312,305.09467338)(109.85546875,305.09467338)
\curveto(108.56054688,305.09467338)(107.53222656,305.53119682)(106.77050781,306.4042437)
\curveto(106.00878906,307.28314995)(105.62792969,308.46088432)(105.62792969,309.93744682)
\curveto(105.62792969,311.51947807)(106.02929688,312.7792437)(106.83203125,313.7167437)
\curveto(107.63476562,314.6542437)(108.71289062,315.1229937)(110.06640625,315.1229937)
\closepath
\moveto(110.01367188,315.07904838)
\lineto(110.01367188,315.07904838)
\closepath
}
}
{
\newrgbcolor{curcolor}{0 0 0}
\pscustom[linestyle=none,fillstyle=solid,fillcolor=curcolor]
{
\newpath
\moveto(116.33300781,318.34857963)
\lineto(117.91503906,318.34857963)
\lineto(117.91503906,305.43744682)
\lineto(116.33300781,305.43744682)
\lineto(116.33300781,318.34857963)
\closepath
}
}
{
\newrgbcolor{curcolor}{0 0 0}
\pscustom[linestyle=none,fillstyle=solid,fillcolor=curcolor]
{
\newpath
\moveto(124.21679688,315.06147026)
\curveto(124.88476562,315.06147026)(125.53222656,314.90326713)(126.15917969,314.58686088)
\curveto(126.78613281,314.27631401)(127.26367188,313.87201713)(127.59179688,313.37397026)
\curveto(127.90820312,312.89936088)(128.11914062,312.34564995)(128.22460938,311.71283745)
\curveto(128.31835938,311.2792437)(128.36523438,310.58783745)(128.36523438,309.6386187)
\lineto(121.46582031,309.6386187)
\curveto(121.49511719,308.68354057)(121.72070312,307.91596245)(122.14257812,307.33588432)
\curveto(122.56445312,306.76166557)(123.21777344,306.4745562)(124.10253906,306.4745562)
\curveto(124.92871094,306.4745562)(125.58789062,306.74701713)(126.08007812,307.29193901)
\curveto(126.36132812,307.60834526)(126.56054688,307.9745562)(126.67773438,308.39057182)
\lineto(128.23339844,308.39057182)
\curveto(128.19238281,308.0448687)(128.0546875,307.65814995)(127.8203125,307.23041557)
\curveto(127.59179688,306.80854057)(127.33398438,306.46283745)(127.046875,306.1933062)
\curveto(126.56640625,305.7245562)(125.97167969,305.40814995)(125.26269531,305.24408745)
\curveto(124.88183594,305.15033745)(124.45117188,305.10346245)(123.97070312,305.10346245)
\curveto(122.79882812,305.10346245)(121.80566406,305.52826713)(120.99121094,306.37787651)
\curveto(120.17675781,307.23334526)(119.76953125,308.42865776)(119.76953125,309.96381401)
\curveto(119.76953125,311.47553276)(120.1796875,312.70307182)(121,313.6464312)
\curveto(121.8203125,314.58979057)(122.89257812,315.06147026)(124.21679688,315.06147026)
\closepath
\moveto(126.73925781,310.89545463)
\curveto(126.67480469,311.58100151)(126.52539062,312.12885307)(126.29101562,312.53900932)
\curveto(125.85742188,313.30072807)(125.13378906,313.68158745)(124.12011719,313.68158745)
\curveto(123.39355469,313.68158745)(122.78417969,313.41791557)(122.29199219,312.89057182)
\curveto(121.79980469,312.36908745)(121.5390625,311.70404838)(121.50976562,310.89545463)
\lineto(126.73925781,310.89545463)
\closepath
\moveto(124.06738281,315.07904838)
\lineto(124.06738281,315.07904838)
\closepath
}
}
\end{pspicture}

    \caption{Desired Hole vs Actual Hole}
  \end{figure}
\end{frame}

\section{Overhangs}
\begin{frame}
  \frametitle{Overhangs}
  For some types of printers (typically those that fuse a powder), overhangs and unsupported sections are not a problem.  For the rest, something may need to be done.
\end{frame}

%
% Need some figures showing overhangs
%

\end{document}
